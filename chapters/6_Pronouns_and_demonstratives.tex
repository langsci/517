\chapter{Pronouns and demonstratives}
\begin{sloppypar}
    
This chapter lays out the pronominal system of Tekht H.\il{Hewramî!Tekht} It lists different forms of pronouns and gives a detailed discussion of their functions. 

\section{Personal pronouns} \label{sect:perso-pro}
Free personal pronouns consist of speech act pronouns\is{speech act pronouns} and third-person pronouns. The latter are subdivided into anaphoric and anaphoric-demonstrative sets (see \S\ref{sect:3perspronouns}). Independent forms of pronouns occur less frequently than bound pronouns and zero anaphora. Their use is limited to specific contexts. For instance, contrastive topic constructions favour free pronouns, as illustrated in (\ref{ex.contr-top}).

\ea \label{ex.contr-top}

\ea[]{
\textit{ađê be zorêş kera.} \\ 
\gll \textbf{ađê} be zor-ê=ş ker-a \\ 
 \textsc{3pl} by force\textsc{-pl.dir=3sg:O} do\textsc{.prs.ind-3pl:A} \\  
\glt `\textbf{They} do it (the job) by force.'
}
\ex[]{
\textit{min be aqiɫîş kerû.} \\
\gll \textbf{min} be aqiɫî=ş ker-û  \\
\textsc{1sg} by wisdom\textsc{.m=3sg:O} do\textsc{.prs.ind-1sg:A} \\
\glt `[However] \textbf{I} do it (the job) by wisdom.' \hfill [ME.93]--[ME.94]
}
\z
\z
\subsection{Speech act pronouns\is{speech act pronouns}}\label{sect:SAP}
Speech act pronouns\is{speech act pronouns} (SAPs) comprise 1st and 2nd person pronouns. They are phonologically non-bound, stress-bearing elements derived from Old Iranian genitive/dative pronouns.\footnote{SAP pronouns still carry on the genitive usage when expressing a possessor in an adnominal possession construction (see \ref{ex.minposessor}). Their use in expressing non-canonical subjects reflects their dative origin (see \ref{ex.emenonc}). Their use in expressing transitive subjects in the past tense reflects the rise of ergativity in Iranian (see \citealt[]{haig_alignment_2008} for details).} The SAP pronouns have lost case distinctions\is{case distinction}, meaning that the same form of the pronoun is used for expressing direct\is{direct case} and oblique-case-marked\is{oblique case} arguments. \tabref{tab:SAPs} exhibits Hewramî\il{Hewramî} SAPs.

\begin{table}[htp]
    
    \begin{tabular}{ll}
    \lsptoprule
person/number & form \\
\midrule
\textsc{1sg}& \textit{min} \\
\textsc{2sg}& \textit{to} \\
\textsc{1pl}& \textit{ême, hême} \\
\textsc{2pl}& \textit{êşme, şime} \\
\lspbottomrule
    \end{tabular}
    \caption{Speech act pronouns\is{speech act pronouns}}
    \label{tab:SAPs}
\end{table}

SAP pronouns express core arguments of the verb, including the intransitive subjects (S), transitive objects (O), transitive subjects (A), and non-canonical subjects\is{non-canonical subjects}. In their function of marking objects of a present-tense verb, SAP pronouns are in complementary distribution with clitic pronouns. In the rest of the functions, they may co-occur with the agreement markers in the same local domain. The use of free SAP pronouns is triggered by contrast and focus\is{focus}.

\ea
S \\
\textit{ême milmê şeşik.} \\ 
\gll ême mi-l-mê şeşik \\ 
 \textsc{1pl} \textsc{ind-}go\textsc{.prs-1pl:S} \textsc{pn} \\ 
\glt `We will go to Shashk.' \hfill [HB.32]
\z 


\ea
A \\
\textit{min şûyş kerû pene.} \\ 
\gll min şû-î=ş ker-û pene \\ 
\textsc{1sg} husband\textsc{.m-sg.obl=3sg:R} do\textsc{.prs.ind-1sg:A} to \\  
\glt `I will marry him.' \hfill [JH.59]
\z 



\ea
O-prs \\
\textit{toyç bera yane.} \\ 
\gll to=îç ber-a yane \\ 
 \textsc{2sg=add} take\textsc{.prs.ind-3pl:A} house\textsc{.m.sg.dir} \\ 
\glt `they will take you to [their] house.' \hfill [HB.38] \label{ex.pro-o.prs}
\z 


\ea
O-past \\
\textit{vatiş, `qurban! qesem pa xway toş epî layiqetî weş kerdenî!'} \\ 
\gll vat=iş qurban qesem p=a xwa-î to=ş e=p=î layiqetî weş kerde=nî \\ 
 say\textsc{.pst=3sg:A} sir.\textsc{voc} oath\textsc{.m} to=\textsc{dem.dist} God\textsc{.m-sg.obl} \textsc{2sg=3sg:A} \textsc{emph=adp=dem.prox} worthiness\textsc{.m} good do\textsc{.pst.ptcp.m=cop.2sg:O} \\ 
\glt `He said, ‘Sir, I swear to God, who has made you with such virtue!’' \label{ex.pro-o.pst} \\ \hfill [ZQ.54] 
\z 


\ea \label{ex.emenonc}
Non-canonical subject\is{non-canonical subjects} \\
\textit{persa, `ême yanew fiɫane kesîma gereka.'} \\ 
\gll pers-a ême yane-û fiɫan-e kes-î=ma gerek=a \\ 
 ask\textsc{.prs.ind-3pl:A} \textsc{1pl} house\textsc{.m-ez.gen} such\_and\_such\textsc{-ez.cmpd} person\textsc{.m-sg.obl=1pl:NC} necessary\textsc{.m=cop.3sg.m:S} \\ 
\glt `[and] asked [people], ‘We are looking for (lit. we want) the house of such-and-such a person.’' \hfill [ZP.84]
\z 

\largerpage
As mentioned, SAP pronouns have lost case distinctions\is{case distinction}. It is, however, notable that they have retained some of the properties of their genitive/dative origin, most notably in subordinate clauses\is{subordinate clauses}. In the following examples, as a reflex of their original syntax, the SAP pronouns are the sole means of expressing an A-past argument.

\ea
\textit{î gîre çêş bî min ward?} \\ 
\gll î gîr=e çêş bî-∅ \textbf{min} ward-∅ \\ 
 \textsc{dem.prox} hook\textsc{.m=dem} what be\textsc{.pst-3sg.m:S} \textsc{1sg} eat\textsc{.pst-3sg.m:O} \\ 
\glt `What is this situation that I am caught in? [Lit. What is this hook that I ate?]' \hfill [HB.23]
\z 


\ea
\textit{maço, `be xwa pađşam seɫamet kuřû şuwaney heke to kuştêw a wextîyekey hukmit daw înîşa îse îna yaneta înîşa.'} \\ 
\gll m-aç-o be xwa pađşa=m seɫamet kuř-û şuwane-î heke \textbf{to} \textbf{kuşt-ê}=û a wext-î-ekey hukm=it da-∅=û înîşa îse îna-∅ yane=ta {} înîşa \\ 
\textsc{ind-}say\textsc{.prs-3sg:A} by God\textsc{.m} king\textsc{.m=1sg:PSR} healthy son\textsc{.m-ez.gen} shepherd\textsc{.m-sg.obl} \textsc{rel} \textsc{2sg} kill\textsc{.pst-cond}=and \textsc{dem.dist} time\textsc{.m-sg.obl-def.m.sg.obl} rule\textsc{.m=2sg:A} give\textsc{.pst-3sg.m:O}=and \textsc{dem.prox.3pl.obl} now \textsc{loc.deic.cop-3sg.m:S} house\textsc{.m=2pl:PSR} {} \textsc{dem.prox.3pl.obl} \\  
\glt `They said [in the letter], ‘Indeed, may my king be healthy, the shepherd’s son who you [ordered to be] killed is now in your house and so on.’' \\ \hfill [KŞ.51]
\z 


\ea
\textit{ane key bê dizî şime kerđen!} \\
\gll ane key b-ê dizî şime kerđe-n \\
\textsc{dem.dist} how be.\textsc{prs-aug.3sg} theft \textsc{2pl} do.\textsc{pst.ptcpl.m=cop.3sg.m:O} \\
\glt `How could what you have committed be considered a theft?' \hfill [ED.107]
\z 

In the same way, the SAP pronouns are the sole way of expressing the agent in clauses with OAV word order\is{word order}, where the focus\is{focus} is on the A-past argument (see \S\ref{sect:differential-A-indexing} for an overview of differential indexing of A-past arguments\is{differential A indexing}). 

\ea
\textit{werêsekê min warđêne} \\
\gll werêse-(e)kê m\`in warđê=ne \\
rope-\textsc{def.f.sg} \textsc{1sg} eat.\textsc{pst.ptcp.f=cop.3sg.f:O} \\
\glt `[The lion said] I have eaten the rope.' \hfill [ÇH.85]
\z 

Similarly, \textit{contrastive topic} constructions may trigger a lack of indexing with an SAP pronoun. Following \citet{dik_typology_1981}, contrastive topic constructions are those constructions where a contrastive parallel is set up between the two subjects. 

\ea
\textit{m\`{i}n rencim daw t\`{o} berđ} \\
\gll min řenc=im da=w to berđ \\
\textsc{1sg} toil=\textsc{1sg:A} give.\textsc{pst}=and \textsc{2sg} take.\textsc{pst} \\
\glt `I toiled, and you took [the credit].' \hfill [YX.16]
\z

SAP pronouns also express non-core arguments, including adpositional complements, non-flagged indirect objects, and possessors. In marking these functions, as is the case with the direct object of a present tense construction, as exemplified in (\ref{ex.pro-o.prs}), SAP pronouns are in complementary distribution with bound pronouns.The choice of free over bound pronouns to express these functions is triggered by factors such as contrast and focus\is{focus}.

\ea
Direct object \\
\textit{toyç bera yane.} \\ 
\gll to=îç ber-a yane \\ 
 \textsc{2sg=add} take\textsc{.prs.ind-3pl:A} house\textsc{.m.sg.dir} \\ 
\glt `They will take you to [their] house.' \hfill [HB.38]
\z 
%

\ea
Adpositional complement \\
\textit{fermawo ađîç, `weɫa min pey şime girewû.'} \\ 
\gll fermaw-o ađ=îç weɫa min pey şime girew-û \\ 
 say\textsc{.prs.ind-3sg:A} \textsc{3sg.m.dir=add} indeed \textsc{1sg} for \textsc{2pl} cry\textsc{.prs.ind-1sg:S} \\  
\glt `He said, ‘I’m crying for you.’' \hfill [BP.155]
\z 
%

\ea
Non-flagged indirect object \\
\textit{nanîç miđa to.} \\ 
\gll nan=îç mi-đ(e)-a to \\ 
 bread\textsc{.m=add} \textsc{ind-}give\textsc{.prs-3pl:A} \textsc{2sg} \\  
\glt `They will give you a meal.' \hfill [HB.40]
\z 
%

\ea \label{ex.minposessor}
Possessor \\
\textit{ane kuřû şuwaneyn serû esbekey to.} \\ 
\gll ane kuř-û şuwane-î=n ser-û esb-ekey to \\ 
\textsc{dem.dist.m.3sg.dir} son\textsc{.m-ez.gen} shepherd\textsc{.m-sg.obl=cop.3sg.m:S} on\textsc{-ez.gen} horse\textsc{-def.m.sg.obl} \textsc{2sg} \\ 
\glt `Look, the shepherd’s son [is] on your horse.' \hfill [KŞ.100]
\z 

\subsection{Third person pronouns} \label{sect:3perspronouns}
The third person pronouns are inflected for case\is{case} and gender\is{gender}. They may be divided into three groups according to their forms: unmarked anaphoric forms, anaphoric demonstratives\is{anaphoric demonstratives}, and independent demonstrative pronouns\is{independent demonstrative pronouns} (\S\ref{sect:ind-dem-pro}). In each of these sets, they come in proximal and distal forms. \tabref{tab:3perspron-remote} and \tabref{tab:3perspron-proximate} lay out unmarked third-person anaphoric pronouns.
\begin{table}[htp]
    \begin{tabular}{lll}
 \lsptoprule
& \textsc{dir}& \textsc{obl} \\ 
\midrule
\textsc{m}&\textit{ađ} &\textit{ađî} \\
\textsc{f}& \textit{ađe}& \textit{adê}\\
\textsc{pl}& \textit{ađê}&\textit{ ađîşa} \\
 \lspbottomrule
    \end{tabular}
    \caption{Anaphoric third person pronouns, the distal set}
    \label{tab:3perspron-remote}
\end{table}

\begin{table}[htp]
    
    \begin{tabular}{lll}
 \lsptoprule
& \textsc{dir}& \textsc{obl} \\ 
\midrule
\textsc{m}&\textit{êđ} &\textit{êđî} \\
\textsc{f}& \textit{êđe}& \textit{êdê}\\
\textsc{pl}& \textit{êđê}&\textit{êđîşa} \\
 \lspbottomrule
    \end{tabular}
    \caption{Anaphoric third person pronouns, the proximal set}
    \label{tab:3perspron-proximate}
\end{table}

The factors conditioning the use of anaphoric pronouns as opposed to other sets of pronouns are not entirely clear. Conversely, as seen in what follows and in \S\ref{sect: dempro}, these three sets may overlap in their functional scope and the types of referents they may encode. However, some tendencies can be outlined here. While anaphoric pronouns track participants that are established and continuing topics (see below), anaphoric demonstratives (most notably in their distal set) establish topic shift. They may be used in contrastive opposition with other nominals. In their use in establishing new topics, anaphoric demonstrative pronouns (partly) overlap with independent demonstrative pronouns (See \S\ref{sect:ind-dem-pro}).

Anaphoric third-person pronouns track participants that are established as topics. This means that they track participants after their second mention. In the following excerpt, the participant \textit{kuřeke}, tracked by \textit{ađ}, has already been mentioned once in the discourse.

\ea
\textit{kuřeke yoşa weşkewte bê kuřû. ađma tehwêɫ gêrt.} \\ 
\gll \textbf{kuř-eke} yo=şa weş kewte b-ê kuř=û \textbf{ađ}=ma tehwêɫ gêrt-{\O}\\ 
 son\textsc{.m-def.m.sg.dir} one\textsc{.m=3pl} good fall\textsc{.pst.ptcp.m} be\textsc{.prs-aug.3sg} son\textsc{.m}=and \textsc{3sg.m.dir}\textsc{=1pl:A} delivery\textsc{.m} grab\textsc{.pst-3sg.m:O}\\ 
\glt `One of the newborn \textbf{boy}s was healthy. We got hold of \textbf{him}.' \\ \hfill [ZQ.17]--[ZQ.18]
\z 

Similarly, anaphoric third-person pronouns express continuing topics. In (\ref{ex.pers-pronouns1})--(\ref{ex.pers-pronouns2}), \textit{ađê} and \textit{ađ} track the topic of the preceding clause as the antecedent.

\ea \label{ex.pers-pronouns1}
\textit{a wextî milarewe maça, `wiɫa î meʕmûrû şime kîyasêbênêta pey îne, înema ane heke şime gerekbê d{ɛ}ma penew. ađê nam{ɛ}w; xerîkû genekarî bênê.'} \\ 
\gll a wext-î mi-l-a=re=we m-aç-a wiɫa î \textbf{meʕmûr}-û şime kîyasê=b-ên-ê=ta pey îne îne=ma ane heke şime gerek b-ê d{ɛ}=ma pene=û \textbf{ađê} n(e)-am{ɛ}=û xerîk-û genekarî b-ên-ê\\ 
 \textsc{dem.dist} time\textsc{.m-sg.obl} \textsc{ind-}go.\textsc{prs-3pl:S=povb=compl} \textsc{ind-}say\textsc{.prs-3pl:A} by\_God \textsc{dem.prox} officer\textsc{.m-ez.gen} \textsc{2pl} send\textsc{.pst.ptcp.pl}=be\textsc{.prs-aug-3pl:O=2pl:A} for \textsc{dem.prox.m.3sg.dir} \textsc{dem.prox.m.3sg.dir=1pl:A} \textsc{dem.dist.m.3sg.dir} if \textsc{2pl} necessary be\textsc{.prs-aug.3sg:S} give\textsc{.pst.3pl:O=1pl:R} to=and \textsc{3pl.dir} \textsc{neg-}come\textsc{.pst.3pl:S}=and busy\textsc{-ez.gen} debauchery\textsc{.m} be\textsc{.prs-aug-3pl:S}\\ 
\glt `Then, they went [to the agha and] said, ‘Indeed, \textbf{the officers} whom you had sent to us, whatever [ taxation] you had asked for, we gave them. [However] \textbf{they} didn’t come back [to you]. They engaged in debauchery.’' \\ \hfill [BP.124]--[BP.125] 
\z 


\ea
\textit{řisq pêsen miɫey şikɫiş pisen î miɫey welê êtir ađ çêw dizo.} \\ 
\gll \textbf{řisq} pêse=n miɫe-î şikɫ=iş pise=n î miɫe-î welê êtir \textbf{ađ} çêw diz-o \\ 
 rat like\textsc{cop.3sg.m:S} mouse\textsc{-sg.obl.m} complexion\textsc{=3sg:PSR} like\textsc{cop.3sg.m:S} \textsc{dem.prox} mouse\textsc{-sg.obl.m} mouse\textsc{-sg.obl.m} but \textsc{disc.ptcl} \textsc{3sg} thing steal\textsc{.prs.ind-3sg:A}\\  
\glt `A \textbf{rat}\textsubscript{\textit{i}} is like a mouse; \textbf{it}\textsubscript{\textit{i}} has the same complexion as a rat, but \textbf{it}\textsubscript{\textit{i}} steals things.' \hfill [DP.41] \label{ex.pers-pronouns2}
\z 

Another set of third-person pronouns are anaphoric demonstratives\is{anaphoric demonstratives}, which come in proximal and distal forms; see \tabref{tab:3proprox} and \tabref{tab:3proremote}. Anaphoric demonstratives\is{anaphoric demonstratives} are distinguished from exophoric demonstratives discussed in \S\ref{sect: dempro}. They are used to track antecedents in previous discourse. Anaphoric demonstratives\is{anaphoric demonstratives} are differentiated from third-person pronouns in the type of referents they can track as antecedents.

\begin{table}[htp]
    
    \begin{tabular}{lll}
 \lsptoprule
Proximal& \textsc{dir}& \textsc{obl} \\ 
\midrule
\textsc{m}&\textit{îđ} &\textit{îđî} \\
\textsc{f}& \textit{îđ(e)}& \textit{îđê} \\
\textsc{pl}& \textit{îđê}& \textit{îđîşa} \\
 \lspbottomrule
    \end{tabular}
    \caption{Anaphoric demonstratives\is{anaphoric demonstratives}, the proximal set}
    \label{tab:3proprox}
\end{table}


\begin{table}[htp]
    
    \begin{tabular}{lll}
 \lsptoprule
Distal & \textsc{dir}& \textsc{obl} \\ 
\midrule
\textsc{m}&\textit{ew} &\textit{ewî}\\
\textsc{f}& \textit{ewe}& \textit{ewê} \\
\textsc{pl}& \textit{ewê, ewêşa}& \textit{ewîşa} \\
 \lspbottomrule
    \end{tabular}
    \caption{Anaphoric demonstratives\is{anaphoric demonstratives}, the distal set}
    \label{tab:3proremote}
\end{table}

Anaphoric demonstratives\is{anaphoric demonstratives} in \tabref{tab:3proprox} and \tabref{tab:3proremote} may reactivate a referent that has occurred some distance in the previous discourse. In the following example, \textit{îđ} and \textit{ew} track referents in a prior discourse. The form \textit{îđ} is used anaphorically for the child who has just died, and \textit{ew} is used anaphorically to refer to the child that the main character left behind in a tree hollow some months earlier. The proximal form expresses more emotional engagement and mental proximity.

\ea
\textit{îđîç luwaw ewîç luwa.} \\ 
\gll \textbf{îđ}=îç luwa=û \textbf{ew}=îç luwa \\ 
 \textsc{3sg.prox.dir.m=add} go\textsc{.pst.3sg:S}=and \textsc{3sg.m.dir=add} go\textsc{.pst.3sg:S} \\ 
\glt `\textbf{He} (the child whom I took with me) died; \textbf{he} (the child whom I left behind) died too!' \hfill [ZQ.32]
\z 

In the following examples, the proximal pronoun is used anaphorically to refer to a nominal at the centre of attention at a particular point in the discourse.

\ea
\textit{milo wextê miđyo zaroɫeke bîyen peřû qiřoɫekey. a wextîyekey îđîçşa arden.} \\ 
\gll mi-l-o wext-ê mi-đy(e)-o \textbf{zaroɫe-(e)ke} bîye=n peř-û qiřoɫ-ekey a wext-î-ekey \textbf{îđ}=îç=şa arde=n\\ 
 \textsc{ind-}go\textsc{.prs-3sg:S} time\textsc{.m-indf} \textsc{ind-}look\textsc{.prs-3sg:S} child\textsc{-def.m.sg.dir} be\textsc{.pst.ptcp.m=cop.3sg.m:S} full\textsc{-ez.gen} hollow\textsc{.m-def.m.sg.obl} \textsc{dem.dist} time\textsc{.m-sg.obl-def.m.sg.obl} \textsc{3sg.m.dir=add=3pl:A} bring\textsc{.pst.ptcp.m=cop.3sg.m:O} \\  
\glt `The father went [closer and] noticed that \textbf{the baby boy} [had grown up so much that] he had filled the tree hollow. They (the family) brought \textbf{him} back.' \hfill [ZB.49]--[ZB.50] 
\z 


\ea
\textit{îđîç hîç minîç hîçim nîyarê.} \\ 
\gll îđ=îç hîç min=îç hîç=im nîy(e)=a=rê\\ 
 \textsc{3sg.prox.dir.m=add} nothing \textsc{1sg=add} nothing\textsc{=1sg:NC} \textsc{neg.exist=cop.3sg.m:S=povb} \\ 
\glt `She [has] nothing, nor do I.' \hfill [ZP.117]
\z 

Anaphoric demonstratives\is{anaphoric demonstratives} may be used in contrastive opposition with other nominals. In (\ref{ex.contrast}), \textit{ew} is set up in contrastive opposition with \textit{yo} `one':

\newpage
\ea
\textit{yoşa gêɫowe milo dimawe. ewşa her mêş çene. bêheya bo. mêş çene.} \\ 
\gll \textbf{yo}=şa gêɫ-o=we mi-l-o dima=we \textbf{ew}=şa her m-ê=ş çene bê-heya b-o m-ê=ş çene\\ 
  one\textsc{.m=3pl:PSR} wander\textsc{.prs.ind-3sg:S=compl} \textsc{ind-}go\textsc{.prs-3sg:S} back\textsc{=post} \textsc{3sg.m.dir=3pl:PSR} \textsc{emph} \textsc{ind-}come\textsc{.prs.3sg:S=3sg:R} with shameless be\textsc{.prs.ind-3sg:S} \textsc{ind-}come\textsc{.prs.3sg:S=3sg:R} with \\ 
\glt `\textbf{One} of them returned [and] went back. \textbf{The other} one was still coming with her. He was shameless. He kept coming with her.' \hfill [ZP.71]--[ZP.72] \label{ex.contrast}
\z 

Anaphoric demonstratives\is{anaphoric demonstratives} may refer to a referent that is not the focus of attention. In this use, they establish new discourse topics. Therefore, the topic of the preceding clause is excluded as a referent. This is different from anaphoric pronouns, seen in (\ref{ex.pers-pronouns1})--(\ref{ex.pers-pronouns2}), which express continuing topics \citep{himmelmann_demonstratives_1996}, which alternatively may be expressed by definite nouns and zero anaphora.

In the following examples, the reference of \textit{ewê} in (\ref{ex.anaph-dem1}), \textit{ew} in (\ref{ex.anaph-dem2}), and \textit{îđ} in (\ref{ex.anaph-dem3}) is not the same as the topics in the previous clause. In other words, the anaphoric demonstratives\is{anaphoric demonstratives} mark a shift of topic. 


\ea
\textit{ħemey ẍeybîyû baba xuađa gina weɫêwe. ewê şûnîşare mila. êtir her werew mila řarê.} \\ 
\gll ħeme-î ẍeybî=û baba xuađa gin-a weɫê=we \textbf{ewê} şûnî=şa=re mi-l-a êtir her wer=ew mi-l-a řa=rê \\ 
\textsc{pn-ez.attr} invisible=and \textsc{pn} \textsc{pn} fall\textsc{.prs.ind-3pl:S} front\textsc{=post} \textsc{3pl.dir} track\textsc{.f=3pl:PSR=post} \textsc{ind-}go\textsc{.prs-3pl:S} \textsc{disc.ptcl} just front\textsc{.m=post} \textsc{ind-}go\textsc{.prs-3pl:S} road\textsc{.f=post} \\ 
\glt `Hama the Invisible and Baba Khwada\textsubscript{\textit{i}} went in front of them. \textbf{They}\textsubscript{\textit{j}} (the people) followed them. They\textsubscript{\textit{i}} kept walking.' \hfill [BP.98]--[BP.99] \label{ex.anaph-dem1}
\z 


\ea
\textit{hereke mêwe cûwab î qisa kero. ewîç goş miđo pene.} \\ 
\gll her-eke m-ê=we cûwab î qis(ê)-a ker-o \textbf{ew}=îç goş mi-đ(e)-o pene \\ 
 donkey\textsc{.m-def.m.sg.dir} \textsc{ind-}come\textsc{.prs.3sg:S=compl} response\textsc{.m} \textsc{dem.prox} talk\textsc{.f-pl.obl} do\textsc{.prs.ind-3sg:A} \textsc{3sg.m.dir=add} ear\textsc{.m} \textsc{ind-}give\textsc{.prs-3sg:A} to \\
\glt `The donkey started to speak; it spoke. \textbf{He} listened to it [i.e., the donkey].' \\ \hfill [HB.44]--[HB.45] \label{ex.anaph-dem2}
\z 

\newpage

\ea
\textit{î birame derdeş, derdêş îna po derdû gulîş maça dađê! îđîç fermawo, luwe ça nîşere.} \\ 
\gll î bira=m=e derde=ş derd(e)-ê=ş îna-∅ p=o derd(e)-û gulî=ş m-aç-a dađ-ê \textbf{îđ}=îç fermaw-o lu-e ça n\stackunder[-10pt]{\^{e}}{\'{}}ş-e=re \\
\textsc{dem.prox} brother\textsc{.m=1sg:PSR=dem} illness\textsc{.f=3sg:R} illness\textsc{.f-indf=3sg:R} \textsc{loc.deic.cop-3sg.m:S} at\textsc{=post} illness\textsc{.f-ez.gen} leprosy\textsc{=3sg:R} \textsc{ind-}say\textsc{.prs-3pl:A} yell\textsc{.m-pl.dir} \textsc{3sg.prox.dir.m=add} say\textsc{.prs.ind-3sg:A} go.\textsc{prs.imp-2sg:S} there sit\textsc{.prs.imp-2sg:S=povb} \\  
\glt `[The man said] `This brother of mine is ill; it is called leprosy; please help!’' \textbf{He} (Sheikh Aladin) said, ‘Go [and] sit there.' \hfill [DG.43]--[DG.44] \label{ex.anaph-dem3}
\z 

%
The choice between the anaphoric pronouns and anaphoric demonstrative pronouns is evident in the following excerpts. The former tracks an already-established topic. The latter expresses a topic shift and new information.

\ea
\textit{ta padşa nebîyen ađ padşaş bîyen. a wextîyekey îđîç aman dawaw padşagerîş kerdêne.} \\ 
\gll ta padşa ne-bîye=n \textbf{ađ} padşa=ş bîye=n a wext-î-ekey \textbf{îđ}=îç ama=n dawa-û padşagerî=ş kerdê=ne \\ 
 until king\textsc{.m} \textsc{neg-}be\textsc{.pst.ptcp.m=cop.3sg.m:S} \textsc{3sg.m.dir} king\textsc{.m=3sg:PSR} be\textsc{.pst.ptcp.m=cop.3sg.m:S} \textsc{dem.dist} time\textsc{.m-sg.obl-def.m.sg.obl} \textsc{3sg.prox.dir.m=add} come\textsc{.pst.ptcp.m=cop.3sg.m:S} demand\textsc{.f-ez.gen} kingship\textsc{.m=3sg:A} do\textsc{.pst.ptcp.f=cop.3sg.f:O} \\
\glt `\textbf{He} (Sibhan Agha) was its king until there were no kings. Then, \textbf{he} (i.e., Jamsher Shah) came [and] claimed kingship.' \hfill [DP.16]--[DP.17]
\z 


\ea
\textit{mila ew kuřekey yoyşa bera. aneşa zilterû ʕalter bo ađî bera. ewî minya qiřoɫû darêwe.} \\ 
\gll mi-l-a ew kuř-ekey yo-î=şa ber-a ane=şa zil-ter=û ʕal-ter b-o \textbf{ađî} ber-a \textbf{ewî} mi-ny(e)-a qiřoɫ-û dar-êwe \\ 
 \textsc{ind-}go\textsc{.prs-3pl:S} \textsc{dem.dist} son\textsc{.m-def.m.sg.obl} one\textsc{.m-sg.obl=3pl:PSR} take\textsc{.prs.ind-3pl:A} \textsc{dem.dist.m.3sg.dir=3pl:PSR} big\textsc{-cmpr}=and good\textsc{-cmpr} be\textsc{.prs.ind-3sg:S} \textsc{3sg.obl.m} take\textsc{.prs.ind-3pl:A} \textsc{3sg.obl.m} \textsc{ind-}put\textsc{.prs-3pl:A} hollow\textsc{.m-ez.gen} tree\textsc{.m-indf} \\
\glt `They went away [and took] that son. They took one of them (i.e., the boys), the one who was bigger and healthier; they took \textbf{him}. They left \textbf{him (i.e., the other one)} in the hole in the tree.' \hfill [ZB.40]--[ZB.41]
\z 

Notably, the proximal set of anaphoric pronouns (\tabref{tab:3perspron-proximate}) seems to have merged in function with the anaphoric demonstrative. Therefore, they can express topic shift. Thus, the generalisation mentioned above for the distinction between anaphoric pronouns and anaphoric demonstratives\is{anaphoric demonstratives} in terms of tracking already-established topics vs. expressing topic shift holds mainly for the distal set of anaphoric pronouns (\ref{tab:3perspron-remote}). In the following excerpt, \textit{êđ} features a topic shift construction.

\ea
\textit{î bizêçe powe awêzane bo mê eçê êđ çeknoşû milo la yagekêş.} \\ 
\gll î bizê=ç=e p=o=we awêzan-e b-o m-ê e=çê \textbf{êđ} çekn-o=ş=û mi-l-o la yag(ê)-ekê=ş \\ 
 \textsc{dem.prox} goat\textsc{.f.sg.obl=add=dem} at=\textsc{dem.dist=post} hanging\textsc{-f} be\textsc{.prs.ind-3sg:S} \textsc{ind-}come\textsc{.prs.3sg:S} in=here \textsc{3sg.prox.dir} suckle\textsc{.prs-3sg:A=3sg:O}=and \textsc{ind-}go\textsc{.prs-3sg:S} to place\textsc{.f-def.f.sg=3sg:PSR} \\ 
\glt `The goat would stay in a hanging position, he (the child) would feed from its udder, and it would go back to its place.' \hfill [ZB.54]
\z 

\section{Clitic pronouns} \label{sect:cliticpers}
Tekht H.\il{Hewramî!Tekht} features a set of bound person pronouns which I call \textsc{clitic pronouns}. The clitic pronouns are derived from non-nominative (accusative and genitive/dative) clitic sets of Old Iranian with identifiable cognates in Old Indic \citep[]{korn_western_2009}{}. The paradigm is presented in \tabref{tab:persclitics}. The plural set is formed by adding the oblique\is{oblique case} plural suffix \textit{-a} to the singular\is{singular} set.
\begin{table}[htp]
    
    \begin{tabular}{ll}
    \lsptoprule
person/number & form \\
\midrule
\textsc{1sg}& \textit{=m} \\
\textsc{2sg}& \textit{=t} \\
\textsc{3sg}& \textit{=ş} \\
\textsc{1pl}& \textit{=ma} \\
\textsc{2pl}& \textit{=ta} \\
\textsc{3pl}& \textit{=şa} \\
\lspbottomrule
    \end{tabular}
    \caption{Clitic pronouns}
    \label{tab:persclitics}
\end{table}

In what follows, the functionality of clitic pronouns (\S\ref{sect:clitic-function}) and their morphosyntactic behaviour (\S\ref{clitic-behaviour}) is discussed. In terms of the former, the single paradigm of clitic pronouns expresses a multitude of clausal and phrasal arguments, sometimes giving rise to subtleties in terms of the agreement vs. anaphora distinction. In terms of morphosyntactic behaviour, clitic pronouns feature typical properties of clitichood, including mobility, freedom of host selection, wide scope over coordination, and realisation external to person suffixes. 

\subsection{Functions} \label{sect:clitic-function}
Clitic pronouns assume different functions, both phrasal and clausal. The phrasal functions include indexing possessors (\ref{ex.clc-func1}), and adposition complements (\ref{ex.clc-func2}) (see \citealt[365--372]{mohammadirad_pronominal_2020} for a brief description of clitic pronouns in Tekht H.\il{Hewramî!Tekht}). In indexing these functions, the clitic pronouns are in complementary distribution with coreferent NPs.

\ea Possessor\\
\textit{yaneş çikon?} \\ 
\gll yane=\textbf{ş} çiko=n\\ 
 house\textsc{.m=3sg:PSR} where\textsc{=cop.3sg.m:S} \\ 
\glt `Where is \textbf{his} house?' \hfill [ZP.85] \label{ex.clc-func1}
\z 


\ea Adposition complement\\
\textit{a esbî zînî kere peym.} \\ 
\gll a esb-î zînî k\'er-e pey=\textbf{m} \\ 
 \textsc{dem.dist} horse\textsc{-m.sg.obl} saddle do\textsc{.prs.imp-2sg:A} for\textsc{=1sg:R} \\ 
\glt `Saddle up the horse for \textbf{me}.' \hfill [ŞC.52] \label{ex.clc-func2}
\z  

Clitic pronouns also assume clausal functions, including indexing the direct object argument of a verb built from the present stem (\ref{ex.clc-func3}), the transitive subject of a past-stem verb, or an A-past (\ref{ex.clc-func4}), and non-flagged indirect objects (\ref{ex.clc-func5}).

\ea O-prs\\
\textit{wizoşare dilû awê.} \\ 
\gll wiz-o=\textbf{şa}=re dil-û awê \\ 
 throw\textsc{.prs.ind-3sg:A=3pl:O=povb} inside\textsc{-ez.gen} water\textsc{.f.sg.obl} \\  
\glt `She delivered \textbf{them} in the water [which had amassed in the tent].'\label{ex.clc-func3} \\ \hfill [ZB.25] 
\z


\ea A-past\\
\textit{ađma tehwêɫ gêrt.} \\ 
\gll ađ=\textbf{ma} tehwêɫ gêrt-∅ \\ 
\textsc{3sg.m.dir=1pl:A} delivery\textsc{.m} grab\textsc{.pst-3sg.m:O} \\ 
\glt `We got hold of him.' \hfill [ZQ.18] \label{ex.clc-func4}
\z 


\ea Non-flagged indirect object\\
\textit{kabra melo waço, `jenîm de!'} \\ 
\gll kabra me-l-o w\'aç-o jenî=\textbf{m} d\'e-(e) \\ 
 man \textsc{neg.ind-}go\textsc{.prs-3sg:S} say\textsc{.prs.sbjv-3sg:A} woman\textsc{.f=1sg:R} give\textsc{.prs.imp-2sg:A} \\ 
\glt `The fellow wouldn’t go [to his parents and] say, ‘Find a girl for me! [Lit. Give \textbf{me} a woman.]’' \hfill [RE.29] \label{ex.clc-func5}
\z 

Clitic pronouns also index non-canonical (or experiencer) subjects\is{non-canonical subjects}. The term \textsc{non-canonical subjects}\is{non-canonical subjects} roughly refers to those subject-like arguments which have some subject properties, e.g., [+ human], but which exert a low level of control over the event of the verb and are marked differently from normal subjects \citep[see][]{onishi_introduction:_2001}. Non-canonical subjects\is{non-canonical subjects} are limited to certain predicate types and centre around certain semantic domains, including (i) possession/existence, (ii) psychological states, (iii) physiological states, (iv) visual/auditory perceptions, (v) modal states of necessity and wanting, (vi) modal states of potentiality, and (vii) uncontrolled events (see \citealt[312]{shibatani_non-canonical_2001} and \citealt[]{hagege_pour_2006}).
In Hewramî\il{Hewramî}, non-canonical subject constructions\is{non-canonical subject constructions} are found in the syntactic expression of possession (\ref{ex.clc-func6}), expression of wanting and necessity (\ref{ex.clc-func7}), and non-controlled internal physical and emotional states, roughly equivalent to ``physiological states" in \citet[]{shibatani_non-canonical_2001},{} (\ref{ex.clc-func8})--(\ref{ex.clc-func11}). In terms of indexing, the non-canonical subjects\is{non-canonical subjects} are distinguished from regular clausal subjects (S and A) in being indexed by clitic pronouns regardless of the tense and the transitivity of the clause.

\newpage

\ea Predicative possession\\
\textit{tifengiş henû îneş henû aneş hen!} \\ 
\gll tifeng=\textbf{iş} hen-∅=û îne=\textbf{ş} hen-∅=û ane=\textbf{ş} {} hen-∅ \\ 
 gun\textsc{=3sg:NC} \textsc{exist-3sg.m:S}=and \textsc{dem.prox.m.3sg.dir=3sg:NC} \textsc{exist-3sg.m:S}=and \textsc{dem.dist.m.3sg.dir=3sg:NC} {} \textsc{exist-3sg.m:S} \\  
\glt `He has a gun, he has this, and he has that!' \hfill [ŞC.12] \label{ex.clc-func6}
\z


\ea Wanting and necessity\\
\textit{ey înîşa ke minta gerekna!} \\ 
\gll ey înîşa ke min=\textbf{ta} gerek=na \\ 
\textsc{voc} \textsc{dem.prox.3pl.obl} \textsc{rel} \textsc{1sg=2pl:NC} necessary\textsc{.m=cop.1sg:S} \\
\glt `O those who want me!' \hfill [KŞ.56] \label{ex.clc-func7}
\z 


\ea Non-controlled internal physical states\\
\textit{serđşa bîyen.} \\ 
\gll serđ=\textbf{şa} bîye=n \\
 cold=\textsc{3pl:NC} be.\textsc{be.pst.ptcp.m=cop.3sg.m:S} \\
\glt `They were cold.' \label{ex.clc-func8} \hfill [DB.16]
\z 


\ea Non-controlled internal physical states\\
\textit{wermîçim mê.} \\ 
\gll werm=îç=\textbf{im} m-ê\\ 
 sleep\textsc{=add=1sg:NC} \textsc{ind-}come\textsc{.prs.3sg:S} \\ 
\glt `I feel sleepy [lit. my sleep comes].' \hfill [BP.183] \label{ex.clc-func9} 
\z 


\ea Non-controlled internal emotional states\\
\textit{qînîşne weroma.} \\
\gll qînî=\textbf{ş}=ne wer-o=ma \\
rage\textsc{.f=3sg:NC=cop.3sg.f:S} eat.\textsc{prs.ind-3sg:A=1pl:O} \\
\glt `It (the demon) is in a rage. It is going to eat us.'  \hfill [WL.35] \label{ex.clc-func10} 
\z 


\ea Non-controlled internal emotional states\\
\textit{fermawaş, `çêştan?'} \\
\gll fermawa=ş çêş=\textbf{ta}=n \\
say.\textsc{pst=3sg:A} what=\textsc{2pl:NC=cop.3sg.m:S} \\
\glt `He said, `What's [wrong] with you?’' \hfill[BB.09] \label{ex.clc-func11}
\z 

The use of clitic pronouns in indexing the direct object of a verb built on the present stem and non-flagged indirect objects is conditioned to the absence of the co-referent NP. By contrast, clitic pronouns tend to obligatorily index transitive subjects and non-canonical subjects\is{non-canonical subjects}, i.e., they exhibit properties of agreement markers. It is, however, notable that the clitic pronouns have not fully grammaticalised into agreement markers in expressing A-past arguments and non-canonical\is{non-canonical subjects} subjects. The assumption within Iranian linguistics is that in languages like clitic pronouns have retained their pronominal nature and are in complementary distribution with an overtly marked A-past argument. Discussing the development of A-indexing clitics in Iranian, \citet[102--103]{haig2020pronoun} states that ``in Middle Iranian, these subject clitic pronouns were in complementary distribution with free NP subjects; this kind of system is still attested in some West Iranian languages to this day.''\footnote{The conditioned indexing of A-past arguments through clitic pronouns is a conservative feature of Tekht H.\il{Hewramî!Tekht}, attested also in Middle Iranian, certain varieties from the Tati\il{Tati} and Taleshi\il{Taleshi} groups (\citealt[]{mohammadirad_pronominal_2020}{}) and some transitional dialects of Northern Kurdish\il{Kurdish!Northern} bordering Central Kurdish\il{Kurdish!Central} (\citealt[]{opengin_pronominal_2022}{}).} This is borne out by the following excerpt: in the first clause, the clitic is absent in the presence of the overt oblique-marked\is{oblique case} NP. In the second clause, the clitic resumes the absent A argument.

\ea

\ea[]{
\textit{min taze padşay kerdena wekêɫ.} \\ 
\gll min taze \textbf{padşa-î} kerde=na wekêɫ \\ 
\textsc{1sg} anyway king\textsc{.m-sg.obl} do\textsc{.pst.ptcp.m=cop.1sg:O} advocate\textsc{.m} \\  
\glt `I -- the king has put me in charge.'
}
\ex[]{
\textit{watenîçiş, `mişyo neberûşo.'} \\
\gll wate=n=îç=\textbf{iş} mişyo ne-ber-û=ş=o  \\
say\textsc{.pst.ptcp.m=cop.3sg.m:O}\textsc{=add}\textsc{=3sg:A} \textsc{aux} \textsc{neg.sbjv-}take\textsc{.prs-1sg:A=3sg:O}\textsc{=compl} \\
\glt `He (the king) has said [to me], ``You shall not take her back.'’'\\ \hfill [ZP.107]--[ZP.108]
}
\z
\z 

In the following examples, the clitic indexing is missing in the presence of an oblique-marked\is{oblique case} third-person A argument:

\ea
\textit{kesî be noʕê pene nezanan.} \\ 
\gll \textbf{kesî} be noʕ-ê pene ne-zana=n \\ 
no\_one\textsc{-m.sg.obl} to kind\textsc{-indf} to \textsc{neg-}know\textsc{.pst.ptcp.m=cop.3sg.m:R} \\ 
\glt `No one really appreciated him.' \hfill [ZP.24]
\z 

\newpage
\ea
\textit{ađî kîyasen dimawe ama.} \\ 
\gll \textbf{ađî} kîyase=n dima=we ama \\ 
 \textsc{3sg.obl.m} send\textsc{.pst.ptcp.m=cop.3sg.m:O} back\textsc{=post} come\textsc{.pst.3sg:S} \\ 
\glt `He (Pir Shaliyar) had sent it back.'\hfill [JP.115]
\z 


\ea
\textit{meselen ađîşa hukim kerdênmê.} \\ 
\gll meselen \textbf{ađîşa} hukim kerdê=nmê \\ 
 for\_example \textsc{3pl.obl} rule\textsc{.m} do\textsc{.pst.ptcp.pl=cop.1pl:O} \\ 
\glt `For instance, they ruled over us.' \hfill[BP.10]
\z 


\ea
\textit{baba \textbf{xway} da.} \\ 
baba xwa-î da-∅ \\ 
\textsc{pn} God\textsc{.m-sg.obl} give\textsc{.pst-3sg:O} \\ 
\glt `God created [lit. gave] Baba.' \hfill[BP.4]
\z 


\ea
\textit{a dey heşpişa dêndêne.} \\ 
\gll a dey \textbf{heşpiş(î)-a} dê=ndê=ne \\ 
\textsc{emph} \textsc{disc.ptcl} lice\textsc{.f-pl.obl} give\textsc{.pst.ptcp.pl=}\textsc{2pl=povb} \\
\glt `Look, you have gotten lice.' \hfill[BP.156]
\z 

However, there are examples in the main text corpus which clearly illustrate that clitic pronouns can co-occur with subject NPs in the same clause. In \S\ref{sect:differential-A-indexing}, following \citet[]{MohammadiradinreviewAindx}, an alternative account of differential A indexing\is{differential A indexing} in Hewramî\il{Hewramî} is given, according to which the presence of indexing is motivated by the subject NP being topical and the lack of clitic indexing is conditioned by the subject NP exhibiting properties related to focus\is{focus}.

\ea
\textit{xway derdema deʕfeş kerdêne.} \\ 
\gll \textbf{xwa-î} derde=ma deʕfe=\textbf{ş} kerdê=ne \\ 
God\textsc{.m-sg.obl} illness\textsc{.f=1pl:PSR} exclusion\textsc{.f=3sg:A} do\textsc{.pst.ptcp.f=cop.3sg.f:O} \\  
\glt `[and] God would protect us from illness.' \hfill [DG.7]
\z 


\ea
\textit{a wextîyekey çareniwîsî niwîsebêş ...} \\ 
\gll a wext-î-ekey \textbf{çareniwîs-î} niwîse=b-ê=\textbf{ş} \\ 
 \textsc{dem.dist} time\textsc{.m-sg.obl-def.m.sg.obl} fortune\_teller\textsc{.m-sg.obl} write\textsc{.pst.ptcp.m}=be\textsc{.prs-aug.3sg:O=3sg:A} \\ 
\glt `Then, the fortune teller had promised [lit. written] ...' \hfill [KŞ.99]
\z 


\ea
\textit{xo ađ mezano hezretû xosî kîyaseniş.} \\ 
\gll xo ađ me-zan-o \textbf{hezret-û} \textbf{xos-î} kîyase=n=\textbf{iş} \\ 
 \textsc{dis.ptcl} \textsc{3sg.m.dir} \textsc{neg.ind-}know\textsc{.prs-3sg:A} his\_highness\textsc{.m-ez.gen} \textsc{pn-m.sg.obl} send\textsc{.pst.ptcp.m=cop.3sg.m:O=3sg:A} \\ 
\glt `Well, he (Pir Shaliyar) did not realise that His Highness Ghaws had sent him (the man).' \hfill [JP.134]
\z 


\subsection{Morphosyntactic behaviour}\label{clitic-behaviour}
 
Clitic pronouns show properties typical of clitics, including a high level of mobility, freedom of host selection, wide scope over coordination, and occurrence external to affixes \citep{halpern_placement_1995,nevis_clitics_2000}{}.
In terms of scope over coordination, the following examples show that clitic pronouns feature coreferential deletion across coordinated clauses. In all the examples, the clitic appears on the second coordinate verb. Examples (\ref{ex.co-refA1}) to (\ref{ex.co-refA4}) are indicative of the coreferential deletion of A-past clitic pronouns.

\ea
\textit{epare beʕzêş piřnaşre berd wisiş řoxane.} \\ 
\gll e=pare beʕzê=ş piřna=ş=re \textbf{berd-\O} \textbf{wis{-\O}=iş} řoxane \\ 
 in=down\_there some\textsc{=3sg:A} break\_off\textsc{.pst=3sg:A=povb} take\textsc{.pst-3sg.m:O} throw\textsc{.pst-3sg.m:O=3sg:A} river \\ 
\glt `Over there, [the flood] took [and] threw some [animals] into the river.' \label{ex.co-refA1} \\ \hfill [ZB.22] 
\z 


\ea
\textit{yoşa berd nîyamne qiřoɫe dareke qurban.} \\ 
\gll yo=şa \textbf{berd-\O} \textbf{nîya{-\O}=m=ne} qiřoɫ-e dar-eke qurban \\ 
 one\textsc{.m=3pl:PSR} take\textsc{.pst-3sg.m:O} put\textsc{.pst-3sg.m:O=1sg:A=povb} hollow\textsc{.m-ez.cmpd} tree\textsc{.m-def.m.sg.dir} sir.\textsc{voc} \\  
\glt `Sir, I took one [and] put him in the tree hollow.' \hfill [ZQ.26]
\z


\ea
\textit{yewaşê berdê kuştêşa.} \\ 
\gll yewaşê \textbf{berd-ê} \textbf{kuşt-ê=şa} \\ 
 then take\textsc{.pst-3pl:O} kill\textsc{.pst-3pl:O=3pl:A} \\ 
\glt `Then they (the king’s men) took them (the pregnant women) [and] killed them.'  \hfill [KŞ.19]
\z 


\ea
\textit{berđ wistşa zînan.} \\
\gll \textbf{berđ-\O} \textbf{wist-\O=şa} zînan \\
take.\textsc{pst-3sg:O} throw.\textsc{pst-3sg:O=3pl:A} prison \\
\glt `They took him and put him in prison.' \hfill [ŞE.76] \label{ex.co-refA4}
\z 

Coreferential deletion also affects other clitic pronoun functions, though apparently less frequently than A-past clitics. Like the co-referential deletion of A-past clitics, the clitic appears on the second coordinate verb.

\ea
\textit{bera keraşane zînaneke.} \\ 
\gll \textbf{ber-a} \textbf{ker-a=şa}=ne zînan-eke \\ 
 take\textsc{.prs.ind-3pl:A} do\textsc{.prs.ind-3pl:A=3pl:O=povb} prison\textsc{.m-def.m.sg.dir} \\ 
\glt `They took [them] and put them in jail.' \hfill [BP.139]
\z 


\ea
\textit{de danê bere zane geřoɫê saqêşa kamênê.} \\ 
\gll de dan(e)-ê b\'er-e z\'an-e \textbf{geřoɫ-ê} \textbf{saq-ê=şa} kam-ê=nê \\ 
ten clf\textsc{clf-pl.dir} take\textsc{.prs.imp-2sg:A} know\textsc{.prs.imp-2sg:A} degraded\textsc{-pl} well\_preserved\textsc{-pl=3pl:PSR} which\textsc{-pl.dir=cop.3pl:S} \\ 
\glt `Take ten [coins], see which ones are debased and which ones are pure. [lit. their debased and their pure are which?]' \hfill[JF.07] 
\z 

Similarly, in the following constructions, where a contrastive parallel is established between two subjects, the clitic pronoun is only expressed once. Notice that, unlike the coordinate verbs seen above, the two subjects refer to different persons in the examples below. These constructions are termed ``contrastive topic'' constructions according to \citegen{dik_typology_1981} terminology. 

\ea
\textit{kê řenciş daw kê berđ.} \\
\gll k\stackunder[-10pt]{\^{e}}{\`{}} řenc=iş da=w k\stackunder[-10pt]{\^{e}}{\`{}} berđ \\
who toil=\textsc{3sg:A} give.\textsc{pst}=and who take.\textsc{pst} \\
\glt `(Look) who toiled, and who took [the credit].' \hfill [YX.15]
\z


\ea
\textit{min rencim daw to berđ.} \\
\gll m\`{i}n řenc=im da=w t\`{o} berđ \\
\textsc{1sg} toil=\textsc{1sg:A} give.\textsc{pst}=and \textsc{2sg} take.\textsc{pst} \\
\glt `[It's] me [who] toiled and [it's] you [who] took [the credit].' \hfill [YX.16]
\z

\newpage
In terms of placement, clitic pronouns attach to the leftmost syntactic element within the verb phrase (VP) as their host. Note that my conception of the VP as the cliticisation domain is not a strict syntactic or theoretical stance; the VP is rather conceived as the (complex) verb, its direct object, and sometimes also its indirect object. On the other hand, the subject NP, clausal adverbs, and clausal conjunctions are considered VP-external elements. In the following examples, the A-past clitic has taken as host the direct object NP (\ref{ex.clc-place1}), light-verb complement (\ref{ex.clc-place2}), derivational preverbs (\ref{ex.clc-place3}), and the verb (\ref{ex.clc-place4}).

\ea
\textit{ħeywanşa wey kerden.} \\ 
\gll ħeywan=\textbf{şa} wey kerde=n \\ 
animal\textsc{.dir.m=3pl:A} raising do\textsc{.pst.ptcp.m=cop.3sg.m:O} \\ 
\glt `They raised animals.' \hfill[JE.2] \label{ex.clc-place1}
\z  


\ea
\textit{maço, `beɫê keçim kerd!’} \\ 
\gll m-aç-o beɫê keç=\textbf{im} kerd{-\O} \\ 
 \textsc{ind-}say\textsc{.prs-3sg:A} yes crooked\textsc{.m=1sg:A} do\textsc{.pst-3sg:O} \\ 
\glt `She (the younger sister), ‘Indeed, I made it less.’' \hfill[JH.49] \label{ex.clc-place2}
\z 


\ea
\textit{hurim gêrtênê.} \\ 
\gll hur=\textbf{im} gêrtê=nê \\ 
 \textsc{pvb=1sg:A} take\textsc{.pst.ptcp.pl=cop.3pl:O} \\ 
\glt `I lifted them.' \hfill[JE.66] \label{ex.clc-place3}
\z 


\ea
\textit{fermawaş, `xeyr beydê!’} \\ 
\gll fermawa=\textbf{ş} xeyr b-e-îdê \\ 
 say\textsc{.pst=3sg:A} goodness\textsc{.m} \textsc{imp-}come\textsc{.prs-2pl:S} \\ 
\glt `He (the Sheikh) said, ‘Welcome!’' \hfill[ZQ.6] \label{ex.clc-place4}
\z 

Preverbal inflectional prefixes are skipped as clitic hosts, leaving syntactic phrases as the anchoring elements. The relevant formatives include the negation prefix (\ref{ex.clc-place5})--(\ref{ex.clc-place6}), the indicative prefix (\ref{ex.clc-place7}), and the subjunctive prefix (\ref{ex.clc-place8}).

\ea
\textit{maço, `nezanam.'} \\ 
\gll m-aç-o ne-zana=\textbf{m} \\ 
 \textsc{ind-}say\textsc{.prs-3sg:A} \textsc{neg-}know\textsc{.pst=1sg:A} \\ 
\glt `He (the man) said, ‘I didn’t understand [his point].’' \hfill[JH.26] \label{ex.clc-place5}
\z 

\newpage
\ea
\textit{meberîm yane milû mizgî.} \\ 
\gll me-ber-î=\textbf{m} yane mi-l-û mizgî \\ 
\textsc{neg.ind-}take\textsc{.prs-2sg:A=1sg:O} house\textsc{.m.sg.dir} \textsc{ind-}go\textsc{.prs-1sg:S} mosque\textsc{.m} \\ 
\glt `[If you do not invite me and] won’t take me home, I will go to the mosque.' \hfill[JH.33] \label{ex.clc-place6}
\z 



\ea
\textit{a wextî maraş. } \\ 
\gll a wext-î m-ar-a=\textbf{ş}\\ 
 \textsc{dem.dist} time\textsc{.m-sg.obl} \textsc{ind-}bring\textsc{.prs-3pl:A=3sg:O} \\
\glt `They brought her.' \hfill[ZP.97] \label{ex.clc-place7}
\z 


\ea
\textit{lodê bardêş!} \\ 
\gll lo-dê b-ar-dê=\textbf{ş} \\ 
 go.\textsc{prs.imp-2pl:S} \textsc{imp-}bring\textsc{.prs-2pl:A=3sg:O} \\ 
\glt `Go [and] bring him!' \hfill[ŞC.20] \label{ex.clc-place8}
\z 

Similarly, clausal conjunctions are skipped as clitic hosts.

\ea
\textit{eger kinaçekêşa don{ɛ} ...} \\ 
\gll eger kinaç(ê)-ekê=\textbf{şa} don-{ɛ} \\ 
 if girl\textsc{.f-def.f.sg=3pl:A} talk\_to\textsc{.pst.cond.aug.3sg:R} \\  
\glt `If they had talked to the girl ...' \hfill[JE.77]
\z 


\ea
\textit{mađam weşeş kerdêbowe ...} \\ 
\gll mađam weş-e=\textbf{ş} kerdê=b-o=we \\ 
 as\_long\_as well\textsc{-f=3sg:O} do\textsc{.pst.ptcp.f}=be\textsc{.prs-3sg:O=compl} \\ 
\glt `Now that it appears that he has healed her [thoroughly] ...' \hfill[JP.261]
\z 


\ea
\textit{welê wesîyetêm hen kerû.} \\ 
\gll welê wesîyet-ê=\textbf{m} hen-∅ ker-û \\ 
but will\textsc{.m-indf=1sg:NC} \textsc{exist-3sg.m:S} do\textsc{.prs.ind-1sg:A} \\  
\glt `However, I have a [last] will to give.' \hfill[BP.184]
\z 

Despite having a seemingly VP-based clitic system, there is evidence that sentence stress has a role in clitic hosting. Therefore, VP-external elements, such as subject arguments and sentence adverbials, which carry nuclear stress, can host clitic pronouns.

\ea
\textit{çîm haɫî nekerdenî?} \\ 
\gll \textbf{ç\stackunder[-10pt]{\^{i}}{\`{}}=m} haɫî ne-kerde=nî\,\suppipe{} \\ 
 why\textsc{=1sg:A} understood \textsc{neg-}do\textsc{.pst.ptcp.m=cop.2sg:O} \\ 
\glt `Why didn't I explain [it] to you?' \hfill[HB.92]
\z 


\ea
\textit{ane kot bera?} \\ 
\gll ane \textbf{k\`o=t} ber-a\,\suppipe{} \\ 
 \textsc{prsv} where\textsc{=2sg:O} take\textsc{.prs.ind-3pl:A} \\ 
\glt `Where are they taking you?' 
\z 


\ea
\textit{î dega toş vînî çoɫe bîyêne.} \\ 
\gll î dega \textbf{t\`o=ş} vîn-î\,\suppipe{} ç\`oɫ-e bîyê=ne \\ 
 \textsc{dem.prox} village{\textsc{.f}} \textsc{2sg=3sg:O} see\textsc{.prs.ind-2sg:A} deserted-\textsc{f} be\textsc{.pst.ptcp.f=cop.3sg.f:S} \\ 
\glt `This village, which you see, was deserted.' \hfill[JE.4]
\z 


\ea
\textit{pokîş maça pene tewenû sireway.} \\ 
\gll \textbf{pok\stackunder[-10pt]{\^{e}}{\`{}}=ş} m-aç-a pene tewen(î)-û {} sireway\,\suppipe{} \\ 
 that\_is\_why\textsc{=3sg:R} \textsc{ind-}say\textsc{.prs-3pl:A} to stone\textsc{.f-ez.gen} {} treatment\textsc{.inf} \\   
\glt `That is why it is called the healing stone.' \hfill[ZP.59]
\z 


\ea
\textit{epaseş penî aman.} \\ 
\gll \textbf{e=pas\`e=ş} penî ama=n\,\suppipe{} \\ 
 \textsc{emph}=such\textsc{=3sg:R} to come\textsc{.pst.ptcp.m=cop.3sg.m:S} \\ 
\glt `It has affected him like this.' \hfill[ŞC.109]
\z 

When occurring on the verb as the sole host, the clitic pronoun forms a sequence with the verbal person suffix. The order in such constellations is uniformly affix=clitic. The sequence applies across the board regardless of the functions clitic pronouns and verbal affixes assume. Thus, in present tense constructions, clitics functioning as a direct object (\ref{ex.clc-place9}), or an adposition complement (\ref{ex.clc-place10}) may form a sequence with the verbal person affixes indexing the A-prs argument.

\ea
\textit{beroma yanew wêşa.} \\ 
\gll ber\textbf{-o=ma }yane-û wê=şa \\ 
take\textsc{.prs.ind-3sg:A=1pl:O} house\textsc{-ez.gen} \textsc{refl=3pl:PSR} \\ 
\glt `He will take us to his [lit. their] house.' \hfill[HB.35] \label{ex.clc-place9}
\z 


\ea
\textit{maçmêş pene tewenû sirewê.} \\ 
\gll m-aç\textbf{-mê=ş} pene tewen(î)-û sirewê \\ 
\textsc{ind-}say\textsc{.prs-1pl:A=3sg:R} to stone\textsc{.f-ez.gen} treatment\textsc{.f} \\ 
\glt `Since the girl was deaf-mute, we call it the healing stone.' \hfill[ZP.51] \label{ex.clc-place10}
\z

On the other hand, in past transitive constructions, an A-past clitic may form a sequence with a verbal person affix indexing transitive objects (\ref{ex.clc-place11})--(\ref{ex.clc-place13}), or adposition complements (\ref{ex.clc-place14}). As can be seen, the ordering of the clitic pronouns relative to the verbal person affixes is fixed, irrespective of the person/number value of the clitics of verbal person affixes.\footnote{The external realisation of clitic pronouns relative to affixes may be indicative of the status of clitic pronouns as syntactic items \citep[]{anderson_aspects_2005}{}.}

\ea
\textit{berđaşa.} \\ 
\gll berđ\textbf{-a=şa} \\ 
take\textsc{.pst-1sg:O=3pl:A} \\ 
\glt `They took me.' \hfill\citep[560]{khan_language_2023} \label{ex.clc-place11}
\z 


\ea
\textit{asîmêt cîyay.} \\
\gll as\textbf{-îmê=t} cîyay \\
let.\textsc{pst-1pl:O=2sg:A} place \\
\glt `You left us behind.' \hfill[ÇK.93] \label{ex.clc-place12}
\z 


\ea
\textit{weznaymre.} \\
\gll wezna\textbf{-î=m}=re \\
drop.\textsc{pst-2sg:O=1sg:A=povb} \\
\glt `I took you down.' \hfill[MR.18] \label{ex.clc-place13}
\z 


\ea
\textit{vatîm pey.} \\ 
\gll vat\textbf{-î=m} pey \\ 
say\textsc{.pst-2sg:R=1sg:A} to \\ 
\glt `I told you.' \label{ex.clc-place14}
\z 

Sequences of clitic pronouns with identical person values are disfavoured. In such cases, one of the clitics is reduced due to haplology. In (\ref{cliticsequence1}), the \textsc{3sg} clitic on the reflexive base expresses two grammatical roles at the same time: possessor and agent. In (\ref{cliticsequence2}), the \textsc{2sg} clitic expresses the possessor and the object simultaneously.

\ea
\textit{şamne a tome pîr şelîyarî wêş arden.} \\ 
\gll şam=ne a tom=e pîr şelîyar-î \textbf{wê=ş} arde=n  \\ 
 \textsc{pn.m.dir=post} \textsc{dem.dist} seed\textsc{.m=dem} \textsc{pn} \textsc{pn-m.sg.obl} \textsc{refl=3sg:PSR/A} bring\textsc{.pst.ptcp.m=cop.3sg.m:O} \\ 
\glt `From Damascus, Pir Shaliyar had brought its seeds. He was in Damascus (for a while).' \hfill[ZP.95] \label{cliticsequence1}
\z 


\ea
\textit{luwe wêt wize milû paɫayeka şê ʕeladînîre.} \\ 
\gll lu-e \textbf{wê=t} w\'iz-e mil-û paɫa-eka şê ʕeladîn-î=re \\ 
 go.\textsc{prs.imp-2sg:S} \textsc{refl=2sg:PSR/O} throw\textsc{.prs.imp-2sg:A} on\textsc{-ez.gen} shoe\textsc{-def.pl.obl} Sheikh\textsc{.m} \textsc{pn-m.sg.obl=post} \\ 
\glt `Go, throw yourself at the feet [lit. shoes] of Sheikh Aladin.' \hfill[DG.27] \label{cliticsequence2}
\z 

\section{Nominal demonstrative pronouns} \label{sect: dempro}
Nominal demonstrative pronouns come in two sets: (i) independent demonstrative pronouns and (ii) demonstrative adjectives. 

\subsection{Independent demonstrative pronouns\is{independent demonstrative pronouns}} \label{sect:ind-dem-pro}
Independent demonstrative pronouns\is{independent demonstrative pronouns} come in proximal and distal forms. 
\begin{table}[htp]
    
    \begin{tabular}{lll}
 \lsptoprule
Proximal& \textsc{dir}& \textsc{obl} \\ 
\midrule
\textsc{m}&\textit{îne} &\textit{îney} \\
\textsc{f}& \textit{în\stackunder[-10pt]{\^{e}}{{}}, înî}& \textit{în\stackunder[-10pt]{\^{e}}{{}}} \\
\textsc{pl}& \textit{în\stackunder[-10pt]{\^{e}}{{}}, înî}& \textit{îna, înîşa} \\
 \lspbottomrule
    \end{tabular}
    \caption{Demonstrative pronouns, the proximal set}
    \label{tab:demprox}
\end{table}

\begin{table}[htp]
    
    \begin{tabular}{lll}
 \lsptoprule
Proximal& \textsc{dir}& \textsc{obl} \\ 
\midrule
\textsc{m}&\textit{ane, ûne} &\textit{aney} \\
\textsc{f}& \textit{an\stackunder[-10pt]{\^{e}}{{}}}& \textit{ane} \\
\textsc{pl}& \textit{an\stackunder[-10pt]{\^{e}}{{}}}& \textit{ana, anîşa} \\
 \lspbottomrule
    \end{tabular}
    \caption{Demonstrative pronouns, the distal set}
    \label{tab:demdis}
\end{table}

Examples below demonstrate the use of demonstrative pronouns in expressing case\is{case}/gender\is{gender} distinctions:

\newpage

\ea
\textit{înî be min mewero kinaçê padşay.} \\ 
\gll \textbf{înî} be min me-wer-o kinaçê padşa-î \\ 
 \textsc{dem.prox.f.3sg.dir} with \textsc{1sg} \textsc{neg.ind-}eat\textsc{.prs-3sg:A} girl\textsc{f.dir} king\textsc{.m-sg.obl} \\ 
\glt `She will not eat with me [under the same roof], the king’s daughter.' \\ \hfill[JP.213]
\z 


\ea
\textit{maço, `îne her hêɫe bî.’} \\ 
\gll m-aç-o \textbf{îne} her hêɫe bî-∅ \\ 
 \textsc{ind-}say\textsc{.prs-3sg:A} \textsc{dem.prox.m.3sg.dir} just egg\textsc{.m} be\textsc{.pst-3sg.m:S} \\ 
\glt `He said, ‘This one was just an egg.’' \hfill[JH.95]
\z 

There are indications that the demonstrative sets are losing case\is{case} and gender\is{gender} inflection, especially in the singular\is{singular} number. The masculine\is{masculine} direct forms \textit{îne} and \textit{ane} may take over the corresponding feminine\is{feminine} forms.

\ea
\textit{ane kinaçekêtne jenîşne.} \\ 
\gll \textbf{ane} kinaç(ê)-ekê=t=ne jenî=ş=ne \\ 
 \textsc{dem.dist.m.3sg.dir} daughter\textsc{.f-def.f.sg=2sg:PSR=cop.3sg.f:S} woman\textsc{.f=3sg:PSR=cop.3sg.f:S} \\ 
\glt `That is your daughter. She is his wife.' \hfill[KŞ.101]
\z 


\ea
\textit{îne padşazayêne.} \\ 
\gll \textbf{îne} padşazayê=ne \\ 
\textsc{dem.prox.m.3sg.dir} princess\textsc{=cop.3sg.f:S} \\ 
\glt `She is a princess.' \hfill[ZP.102]
\z 

The remainder of this section describes the functional scope of demonstratives roughly following the classification in \citet{diessel_demonstratives_1999}.

\subsubsection{Exophoric\is{exophoric} use}
Exophoric\is{exophoric} demonstratives focus the hearer’s attention on entities in the situation surrounding the interlocutors \citep[94]{diessel_demonstratives_1999}. The exophoric\is{exophoric} use is the same as the deictic use of demonstratives, where the speaker is at the deictic centre, and a deictic contrast is established based on a distance scale. The proximal deixis refers to items near the speaker, and the distal deixis refers to items that are distant from the speaker. The following examples indicate the exophoric\is{exophoric} use of demonstratives.

\ea
\textit{ane kuřû şuwaneyn serû esbekey to.} \\ 
\gll \textbf{ane} kuř-û şuwane-î=n ser-û esb-ekey to \\ 
 \textsc{dem.dist.m.3sg.dir} son\textsc{.m-ez.gen} shepherd\textsc{.m-sg.obl=cop.3sg.m:S} on\textsc{-ez.gen} horse\textsc{-def.m.sg.obl} \textsc{2sg} \\ 
\glt `That is the shepherd’s son on your horse.' \hfill[KŞ.100]
\z 


\ea
\textit{min metawû îne nigebanîye kerû.} \\ 
\gll min me-taw-û \textbf{îne} nigebanî-e k\'er-û \\ 
 \textsc{1sg} \textsc{neg.ind-}can\textsc{.prs-1sg:A} \textsc{dem.prox.m.3sg.dir} guardian\textsc{-f} do\textsc{.prs.sbjv-1sg:A} \\ 
\glt `I cannot take care of her.' \hfill[ZP.103]
\z 

The exophoric\is{exophoric} use of demonstratives often comes with a gesture pointing to the referent. In Tekht H.\il{Hewramî!Tekht}, the use of gestures with demonstratives can be extended to express the height of a referent. In such cases, the distal masculine\is{masculine} form \textit{ane} is used by default to indicate the height. In the following example, the speaker indicates the height of the bed using his hands and by expressing \textit{ane}:

\ea
\textit{wextê miđya îna ane ce zemînî hewawe wuten.} \\ 
\gll wext-ê mi-đy(e)-a în(e)=a \textbf{ane} ce zemîn-î hewa=we wute=n \\ 
 time\textsc{.m-indf} \textsc{ind-}look\textsc{.prs-3pl:S} \textsc{dem.prox.dir.m.3sg=ptcl} \textsc{dem.dist.m.3sg.dir} from earth\textsc{.m-sg.obl} air\textsc{=post} sleep\textsc{.pst.ptcp.m=cop.3sg.m:S} \\  
\glt `They saw that he (Pir Shaliyar) had levitated and slept aloft in the air at this height [lit. that much].' \hfill[JP.68]
\z 

In the following example, the speaker indicates the height of the sacks of flour using the distal demonstrative \textit{ane}:

\ea
\textit{awî luw{ɛ}nêrew serû ane hardîşa bîyênê.} \\ 
\gll awî luw{ɛ}=nê=re=û ser-û \textbf{ane} hardî=şa bîyê=nê\\ 
 water\textsc{.f} go\textsc{.pst.ptcp.pl=cop.3pl:S=povb}=and on\textsc{-ez.gen} \textsc{dem.dist.m.3sg.dir} flour\textsc{.f.pl.dir=3pl:NC} be\textsc{.ptcp.pl=cop.3pl:S} \\ 
\glt `The water had gone up to the level of the flour [sacks] that they had.' \\ \hfill[ZB.28]
\z 
 
\subsubsection{Anaphoric use}
Demonstrative pronouns may be used to track participants in the previous discourse. In this use, they overlap with anaphoric demonstratives\is{anaphoric demonstratives} in \S\ref{sect:3perspronouns} (see \tabref{tab:3proprox} and \tabref{tab:3proremote}).
In the following example, the referent is physically absent from the speech situation. The proximal demonstrative is used to track a referent mentioned in the previous discourse. The use of the proximal demonstrative conveys emotional engagement with the referent.

\ea
\textit{perso, `ceryan çikonû? î yanew kabray kênû îne kên?'} \\ 
\gll pers-o ceryan çiko=n=û î yane-û kabra-î kê=n=û \textbf{îne} kê=n \\ 
 ask\textsc{.prs.ind-3sg:A} story\textsc{.m} where\textsc{=cop.3sg.m:S}=and \textsc{dem.prox} house\textsc{.m-ez.gen} man\textsc{-m.sg.obl} who\textsc{=cop.3sg.m:S}=and \textsc{dem.prox.m.3sg.dir} who\textsc{=cop.3sg.m:S} \\ 
\glt `[He came to the village.] He asked [people], ‘What is the story? Where is the house of the fellow? Who is \textbf{he}?’' \hfill[JP.204]
\z 

The anaphoric use of demonstratives is also visible in relative clauses. The pronominal head of such clauses can only be a demonstrative pronoun.

\ea
\textit{mila ew kuřekey yoyşa bera. aneşa zilterû ʕalter bo ađî bera.} \\ 
\gll mi-l-a ew kuř-ekey yo-î=şa ber-a \textbf{ane}=şa zil-ter=û ʕal-ter b-o ađî ber-a \\ 
 \textsc{ind-}go\textsc{.prs-3pl:S} \textsc{dem.dist} son\textsc{.m-def.m.sg.obl} one\textsc{.m-sg.obl=3pl:PSR} take\textsc{.prs.ind-3pl:A} \textsc{dem.dist.m.3sg.dir=3pl:PSR} big\textsc{-cmpr}=and good\textsc{-cmpr} be\textsc{.prs.ind-3sg:S} \textsc{3sg.obl.m} take\textsc{.prs.ind-3pl:A} \\ 
\glt `They went away [and took] that son. They took one of them (i.e., the boys), \textbf{the one who} was bigger and healthier; they took him.' \hfill[ZB.40]
\z 


\ea
\textit{ane ke berdma şiş mangê menn.} \\ 
\gll \textbf{ane} ke berd-∅=ma şiş mang(e)-ê menn \\ 
 \textsc{dem.dist.m.3sg.dir} \textsc{rel} take\textsc{.pst-3sg.m:O=1pl:A} six month\textsc{.f-pl.dir} remain\textsc{.pst.3sg.m:S} \\ 
\glt `\textbf{The one} whom we took [with us] lived for six months.' \hfill[ZQ.29]
\z 


\ea
\textit{îne maçmêş pene şalîyare sîyaw lalow kinaçekên.} \\ 
\gll \textbf{îne} m-aç-mê=ş pene şalîyar-e sîyaw lalo-û kinaç(ê)-ekê=n \\ 
 \textsc{dem.prox.m.3sg.dir} \textsc{ind-}say\textsc{.prs-1pl:A=3sg:R} to \textsc{pn-ez.cmpd} black maternal\_uncle\textsc{.m-ez.gen} girl\textsc{.f-def.f.sg=cop.3sg.m:S} \\ 
\glt `\textbf{This [person]}, whom we call Shaliyar Siya, was the maternal uncle of the [king’s] daughter.'  \hfill[ZP.36]
\z 

\subsubsection{Discourse presentative\is{discourse presentative} use}
The demonstrative pronouns \textit{îne} and \textit{ane} may be used as sentential demonstratives to draw attention to a proposition and give it a sense of immediacy. This use of demonstratives may be called presentative\is{discourse presentative}.

\ea
\textit{maço `qurban, îne çêşit kerd?'} \\ 
\gll m-aç-o qurban \textbf{îne} çêş=it kerd{-\O} \\ 
 \textsc{ind-}say\textsc{.prs-3sg:A} sir.\textsc{voc} \textsc{prsv} what\textsc{=2sg:A} do\textsc{.pst-3sg.m:O}\\ 
\glt `They said, ‘Sir, what have you done?’' \hfill[PM.29]
\z 


\ea
\textit{maça, `çî girewî ane miheme?’} \\ 
\gll m-aç-a çî girew-î \textbf{ane} miheme \\ 
 \textsc{ind-}say\textsc{.prs-3pl:A} why cry\textsc{.prs.ind-2sg:S} \textsc{prsv} \textsc{pn} \\ 
\glt `They said, `Muhammad, why are you crying?’' \hfill[BP.154]
\z 

In the presentative\is{discourse presentative} use, the demonstrative pronouns can also draw attention to a referent in the extralinguistic situation. In this usage, they are combined with the attention-drawing particle \textit{=a}.

\ea
\textit{maça, `qurban wiɫa fiɫane kes toryanû ana.'} \\ 
\gll m-aç-a qurban wiɫa fiɫan-e kes torya=n=û \textbf{an(e)=a} \\ 
 \textsc{ind-}say\textsc{.prs-3pl:A} sir.\textsc{voc} by\_God such\_and\_such\textsc{-ez.cmpd} person\textsc{.m} get\_offended\textsc{.pst.ptcp.m=cop.3sg.m:S}={and} \textsc{dem.dist.m.3sg.dir=ptcl} \\ 
\glt `They said, ‘Sir, indeed, the fellow has got offended; there [he is].’' \hfill[HB.76]
\z 


\ea
\textit{maço, `amano îna.’} \\ 
\gll m-aç-o ama=n=o \textbf{în(e)}=a \\ 
 \textsc{ind-}say\textsc{.prs-3sg:A} come\textsc{.pst.ptcp.m=cop.3sg.m:S=compl} \textsc{dem.prox.dir.m.3sg=ptcl} \\  
\glt `He (the sultan) said, ‘He has returned; there [he is].’' \hfill[JH.111]
\z 

\subsubsection{Empathetic\is{empathetic} use}
The demonstrative pronouns may be used to express the emotional engagement of the speaker towards a referent in the speech situation. In this usage, the near deixis usually expresses the speaker's positive attitude towards the referent, while the distal deixis expresses the speaker's negative attitude. In the following example, the speaker points to his son, sitting a meter away from the speaker, using the distal deixis to express his negative feelings towards the son smoking cigarettes. 


\ea
\textit{ane wero.} \\
\gll ane wer-o \\
\textsc{dem.dist.m.3sg.dir} smoke.\textsc{prs.ind-3sg:S} \\
\glt `He smokes.' \hfill [hearsay]
\z 


\subsubsection{Predicative use}
Demonstrative pronouns can be used predicatively. In this usage, they occur in the topic position of identificational clauses with a nominal predicate \citep{diessel_predicative_1997}. The predicative demonstratives correspond to English\il{English} `this is', `that is', and `there goes'. The near demonstrative \textit{îne} has a predicative function in the following excerpt.

\newpage
\ea
\textit{îne serew êmew îne şimşêrû şime.} \\ 
\gll \textbf{îne} sere-û ême=û \textbf{îne} şimşêr-û şime \\ 
 \textsc{dem.prox.m.3sg.dir} head\textsc{.m-ez.gen} \textsc{1pl}=and \textsc{dem.prox.m.3sg.dir} sword\textsc{.m-ez.gen} \textsc{2pl} \\ 
\glt `Here are their heads and your swords. [Lit. \textbf{These are} our heads, and \textbf{these are} your swords.]' \hfill[BP.97] 
\z 

\subsection{Demonstrative determiners\is{demonstrative determiners}} \label{sect:demonstrative-determiner}
Tekht H.\il{Hewramî!Tekht} features several demonstrative determiners\is{demonstrative determiners}, which are by default discontinuous, consisting of the demonstrative determiner to the left and deictic clitic \textit{=e} to the right of a noun phrase. 

\begin{table}[htp]
    
    \begin{tabular}{ll}
 \lsptoprule
& \textsc{dir} and \textsc{obl} \\ 
\midrule
Proximal&\textit{î ... (e)} \\
Distal 1& \textit{a / û ... (e)} \\
Distal 2& \textit{ew ... (e)} \\
 \lspbottomrule
    \end{tabular}
    \caption{Demonstrative determiners}
    \label{tab:demproxadj}
\end{table}

The deictic clitic \textit{=e} is deleted following fusional gender/case/number suffixes, presumably to avoid vowel hiatus. However, it resurfaces when another clitic is added to a base noun, including a clitic pronoun and an additive clitic.
\TabPositions{1.5cm,3cm,4.5cm,7cm}
\ea \textit{kuř} `boy, son'; \textit{beg} `chief' \label{ex.kur}\\
\textsc{sg.dir}\tab \textit{\textbf{î} kuř=\textbf{e}} \tab [KŞ.80] \\
\textsc{sg.obl}\tab \textit{\textbf{î} kuř-î} \tab[KŞ.56]  \tab \textit{p=\textbf{î} kuř-î=m=\textbf{e}}\tab [RE.7] \\
\textsc{pl.dir}\tab \textit{\textbf{î} begê} \tab [ZP.67] \tab \textit{\textbf{î} begê=ma} \tab [JP.77]\\
\textsc{pl.obl}\tab \textit{\textbf{a} bega} \tab [BP.121] \\
\z

\ea \textit{kinaçê} `girl' \label{ex.kiç}\\
\textsc{sg.dir}\tab \textit{\textbf{î} kinaçê} \tab [ZP.44] \tab \textit{\textbf{î} kinaçê=t=\textbf{e}} \tab [JP.163]\\
\textsc{sg.obl}\tab \textit{\textbf{î} kinaçê} \tab [ZQ.39] \tab \textit{\textbf{î} kinaçê=t=\textbf{e}}\tab [RE.7] \\
\z


\ea \textit{pîya} `man' \label{ex.piya}\\
\textsc{sg.dir}\tab \textit{\textbf{î} pîya} \tab [ZP.67] \tab \textit{\textbf{û} pîya=îç=\textbf{e}} \tab [JP.77]\\
\textsc{sg.obl}\tab \textit{\textbf{î} pîya-î} \tab [ZP.44] \tab  \\
\textsc{pl.dir}\tab \tab \tab \textit{\textbf{î} pîy{ɛ}=ma} \tab [BP.121] \\
\textsc{pl.obl}\tab \tab \tab \textit{\textbf{a} pîya-ya} \tab [HM.39]
\z

Example (\ref{ex.dem-obl}) shows that the deictic clitic \textit{=e} is deleted after the oblique case on \textit{pîya}. In (\ref{ex.dem-obl2}), by contrast, when the same noun is followed by the additive clitic \textit{=îç}, the deictic clitic resurfaces.

\ea \label{ex.dem-obl}
\textit{to eger î kinaçê berdet xizmetû î pîyay ...} \\ 
\gll to eger î kinaçê berd-e=t xizmet-û \textbf{î} \textbf{pîya-î} \\ 
 \textsc{2sg} if \textsc{dem.prox} daughter\textsc{f.dir} take\textsc{.pst-3sg.f:O=2sg:A} service\textsc{.m-ez.gen} \textsc{dem.prox} man\textsc{.m-sg.obl} \\ 
\glt `When you have taken this girl into the presence of this man ...' \hfill [ZP.44]
\z 


\ea \label{ex.dem-obl2}
\textit{û pîyayçe řasiş wat.} \\ 
\gll û pîya=îç=e řas=iş wat \\ 
 \textsc{dem.dist} man\textsc{.m=add=dem} truth\textsc{=3sg:A} say\textsc{.pst} \\ 
\glt `And the man (the servant) told the truth.' \hfill [JP.77]
\z 

The deictic clitic exhibits variation in appearing with the demonstrative determiners\is{demonstrative determiners} when preceded by nouns such as \textit{zeman} `time' and \textit{sefer} `time' that often can be used adverbially in the sense of `back then’, `time’ (e.g., this time). Contrast the examples in (\ref{ex.dem.det}).
\TabPositions{2.5cm}
\ea \label{ex.dem.det}
\textit{a zeman=e}\tab [JM.43] \\
\textit{a zeman}\tab [BP.9] \\
\textit{î sefer}\tab [BP.173] \\
\z

Demonstrative determiners\is{demonstrative determiners} feature the same set of functions as independent demonstrative pronouns\is{independent demonstrative pronouns}. Notably, the set with \textit{ew} appears primarily in contrast with \textit{î}, expressing the opposite direction.

\ea
\textit{neweşe gino herçîw duktirû î law ew la beraş duktir hîç duktirê ʕelêceş mekero.} \\ 
\gll neweş-e gin-o herçîw duktir=û \textbf{î} \textbf{la}=û \textbf{ew} \textbf{la} ber-a=ş duktir hîç duktir-ê ʕelêce=ş me-ker-o \\ 
 ill\textsc{-f} fall\textsc{.prs.ind-3sg:S} any physician\textsc{.m}=and \textsc{dem.prox} side=and \textsc{dem.dist} side take\textsc{.prs.ind-3pl:A=3sg:O} physician\textsc{.m} no physician\textsc{.m-indf} treatment\textsc{.f=3sg:O} \textsc{neg.ind-}do\textsc{.prs-3sg:A} \\ 
\glt `No matter how often she was taken to doctors here and \textbf{there} (lit. this side and that side), no doctor could cure her.' \hfill[ZP.26]
\z 


\ea
\textit{eçê milo bilo pey ew dîmû kelî.} \\ 
\gll e=çê mi-l-o bi-l-o pey ew dîm-û kel-î \\ 
 from=here \textsc{ind-}go\textsc{.prs-3sg:S} \textsc{sbjv-}go\textsc{.prs-3sg:S} for \textsc{dem.dist} side\textsc{.m-ez.gen} mountain\textsc{-m.sg.obl} \\ 
\glt `He went from here to the other side of the mountain.' \hfill[JH.5]
\z 

The exophoric\is{exophoric} use of demonstrative determiners\is{demonstrative determiners} is seen in the following example.

\ea
\textit{da dey hurbêze î mexlûqî girdiş gêɫeş pone.} \\ 
\gll da dey hur-b-êz-e î mexlûq-î gird=iş g\stackunder[-10pt]{\^{e}}{\'{}}ɫ-e=ş pone \\ 
 \textsc{hort} \textsc{disc.ptcl} \textsc{pvb-imp-}rise\textsc{.prs-2sg:S} \textsc{dem.prox} people\textsc{-m.sg.obl} all\textsc{=3sg:PSR} wander\textsc{.prs.imp-2sg:S=3sg:R} at \\ 
\glt `Come on, get up and search among all these people.' \hfill[HB.90]
\z 

The distal determiner \textit{a} is used in the following example in an anaphoric function. Here, the proximal determiner is used after the first mention of `lion' to establish it as a discourse participant.

\ea
\textit{pase zanû şêrê luwan dilêş! wêş zanî be ʕemeɫekaş. ađîç wat, `a şêre minna.'} \\ 
\gll pase zan-û \textbf{şêr-ê} luwa=n dilê=ş wê=ş zanî be ʕemeɫ-eka=ş ađ=îç wat \textbf{a} \textbf{şêr=e} min=na \\ 
 like know\textsc{.prs.ind-1sg:A} lion\textsc{.m-indf} go\textsc{.pst.ptcp.m=cop.3sg.m:S} inside\textsc{=3sg:R} \textsc{refl=3sg:A} know\textsc{.pst} by deed\textsc{.m-def.pl.obl=3sg:PSR} \textsc{3sg.obl.m=add} say\textsc{.pst} \textsc{dem.dist} lion\textsc{.m=dem} \textsc{1sg=cop.1sg:S} \\ 
\glt `[He (Hayas) said, ‘Sir, why should I go back (home)? After a long time, my garden has produced [fruit]]. Now, to me, it is as if \textbf{a lion} has trespassed in it.' He (the sultan) understood [these words] were meant to [refer to] his deed. He (the sultan) said, ‘I am \textbf{that lion}.’' \\ \hfill[JH.114]--[JH.116]
\z 

The following excerpt exemplifies the empathetic\is{empathetic} use of the demonstrative determiner\is{demonstrative determiners} \textit{î}. Here, the proximal demonstrative is used for a referent who is not physically present in the speech situation. The use of the proximal demonstrative conveys the speaker's emotional engagement towards the referent.

\ea
\textit{î birame derdeş, derdêş îna po derdû gulîş maça dađê!} \\ 
\gll \textbf{î} \textbf{bira=m=e} derde=ş derd(e)-ê=ş îna-∅ p=o derd(e)-û gulî=ş m-aç-a dađ-ê \\ 
 \textsc{dem.prox} brother\textsc{.m=1sg:PSR=dem} illness\textsc{.f=3sg:R} illness\textsc{.f-indf=3sg:R} \textsc{loc.deic.cop-3sg.m:S} at\textsc{=post} illness\textsc{.f-ez.gen} leprosy\textsc{=3sg:R} \textsc{ind-}say\textsc{.prs-3pl:A} yell\textsc{.m-pl.dir} \\ 
\glt `This brother of mine is ill; it is called leprosy; please help!' \hfill[DG.43]
\z 

The demonstrative determiners\is{demonstrative determiners} may have a recognitional use, expressing shared knowledge between the speaker and the addressee. Recognitional demonstratives mark information that is discourse-new and hearer-old \citep[106]{diessel_demonstratives_1999}{}. In the following excerpt, the use of the demonstrative with `mountains' conveys shared information between the speaker and the listener. Note that the referent of `mountains' is invisible (see \S\ref{sect:loc-adv-dem}).

\ea
\textit{be beqêwew tifengê hînê tifengû a wextî herçê bîyen pa keşane gêɫan.} \\ 
\gll be beq-êwe=û tifeng-ê hîn-ê tifeng-û a wext-î herçê bîye=n p=\textbf{a} keş-a=ne gêɫa=n \\ 
 with male\_partridge\textsc{.m-indf}=and gun\textsc{-indf} \textsc{fill-indf} gun\textsc{-ez.gen} \textsc{dem.dist} time\textsc{.m-sg.obl} whatever be\textsc{.pst.ptcp.m=cop.3sg.m:S} at=\textsc{dem.dist} mountain\textsc{.m=dem.dist=post} wander\textsc{.pst.ptcp.m=cop.3sg.m:S} \\ 
\glt `[He had] a partridge and a gun -- of whatever type it was back then -- [and] would wander in those mountains.' \hfill[ŞC.7]
\z 

\subsection{Local adverbial demonstratives\is{local adverbial demonstratives}} \label{sect:loc-adv-dem}
Local adverbial demonstratives\is{local adverbial demonstratives} point to a place. They are distinguished in terms of spatial function and visibility. In terms of marking, these demonstratives are derived either from nominal demonstratives followed by a derivational place suffix, or from demonstratives enclosed within a circumpositional phrase (see \tabref{tab:locadvdem}).

\begin{table}[htp]
    \begin{tabular}{lll}
    \lsptoprule
`here' & \textit{çêge, êge} & < \textit{ç} `in' + \textit{ê} `this' + \textit{-ge} \\
`there (immediate context)& \textit{çage}& < \textit{ç} `in' + \textit{a} `that' + \textit{-ge} \\
`there (distant but visible)' & \textit{çoge, oge} & < \textit{o} `that' + \textit{-ge} \\
`there (invisible)' & \textit{pagene, epagewe} & <\textit{p} `at' + \textit{a} `that' + \textit{-ge} + \textit{=ne} `in'\\
\lspbottomrule
    \end{tabular}
    \caption{Local adverbial demonstratives}
    \label{tab:locadvdem}
\end{table}

The local adverbial demonstratives\is{local adverbial demonstratives} can further be distinguished based on elevation; that is, the demonstratives indicate whether the referent is at a higher or lower elevation relative to the deictic centre \citep[][42]{diessel_demonstratives_1999}{}{}. In such cases, the local adverbial demonstratives\is{local adverbial demonstratives} are followed by a postposition \textit{-er} `down' and \textit{-hur} `up' indicating elevation.
\TabPositions{4cm}
\ea
`down here'\tab \textit{epêre} \\
`down there (invisible) \tab \textit{epare} \\
\z

\ea
`up there (visible)'\tab \textit{ewagehur}  \\
`up there (invisible)'\tab \textit{epagehur}  \\
\z

The following examples illustrate the use of local adverbial demonstratives\is{local adverbial demonstratives} in running discourse.

\ea
\textit{keɫeşîrê gêrten bawşîşo qoqeqoqû epagehur.} \\ 
\gll keɫeşîr-ê gêrte=n bawşî=ş=o qoqeqoq=û \textbf{epagehur} \\ 
 rooster\textsc{.m-indf} grab\textsc{.pst.ptcp.m=cop.3sg.m:O} embrace\textsc{.f=3sg:PSR=post} \textsc{onom}=and up\_there \\  
\glt `He [the girl’s father] would grab a rooster in his arms [to present to the nobleman], [while the rooster] was crowing from this side to \textbf{up there} (i.e., the top of the village).' \hfill[RE.48]
\z 


\ea
\textit{epare beʕzêş piřnaşre berd wisiş řoxane.} \\ 
\gll \textbf{epare} beʕzê=ş piřna=ş=re berd-∅ wis-∅=iş řoxane \\ 
 in=down\_there some\textsc{=3sg:A} break\_off\textsc{.pst=3sg:A=povb} take\textsc{.pst-3sg.m:O} throw\textsc{.pst-3sg.m:O=3sg:A} river \\ 
\glt `\textbf{Down there}, [the flood] took [and] threw some [animals] into the river.' \\\hfill[ZB.22]
\z 

\subsection{Manner adverbial demonstratives\is{manner adverbial demonstratives}} \label{sect:man-adv-dem}
Together with local demonstratives\is{local adverbial demonstratives}, manner adverbial demonstratives\is{manner adverbial demonstratives} constitute demonstrative adverbs. Manner demonstratives\is{manner adverbial demonstratives} contain the demonstrative elements \textit{ê} and \textit{a} (see \tabref{tab:mannerdem1} and \tabref{tab:mannerdem2}). They have both prepositional functions and adverbial functions. In the prepositional function, they are equivalent to English\il{English} `like' and modify a noun phrase. In this usage, the preposition may appear with the genitive ezafe\is{genitive ezafe} linker \textit{-û}.

\begin{table}[htp]
    \begin{tabular}{lll}
    \lsptoprule
\multirow{2}{*}{Set 1}& \textit{pêse, epêse} & `like this, in this way' \\
& \textit{pase, epase} & `like that, in that way'  \\
\lspbottomrule
    \end{tabular}
    \caption{Manner adverbial demonstratives\is{manner adverbial demonstratives}, Set 1}
    \label{tab:mannerdem1}
\end{table}
 
 \ea 
\textit{cuwanê bê pêsew î girdû cuwana.} \\ 
\gll cuwan-ê b-ê pêse-û î gird-û cuwan-a \\
 youth\textsc{.m-indf} be\textsc{.prs-aug.3sg:S} like\textsc{ez.gen} \textsc{dem.prox} all\textsc{-ez.gen} youth\textsc{.m-pl.obl} \\ 
\glt `He is a young man like all young men.' \hfill[ZQ.54]
\z 

 \ea 
\textit{to pêsenî emîrî.} \\ 
\gll to pêse=nî emîr-î \\
 \textsc{2sg} like=\textsc{cop.2g:S} \textsc{pn-m.sg.obl} \\ 
\glt `You are like Emir.' \hfill[hearsay]
\z 

In the adverbial use, manner adverbial demonstratives\is{manner adverbial demonstratives} are typically used to refer to a proposition in a surrounding discourse.

\ea
\textit{pase jîwy{ɛ}nê.} \\ 
\gll pase jîwy{ɛ}=nê \\ 
 like\_that live\textsc{.pst.ptcp.pl=cop.3pl:S} \\  
\glt `They (people) used to live like that.' \hfill[RE.72]
\z 


\ea
\textit{ême e pêsema ser qomîyan.} \\ 
\gll ême e=pêse=ma ser qomya=n \\ 
\textsc{1pl} \textsc{emph}=such\textsc{=1pl:R} on happen\textsc{.pst.ptcp.m=cop.3sg.m:S} \\ 
\glt `We have been through such and such.’'  \hfill[BP.202]
\z

There is another set of manner adverbials, which are used less frequently than Set 1 seen above. These demonstratives are formed by a combination of the emphatic particle \textit{e}\footnote{The particle \textit{e} can alternatively be interpreted as the preposition \textit{e=}.} and the preposition \textit{ç} with the demonstratives \textit{îne} and \textit{ane}.

\begin{table}[htp]
    
    \begin{tabular}{lll}
    \lsptoprule
\multirow{2}{*}{Set 2}& \textit{eçîne} & `like this, in this way'\\
& \textit{eçane}& `like that, in that way' \\
\lspbottomrule
    \end{tabular}
    \caption{Manner adverbial demonstratives\is{manner adverbial demonstratives}, Set 2}
    \label{tab:mannerdem2}
\end{table}

Like Set 1, Set 2 manner demonstratives\is{manner adverbial demonstratives} can be used adverbially to refer to a chunk of the proposition in the surrounding discourse.

\ea
\textit{ême zemanma řisûmatû ewsayma eçîne bîyen.} \\ 
\gll ême zeman=ma řisûmat-û ewsa-î=ma \textbf{e=çîne} bîye=n \\ 
 \textsc{1pl} time\textsc{.m=1pl:PSR} traditions\textsc{.m-ez.gen} past\textsc{.m-sg.obl=1pl:PSR} \textsc{emph}=like\_this be\textsc{.pst.ptcp.m=cop.3sg.m:S} \\  
\glt `In the old times, our customs were like this.' \hfill[RE.42]
\z 


\ea
\textit{maça, `eçînew epase.’} \\ 
\gll m-aç-a e=çîne=û e=pase \\ 
 \textsc{ind-}say\textsc{.prs-3pl:A} \textsc{emph}=like=and \textsc{emph}=like \\ 
\glt `They said, ‘[The story is] like this and like that.’' \hfill[HB.77]
\z 

Set 2 manner demonstratives\is{manner adverbial demonstratives} can be used adjectivally.

\ea
\textit{be xwa fiɫan heyas amano çîwê eçînew eçane ʕaliş arden.} \\ 
\gll be xwa fiɫan heyas ama=n=o çîw-ê \textbf{e=çîne}=û \textbf{e=çane} ʕal=iş arde=n \\ 
 by God\textsc{.m} such\_and\_such \textsc{pn} come\textsc{.pst.ptcp.m=cop.3sg.m:S=compl} thing\textsc{-indf} \textsc{emph}=like\_this{=and} \textsc{emph}=like\_that good\textsc{=3sg:A} bring\textsc{.pst.ptcp.m=cop.3sg.m:O} \\ 
\glt `Such-and-such Hayas has returned. He has brought [with him] such a nice thingummy (i.e., woman). [Lit. He has brought a thing like this and like that, very nice.]' \hfill[JH.66]
\z 
 
\section{Reflexive pronouns} \label{sect:reflexivepronoun}
Reflexive pronouns are anaphoric pronouns that are specialised for coreferential use within a clause \citep[][]{janic_comparing_2023}{}{}. In Tekht H.\il{Hewramî!Tekht}, the intended reference of the reflexive pronoun\is{reflexive pronouns} must be the subject of the clause. The reflexive pronoun\is{reflexive pronouns} consists of the reflexive base \textit{wê} and the clitic pronouns, as seen below. 

\begin{table}[htp]
    
    \begin{tabular}{ll}
    \lsptoprule
person/number & form \\
\midrule
\textsc{1sg}& \textit{wê=m} \\
\textsc{2sg}& \textit{wê=t} \\
\textsc{3sg}& \textit{wê=ş} \\
\textsc{1pl}& \textit{wê=ma} \\
\textsc{2pl}& \textit{wê=ta} \\
\textsc{3pl}& \textit{wê=şa} \\
\lspbottomrule
    \end{tabular}
    \caption{reflexive pronouns\is{reflexive pronouns}}
    \label{tab:reflpro}
\end{table}

The forms in \tabref{tab:reflpro} were attested in Hewraman Tekht. In the variety of Silên, spoken 25 km south of Hewraman Tekht, and less so in the Tekht vernacular of Nwên, the reflexive pronoun\is{reflexive pronouns} may also be the non-inflecting \textit{wê}, which can occur independently in a direct object position, see (\ref{ex.wê1})--(\ref{ex.wê3}).

Following \citet[20]{janic_comparing_2023}{}, a reflexive construction is a grammatical construction (i) that can only be used when two argument positions of a clause require coreference (ii) and that contains a particular form (a reflexiviser) that signals this coreference.
In Tekht H.\il{Hewramî!Tekht}, the reflexive pronouns\is{reflexive pronouns} can occur in the object position (\ref{ex.refl.1}), adposition complement (\ref{ex.refl.2}). and possessor in a possessive phrase (\ref{ex.refl.3}).

\ea
\textit{wêş wizone tewêɫeke.} \\ 
\gll \textbf{wê=ş} wiz-one tewêɫe-(e)ke \\ 
\textsc{refl=3sg:PSR} throw\textsc{.prs.ind-3sg:A} stable\textsc{.m-def.m.sg.dir} \\ 
\glt `He threw himself into the stable.'  \hfill[ŞC.64] \label{ex.refl.1}
\z 


\ea
\textit{kuřma berardû berdma çenû wêma.} \\ 
\gll kuř=ma ber-ard-∅=û berd-∅=ma çenû wê=ma\\ 
 boy\textsc{.m=1pl:A} out-bring\textsc{.pst-3sg.m:O}=and take\textsc{.pst-3sg.m:O=1pl:A} with\textsc{.ez.gen} \textbf{\textsc{refl=1pl:PSR}} \\ 
\glt `We took out the son, and took him with us.' \hfill[ZQ.53] \label{ex.refl.2}
\z 


\ea
\textit{bereşo memliketû wêta, welê serma deydê.} \\ 
\gll ber-e=ş=o memliket-û \textbf{wê=ta} welê ser=ma d\'e-îdê \\ 
 take\textsc{.prs.imp-2sg:A=3sg:O=compl} country\textsc{.m-ez.gen} \textsc{refl=2pl:PSR} but head\textsc{.m=1pl:R} give\textsc{.prs.imp-2pl:A} \\ 
\glt `Take him to your region, but pay us a visit.' \hfill[DG.68] \label{ex.refl.3}
\z 

The reflexive pronouns\is{reflexive pronouns} may also occur in the subject position, mainly to give extra emphasis to the subject argument.

\ea
\textit{şewê padşa wêş werm wîno.} \\ 
\gll şew(e)-ê padşa \textbf{wê=ş} werm wîn-o \\ 
 night\textsc{.f-indf} king\textsc{.m} \textsc{refl=3sg:PSR} sleep see\textsc{.prs.ind-3sg:A} \\ 
\glt `One night, the king had a dream [lit. saw sleep].'  \hfill[ZP.30]
\z 

In the vernaculars of Silên and Nwên, the reflexive pronoun\is{reflexive pronouns} is used in its bare\is{bare} form to express the direct object and, less so, the prepositional object; see (\ref{ex.wê1})--(\ref{ex.wê3}) and (\ref{wêadp}). The non-inflecting reflexive pronoun\is{reflexive pronouns} \textit{wê} can occur in the object position. The form may be considered a ``reflexive pronominoid" according to the terminology in \citet[]{janic_comparing_2023}{}.

\ea
\textit{wê suçno.} \\ 
\gll wê suçn-o \\ 
 \textsc{refl} burn.\textsc{prs.ind-3sg:A} \\ 
\glt `He (the demon) burnt himself.'  \label{ex.wê1} \hfill[ÇK.84]
\z 


\ea
\textit{zatşa mebo wê aşkera kera.} \\ 
\gll zat=şa me-b-o wê aşkera ker-a \\ 
 fear=\textsc{3pl:NC} \textsc{neg.ind-}be.\textsc{prs-3sg:S} \textsc{refl} disclosed do.\textsc{prs.ind-3pl:A} \\ 
\glt `They were afraid to make themselves visible.' \label{ex.wê2} \hfill[ÇK.67]
\z 


\ea
\textit{wê keř kero.} \\ 
\gll wê keř ker-o \\ 
 \textsc{refl} deaf do.\textsc{prs.ind-3sg:A} \\ 
\glt `He turned a deaf ear [lit. He deafened himself].' \label{ex.wê3} \hfill[JL.56] 
\z 

For other argument positions, the inflected reflexive pronouns\is{reflexive pronouns} are used:

\ea
\textit{minyoş piştû wêş.} \\ 
\gll mi-ny(e)-o=ş pişt-û wê=ş \\ 
 \textsc{ind}-put.\textsc{prs-3sg:A=3sg:O} behind\textsc{-ez.gen} \textsc{refl=3sg:PSR} \\ 
\glt `He got her riding behind him.'  
\z


\ea
\textit{berût yanew wêm.} \\ 
\gll ber-û=t yane-û wê=m \\ 
 take.\textsc{prs.ind-1sg:A=2sg:O} home\textsc{-ez.gen} \textsc{refl=1sg:PSR} \\ 
\glt `I will take you to my home.'  
\z


\ea
\textit{wêt zanî çêşa.} \\ 
\gll wê=t zan-î çêş=a \\ 
 \textsc{refl=2sg:PSR} know.\textsc{prs.ind-2sg:A} what=\textsc{cop.3sg.m:S}\\ 
\glt `You will understand what [the story] is.'  
\z

In the following example from the vernacular of Nwên, the bare\is{bare} form \textit{wê} has the same reference as the subject.

\ea \label{wêadp}
\textit{lûwa zemînêweş da werû wê.} \\
\gll lûwa zemîn-êwe=ş da wer-û wê \\
go.\textsc{pst.3sg:S} land.\textsc{indf=3sg:A} give.\textsc{pst} on-\textsc{ez.gen} \textsc{refl} \\
\glt `He went to a field [lit. gave a land on himself].' \hfill[ED.30]
\z 

\section{Reciprocal pronouns\is{reciprocal pronouns}} \label{sect:reciprocalpronoun}
The reciprocal pronoun is \textit{yotirîn}. It is grammatically derived from the numeral `one' and the superlative\is{superlative} suffix \textit{-tirîn}. \textit{yotirîn} can behave like a nominal, e.g., it can take the oblique suffix\is{oblique case}.

\ea
\textit{ađê êtir şađê bîyê be yotirînî.} \\
\gll ađê êtir şađ-ê bî-ê be yotirîn-î \\
\textsc{3pl} well happy-\textsc{pl} be.\textsc{pst-3pl:S} with \textsc{recp-m.sg.obl} \\
\glt `They became happy with each other.' \hfill[ME.202]
\z


\ea
\textit{watşa be yotirînî.} \\
\gll wat=şa be yotirîn-î \\
say\textsc{.pst=3pl:A} to \textsc{recp-m.sg.obl} \\
\glt `They talked to each other.' 
\z

\section{Indefinite pronouns\is{indefinite pronouns}}
Indefinite pronouns\is{indefinite pronouns} appear in four sets: ordinary, specific, free-choice, and negative. This classification is inspired by \citegen{haspelmath_grammar_1993} description of indefinite pronouns\is{indefinite pronouns} in Lezgian\il{Lezgian}. 
In the formation of ordinary and specific indefinite pronouns\is{indefinite pronouns}, the indefinite suffix \textit{-ê, -êwe} is added to the nominal (see \tabref{tab:indfpro}).

\begin{table}[htp] 
\begin{tabular}{llll}
\lsptoprule
\multicolumn{2}{c}{Ordinary} & \multicolumn{2}{c}{Specific} \\ \midrule 
\textit{kesê} & `someone' & \textit{kesê} & `a person' \\
\textit{çîwê} & `something' & \textit{çîwê} &`a thing' \\
\textit{çin carê} & `sometimes' & \textit{carê} & `once' \\
\textit{yagêwe} & `somewhere' &\textit{yagêwe} & `a place' \\
\lspbottomrule
\end{tabular}
\caption{Indefinite pronouns-- ordinary and specific sets}
\label{tab:indfpro}
\end{table}
 
As seen in \tabref{tab:indfpro}, ordinary and specific pronouns are formed identically. With ordinary indefinites, a sense of `some' is conveyed; see (\ref{ex.indf.pro1})--(\ref{ex.indf.pro2}).

\ea
\textit{vatiş, `dey tate dey ba çîwêweş pey bere!’} \\ 
\gll vat=iş dey tate dey ba \textbf{çîw-êwe}=ş pey b\'er-e \\ 
 say\textsc{.pst=3sg:A} \textsc{disc.ptcl} father\textsc{.m} \textsc{disc.ptcl} \textsc{hort} thing\textsc{-indf=3sg:R} for take\textsc{.prs.imp-2sg:A} \\ 
\glt `She said, ‘Father, get him something.’' \hfill[JH.38] \label{ex.indf.pro1}
\z 


\ea
\textit{yo kabrayêwe çowewe aman řakêre.} \\ 
\gll yo \textbf{kabra-êwe} ç=o=we ama=n řa-(e)kê=re \\ 
 one\textsc{.m} fellow\textsc{-indf} in\textsc{=dem.dist=post} come\textsc{.pst.ptcp.m=cop.3sg.m:S} road\textsc{.f-def.f.sg=post} \\ 

\glt `Someone came from the other direction onto their road.' \hfill[HB.49] \label{ex.indf.pro2}
\z

With specific indefinites, both specific referents (\ref{ex.indf.pro3}) and non-specific referents (\ref{ex.indf.pro4}) can be expressed.

\ea
\textit{yagê hen ...} \\ 
\gll \textbf{yag(ê)-ê} hen-∅ \\ 
 place\textsc{.f-indf} \textsc{exist-3sg.m:S} \\ 
\glt `There is a place ...' \hfill[JP.20] \label{ex.indf.pro3}
\z 


\ea
\textit{ne kesê hen be weş dađiş berî la.} \\ 
\gll ne \textbf{kes-ê} hen-∅ be weş dađ=iş b\'er-î la \\ 
 nor person\textsc{.m-indf} \textsc{exist-3sg.m:S} by good cry\textsc{.m=3sg:R} take\textsc{.prs.sbjv-2sg:A} to \\ 
\glt `Nor is there a person one can take counsel with.' \hfill[ZB.60] \label{ex.indf.pro4}
\z 

Free-choice indefinites\is{free-choice indefinites} are formed by adding the particle \textit{her, heç} `every, any' or \textit{girđ} `all' to a general noun (see \tabref{tab:freechoice}).

\begin{table}[htp]
    \begin{tabular}{lll}
    \lsptoprule
\textit{heçkes} & `whoever, anyone' \\
\textit{girđkes} & `everyone' \\
\textit{herçê} & `whatever, anything' \\
\textit{her yagêwe} & `anywhere' \\
\textit{her noʕê} & `any kind' \\
\lspbottomrule
    \end{tabular}
    \caption{Free choice indefinite pronouns}
    \label{tab:freechoice}
\end{table}

Free-choice indefinites\is{free-choice indefinites} express clauses containing possibilities where all options are permitted:

\ea
\textit{heçkes xelêş bê, xelêş çane san{ɛ}nê.} \\ 
\gll heçkes xel(e)-ê=ş b-ê xel(e)-ê=ş çane san{ɛ}=nê \\ 
 whoever grain\textsc{.m-pl.dir=3sg:NC} be\textsc{.prs-aug.3sg:S} grain\textsc{.m-pl.dir=3sg:A} from take\textsc{.pst.ptcp.pl=cop.3pl:R} \\ 
\glt `Whoever had grain, they would collect the grain from him as tax.' \\\hfill[BP.15]
\z 


\ea
\textit{êtir pagene girđkes ħeywandarîş kerden.} \\ 
\gll êtir pagene girđkes ħeywandarî=ş kerde=n \\ 
 \textsc{disc.ptcl} there everyone animal\_husbandry\textsc{=3sg:A} do\textsc{.pst.ptcp.m=cop.3sg.m:O} \\ 
\glt `There, everyone tended animals.' \hfill[JE.11]
\z 


\ea
\textit{herkesê herçêş lawe bo, misanaş çene.} \\ 
\gll herkes-ê herçê=ş la=we b-o mi-san-a=ş çene \\ 
 everyone\textsc{-indf} whatever\textsc{=3sg:R} with\textsc{=post} be\textsc{.prs.sbjv-3sg:S} \textsc{ind-}buy\textsc{.prs-3pl:A=3sg:R} from \\ 
\glt `They would take whatever possessions anyone had.' \hfill[BP.30]
\z 
Negative indefinite pronouns\is{negative indefinite pronouns} are generally derived from the particle \textit{hîç} combined with a general noun (see \tabref{tab:negindf}). When used attributively, occasionally \textit{hîç} is followed by the indefinite suffix\is{indefinite suffix} \textit{-ê}.

\begin{table}[htp]
    \begin{tabular}{lll}
    \lsptoprule
\textit{hîç} & `nothing' \\
\textit{kes, hîçkes} & `no one' \\
\textit{hîçke}& `never' \\
\textit{hîçkam}& `none' \\
\lspbottomrule
    \end{tabular}
    \caption{Negative indefinite pronouns}
    \label{tab:negindf}
\end{table}

Negative indefinites\is{negative indefinite pronouns} are used in clauses with negated verbs.

\ea
\textit{îđîç hîç minîç hîçim nîyarê.} \\ 
\gll îđ=îç \textbf{hîç} min=îç \textbf{hîç}=im nîy(e)=a=rê \\ 
 \textsc{3sg.prox.dir.m=add} nothing \textsc{1sg=add} nothing\textsc{=1sg:NC} \textsc{neg.exist=cop.3sg.m:S=povb} \\ 
\glt `She [has] nothing, nor do I.' \hfill[ZP.117]
\z 


\ea
\textit{kesî be noʕê pene nezanan.} \\ 
\gll \textbf{kes-î} be noʕ-ê pene ne-zana=n \\ 
 no\_one\textsc{-m.sg.obl} to kind\textsc{-indf} to \textsc{neg-}know\textsc{.pst.ptcp.m=cop.3sg.m:R} \\ 
\glt `No one really appreciated him.' \hfill[ZP.24]
\z 


\ea
\textit{î merasême her menowe, hîçkeyç kem mekero.} \\ 
\gll î merasêm=e her men-o=we \textbf{hîçke}=îç kem me-ker-o \\ 
 \textsc{dem.prox} ceremony\textsc{.m=dem} \textsc{emph} remain\textsc{.prs.ind-3sg:S=compl} never\textsc{=add} little \textsc{neg.ind-}do\textsc{.prs-3sg:A} \\ 
\glt `This ceremony will be repeated; it will never diminish.' \hfill[JP.245]
\z 


\ea
\textit{hîçkamma mihtajê mebîmê pîney.} \\ 
\gll \textbf{hîçkam}=ma mihtaj-ê me-b-îmê {} p=îney \\ 
 no\_one\textsc{=1pl:PSR} needy\textsc{-pl} \textsc{neg.ind-}be\textsc{.prs-1pl:S} {} to=\textsc{dem.prox.m.3sg.obl} \\  
\glt `None of us would be in need of these [products].' \hfill[PM.39]
\z 

\section{Interrogative pronouns\is{interrogative pronouns}} \label{secr:interogative_pronoun}
Interrogative pronouns\is{interrogative pronouns} form a closed class. They can, in principle, inflect for gender\is{gender}, case\is{case}, and number\is{number}, though the distinction is only visible in forms ending in a consonant. \tabref{tab:interpro} lists basic interrogative pronouns\is{interrogative pronouns}.
\begin{table}[htp]
    \begin{tabular}{lll}
    \lsptoprule
\textit{çêş}& what?\\
\textit{kê}& who? \\
\textit{çî}& why? \\
\textit{ko}& where? \\
\textit{çenî(n)} & how? \\
\textit{kam}& which? \\
\textit{çinne}& how many, how much? \\
\textit{key}& when? \\
\lspbottomrule
    \end{tabular}
    \caption{Interrogative pronouns}
    \label{tab:interpro}
\end{table}

\subsection{çêş `what?'}
\textit{çeş} refers to inanimate entities in the discourse. It occurs in the following forms in the corpus: \textit{çêş}; \textit{çêşe}; \textit{çêşî}.

\textit{çêş} is the unmarked form of the pronoun. It is used as the object of transitive clauses and subject in copula clauses.

\ea
\textit{mezano çêş kero.} \\ 
\gll me-zan-o çêş k\'er-o \\ 
 \textsc{neg.ind-}know\textsc{.prs-3sg:A} what do\textsc{.prs.sbjv-3sg:A} \\ 
\glt `He did not know what to do.' \hfill[ZP.111]
\z 


\ea
\textit{neweşîyekeyş çêş bo?} \\ 
\gll neweşî-ekey=ş çêş b-o \\ 
 illness\textsc{.m-def.m.sg.obl=3sg:PSR} what be\textsc{.prs.ind-3sg:S} \\  
\glt `What was her illness?' \hfill[ZP.27]
\z 

\textit{çêşe} refers back to a feminine\is{feminine} referent.

\ea
\textit{înê î qisê çêşene?} \\ 
\gll înê î qisê çêşe=ne \\ 
\textsc{dem.prox.f.3sg.dir} \textsc{dem.prox} talk\textsc{.f.sg} what\textsc{.f=cop.3sg.f:S} \\ 
\glt `What is this talk?' \hfill[JP.223]
\z 

In some cases, it seems that the feminine\is{feminine} form \textit{çêşe} extends its use and refers back to a masculine referent. Alternatively, the final \textit{e} in \textit{çeşe} in such cases can be considered a frozen definite suffix \textit{e}.

\ea
\textit{maço be tûwaɫekeyşa, `îne çêşen?’} \\ 
\gll m-aç-o be tûwaɫe-(e)key=şa îne çêş-e=n \\ 
 \textsc{ind-}say\textsc{.prs-3sg:A} to shell\textsc{-def.m.sg.obl=3pl:PSR} \textsc{dem.prox.m.3sg.dir} what\textsc{-f=cop.3sg.m:S} \\ 
\glt `She pointed to the (egg)shells [and] said, `What are these? [Lit. What is this?]’'  \hfill[JH.86]
\z 

The form \textit{çêşî} occurs as the genitive (\ref{ex.what-1}) and prepositional complement (\ref{ex.what-2}).

\ea
\textit{ewêç maço, `waranû çêşî?'} \\ 
\gll ewê=ç m-aç-o waran-û çêş-î \\ 
 \textsc{3sg.f.dir=add} \textsc{ind-}say\textsc{.prs-3sg:A} rain\textsc{-ez.gen} what\textsc{-m.sg.obl} \\  
\glt `She said, ‘What rain?!’' \hfill[ZB.16] \label{ex.what-1}
\z 


\ea
\textit{werû çêşî?} \\ 
\gll wer-û çêş-î \\ 
 out\_of\textsc{-ez.gen} what\textsc{-m.sg.obl} \\ 
\glt `for what reason?' \hfill[ŞC.106] \label{ex.what-2}
\z 

When used attributively, the particle has the form \textit{çi} `what':

\ea
\textit{mezanû ađ çi meẍzêweş henû ême çi meẍzêma hen.}\\
\gll me-zan-û ađ çi meẍz-êwe=ş hen-{\O}=û ême çi meẍz-ê=ma hen-{\O}\\
\textsc{neg.ind-}know.\textsc{prs-1sg:A} \textsc{3sg} what brain\textsc{-indf=3sg:NC} \textsc{exist-3sg:S=}and \textsc{1pl} what brain\textsc{-indf=1pl:NC} \textsc{exist-3sg:S} \\
\glt `I don't know what a brain he has and what a brain we have!' \hfill[XŞ.15]
\z

\subsection{kê `who?'}
The interrogative pronoun \textit{kê} refers to animate entities. It is of the invariable form \textit{kê} in different syntactic positions.

\ea
\textit{maço, `to kênî?'} \\ 
\gll m-aç-o to kê=nî \\ 
 \textsc{ind-}say\textsc{.prs-3sg:A} \textsc{2sg} who\textsc{=cop.2sg:S} \\
\glt `He (the king) said, ‘Who are you?’' \hfill[KŞ.10']
\z 


\ea
\textit{îse kê be zûwanû herî zano bêcge to?} \\ 
\gll îse kê be zûwan-û her-î zan-o bêcge to \\ 
 now who in language\textsc{.m-ez.gen} donkey\textsc{.m-sg.obl} know\textsc{.prs.ind-3sg:A} except\_for \textsc{2sg} \\  
\glt `Now who can understand the language of donkeys except you?' \hfill[HB.86]
\z 


\ea
\textit{kinaçêw padşayt îse pey kê niwîse?} \\ 
\gll kinaçê-û padşa-î=t îse pey kê niwîs-e \\ 
daughter\textsc{.f.sg.dir-ez.gen} king\textsc{.m-sg.obl=2sg:A} now to who write\textsc{.pst-3sg.f:O} \\  
\glt `Whom have you considered [as the future husband] for the king’s daughter? [Lit. Whom did you write the king’s daughter for?]' \hfill[KŞ.12]
\z 

When used as a genitive, it is equivalent in meaning to English\il{English} `whose':

\ea
\textit{fermawo, `pîyew kêndê?'} \\ 
\gll fermaw-o pîye-û kê=ndê \\ 
 say\textsc{.prs.ind-3sg:A} man\textsc{.m-ez.gen} who\textsc{=cop.2pl:S} \\  
\glt `He said, ‘Whose men are you?’' \hfill[ŞC.33]
\z 

Despite being invariable in form, \textit{kê} can trigger no indexing on the verb when used as a transitive subject in clauses built on a past stem verb. This typically occurs with \textit{kê} carrying nuclear stress (see \S\ref{sect:differential-A-indexing}).

\ea
\textit{î zeře çermeme kê berden?} \\
\gll î zeř-e çerme=m=e kê berde=n \\
\textsc{dem.prox} coin-\textsc{ez.cmpd} white=\textsc{1sg:PSR=dem} who take.\textsc{pst.ptcp.m=cop.3sg.m:O} \\
\glt `Who has taken my white coins?' \hfill[PK.29]
\z 

\subsection{çî `why?'}
\textit{çî} is an invariable form used to express `why'.

\ea
\textit{çî ʕaciz bîyenî?’} \\ 
\gll çî ʕaciz bîye=nî \\ 
why upset\textsc{.m} be\textsc{.pst.ptcp.m=cop.2sg:S} \\  
\glt `Why did you get annoyed?’' \hfill[HB.81]
\z 

\subsection{ko `where?'}
\textit{ko} has an invariable form and is used both in the sense of direction and location.

\ea
\textit{watim peneş, `to milî ko?'} \\ 
\gll wat=im pene=ş to mi-l-î ko \\ 
 say\textsc{.pst=1sg:A} to\textsc{=3sg:R} \textsc{2sg} \textsc{ind-}go\textsc{.prs-2sg:S} where \\ 
\glt `I said to him, `Where are you going?’' \hfill[JH.31]
\z 


\ea
\textit{îne konê ewê? çêşşa pene ama?} \\ 
\gll îne ko=nê ewê çêş=şa pene ama-∅ \\ 
 \textsc{prsv} where\textsc{=cop.3pl:S} \textsc{3pl.dir} what\textsc{=3pl:R} to come\textsc{.pst-3sg.m:S} \\ 
\glt `Where are they? What happened to them?' \hfill[BP.112]
\z 
 
The particle also occurs in combination with an adposition. The resultant forms sometimes yield different shades of meaning. The forms \textit{çiko}, \textit{çikowe}, \textit{kowe}, and \textit{koge} are attested in the text corpus.

\ea
\textit{mênê maça, `şê ʕumer çikon?'} \\ 
\gll m-ê-nê m-aç-a şê ʕumer çiko=n \\ 
 \textsc{ind-}come\textsc{.prs-3pl:S} \textsc{ind-}say\textsc{.prs-3pl:A} Sheikh\textsc{.m} \textsc{pn} where\textsc{=cop.3sg.m:S} \\ 
\glt `They went [and] said, ‘Where is Sheikh Omer?’' \hfill[ŞC.10]
\z 


\ea
\textit{nîşdêrew fermawaş, `çikowe meydê?'} \\ 
\gll n\stackunder[-10pt]{\^{i}}{\'{}}ş-dê=re=û fermawa=ş çi=ko=we m-e-îdê \\ 
 sit\textsc{.prs.imp-2pl:S=povb}=and say\textsc{.pst=3sg:A} from=where\textsc{=compl} \textsc{ind-}come\textsc{.prs-2pl:S} \\ 
\glt `Sit down!' [the Shah] said, ‘Where do you come from?’' \hfill[ZQ.4]
\z 


\ea
\textit{ey wêm kowe bilû?} \\ 
\gll ey wê=m ko=we bi-l-û \\ 
\textsc{intj} \textsc{refl=1sg:PSR} where\textsc{=post} \textsc{sbjv-}go\textsc{.prs-1sg:S} \\
\glt `Where should I go myself?' \hfill[DP.23]
\z 

 \ea 
\textit{îse koge\footnotemark bilî?} \\ 
\gll îse koge bi-l-î \\ 
now where \textsc{sbjv-}go\textsc{.prs-2sg:S} \\ 
\glt `Where might you be going?' \hfill[JH.17]
\z\footnotetext{Additionally, the phonologically similar \textit{kûgey} `where from' has been attested, e.g., \textit{kûgeynî?} `Where are you from?'} 
 
\subsection{çenî `how?'}
The particle has two variants: \textit{çenîn}, and \textit{çenî}. The former is used with the present copula. The latter is used in all other contexts.

\ea
\textit{vatiş, `çenîna?' cafir san.} \\ 
\gll vat=iş çenîn=a cafir san \\ 
 say\textsc{.pst=3sg:A} how\textsc{=cop.3sg.m:S} \textsc{pn} \textsc{pn} \\ 
\glt `Jafir San said, ‘What is he like?’' \hfill[ŞC.16]
\z 

 \ea 
\textit{maço, `tamû mezeşa çenî bî yerêşa?'} \\ 
\gll m-aç-o tam=û meze=şa çenî bî-∅ yerê=şa \\ 
 \textsc{ind-}say\textsc{.prs-3sg:A} taste=and taste\textsc{=3pl:PSR} how be\textsc{.pst-3sg.m:S} three\textsc{=3pl:PSR} \\ 
\glt `[and] said, ‘How did the three of them (the eggs) taste?’' \hfill[JH.97]
\z 
 
\subsection{kam `which?'}
\textit{kam} is morphologically inflected. The forms \textit{kam}, \textit{kamê}, and \textit{kamî} have been attested.

\ea
\textit{a firmande wêş zano heqû ême kama.} \\ 
\gll a firmande wê=ş zan-o heq-û ême kam=a \\ 
\textsc{dem.dist} leader\textsc{.m} \textsc{refl=3sg:PSR} know\textsc{.prs.ind-3sg:A} right\textsc{.m-ez.gen} \textsc{1pl} which\textsc{=cop.3sg.m:S} \\ 
\glt `[Your] boss knows which portion [of land] is mine.' \hfill[PM.20]
\z 


\ea
\textit{xo haɫît bo hêɫê yanê kamênê?} \\ 
\gll xo haɫî=t b-o hêɫ(e)-ê yanê kam-ê=nê \\ 
 \textsc{dis.ptcl} understood\textsc{=2sg:NC} be\textsc{.prs.ind-3sg:S} egg\textsc{.m-pl.dir} that\_means which\textsc{-pl.dir=cop.3pl:S} \\  
\glt `Do you understand what [the word] \textit{hêɫê} `eggs' [is]? [Do you know] what they are?' \hfill[JH.82]
\z 


\ea
\textit{daxom î jenî pey kamî bo?} \\
\gll daxom î jenî pey kam-î b-o \\
\textsc{q.ptcl} \textsc{dem.prox} woman for which-\textsc{obl.m} be.\textsc{prs.sbjv-3sg:S} \\
\glt `Which one would the girl be [married to]?' \hfill[SH.147]
\z 

\subsection{çinne `how many, how much'}
The particle has the direct form \textit{çinne}.

\ea
\textit{mareyî çinne bê?} \\
 \gll mareyî çinne b-ê \\
 bride\_price how\_much be\textsc{.prs-aug.3sg:S} \\ 
\glt `How much was the bride price?'
\z 

When used as a genitive, the particle has the form \textit{çinnê} and is equivalent in meaning to English\il{English} `which'.

\ea
\textit{êtir meyanû saɫû çinnê bê.} \\ 
\gll êtir me-zan-û saɫ(e)-û çinê b-ê \\ 
 \textsc{disc}.\textsc{ptcl} \textsc{neg.ind-}know\textsc{.prs-1sg:A} year\textsc{.f}\textsc{-ez.gen} how\_many\textsc{.obl} be\textsc{.prs-aug.3sg:S} \\ 
\glt `I don’t know which year it was.' \hfill[JM.49]
\z 

When used attributively, the particle has the reduced form \textit{çin}.

\ea
\textit{zemawine çin řowê bê?} \\ 
\gll zemawine çin řo-ê b-ê \\ 
 wedding\_ceremony how\_many day\textsc{.m-pl.dir} be\textsc{.prs-aug.3sg:S} \\ 
\glt `How many days was the wedding ceremony?'
\z 


\subsection{key `when'}
The particle has the invariable form \textit{key}.

\ea
\textit{key bî çoɫ?} \\
\gll key bî-∅ çoɫ \\
when be.\textsc{pst-3sg.m:S} deserted \\
\glt `When did it become deserted?' \hfill[SH.251]
\z 

\end{sloppypar}
