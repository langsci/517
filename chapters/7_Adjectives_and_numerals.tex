\chapter{Adjectives and numerals\is{numerals}}
\begin{sloppypar}
\section{Adjectives}
The category of adjectives is characterised by carrying agreement in gender\is{gender agreement} and number\is{number agreement} when used attributively or predicatively. When substantivised, adjectives can occur as the head of a noun phrase, inflecting for case\is{case} as well. 
In terms of semantic content, the adjectives in Tekht Hewramî\il{Hewramî!Tekht} encompass semantic fields commonly found across world languages \citep{dixon_adjective_2004}, including dimension (\ref{ex.dimension}), age (\ref{ex.age}), value (\ref{ex.value}), colour (\ref{ex.colour}), physical property (\ref{ex.ph-property}), human propensity (\ref{ex.hum-prop}), qualification (\ref{ex.qualification}), and order (\ref{ex.order}). For all the adjectives listed below, the masculine form of the adjective has been given. The feminine form has an additional \textit{-e}, and the plural form has \textit{-ê}; e.g., \textit{xas} `good (\textsc{m})'; \textit{xas-e} `good (\textsc{f})'; \textit{xas-ê} `good (\textsc{pl})'.\\
\textbf{I. Dimension}
\TabPositions{2.75cm,6cm,7.15cm,9cm}
\ea \label{ex.dimension}
\textit{zil}\tab `big, giant’ \tab \textit{pan}\tab `wide (of road)’\\
\textit{wiçkile}, \textit{miçkile}\tab `small (of object)’ \tab \textit{barîk}\tab `narrow (of road)’\\
\textit{řêz}\tab `tiny’ \tab \textit{xirt}\tab `round (of tray)’\\
\textit{dirêj}\tab `tall’ \tab \textit{teng}\tab `not spacious (of house)’\\
\textit{kuɫ}\tab `short’ \tab \textit{tesk}\tab `narrow (of trousers)’\\
\textit{berz}\tab `high’\tab  \textit{tûl}\tab `vast (of area)’\\
\textit{nizm}, \textit{kuɫ}\tab `low (wall)’ \tab  \textit{şoř}\tab `loose, pendant’\\
\z

\textbf{II. Age}
\TabPositions{1.5cm,4.5cm,7.15cm,9cm}
\ea \label{ex.age}
\textit{pîr}\tab `old’ \tab \textit{taze}\tab `new (of object)’ \\ 
\textit{cuwan}\tab `young’ \tab \textit{teř}\tab `wet, fresh (of human)’\\
\textit{kone}\tab `old (of object)’ \tab \textit{gewre}\tab `old (of age)’\\
\tab  \tab \textit{wiçkile}, \textit{miçkile}\tab `little (of age)’\\
\z

\textbf{III. Value}

\ea \label{ex.value}
\textit{ʕal}\tab `good (of action)’ \tab  \textit{gul}\tab `bad, dirty’\\
\textit{weş}\tab `good (of taste)’ \tab  \textit{xirab}\tab `bad (of object)’\\
\textit{xas}\tab `good (of human)’ \tab  \textit{giran}\tab `expensive’ \\
\textit{beđ}\tab `bad (of action)’ \tab  \textit{herzan}\tab `cheap’ \\
\z

\textbf{IV. Colour}
\TabPositions{1.95cm,4.75cm,7.75cm,9cm}
\ea \label{ex.colour}
\textit{sîyaw}\tab `black’ \tab \textit{sewz}\tab `green’\\
\textit{çerme}\tab `white’ \tab \textit{zer}\tab `yellow’\\
\textit{sûr}\tab `red’ \tab \textit{qaweyî}\tab `brown’ \\
\textit{kewe}\tab `blue’ \tab \textit{çermeɫe}, \textit{çermeɫê} \tab `whitish’\\
\tab \tab \textit{bazbaz}\tab `black and white’ \\
\z

\textbf{V. Physical property}
\TabPositions{1.95cm,4.75cm,7.75cm,9cm}
\ea \label{ex.ph-property}
\textit{têj}\tab `sharp (of knife)’\tab \textit{wişk}\tab `dry’\\
\textit{kul}\tab `blunt’\tab \textit{pîs}\tab `dirty’\\
\textit{tiş}\tab `sour’\tab \textit{gena} (\textsc{m}); \textit{genê} (\textsc{f})\tab `rotten’\\
\textit{soɫ}\tab `salty’\tab \textit{xîş}\tab `fat’\\
\textit{şîrîn}\tab `sweet’\tab \textit{hejar}\tab `thin’ (of human)\\
\textit{taɫ}\tab `bitter’\tab \textit{leř}\tab  `thin'\\
\textit{saf}\tab `flat (of level)’\tab \textit{qaym}, \textit{biřik}\tab `thick’ (of object)\\
\textit{germ}\tab `warm’\tab \textit{nasik}\tab `thin (e.g., of object)’\\
\textit{serđ}\tab `cold’\tab \textit{keř}\tab `deaf’ \\
\textit{qurs}\tab `heavy’\tab \textit{laɫ}\tab `mute’ \\
\textit{sûk}\tab `light’\tab \textit{qoɫ}, \textit{qûɫ}\tab `deep’\\
\textit{nerm}\tab `soft’\tab \textit{puxte}\tab `clean’\\
\textit{nemdar}, \textit{teř}\tab `wet’\tab \textit{řeq}, \textit{pitew}\tab `hard’ (e.g., of stone)\\
\textit{kaɫ}\tab `unripe (fruit)’\tab \textit{lem(e)-dare}\tab `pregnant’\\
\tab \tab \textit{duwe gîyane}\tab `pregnant’\\
\z

\textbf{VI. Human propensity}

\ea \label{ex.hum-prop}
\textit{adiz}\tab `upset’\tab \textit{weşɫe} \tab `cute’\\
\textit{weşhaɫ}, \textit{kok}\tab `happy’\tab \textit{jîr}\tab `wise’\\
\textit{nařihet}\tab `sad’\tab \textit{ezem} / \textit{ezeme} \tab `unmarried, single’\\
\textit{keç}\tab `crooked’\tab \textit{har}\tab `restless’\\
\textit{cuwanxas}\tab `good-looking’\tab \textit{qûwet}\tab `strong’\\
\textit{zoɫ}\tab `cunning’\tab \textit{zeʕîf}, \textit{zayf}, \textit{kiz}\tab `weak’\\
\textit{manya}\tab `tired’\tab  \textit{zerîf}\tab `beautiful (of woman)’\\
\z

\textbf{VII. Qualification}
\TabPositions{3cm,4.75cm,7.75cm,9cm}
\ea \label{ex.qualification}
\textit{řas}\tab `true’ \tab \textit{ẍeɫet}, \textit{qeret}\tab `wrong’\\
\textit{meʕlûm}\tab `obvious’ \tab \textit{sehl}\tab `easy’\\
\tab \tab \textit{tûl}\tab `long (of story)' \\
\z

\textbf{VIII. Order}

\ea \label{ex.order}
\textit{yomîn, yekemîn}\tab `first’ \tab \textit{duwê}\tab `second’\\
\tab \tab \textit{axirîn}\tab `last’ \\
\z

\subsection{Adjectival inflection}\label{adjective-infl}

The class of adjectives is inflectionally distinct from that of nouns because while nouns are inherently specified for one gender\is{gender}, adjectives have no inherent gender\is{gender}. That is, adjectives agree with the head noun in gender\is{gender agreement} but do not bear gender\is{gender} distinctions per se. 

The defining feature of adjectives is that they agree in gender and number when used attributively (see \S\ref{sect:attributive-adj}) or predicatively (see \S\ref{sect:predicative-adj}). When substantivised, adjectives take nominal inflection (see \S\ref{sect:substantive}). 

Similarly, the category of participles as deverbal adjectives inflect for gender\is{gender} and number\is{number}, as can be seen in the inflection of the resultative participle form of \textit{witey} `to sleep’ in (\ref{ex.witey}), and in example (\ref{ex.ptcppl}):
\TabPositions{1.5cm}
\ea Inflected participial forms of \textit{witey} `to sleep' \label{ex.witey}\\
\textsc{m.sg}\tab  \textit{wit-e} \\
\textsc{f.sg}\tab  \textit{wit-ê} \\
\textsc{pl}\tab  \textit{wit-ê} \\
\z


\ea \label{ex.ptcppl}
\textit{paɫê diřyê} \\
\gll paɫ(a)-ê diřy(a)-ê \\
shoe-\textsc{pl} tear.\textsc{pst.ptcp-pl} \\
\glt `torn shoes’ 
\z  


\subsubsection{Attributive adjectives}\label{sect:attributive-adj}
When used in a head-modifier relation (i.e., when used attributively), adjectives agree in gender and number with the head noun. The relevant inflectional formatives are \textit{-\O} [\textsc{m.sg}] and \textit{-e} [\textsc{f.sg}] cumulatively expressing gender and number, and \textit{-ê} [\textsc{pl}] expressing number agreement in the plural. 
\begin{itemize}
    \item Examples of gender agreement\is{gender agreement}:

\ea
\textit{kuřê cuwanxas} \\
\gll kuř-ê cuwanxas-\O \\
boy.\textsc{m-indf} good.looking-\textsc{m} \\
\glt `a good-looking boy’ \hfill[KŞ.68] \\
\z 


\ea
\textit{jenê xase} \\
\gll jen(î)-ê xas-e \\
woman.\textsc{f-indf} nice-\textsc{f}\\
\glt `a nice wife’ \hfill[JH.64] \\
\z 


\ea
\textit{yagê cîyakare} \\
\gll yag(e)-ê cîyakar-e\\
place\textsc{.f-indf} separate\textsc{-f} \\
\glt `a separate place’ \hfill[JP.63] \\
\z 

\item Examples of number agreement\is{number agreement}:

\ea
\textit{însanê bêʕeqɫê} \\ 
\gll însan-ê bêʕeqɫ-ê \\ 
 human\textsc{.m}\textsc{}\textsc{-pl} silly\textsc{\textsc{-pl}}  \\ 
\glt `silly men' \hfill[ŞC.28]
\z  


\ea
\textit{karê xirabê} \\
\gll kar-ê xirab-ê \\
thing\textsc{.m}-\textsc{pl} bad-\textsc{pl} \\
\glt `bad things’ \hfill\citep[439]{khan_language_2023} \\
\z 

\end{itemize}

\subsubsection{Predicative adjectives}\label{sect:predicative-adj}
In their predicative use, adjectives agree in gender\is{gender agreement} and number\is{number agreement} with the subject of copula clauses. Examples of gender agreement\is{gender agreement}:

\ea
\textit{kinaçekê felece bo.} \\ 
\gll kinaç(ê)-ekê felec-e b-o \\ 
 girl\textsc{.f-def.f.sg} disabled\textsc{-f} be\textsc{.prs.ind-3sg:S} \\ 
\glt `The girl was disabled.' \hfill[JP.149] 
\z 


\ea
\textit{řama dûrene.} \\ 
\gll řa=ma dûr-e=ne \\ 
 road\textsc{.f=1pl:PSR} far\textsc{-f=cop.3sg.f:S} \\ 
\glt `We have a long way [to go]. [Lit. Our way is far.]' \hfill[BP.191]
\z 

Example (\ref{agr.n.adj}) features a predicative adjective agreeing in number\is{number agreement} with the subject argument.

\ea
\textit{ême firê bênmê.} \\ 
\gll ême fir(e)-ê b-ên-mê. \\ 
 \textsc{1pl} a\_lot\textsc{-pl} be.\textsc{prs-aug-1pl:S} \\  
\glt `We were a large number.' \hfill[BP.110] \label{agr.n.adj}\\
\z 

The examples below illustrate the agreement pattern of coordinate predicative adjectives, linked by the conjunction \textit{=û} `and' (see \S\ref{sect:conjunction}). In most cases, only the second coordinate adjective carries agreement with the head subject argument, see (\ref{coord-adj1})--(\ref{coord-adj2}), though it is possible that both coordinate adjectives carry agreement markers (\ref{coord-adj3}).

\ea
\textit{êtir kinaçê keřû laɫe bo.} \\ 
\gll êtir kinaçê keř(e)=û laɫ-e b-o \\ 
 \textsc{disc.ptcl} daughter\textsc{.f.dir} deaf\textsc{.f}=and mute\textsc{-f} be\textsc{.prs.ind-3sg:S}  \\ 
\glt `The girl was deaf-mute.' \hfill[ZP.51] \label{coord-adj1}
\z 


\ea
\textit{pîsû poxɫêndê.}\\
\gll pîs=û poxɫ-ê=ndê \\
dirty=and grubby\textsc{-pl}=\textsc{cop.2pl:S} \\
\glt `You (pl.) are dirty and grubby.' \label{coord-adj2}  \hfill[ÇK.20]
\z 


\ea
\textit{ênneç zerîfew mehbûbe bê.} \\
\gll ênne=ç zerîf-e=û mehbûb-e b-ê \\
so\_much\textsc{=add} beautiful-\textsc{f}=and comely-\textsc{f} be.\textsc{prs-aug.3sg:S} \\
\glt `She was so beautiful and comely.' \hfill[DB.175] \label{coord-adj3} 
\z

It is notable that inherently plural\is{plural} mass nouns, which are grammatically feminine (see \S\ref{inh-pl-section}), trigger plural\is{plural} marking on predicative adjectives in non-verbal clauses.

\ea
\textit{mekî soɫênê.} \\
\gll mekî soɫ-ê꞊nê \\
salt.\textsc{f} salty-\textsc{pl꞊cop.3pl:S} \\
\glt `The salt is salty.’ 
\z 


\ea
\textit{vamî wişkênê.} \\
\gll vamî wişk-ê꞊nê \\
almond.\textsc{f} dry-\textsc{pl꞊3pl:S} \\
\glt `The almonds are dry.’ 
\z 


\ea
\textit{qijê sîyawênê.} \\
\gll qijê sîyaw-ê꞊nê \\
hair.\textsc{f} black-\textsc{pl꞊cop.3pl:S} \\
\glt `The hair is black.’ 
\z 


\subsubsection{Substantivised adjectives}\label{sect:substantive}
When substantivised, adjectives inflect for number\is{number}, case\is{case}, and gender\is{gender} (only in the singular) through fusional endings. The inflection of substantivised adjectives is identical to that of nouns, hence there is one underlying inflectional class for nominal/adjectival inflection. This inflectional class manifests itself only in the adjectives' substantive use. The inflection of \textit{cuwan} `young’ in (\ref{ex.adjinfl1}) illustrates the underlying fusional suffixes for substantivised adjectives.
\TabPositions{1.5cm,3.5cm,6.5cm,8.5cm}
\ea \textit{cuwan} `young' \label{ex.adjinfl1}\\
\tab \textsc{m}\tab \tab \textsc{f} \\
\textsc{sg.dir}\tab \textit{cuwan}\tab \tab \textit{cuwane} \\
\textsc{sg.obl}\tab \textit{cuwan-î}\tab \tab \textit{cuwan-ê} \\
\textsc{pl.dir}\tab \tab \textit{cuwan-ê}  \\
\textsc{pl.obl}\tab \tab \textit{cuwan-a} [ZQ.54] \\
\z 

The underlying inflection undergoes phonologically-conditioned allomorphy when the adjective ends in a vowel. The inflection patterns of \textit{gewre} `big, old (of age)’ (\ref{ex.adjinfl2}) and \textit{awra, wira} `hungry' (\ref{ex.adjinfl3}) are given as examples.

\ea \textit{gewre} `big, old (of age)’ \label{ex.adjinfl2}\\
\tab \textsc{m}\tab \tab \textsc{f} \\
\textsc{sg.dir}\tab \textit{gewre}\tab \tab \textit{gewrê} \\
\textsc{sg.obl}\tab \textit{gewrey}\tab \tab \textit{gewrê} \\
\textsc{pl.dir}\tab \tab \textit{gewrê} \\
\textsc{pl.obl}\tab \tab \textit{gewra}\\
\z


\ea \textit{awra} `hungry' \label{ex.adjinfl3}\\
\tab \textsc{m}\tab \tab \textsc{f} \\
\textsc{sg.dir}\tab \textit{awra}\tab \tab \textit{awra} \\
\textsc{sg.obl}\tab \textit{awra-y}\tab \tab \textit{awrɛ} \\
\textsc{pl.dir}\tab \tab \textit{awrɛ} \\
\textsc{pl.obl}\tab \tab \textit{awra-ya}\\
\z


\subsubsection{Adjectives in the NP}
Adjectives link to the head noun via a head-linking strategy called ezafe within Iranian linguistics (see \S\ref{sect:attr.ez}). In the material from the text corpus, the ezafe linker is dropped following indefinite\is{indefinite suffix} and plural\is{plural} suffixes (see \S\ref{sect:attr.ez} for details and exceptions). The attributive ezafe\is{attributive ezafe} linker \textit{-î} connects the head noun to the adjective, when the head noun is either bare (\ref{ex.bare-adj}), or a proper noun (\ref{ex.pn-adj}). The linker \textit{-î} also links relative clauses in present tense constructions to a 3rd person pronominal head, see (\ref{ex.aney-reladj}).

\ea \label{ex.bare-adj}
\textit{diɫî beđ} \\
\gll diɫ-î beđ \\
heart.\textsc{m-ez.attr} bad \\
\glt `anger [lit. bad heart]’ \hfill[DP.38] \\
\z 


\ea \label{ex.pn-adj}
\textit{heyasî jîr} \\
\gll heyas-î jîr \\ 
\textsc{pn}\textsc{-ez.attr} shrewd \\ 
\glt `Hayas the Wise' \hfill[JH.2]
\z 


\ea \label{ex.aney-reladj}
\textit{aney pa milo řaw hatîre bo patşa.} \\
\gll ane-\textbf{î} pa mi-l-o řa-û hat-î=re b-o patşa \\
\textsc{dem.dist.m.3sg.dir-ez.attr} from\_there \textsc{ind-}go.\textsc{prs-3sg:S} way\textsc{.f-ez.gen} fortune-\textsc{m.sg.obl=post} become.\textsc{prs.ind-3sg:S} king \\
\glt `The one who went to the way of good fortune became a king.' [DB.111]\\
\z 

\subsection{Adjectival derivation} \label{sect:adj-derivation}
Adjectives can be derived from other word categories by adding an affix, often a suffix. The most common affixes found in the corpus are listed below. 

\subsubsection{-î} 
The derivational suffix \textit{-î} is the most productive adjectivising suffix in Hewramî. It can derive adjectives from nouns (\ref{ex.-i}), and from adjectives (\ref{ex-ii}). In the former, \textit{-î} derives gentilic adjectives from nouns. It can also convey the meaning `affiliation’.
\TabPositions{2.5cm}
\ea \label{ex.-i}
\textit{ʕereb-î}\tab  `Arabic’\il{Arabic} \\
\textit{kirmaşan-î}\tab  `from Kermanshah’ \\
\textit{şerʕ-î}\tab  `lawful’ \\
\textit{naşerʕ-î}\tab  `unlawful’ \\
\textit{xwa-yî}\tab  `divine' \\
\z

The suffix sometimes derives an adjective from another adjective. This is particularly the case with Arabic \il{Arabic}loans, whose word class was probably obscure when borrowed. Thus, adding \textit{-î} ensured their being classed as adjectives.

\ea \label{ex-ii}
\textit{muweqet-î}\tab  `temporary’ \\
\z

\subsubsection{-in} 
The adjectivising suffix \textit{-in} derives adjectives indicating a characteristic typical of a person.
\TabPositions{2cm,4cm,6cm}
\ea 
\textit{çiɫk-in}\tab `dirty’\tab cf. \textit{çiɫk}\tab `microbe’ \\
\textit{çiɫm-in}\tab `snotty’\tab cf. \textit{çiɫm}\tab `mucous’ \\
\textit{mirxin}\tab `snorting’\tab cf. \textit{mirxe}\tab `snort’ \\
\z

\subsubsection{-dar} 
The adjectivser suffix \textit{-dar} derives adjectives from nouns with roughly the meaning of `having’.
\TabPositions{2cm,5cm,7cm}
\ea
\textit{lemdare}\tab `pregnant’\tab cf. \textit{leme}\tab `belly’ \\
\textit{nemdar}\tab `wet’\tab cf. \textit{nem}\tab `humidity’ \\
\textit{zamdar}\tab `wounded’\tab cf. \textit{zam} \tab `wound’ \\
\textit{heyadar}\tab  `self-conscious'\tab  cf. \textit{heya}\tab  `decency' \\
\textit{qûwedar}\tab  `strong, powerful' \tab  cf. \textit{qûwe} \tab `strength' \\
\textit{samdar}\tab  `formidable' \tab  cf. \textit{sam} \tab `awe' \\
\textit{ʕeybdar}\tab  `defective, faulty' \tab  cf. \textit{ʕeyb} \tab `fault' \\
\z

\subsubsection{-oɫ}
The derivational suffix \textbf{-oɫ}, likely to be cognate with the diminutive \textit{oɫe}, derives adjectives from nouns. When added to the base, it conveys the meaning `the quality of' or `full of'.

\ea
\textit{kirm(î)-oɫ}\tab `wormy’\tab cf. \textit{kirmî}\tab  `worm’ \\
\textit{geřoɫ}\tab  `debased (of coin)'\tab  cf. \textit{geř} \tab  `curve, circle' \\
\z

\subsubsection{ne-} 
The adjectivising prefix \textit{ne-} is a negative prefix that derives adjectives from nouns and adjectives alike.

\ea
\textit{nefam}\tab `inexperienced’\tab cf. \textit{fam}\tab  `understanding’ \\
\textit{neweş}\tab `ill’\tab cf. \textit{weş}\tab  `well, nice’ \\
\z

\subsubsection{na-} 
The prefix \textit{na-}, like \textit{ne} is an adjectivising suffix with the core meaning `not’. However, it seems to be only deriving adjectives from adjectives.
\TabPositions{2cm,5.75cm,7.5cm}
\ea
\textit{nařihet}\tab `sad’\tab cf. \textit{řihet}\tab `relaxed’ \\
\textit{namerđ}\tab `ungallant’\tab cf. \textit{merd}\tab `gallant’ \\
\textit{naşerʕî}\tab `unlawful’\tab cf. \textit{şerʕî}\tab `lawful’ \\
\textit{nabeɫe}\tab  `unskilled, unfamiliar' \tab cf. \textit{beɫe}\tab  `skilled, familiar'\\
\textit{naxafiɫ}\tab  `unexpected' \\
\z

\subsubsection{Other derivational affixes} \label{deriv-adj}
Less common derivational affixes include the negation prefix \textit{bê-}, the diminutive\is{diminutive} \textit{-ɫe} (\textsc{m.sg}) /\textit{-ɫê} (\textsc{f.sg, pl}), and \textit{-mek}.

\ea
\textit{bê-heya}\tab `indecent’\tab cf. \textit{heya}\tab `decency’ \\
\textit{bê-ʕeqɫ}\tab `silly’\tab cf. \textit{ʕeqɫ}\tab `wisdom’ \\
\textit{bê-xeber}\tab  `unaware' \tab  cf. \textit{xeber}\tab  `news' \\
\textit{bêkar}\tab  `unemployed' \tab  cf. \textit{kar}\tab  `job, task' \\
\textit{weş-ɫe}\tab `cute' (\textsc{m})\tab cf. \textit{weş}\tab `well, nice’ \\
\textit{weş-mek}\tab `amusing’\tab cf. \textit{weş}\tab `well, nice’ \\
\z

The occurrence of the diminutive\is{diminutive} suffixes \textit{-ɫe/-ɫê} with the adjectives gives a meaning of intensification (see \citealt[]{jurafsky_universal_1996} for a typology of diminutive suffixes):
\TabPositions{2.5cm,5.75cm,7.5cm}
\ea
\textit{hejar-le}\tab `emaciated’\tab cf. \textit{hejar}\tab `thin’ \\
\textit{feqîr-le}\tab `destitute’\tab cf. \textit{feqîr}\tab `poor’ \\
\textit{denûle/ denale}\tab  `tiny'\tab  cf. \textit{dane} \tab  `seed' \\ 
\z

\subsection{Compound adjectives}
Like nouns, adjectives can be formed through compounding. The most common strategy for deriving compound adjectives is to combine a nominal root with an adjective. Like simple adjectives, compound adjectives show agreement with the nominal head, as shown by \textit{cîyakare} in (\ref{ex.adjcomp1}) < \textit{cîya} `separate’ + \textit{kar} `job’ + \textit{-e} (\textsc{f}), and \textit{lemepeře} `pregnant’ in (\ref{ex.adjcomp2}) < \textit{leme} `belly’ + \textit{peř} `full’ + \textit{-e} (\textsc{f}).

\ea \label{ex.adjcomp1}
\textit{yagê cîyakare} \\
\gll yag(ê)-ê cîyakar-e\\
place\textsc{.f-indf} separate\textsc{-f} \\
\glt `a separate place’ \hfill[JP.63] 
\z 



\ea \label{ex.adjcomp2}
\textit{ jenê lemepeře} \\ 
\gll jen(î)-ê lemepeř-e \\ 
 woman\textsc{.f}\textsc{-indf} pregnant\textsc{-f} \\ 
\glt `a pregnant woman' \hfill[KŞ.17]
\z 

\subsubsection{Noun + adjective compounds}
In Noun + adjective compounds, an adjective (\ref{ex.cmpdadj}) or a participle (\ref{ex.cmpdptcp}) is juxtaposed to the nominal root.
\TabPositions{2.25cm,5.25cm,8.5cm,9.5cm}
\ea \label{ex.cmpdadj}
\textit{lemepeře}\tab `pregnant’\tab < \textit{leme} `belly’ + \textit{peře} `full’ \tab [KŞ.17] \\
\textit{payebiɫin}\tab `grand’\tab < \textit{paye} `leg’ + \textit{biɫin} `high' \tab [DG.4] \\
\textit{serberz}\tab  `proud, dignified'\tab  < \textit{ser} `head' + \textit{berz} `high'\\
\z


\ea \label{ex.cmpdptcp}
\textit{lemdiřya}\tab `glutton’\tab < \textit{leme} `belly’ + \textit{diřya} `torn’
\z

\subsubsection{Adjective + noun compounds}
Adjectives can be formed equally by combining with a noun through simple compounding or, less commonly, through the compound marker \textit{-e} (see \S\ref{sect:cmpdez}).
\TabPositions{1.85cm,5.15cm,8.65cm,9.75cm}
\ea
\textit{weşkelam}\tab `eloquent’\tab < \textit{weş} `good’ + \textit{kelam} `word’ \tab [KŞ.42] \\
\textit{beđbext}\tab `poor, unlucky’\tab < \textit{beđ} `bad’ + \textit{bext} `luck’ \tab [JE.43] \\
\textit{cîyakar}\tab `separate’\tab < \textit{cîya} `separate’ + \textit{kar} \tab [JP.63] \\
\textit{noxet}\tab `just grown beard’\tab < \textit{no} `new’ + \textit{xet} `line’ \tab [KŞ.74] \\
\textit{berz-e pey}\tab `standing’\tab < \textit{berz} `high’ + \textit{-e} + \textit{pey} `leg’\tab  [ZB.26]\\
\textit{weşhaɫ}\tab `happy’\tab < \textit{weş} `good’ + \textit{haɫ} `state’\tab  \\
\textit{kemqûwe}\tab  `weak'\tab < \textit{kem} `little' + \textit{qûwe} `strength' \z


\ea
\textit{berzepey miđrarê.}\\ 
\gll berzepey miđr-a=rê\\ 
 standing stop\textsc{.prs-3pl:S=povb}\\ 
\glt `They remained [Lit. stopped] [in the tent] standing up.' \hfill[ZB.26]
\z 

\subsubsection{Other types of compounds}
Compound adjectives can be formed less commonly by combining adjectives with other parts of speech, e.g., participles and numerals\is{numerals}.
\TabPositions{2.5cm,4.35cm,9.98cm}
\ea
\textit{weşkewte}\tab `healthy’\tab < \textit{weş} `good’ + \textit{kewte} `fallen (\textsc{m}) \tab [ZQ.17] \\
\textit{duwe gîyane}  \tab `pregnant’\tab < \textit{duwê} `two’ + \textit{gîyan} `soul’ + \textit{-e} (\textsc{f})  \\
\z

Far less common are compound adjectives in which neither of the component parts is from the category of adjectives. In the following example, the component parts are joined by the compound marker -e.
\TabPositions{1.75cm,4.5cm,10cm}
\ea
\textit{tersezaɫ}\tab `timid, coward’\tab < \textit{ters} `fear’ + \textit{-e} + \textit{zaɫ} `bladder?’\tab [BP.100] \\ 
\z 

A number of adjectives can be formed through a reduced clause. An example is \textit{taze yawa penew}, which has a sense of `just hitting puberty', as glossed below (See \S\ref{sect:n-v-compounds} for similar clauses in nominal compounding).

\ea
\textit{miđyo kuřê noxetû cuwanxasû taze yawa pene.} \\ 
\gll mi-đy(e)-o kuř-ê noxet=û cuwanxas=û taze-yawa-pene \\ 
 \textsc{ind-}look\textsc{.prs-3sg:S} boy\textsc{.m-indf} just\_grown\_beard=and gentlemanly=and just-arrive\textsc{.pst.ptcp.m}-to \\ 
\glt `She saw a good-looking young man [who had] just grown a beard and [had] just hit puberty.' \hfill[KŞ.74]
\z 

\subsection{Comparison of adjectives}
Adjectives mark three degrees of comparison. Three degrees of comparison are plain (unmarked), comparative\is{comparative}, and superlative\is{superlative}. The paradigm in (\ref{ex.adj.degree}) illustrates the three forms of comparison on \textit{gewre} `big'.
\TabPositions{2.5cm}
\ea \label{ex.adj.degree}
\textit{gewre}\tab `big’\\
\textit{gewre-ter}\tab  `bigger'\\
\textit{gewre-terîn}\tab  `biggest'\\
\z

The comparative\is{comparative} is expressed by the suffix \textit{-ter}, which can, in principle, inflect for gender\is{gender}.
\TabPositions{2.5cm,4cm}
\ea
\textit{zîyater}\tab `more’\tab [ZB.57] \\
\textit{dewɫetmenter}\tab  `richer'\tab  [JM.43] \\
\textit{xaster}\tab  `better'\tab  [JM.55] \\
\z

Example (\ref{cmpr-agr}) illustrates gender agreement\is{gender agreement} where the comparative adjective has been used predicatively.

\ea
\textit{sinʕeş gewretere bê.}\\
\gll \textbf{sinʕe}=ş \textbf{gewre-ter-e} b-ê \\
age\textsc{.f=3sg:PSR} old-\textsc{cmpr-f} be.\textsc{prs-aug.3sg:S} \\
\glt `He was older. [Lit. His age was older.]' \hfill[HS.47] \label{cmpr-agr}
\z 

The comparative adjectives can also be used attributively, as shown in (\ref{cmpr-agr-attr}).

\ea \label{cmpr-agr-attr}
\textit{kitêwe ʕaltere misanû.} \\
\gll kit(e)-êwe ʕal-ter-e mi-san-û\\
cat\textsc{.f-indf} good-\textsc{cmpr-f} \textsc{ind-}buy.\textsc{prs-1sg:A} \\
\glt `I will buy a better cat.' 
\z 

The comparative\is{comparative} suffix \textit{-ter} is repeated on adjectives connected by the conjunction particle \textit{=û}, pointing to its affixal status.

\ea
\textit{aneşa zilterû ʕalter bo ...} \\ 
\gll ane=şa \textbf{zil-ter=û} \textbf{ʕal-ter} b-o \\ 
\textsc{dem.dist.m.3sg.dir=3pl:PSR} big\textsc{-cmpr}=and good\textsc{-cmpr} be\textsc{.prs.ind-3sg:S} \\ 
\glt `the one who was bigger and healthier ...' \hfill[ZB.40]
\z 

The standard of comparison is expressed by a prepositional phrase headed by the preposition \textit{ce} `from’ (or its morphological allomorph \textit{ç=}), hence marked by the oblique case\is{oblique case}.

\ea
\textit{ce taranî kuweyt dewɫetmenter bê.} \\ 
\gll ce taran-î kuweyt dewɫetmen-ter b-ê \\ 
from \textsc{pn}\textsc{-obl}\textsc{.m} \textsc{pn} rich\textsc{-cmpr} be\textsc{.prs}\textsc{-aug}\textsc{.3sg:S} \\ 
\glt `Kuwait was more affluent than Tehran.' \hfill[JM.44]\\ 
\z 

\newpage
\ea
\textit{şiş kîlowê ađ çewî zîyater bîyebê.} \\ 
\gll şiş kîlo-ê ađ ç=ewî zîya-ter bîye=b-ê \\ 
six kilo\textsc{.m-pl.dir} \textsc{3sg.m.dir} from=\textsc{3sg.obl.m} a\_lot\textsc{-cmpr} be\textsc{.pst.ptcp.m}=be\textsc{.prs-aug.3sg:S} \\ 
\glt `He weighed six kilos more than him (the other son).' \hfill[ZB.57]
\z 

The comparative\is{comparative} degree can be expressed periphrastically through a copular clause linked to the standard of comparison via the preposition \textit{ta} `than’. This construction has been attested in the speech of very old speakers.

\ea
\textit{min serberzna ta to.} \\
\gll min serberz꞊na ta to \\
 \textsc{1sg} successful\textsc{.m=cop.1sg:S} than \textsc{2sg} \\
\glt `I (m.) am more successful than you.’ \hfill[ŞŞ.02]
\z 

Superlatives\is{superlative} can be expressed in different ways. They can be morphologically expressed by adding \textit{-terîn} to the bare\is{bare} form of the adjective. The suffix \textit{-terîn} is cognate with \textit{-tirîn} in CK\il{Kurdish!Central} and SK\il{Kurdish!Southern}, and \textit{-tarin} in Persian\il{Persian}. It can be analysed as composing of the comparative suffix \textit{-ter} + \textit{-în}. This strategy seems to be more prevalent among the younger generation.

\ea
\textit{bihterîn yagêw hewramanî} \\
\gll bih-terîn yagê꞊w hewraman-î \\
 good-\textsc{supr} place-\textsc{ez.gen} \textsc{pn-m.sg.obl} \\
 \glt `the best place in Hewraman’ \hfill[hearsay]
 \z 
Another strategy for expressing superlative\is{superlative} degree is to have the standard of comparison introduced by the phrase `from all’, with the noun bearing definite marking, and the adjective marked by the comparative\is{comparative} suffix \textit{-ter}. In (\ref{ex.superlative}) the stand of comparison \textit{kêşwer-eka} has definite marking.

\ea \label{ex.superlative}
\textit{a zemane kuweyt ce girđû kêşwereka dewɫetmenter bê.} \\ 
\gll a zeman=e kuweyt ce girđ-û kêşwer-eka dewɫetmen-ter b-ê \\ 
 \textsc{dem.dist} time\textsc{.m}\textsc{=dem} \textsc{pn} from all\textsc{-ez.gen} country\textsc{-def.pl.obl} rich\textsc{-cmpr} be\textsc{.prs}\textsc{-aug}\textsc{.3sg:S} \\ 
\glt `Back then, Kuwait was the richest of all countries.' \hfill[JM.43] 
\z 

A similar strategy is to use a nominalised adjective as the head of the noun phrase to express the superlative\is{superlative} in what is equivalent to the English\il{English} `the youngest of all':

\ea
\textit{wirđîkɫew girđîn.} \\
\gll wirđîkɫe-û girđ-î=n \\
young-\textsc{ez.gen} all-\textsc{obl.m=cop.m} \\
\glt `He was the youngest of all.' \hfill[ME.24]
\z 

\subsection{Adverbial function of adjectives}
A subset of adjectives, e.g., \textit{fire} `many’, can be used adverbially to modify another adjective.

\ea
\textit{dastanê fire tûle} \\ 
\gll dastane-ê fire tûl-e \\ 
 story.\textsc{f-indf} very long\textsc{-f} \\ 
\glt `a very long tale' \hfill[BP.158] 
\z 

A small number of adjectives can be used adverbially to modify verbs. These typically include value and qualification adjectives such as \textit{xas} `good’, \textit{řas} `true’, and quantity adjectives such as \textit{fire} `many’, and \textit{kem} `small’. 

\ea
\textit{welî emaneketê xas xeɫk bawiřiş pene mekero.} \\ 
\gll welî emaneketê xas xeɫk bawiř=iş pene me-ker-o \\ 
but but well people\textsc{.m} belief\textsc{=3sg:R} in \textsc{neg.ind-}do\textsc{.prs-3sg:A} \\  
\glt `But people still did not believe in him that much [as a leader].' \hfill[JP.144] \\
\z


\ea
\textit{fire mûsaw kem mûsa.} \\ 
\gll fire m-ûs-a=û kem m-ûs-a \\ 
a\_lot \textsc{ind-}sleep\textsc{.prs-3pl:S}=and little \textsc{ind-}sleep\textsc{.prs-3pl:S} \\ 
\glt `They slept a lot; They slept a little.' \hfill[JP.66] 
\z 

\section{Numerals\is{numerals}}
This section studies numerals\is{numerals}. Like neighbouring Iranian and Semitic languages (e.g., North-Eastern Neo-Aramaic), Tekht Hewramî\il{Hewramî!Tekht} has a decimal numeral system.

\subsection{Cardinal numerals\is{cardinal numerals}} \label{sect:cardinal-numerals}
The numeral one marks gender\is{gender} distinction: \textit{yo} `one (\textsc{m})'; \textit{yuwe} `one (\textsc{f})'. In cardinal numerals\is{numerals} above twenty containing `one', \textit{yek} is used, e.g., \textit{wîs=û yek} `twenty-one'. However, in ordinal numbers above twenty, \textit{yuwe} is used, e.g., \textit{şew(e)=û wîs=û yuwe-m} [night\textsc{.f=ez.gen} twenty=and one-\textsc{ord}] `the twenty first night'. Numerals\is{numerals} `two' and `three' have plural\is{plural} inflection when used nominally (\ref{ex.numhead}) or when modifying a head noun (\ref{ex.numn}).

\ea
\textit{her yerê anêya ça.}\\
\gll her yerê anêya ça\\
all three \textsc{deic.3pl:S} there \\
\glt `All three were [lying] there.' \hfill[SH.125] \label{ex.numhead}
\z 


\ea \label{ex.numn}
\textit{duwê dêwê în{ɛ} dewrşane.} \\ 
\gll \textbf{duwê} \textbf{dêw-ê} în{ɛ} dewr=şa=ne \\ 
two ogre\textsc{.m-pl.dir} \textsc{deic.3pl:S} around\textsc{=3pl:PSR=post} \\ 
\glt `They were surrounded by two ogres.[Lit. Two ogres were in their surroundings.]' \hfill[ZP.67]
\z 

However, when used following a demonstrative, the base forms \textit{duwe} `two' and \textit{yere} `three' are used, see (\ref{base-two})--(\ref{base-three}). The base forms end in \textit{-e}, also suggested by the adjective \textit{duwe g\^yane} `pregnant' (lit. two souls). Similarly, when used as an adjective of order, the base form is used, see (\ref{ex.orderadj}).

\ea
\textit{nitqşa girtêne î duwe dêwene.} \\ 
\gll nitq=şa girtê=ne î \textbf{duwe} dêw=e=ne \\ 
speech\textsc{=3pl:A} grab\textsc{.pst.ptcp.f=cop.3sg.f:O} \textsc{dem.prox} two ogre\textsc{=dem=cop.3sg.f:S} \\ 
\glt `They had muted her speech, these two ogres.' \hfill[JP.178] \label{base-two}
\z 


\ea
\textit{î yere nefere} \\
\gll î \textbf{yere} nefer=e \\
\textsc{dem.prox} three person\textsc{dem} \\
\glt `these three people' \hfill[SH.114] \label{base-three}
\z 


\ea \label{ex.orderadj}
\textit{dehfew yerey } \\ 
\gll dehfe-û yere-î \\ 
time\textsc{.f-ez.gen} three\textsc{-m.sg.obl} \\ 
\glt `the third time' [KŞ.27]
\z 
Numerals\is{numerals} above three are invariable. The numbers 11--19 are formed by adding numbers 1 to 9 (see Table \ref{tab:card-numerals}) on 10, hence \textit{çwarde} `fourteen': \textit{çwar} `four' + \textit{de} `ten'. The initial segment in \textit{de} `ten' is assimilated to the final segment in numerals such as \textit{heve} `seventeen', and \textit{noze} `nineteen'. Numerals \textit{yaŋze} `eleven', \textit{dwaŋze} `twelve', \textit{paŋze} `fifteen', and \textit{şaŋze} `sixteen' come with additional segments between what is assumed to be the numeral + \textit{de} `ten'. The numbers above twenty, which occur between multiples of ten, are formed either by simple compounding or coordination, through the coordinator conjunction \textit{=û} (see \S\ref{sect:conjunction}). Table \ref{tab:card-numerals} lists cardinal numerals 1--40. 
\begin{table}
\begin{tabular}{llll}
\lsptoprule
1& \textit{yo} (\textsc{m}), \textit{yuwe} (\textsc{f}) & 21 & \textit{wîs(=û) yek}\\
2& \textit{duw\^e}& 22 & \textit{wî(=û) duw\^e}\\
3& \textit{yer\^e}& 23 & \textit{wîs(=û) yer\^e}\\
4& \textit{çwar}& 24 & \textit{wîs(=û) çwar}\\
5& \textit{penc}& 25 & \textit{wîs(=û) penc}\\
6& \textit{şiş}& 26 & \textit{wîs(=û) şiş}\\
7& \textit{ħewt}& 27 & \textit{wîs(=û) ħewt}\\
8& \textit{heşt}& 28 & \textit{wîs(=û) heşt}\\
9& \textit{no}& 29 & \textit{wîs(=û) no}\\
10& \textit{de}& 30 & \textit{sî}\\
11& \textit{yaŋze}& 31 & \textit{sî(=û) yek}\\
12&\textit{dwaŋze} & 32 & \textit{sî(=û) duw\^e}\\
13& \textit{s\^eŋze}& 33 & \textit{sî(=û) yer\^e}\\
14& \textit{çwarde}& 34 & \textit{sî(=û) çwar}\\
15& \textit{paŋze}& 35 & \textit{sî(=û) penc}\\
16& \textit{şaŋze}& 36 & \textit{sî(=û) şiş}\\
17& \textit{ħeve}& 37 & \textit{sî(=û) ħewt}\\
18& \textit{hejde}& 38 & \textit{sî(=û) heşt}\\
19& \textit{noze}& 39 & \textit{sî(=û) no}\\
20&\textit{wîs} & 40 & \textit{çil}\\
\lspbottomrule
\end{tabular}
    \caption{Cardinal numerals\is{numerals} 1--40}
    \label{tab:card-numerals}
\end{table}



The same speaker may use the form with and without the coordination, as suggested by (\ref{ex.num1})--(\ref{ex.num2}) below. However, the compounding strategy seems more frequent among the older generation. The coordination strategy is used more commonly in the speech of younger speakers. 

\newpage
\ea
\textit{řowê sî penc timenê kar kerênmê.} \\ 
\gll řo-ê \textbf{sî} \textbf{penc} timen-ê kar ker-ên-mê \\ 
day\textsc{.m-indf} thirty five \textsc{pn-pl.dir} task\textsc{.m} do\textsc{.prs-aug-1pl:A} \\ 
\glt `We used to work for a daily salary of thirty-five tomans.' \hfill[JM.46] \label{ex.num1}
\z 


\ea
\textit{şesû şiş saʕetê} \\ 
\gll \textbf{şes-û} \textbf{şiş} saʕet-ê \\ 
 sixty\textsc{-ez.gen} six hour\textsc{.m-pl.dir} \textsc{1pl} \\ 
\glt `sixty-three hours' \hfill[JM.33] \label{ex.num2}
\z 
 
Numerals\is{numerals} above one trigger plural\is{plural} marking on nouns. The noun occurs with the plural direct\is{direct case} suffix.

\ea
\textit{penc bizê menênê.} \\ 
\gll penc biz(e)-ê menê=nê \\ 
five goat\textsc{.f-pl.dir} remain\textsc{.pst.ptcp.pl=cop.3pl:S} \\  
\glt `Five goats had survived [from the flood].' \hfill[ZB.27] 
\z


\ea
\textit{duwanze hêɫêş ardê.} \\ 
\gll duwanze hêɫ(e)-ê=ş ard-ê \\
twelve egg-\textsc{pl.dir}=\textsc{3sg:A} take.\textsc{pst-\textbf{3pl:O}} \\
\glt `She took twelve eggs.' \label{threeeggs} \hfill[HR.44]
\z 

In clauses with present stem verbs, numerals\is{numerals} trigger plural\is{plural} marking on the direct object, which blocks oblique\is{oblique case} marking on the direct object argument:

\ea
\textit{jenêç nîşore duwê zaroɫê wîno.} \\ 
\gll jen(î)-ê=ç nîş-o=re \textbf{duwê} \textbf{zaroɫ(e)-ê} wîn-o \\ 
 woman\textsc{-f.sg.obl=add} sit\textsc{.prs.ind-3sg:S=povb} two child\textsc{-pl.dir} see\textsc{.prs.ind-3sg:A} \\ 
\glt `The wife gave birth to two babies. [Lit. She sat down [and] saw two babies.]' \hfill[ZB.24]
\z 


\ea
\textit{ba yerê tîrê teqnû.} \\
\gll ba yerê tîr-ê t\'eqn-û \\
\textsc{hort} three bullet\textsc{-pl.dir} fire.\textsc{prs.sbjv-1sg} \\
\glt `I shall fire three bullets.' \hfill[DB.155]
\z 

\subsection{Substantivised numerals\is{substantivised numerals}}
Cardinal numerals\is{cardinal numerals} may be used independently as the head of the noun phrase or as a genitive, in which case they inflect for case\is{case}.

\ea
\textit{řowê ce řowa yuwe mê yoyç mê xizmetû şê ʕeladînî.} \\ 
\gll řo-ê ce řo-a \textbf{yuwe} m-ê \textbf{yo}=îç m-ê xizmet-û şê ʕeladîn-î \\ 
 day\textsc{.m-indf} from day\textsc{.m-pl.obl} one\textsc{.f} \textsc{ind-}come\textsc{.prs.3sg:S} one\textsc{.m=add} \textsc{ind-}come\textsc{.prs.3sg:S} service\textsc{.m-ez.gen} sheikh\textsc{.m} \textsc{pn-m.sg.obl} \\ 
\glt `Once, [lit. One day of days.] a woman and a man [lit. One (\textsc{f}) came, one (\textsc{m})] came to the service of Sheikh Aladin.' \hfill[ZB.1]
\z 


\ea
\textit{mila ew kuřekey yoyşa bera.} \\ 
\gll mi-l-a ew kuř-ekey \textbf{yo-î}=şa ber-a \\ 
\textsc{ind-}go\textsc{.prs-3pl:S} \textsc{dem.dist} son\textsc{.m-def.m.sg.obl} one\textsc{.m-sg.obl=3pl:PSR} take\textsc{.prs.ind-3pl:A} \\ 
\glt `They went away [and took] that son. They took one of them (i.e., the boys).' \hfill[ZB.40]
\z 


\ea
\textit{mêmanû yoy bîyênê.} \\ 
\gll mêman-û \textbf{yo-î} bîyê=nê \\ 
guest\textsc{.m-ez.gen} one\textsc{.m-sg.obl} be\textsc{.pst.ptcp.pl=cop.3pl:S} \\ 
\glt `They were each a guest of one person.' \hfill[BP.44]
\z 

Substantivised numerals\is{substantivised numerals} above `one' trigger plural agreement\is{number agreement} on the verb.

\ea
\textit{çwar ba penc ba ...} \\ 
\gll çwar b-a penc b-a\\
four be\textsc{.prs.ind-3pl:S} five be\textsc{.prs.ind-3pl:S}\\ 
\glt `[No matter] if they were four or five [guests] ...' \hfill[BP.70]
\z

\subsection{Ordinal numerals\is{ordinal numerals}}
Ordinal numerals\is{ordinal numerals} are expressed in different ways. The first strategy, which is attested in the speech of the older generation, is to use the suffix \textit{-ê} to express ordinal numbers. In (\ref{ex.ord-num1}), the ordinal suffix is attached to the Arabic\il{Arabic} borrowing \textit{eweɫ} `first'. In (\ref{ex.ord-num2}), it is merged with the final vowel in \textit{duw\^e}.

\ea
\textit{eweɫêne duwê jenî kîyana.} \\
\gll eweɫ-ê=ne duwê jenî kîyan-a \\
first-\textsc{ord=post} two woman.\textsc{pl.dir} send.\textsc{prs.ind-3pl:A} \\
\glt `At first, they (i.e., the family of the boy) send two women [to the family of the girl].’ \hfill\citep[557]{khan_language_2023} \label{ex.ord-num1}
\z


\ea
\textit{min kuřû duwêna.} \\ 
\gll min kuř-û duw(ê)-ê=na \\ 
 \textsc{1sg} son\textsc{.m-ez.gen} two\textsc{-ord=cop.1sg:S} \\  
\glt `I am the son of the second [wife].' \hfill[JM.3] \label{ex.ord-num2}
\z

Another strategy is to use the base form of the numeral. In (\ref{ord.num-3}), the base form of three in the oblique case\is{oblique case} expresses the ordinal number.

\ea
\textit{dehfew yerey} \\ 
\gll dehfe-û yere-î \\ 
time\textsc{.f-ez.gen} three\textsc{-m.sg.obl} \\ 
\glt `the third time' \hfill[KŞ.27] \label{ord.num-3}
\z 

In the speech of the younger generation, the suffix \textit{-(e)m, -mîn} tends to express ordinal numerals\is{ordinal numerals}. The following examples are from \citet[120]{khan_language_2023}.

\ea
\textit{yomîn pîya} \\
\gll yo-mîn pîya \\
one\textsc{.m-ord} man \\
\glt `the first man'  
\z 



\ea
\textit{yuwemîn jenî} \\
\gll yuwe-mîn jenî \\
one\textsc{.f-ord} woman \\
\glt `the first woman'  
\z 


\ea
\textit{jenî yuwem} \\
\gll jen(î)-î yuw(e)-em \\
woman-\textsc{ez.attr} one\textsc{.f-ord} \\
\glt `the first woman'  
\z 


With numeral `two', occasionally the suffix -\textit{îşne}, of unknown origin is used:

\ea
\textit{jenî duwîşne} \\
\gll jen(î)-î duw(e)-işne \\
woman-\textsc{ez.attr} two-\textsc{ord}\\
\glt `the second woman'  
\z 


\end{sloppypar}
