\chapter{Grammatical relations}

\begin{sloppypar}
This chapter lays out argument indexing\is{argument indexing} and argument flagging\is{argument flagging}. Following \citet{haspelmath_argument_2013}, I use the term `indexing' as a cover term for the agreement phenomenon encompassing both ``grammatical agreement" and ``anaphoric agreement". The term agreement is reserved for cases where the index obligatorily indexes an argument regardless of the presence or the absence of the coreferent NP in the clause. Agreement in this sense parallels what \citet{haig_grammaticalization_2018} refers to as ``obligatory" indexing. On the other hand, the term ``alternating indexing" is used to refer to the occurrence of indexes as anaphora. This refers to cases where contextual factors have an effect on the presence or absence of the person index, e.g., the presence or absence of the co-referent NP, word order\is{word order}, clausal focus\is{focus}, etc. 

The term flagging is reserved for case marking on clausal arguments and marking by adpositions. Thus, a clausal argument can be flagged by an oblique case, an adposition, or both. Following \citet[]{Haspelmath+2019+313+334}, argument flagging\is{argument flagging} and argument indexing\is{argument indexing} are together called argument marking\is{argument marking}.

Some other terminological conventions are in order. In discussing alignment types, I use the following abbreviations to refer to the core arguments within the clause: S\_ single argument of an intransitive verb; A\_ agent-like argument of a transitive clause: P\_ patient-like argument of a transitive clause; T\_theme-like argument in ditransitive clauses; R\_recipient-like argument in ditransitive clauses. The term R here is extended to non-core oblique arguments\is{oblique arguments}, including recipients\is{recipient}, beneficiaries, possessors, and comitatives. 

Hewramî\il{Hewramî} is a language with a split ergative system, where ergative alignment\is{ergative alignment} is conditioned by tense. The alignment is accusative\is{accusative alignment} in clauses built on the present stem of the verb, but ergative in clauses built on the past stem of the verb. Alignment\is{alignment} in Hewramî is relevant in terms of agreement and case marking\is{case marking}. While the alignment system is of a tense-sensitive type, there are certain language-internal deviations from tense-sensitive alignment, both in terms of agreement and case marking. In terms of agreement, the system offers differential A indexing\is{differential A indexing} (\S\ref{sect:differential-A-indexing}) and differential P indexing\is{differential P indexing} (\S\ref{sect:DOI}), limited to transitive clauses built on past stem verbs. As for case marking\is{case marking}, differential A flagging\is{differential A flagging} in transitive clauses built on past stem verbs (see \S\ref{sect:DAM}, and differential P flagging\is{differential P flagging}, most notably in clauses built on the present stem of the verb (\S\ref{sect:DOM}), are attested. Hewramî also features differential case marking on non-core arguments (\S\ref{sect:diffobl}). The full investigation of differential argument indexing\is{differential argument indexing} and differential argument flagging\is{differential argument flagging} requires detailed corpus-based studies. Here, I lay out some basic patterns. 



\section{Argument indexing\is{argument indexing}}
Argument indexing\is{argument indexing} has accusative alignment\is{accusative alignment} in TAM categories based on the present stem of the verb and ergative alignment\is{ergative alignment} in TAM categories based on the past stem of the verb. Agreement is manifested by using different indexes for indexing S, A, and P, summarised in Table \ref{tab:argumentindexing}. The bound indexing of R is the same as the indexing of P in present tense constructions. In past tense constructions, it could be realised as a verbal person/number suffix or a clitic pronoun, both being alternating indexes. Recall from \S\ref{bound PMs} that the verbal person suffixes have partially different paradigms in the present and past tenses. On the other hand, one and the same paradigm of clitic pronouns expresses different clausal, e.g., A, P, and phrasal arguments, e.g., possessor (see \S\ref{sect:clitic-function} for detailed discussion). 
\begin{table}[htp]
    \begin{tabular}{ccll}
    \lsptoprule
   TAM &Argument &Index& Indexing type \\
    \midrule
    \multirow{2}{*}{\textsc{prs} stem}&  S, A & verbal person/number suffix & obligatory \\\cline{2-4}
    &P, T, R& clitic pronoun& alternating\\\cline{2-4}
    \\
    \multirow{5}{*}{\textsc{pst} stem}&  S& verbal person/number suffix& obligatory\\\cline{2-4}
    &P, T& verbal person/number suffix& (nearly) obligatory\\\cline{2-4}
    &R&verbal person/number suffix& alternating\\\cline{2-4}
    &A&clitic pronoun& (nearly) obligatory \\\cline{2-4}
    &R& clitic pronoun& alternating \\\cline{2-4}
    \lspbottomrule
    \end{tabular}
    \caption{Morphological indexing of arguments and indexing type}
    \label{tab:argumentindexing}
\end{table}

Table \ref{tab:argumentindexing} lays out the morphological expression of arguments across different TAM constructions, and the indexing type for each argument. The functional status of indexes as obligatory indexes or alternating indexes is fully discussed in \S\ref{sect:arginx.prs} and \S\ref{sect:arginx.pst}. As can be seen from Table \ref{tab:argumentindexing}, the morphophonological form of the indexes is not a good predictor of their functions. For instance, historical agreement suffixes in the past tense have become degrammaticalised and can now be alternating indexes of R arguments, such as possessors, beneficiaries, recipients, etc. Likewise, the clitic pronouns that index A-past arguments have retained some of their pronominal origins to some degree but increasingly show traits of agreement indexes (see \S\ref{sect:differential-A-indexing}). 

As seen in Table \ref{tab:argumentindexing}, among the core arguments of the clause, intransitive subjects are regularly indexed by verbal person suffixes. The following examples illustrate agreement with S: (\ref{ex.Sprs}) with a co-referent subject argument, and (\ref{ex.Sprsnull}) with a zero subject argument. The S argument is a coordinated noun phrase in (\ref{ex.Scoor}). Yet, the copula verb agrees only with the second coordinate.

\ea \label{ex.Sprs}
\textit{bizekê cîya leweřyayşa hurpiřa.} \\ 
\gll \textbf{bize-(e)kê} cîya leweřyay=şa hur-piř-\textbf{a} \\ 
 goat\textsc{.f-def.pl.dir} instead\_of graze\textsc{.inf=3pl:PSR} \textsc{pvb-}jump\textsc{.prs.ind-3pl:S} \\ 
\glt `The goats were dancing instead of grazing.' \hfill[JP.55]
\z 


\ea \label{ex.Sprsnull}
\textit{milawe pey germîyanî.} \\ 
\gll mi-l-\textbf{a}=we pey germîyan-î \\ 
\textsc{ind-}go\textsc{.prs-3pl:S=compl} for \textsc{pn-m.sg.obl} \\ 
\glt `They went to Garmiyan.'\hfill[ZB.39]
\z 


\ea
\textit{eđaw tatew min her beđbext bîyen.} \\ 
\gll \textbf{eđa=û} \textbf{tate-û} \textbf{min} her beđbext bîye=\textbf{n} \\ 
 mother\textsc{.f}=and father\textsc{.m-ez.gen} \textsc{1sg} \textsc{emph} poor be\textsc{.pst.ptcp.m=cop.3sg.m:S} \\ 
\glt `My parents were poor.' \label{ex.Scoor} \hfill[JE.43]
\z 

\subsection{Argument indexing\is{argument indexing} in TAM constructions built on the present stem of the verb}\label{sect:arginx.prs}
Accusative alignment\is{accusative alignment} is found in clauses built on the present stem of the verb. Verbal person/number affixes carry out the expression of S and A. In contrast, clitic pronouns express P. Accusative alignment\is{accusative alignment} is also reflected in the obligatoriness of the indexes used to express the core arguments within the clause. While the indexing of A and S is obligatory, the indexing of P is conditioned by the absence of the co-referent NP. In other words, the P-indexing clitic pronoun is mutually exclusive with an overt direct object in the same syntactic domain.

\newpage
\ea
\textit{ême milmê şeşik.} \\ 
\gll ême mi-l-\textbf{mê} şeşik \\ 
 \textsc{1pl} \textsc{ind-}go\textsc{.prs-1pl:S} \textsc{pn} \\ 
\glt `We will go to Shashk.' \hfill[HB.32]
\z 


\ea
\textit{îney maçmê.} \\ 
\gll îney m-aç-\textbf{mê} \\ 
 \textsc{\textsc{dem.prox}.obl.m.3sg} \textsc{ind-}say\textsc{.prs-1pl:A} \\ 
\glt `We will say this.' \hfill[BP.69]
\z 


\ea
\textit{beroma yanew wêşa.} \\ 
\gll ber-o=\textbf{ma} yane-û wê=şa \\ 
take\textsc{.prs.ind-3sg:A=1pl:O} house\textsc{-ez.gen} \textsc{refl=3pl:PSR} \\ 
\glt `He will take us to his [lit. their] house.' \hfill[HB.35]
\z 

Similarly, the same set of clitic pronouns that express P also express R arguments. In doing so, clitic pronouns are alternating with the coreferent nominal. Note that the placement of R-indexing clitics follows the clitic placement rule set out in \S\ref{sect:absolute_prep} and \S\ref{clitic-behaviour}. In short, the clitic pronouns land on the leftmost element within the VP as their host, thus excluding the subject NP as a possible host.

\ea \label{ex.r-clc1}
\textit{dey to waçem la.} \\ 
\gll dey to waç-e=\textbf{m} la \\ 
 \textsc{disc.ptcl} \textsc{2sg} say\textsc{.imp-2sg:A=1sg:R} to \\ 
\glt `Tell me!' \hfill[ÇK.107] 
\z 



\ea \label{ex.r-clc2}
\textit{êtir mebo ême karêş pene kermê.} \\ 
\gll êtir me-b-o ême kar-ê=\textbf{ş} pene k\'er-mê \\ 
 no\_longer \textsc{neg.ind-}be\textsc{.prs-3sg:S} \textsc{1pl} task\textsc{.m-indf=3sg:R} for do\textsc{.prs.sbjv-1pl:A} \\ 
\glt `We can no longer do anything for him.' \hfill[JP.75]
\z 

\subsection{Argument indexing\is{argument indexing} in TAM constructions built on the past stem of the verb} \label{sect:arginx.pst}
\largerpage
As remarked above, clauses built on the past stem of the verb feature ergative alignment\is{ergative alignment}. In canonical ergative constructions, the verbal person/number affixes agree with P and S marked in the direct case\is{direct case}. In (\ref{ex.ergative.n.dir})--(\ref{ex.ergative.3sg.dir}), the verb agrees with a direct-marked object NP\is{direct case}. which is nominal in (\ref{ex.ergative.n.dir}) and pronominal in (\ref{ex.ergative.3sg.dir}). S is indexed by the same set of verbal person/number affixes as P (\ref{ex.ergative.S}). On the other hand, A is indexed by clitic pronouns in (\ref{ex.ergative.n.dir})--(\ref{ex.ergative.3sg.dir}).

\ea
\textit{pase zawlêşa wey kerdê.} \\ 
\gll pase \textbf{zawlê}=şa wey kerd-\textbf{ê} \\ 
 like\_this child\textsc{.pl.dir=3pl:A} raising do\textsc{.pst-3pl:O} \\ 
\glt `They raised children in this way.' \hfill[JE.55] \label{ex.ergative.n.dir}
\z 


\ea
\textit{ađma tehwêɫ gêrt.} \\ 
\gll \textbf{ađ}=ma tehwêɫ gêrt-\textbf{\O} \\ 
 \textsc{3sg.m.dir=1pl:A} delivery\textsc{.m} grab\textsc{.pst-3sg.m:O} \\ 
\glt `We got hold of him.' \hfill[ZQ.18] \label{ex.ergative.3sg.dir}
\z 


\ea
\textit{karewanîyê amêyanê serere.} \\ 
\gll \textbf{karewanî-ê} amêya=\textbf{nê} sere=re. \\ 
 caravan\_people-\textsc{pl.dir} come\textsc{.pst.ptcp.pl=cop.3sg.pl:S} top=\textsc{postp} \\ 
\glt `Some passers-by had stayed there.' \hfill[DB.15]\label{ex.ergative.S}
\z 

In terms of obligatoriness, the indexation of S is fully obligatory, i.e., every past intransitive clause is required to have an index for S, whether or not the coreferent NP is present. P is indexed by the same set of person/number suffixes and is nearly obligatory (see \S\ref{sect:DOI} for cases where the indexing is absent). Similarly, the indexation of A-past arguments via clitic pronouns is nearly obligatory. In the last two cases, the indexes feature transitional behaviour between `agreement' and `anaphora', meaning they are neither fully-fledged agreement indexes nor pronominal indexes (see \S\ref{sect:differential-A-indexing}). 

As outlined in \S\ref{sect:perso-pro}, speech act pronouns\is{speech act pronouns} have lost the case distinction\is{case distinction}. Nonetheless, they trigger agreement on the verb when occurring as a direct object in an ergative construction. Examples (\ref{ex.Pindexsap1})--(\ref{ex.Pindexsap3}) represent P indexing when P is an SAP pronoun.

\ea \label{ex.Pindexsap1}
\textit{to minit quɫ kerđa.} \\
\gll to \textbf{min}=it quɫ kerđ-\textbf{a} \\
\textsc{2sg} \textsc{1sg=2sg:A} pierced do.\textsc{pst-1sg:O} \\
\glt `You disabled me.' \hfill[PW.30]
\z 


\ea \label{ex.Pindexsap2}
\textit{qesem pa xway toş epî layiqetî weş kerdenî.} \\ 
\gll qesem p=a xwa-î \textbf{to}=ş e=p=î layiqetî weş kerde=\textbf{nî} \\ 
 oath\textsc{.m} to=\textsc{dem.dist} God\textsc{.m-sg.obl} \textsc{2sg=3sg:A} \textsc{emph=adp=dem.prox} worthiness\textsc{.m} good do\textsc{.pst.ptcp.m=cop.2sg:O} \\ 
\glt `I swear to God, who has made you with such virtue.' \hfill[ZQ.54]
\z 


\ea \label{ex.Pindexsap3}
\textit{ça toşa şîrîne kerđî ça çageyç minşa sîyaw kerđa.} \\
\gll ça \textbf{to}=şa şîrîn-e kerđ-\textbf{î} ça çage=yç \textbf{min}=şa sîyaw-{\O} kerđ-\textbf{a} \\
there \textsc{2sg=3pl:A} sweet-\textsc{f} do.\textsc{pst-2sg:O} there there\textsc{=add} \textsc{1sg=3pl:A} black-\textsc{m} do.\textsc{pst-1sg:O} \\
\glt `[The husband said to his wife], there where they sweetened you, they also blackened me.' \hfill[XX.87]
\z 

The similarity between S and P indexes as opposed to A-past clitic pronouns is also reflected in syntactic criteria. One such case criterion is deletion under same-referent clause coordination, which differentiates between the A index, on the one hand, and the S and P indexes, on the other hand. In the following examples, the indexes of S and P occur on each of the coordinate verbs, but the coreferential A index is deleted in the first coordinate verb. In \S\ref{sect:cliticpers}, this behaviour of the A-past index was argued to reflect its status as a `clitic', as opposed to the affixal status of S and P indexes.

\ea
\textit{ehmeđe dizeyç hurêst lûwawe lûlejenay.} \\ 
\gll ehmeđ-e diz-e=yç hur-êst-\textbf{\O} lûwa-\textbf{\O}=we lûlejenay\\ 
 \textsc{pn-ez.cmpd} thief\textsc{-def=add} \textsc{pvb-}rise\textsc{.pst-\textbf{3sg:S}} go\textsc{.pst-\textbf{3sg:S}=compl} flute\_playing\\ 
\glt `Ahmad the Thief got up [and] went back [to the palace] to play flute.' \\ \hfill[ED.255]
\z


\ea
\textit{yewaşê berdê kuştêşa.} \\ 
\gll yewaşê berd-\textbf{ê} kuşt-\textbf{ê}=\textbf{şa} \\ 
 then take\textsc{.pst-3pl:O} kill\textsc{.pst-3pl:O=3pl:A} \\ 
\glt `Then they (the king’s men) took them (the pregnant women) [and] killed them.' \hfill[KŞ.19]
\z 
 
As a reflex of a historical construction dating back to Middle Iranian, the expression of P via verbal affixes/copula affixes gets extended to R arguments under `affix co-optation' \citep{haig_grammaticalization_2018}. R arguments undergoing this process include possessors (\ref{ex.external-possessor}), recipients\is{recipient} (\ref{ex.external-recipient}), human goals\is{goal} (\ref{ex.external-goal}), addressees\is{addressee} (\ref{ex.external-addressee}), sources (\ref{ex.external-source}), comitatives (\ref{ex.external-comitative}), etc. This results in externally realised R arguments, e.g., external possession\is{external possession}, where the possessor is realised at a distance from its possessed noun. Unlike the indexing of O, which tends to be obligatory (see \S\ref{sect:DOI}), the indexation of R arguments alternates with the presence of the coreferent NPs.

\ea
\textit{xeberêm nezanênê.} \\ 
\gll xeber-ê=m ne-zanê=\textbf{nê} \\ 
news\textsc{.m-pl.dir=1sg:A} \textsc{neg-}know\textsc{.pst.ptcp.pl=cop.3pl:PSR} \\ 
\glt `I didn't know their news.' \hfill[JM.30] \label{ex.external-possessor}
\z 


\ea
\textit{waçe her kesî nan danî.}\\ 
\gll w\'aç-e her kes-î nan da=\textbf{nî} \\ 
say\textsc{.prs.imp-2sg:A} every person\textsc{.m-sg.obl} bread\textsc{.m} give\textsc{.pst.ptcp.m=cop.2sg:R} \\ 
\glt `Tell whoever gave you food.' \hfill[BP.165] \label{ex.external-recipient}
\z


\ea
\textit{xway ketê pey kîyasen.} \\ 
\gll xwa-î ket-ê pey kîyase=\textbf{n} \\ 
 God\textsc{.m-sg.obl} bed\textsc{.m-indf} to send\textsc{.pst.ptcp.m=cop.\textbf{3sg.m:R}} \\ 
\glt `[As if] God had sent him a bed.' \hfill[JP.69] \label{ex.external-goal}
\z 


\ea
\textit{be xwa xuřyeymêş venî.} \\ 
\gll be xwa xuřî-\textbf{eymê}=ş venî \\ 
by God\textsc{.m} shout\textsc{.pst-1pl:R=3sg:A} at \\ 
\glt `By God, he shouted at us.' \hfill[ŞC.44] \label{ex.external-addressee}
\z 


\ea
\textit{sanɛnêşa çene.} \\ 
\gll sanɛ=\textbf{nê}=şa çene \\ 
 take\textsc{.pst.ptcp.pl=cop.3pl:R=3pl:A} from \\ 
\glt `They used to collect [tax] from them (people of Hewraman).' \hfill[BP.25] \label{ex.external-source}
\z 


\ea
\textit{mameɫeş nekerđe çenî.} \\
\gll mameɫe=ş ne-kerđ-\textbf{e} çenî \\
deal=\textsc{.m=3sg:A} \textsc{neg-}do\textsc{.pst-\textbf{3sg.f:R}} with \\
\glt `He didn't have dealings with her.' \hfill[SH.254] \label{ex.external-comitative}
\z 

The indexing of non-core argument on past stem verbs takes over the object index slot. Thus in (\ref{ex.external-comitative}) the verb doesn't agree with the masculine direct object \textit{mameɫe}, indexing instead the comitative (see \S\ref{sect:DOI} for more details). 

Alternatively, R arguments may be expressed by clitic pronouns, in which case the argument is realised locally on the governing adposition. This indexing strategy is less frequent in the text corpus than indexation via verbal affixes.

\ea
\textit{watim peneş, `to milî ko?'} \\ 
\gll wat=im pene=\textbf{ş} to mi-l-î ko \\ 
 say\textsc{.pst=1sg:A} to\textsc{=3sg:R} \textsc{2sg} \textsc{ind-}go\textsc{.prs-2sg:S} where \\ 
\glt `I said to him, `Where are you going?’' \hfill[JH.31]
\z 


\ea
\textit{aẍekew ʕêraqî, berdenşawe peyşa.} \\ 
\gll aẍe-(e)ke-û ʕêraq-î berde=n=şa=we pey=\textbf{şa} \\ 
 agha\textsc{.m-def.m.sg.dir-ez.gen} \textsc{pn-m.sg.obl} take\textsc{.pst.ptcp.m=cop.3sg.m:O=3pl:A=compl} to\textsc{=3pl:R} \\ 
\glt `The aghas in Iraq, they took it back [the taxes] to them and so on.'\\ \hfill[BP.28] \label{ex.insitu-goal}
\z 
\subsubsection{Differential A indexing\is{differential A indexing}} \label{sect:differential-A-indexing}
A feature of the Iranian languages is that following the shifts in the verbal and nominal morphology and the rise of ergativity in Middle Iranian, the historical clitic pronouns came to express A arguments in clauses built on the past stem of the verb.\footnote{This shift concerns the loss of finite perfective verb forms in late Old Iranian and their replacement by the resultative participle. In terms of its argument structure, the resultative participle agreed with the direct-marked\is{direct case} P argument as it did with S of intransitive constructions. However, the expression of A was carried out by oblique\is{oblique case} NPs or through clitic pronouns (see \citealt{haig_alignment_2008,haig_deconstructing_2017} for a detailed discussion, and \citealt{mohammadirad_pronominal_2020} for the fate of A-past indexing in Western Iranian languages).} A-indexing clitic pronouns have fully turned into obligatory indexes of A-past arguments in neighbouring Central Kurdish\il{Kurdish!Central} (though they have preserved their mobility). But in Hewramî\il{Hewramî}, as well as in some other Iranian languages, A-past indexing clitics have retained, to some extent, their pronominal origin (\citealt{jugel_les_2016}; \citealt{mohammadirad_pronominal_2020}), as their precursors did in Middle Iranian \citep{jugel_entwicklung_2015}. In other words, clitic pronouns have remained alternating indexes of A arguments in clauses built on the past stem of the verb.

This implies that not all A arguments are indexed via a clitic pronoun. Our frequency count based on the corpus in \citet{mohammadirad_speking_the_past}, suggests that overall 12\% of past transitive constructions (45 out of 379 clauses) are unindexed (see Table \ref{tab:agentindexing}). See \citet{MohammadiradinreviewAindx} for a detailed study of differential A indexing\is{differential A indexing} in Hewramî\il{Hewramî} based on a much larger corpus. This suggests that, in terms of token frequency, clitic pronouns have nearly grammaticalised as agreement markers.  
\begin{table}[htp]
    \begin{tabular}{lcccc}
\lsptoprule
& n. past tr. clauses & A is indexed & A is not indexed \\
\midrule
Total corpus& 379 & 88\% & 12\%  \\
\lspbottomrule
    \end{tabular}
    \caption{Indexing A-past arguments}
    \label{tab:agentindexing}
\end{table}

The differential indexing\is{differential A indexing} of A in Hewramî\il{Hewramî} and related languages has been assumed to be conditioned by the complementarity between the clitic pronoun and the oblique-marked\is{oblique case} A argument in the same syntactic domain. For instance, discussing the development of A-indexing clitics in Iranian, \citet[102--103]{haig2020pronoun} suggests that: ``In Middle Iranian, these subject clitic pronouns were in complementary distribution with free NP subjects; this kind of system is still attested in some West Iranian languages to this day.'' This is exactly the case in the following excerpt from Hewramî\il{Hewramî}. In the first clause, the clitic pronoun is absent in the presence of the overt-oblique-marked\is{oblique case} NP. In the second clause, the clitic pronoun resumes the absent A argument.

\ea

\ea[]{
\textit{min taze padşay kerdena wekêɫ.} \\ 
\gll min taze \textbf{padşa-î} kerde=na wekêɫ \\ 
\textsc{1sg} anyway king\textsc{.m-sg.obl} do\textsc{.pst.ptcp.m=cop.1sg:O} advocate\textsc{.m} \\ 
\glt `I—\textbf{the king} has put me in charge [lit. made me an advocate].'
}
\ex[]{
\textit{watenîçiş, `mişyo neberûşo.'} \\
\gll wate=n=îç=\textbf{iş} mişyo ne-ber-û=ş=o \\
say\textsc{.pst.ptcp.m=cop.3sg.m:O}\textsc{=add}\textsc{=3sg:A} \textsc{aux} \textsc{neg.sbjv-}take\textsc{.prs-1sg:A=3sg:O}\textsc{=compl} \\
\glt `\textbf{He (the king)} has said [to me], ``You shall not take her back.'’' \\
\hfill[ZP.107]--[ZP.108]
}
\z
\z

Yet, the data from the corpus show that the complementarity between the clitic and the oblique-marked\is{oblique case} overt NP only partly accounts for differential A indexing\is{differential A indexing} in Hewramî\il{Hewramî}, as suggested by the following examples. More importantly, it does not explain why the clitic pronoun occurs despite the oblique-marked\is{oblique case} A NP being present.

\ea
\textit{pađşay desûriş dan be min.} \\ 
\gll \textbf{pađşa-î} desûr=\textbf{iş} da=n be min \\ 
 king\textsc{.m-sg.obl} order\textsc{.m=3sg:A} give\textsc{.pst.ptcp.m=cop.3sg.m:O} to \textsc{1sg} \\ 
\glt `The king has ordered me [to do this].' \hfill[JP.209] \label{ex.A-clc-complementarity1}
\z 


\newpage
\ea
\textit{heta xway dinyaş weşe kerdêne î merasême kem mekero.} \\ 
\gll heta \textbf{xwa-î} dinya=\textbf{ş} weş-e kerdê=ne î merasêm=e kem me-ker-o \\ 
 as\_long\_as God\textsc{.m-sg.obl} world\textsc{.f.sg.dir=3sg:A} well\textsc{-f} do\textsc{.pst.ptcp.f=cop.3sg.f:O} \textsc{dem.prox} ceremony\textsc{.m=dem} little \textsc{neg.ind-}do\textsc{.prs-3sg:A} \\ 
\glt `So long as God's world continues to exist, this ceremony will keep on being held [each year].' \hfill[ZP.130] \label{ex.A-clc-complementarity2}
\z 

In the rest of this section, I provide some tendencies regarding differential A indexing (DAI)\is{differential A indexing} in Hewramî\il{Hewramî}, without any claim to comprehensiveness. Interested readers are encouraged to consult the detailed analysis of DAI in Hewramî\il{Hewramî} proposed by \citet[]{MohammadiradinreviewAindx}. Differential A indexing\is{differential A indexing} in Hewramî\il{Hewramî} is primarily triggered by the A NP displaying properties related to focus\is{focus}. According to \citet[]{LambrechtPolinsky1997}, the following properties are characteristics of focused A arguments cross-linguistically:
\begin{itemize}
    \item prosodic prominence
    \item specific linear position
    \item non-nominative case marking
    \item lack of grammatical agreement
\end{itemize}

These properties characterise differential A indexing\is{differential A indexing} in Hewramî\il{Hewramî} too.\footnote{Similarly, \citet[159--162]{siewierska_person_2004} reports that in some languages, differential A indexing\is{differential A indexing} is triggered by focus\is{focus}.} A focused A NP in Hewramî\il{Hewramî} may feature non-contrastive or contrastive focus\is{focus}. Non-contrastive focus\is{non-contrastive focus} is further divided into wh-focus and completive focus\is{completive focus}. In the following wh-focus constructions, A has nuclear focus\is{nuclear focus}, and the A-indexing clitic pronoun \textit{=ş} is missing:

\ea
\textit{î zeře çermeme kê berden eçêge?} \\
\gll î zeře çerme=m=e \textbf{k\stackunder[-10pt]{\^{e}}{\`{}}} berde=n eçêge \\
\textsc{dem.prox} money-\textsc{ez.cmpd} white\textsc{=1sg:PSR=deic} who take.\textsc{pst.ptcp.m=cop.3sg.m:O} here \\
\glt `\textbf{Who} has taken my white money [that is now] here?' \hfill[PK.29]
\z 


\ea
\textit{watşa, `daxom kê etik kerđêbo?'} \\
\gll wat=şa daxom \textbf{k\stackunder[-10pt]{\^{e}}{\`{}}} etik kerđê=b-o \\
say.\textsc{pst=3pl:A} \textsc{q.ptcl} who disgrace do.\textsc{pst.ptcp.f}=be\textsc{.prs.3sg:O} \\
\glt `They said, `We wonder \textbf{who} might have disgraced her?’' \hfill[ED.195]
\z 

When A is in completive focus\is{completive focus}, it is not indexed via mobile clitic pronouns. In (\ref{ex.completive-focus}), in response to the wh-question, the focused \textsc{2sg} A argument is not indexed.

\ea
\textit{maço `miɫk ehmeđ! î dijmenême kê kuştênê? to kuştênê?'} \\
\gll m-aç-o miɫk ehmeđ î dijmen-ê-m=e \textbf{k\stackunder[-10pt]{\^{e}}{\`{}}} kuştê=nê \textbf{t\`o} kuştê=nê \\
\textsc{ind}-say.\textsc{prs-3sg:S} \textsc{pn} \textsc{pn} \textsc{dem.prox} enemy\textsc{-pl.dir=1sg:PSR=deic} who kill\textsc{.pst.ptcp.pl=cop.3pl:O} \textsc{2sg} kill\textsc{.pst.ptcp.pl=cop.3pl:O} \\
\glt `He (the king) said, `Oh Milk Ahmad! \textbf{Who} has killed my enemies? \textbf{You} have killed them?’' \hfill[ME.150] \label{ex.completive-focus}
\z 

A-indexing is also absent when the A argument has a contrastive focus\is{contrastive focus}. The following examples feature `contrastive topic constructions' \citep{dik_typology_1981}, characterised by a parallel being set up between two subjects. In each example, the indexing is absent for the second subject. Here, DAI seems to be triggered by the unexpectedness of the second subject, which also carries nuclear stress.\footnote{It has been suggested that the low-frequency members of a particular grammatical relation are generally those that exhibit additional marking (cf. \citeauthor{haspelmath2021role}'s \citeyear{haspelmath2021role} suggestions regarding the relationship between coding length and predictability). In differential argument flagging\is{differential argument flagging}, for example, it is deviations from expected norms which are penalised through longer coding. However, for DAI in Hewramî, it is the least expected subject type that lacks indexing, while a typical topical subject is indexed. }

\ea
\textit{kê řenciş daw kê berd!} \\
\gll kê řenc=iş da=w k\stackunder[-10pt]{\^{e}}{\`{}} berd \\
who toil=\textsc{3sg:A} give.\textsc{pst=}and who take.\textsc{pst} \\
\glt `[Look] who toiled and who took [the credit]!' \hfill[YX.14]
\z 


\ea
\textit{min řencim daw to berd!} \\
\gll min řenc=im da=w \textbf{t\`o} berd \\
\textsc{1sg} toil=\textsc{1sg:A} give.\textsc{pst=}and \textsc{2sg} take.\textsc{pst} \\
\glt `I toiled, and you took [the credit]!' \hfill[YX.15] \label{ex.DAI-SPA1}
\z 

Another illustration of contrastive focus\is{contrastive focus} is `counter-presuppositional focus', which also triggers a lack of indexing for the A argument. The term counter-presuppositional focus\is{counter-presuppositional focus} is used for situations where there is a contrast between the addressee's presupposition and the speaker's assertion. In the following example, the speaker's assertion counters the addressee's assumption that the first subject has killed the speaker's son.

\ea \label{ex.oavcont}
\textit{kuřû min řozgarîya nekuşten xway kuşten.} \\
\gll kuř-û min řozgarî-a ne-kuşte=n \textbf{xwa-î} kuşte=n \\
son-\textsc{ez.gen} \textsc{1sg} \textsc{pn-pl.obl} kill.\textsc{pst.ptcp.m=cop.3sg.m:O} God\textsc{-m.sg.obl} kill.\textsc{pst.ptcp.m=cop.3sg.m:O} \\
\glt `My son, the Rozgaris did not kill him, \textbf{God} killed him.' \hfill[HM.08]
\z 

In most examples with no A indexing, A is placed in the immediate preverbal position, the default focus\is{focus} position in Hewramî\il{Hewramî}. This is further borne out by (\ref{ex.DAI-OAV1})--(\ref{ex.DAI-OAV2}), which feature OAV word order\is{word order}, and neither of which employs clitic pronouns to index the immediate preverbal A NP. 

\ea
\textit{heywane awê berde} \\ 
\gll heywane aw\stackunder[-10pt]{\^{e}}{\`{}} berd-e \\ 
 animal\textsc{.f.sg.dir} water\textsc{f.sg.obl} take\textsc{.pst-3sg}\textsc{.f:O} \\ 
\glt `The flood [lit. water] took away the animals.' \hfill[ZB.21] \label{ex.DAI-OAV1} 
\z


\ea
\textit{ku\v{r}û min xuđay ku\c{s}ten.} \\
\gll ku\v{r}-û min \textbf{xuđa-î} ku\c{s}te=n \\
son-\textsc{ez.gen} \textsc{1sg} God-\textsc{obl.m} kill.\textsc{pst.ptcp.m=cop.3sg.m:O} \\
\glt `God killed my son.' \label{ex.DAI-OAV2} \hfill[HM.95]  
\z 

Figure \ref{fig:oav1} represents pitch alignment for example (\ref{ex.DAI-OAV2}). As in (\ref{ex.oavcont}), the discourse context of the sentence is one of contrastive focus: the death of someone's son has been associated with God, not some particular people in the story. As can be seen, there is a pitch rise on \textit{xuđa} which indicates that \textit{xuđa} has prosodical prominence in (\ref{ex.DAI-OAV2}). 

\begin{figure}[htp]
    \includegraphics[width=.6\textwidth]{figures/hew-hacimehmu-rizgarya-speaker01ku.png}
    \caption{`God killed my son.'}
    \label{fig:oav1}
\end{figure}

By contrast, when the oblique-marked\is{oblique case} A argument is in the topic position and not focused, the clitic pronoun resumes it. This explains the co-occurrence of the A-indexing clitic pronoun and the oblique-marked\is{oblique case} A argument in (\ref{ex.A-clc-complementarity1})--(\ref{ex.A-clc-complementarity2}), and in the following example, where the focus is on the verb in boldface.

\ea
\textit{î pîyay tawaş î kinaçêşe kerde be qerarê} \\ 
\gll î pîya-î \textbf{taw\`a}=ş î kinaçê=ş=e kerd-e be qerar-ê\\ 
 \textsc{dem.prox} man\textsc{.m-sg.obl} can\textsc{.pst=3sg:A} \textsc{dem.prox} girl\textsc{.f.sg=3sg:A=dem} do\textsc{.pst-3sg.f:O} to settlement\textsc{-indf} \\ 
\glt `[and if] the man has been \textbf{able} to cure the girl' \hfill[ZP.45]
\z 

A indexing being sensetive to focus is also evident in the following examples, both featuring AV order. (\ref{ex.rosem}) is a topic-reaffirmation construction: the reference of Rosam is the same as the reference of `He' in the preceding clause. On the other hand, (\ref{ex.jene}) features a topic-shift construction. Note further that the additive particle marks the topic shift here (see \S\ref{sect:additiveclitic}).

\ea
He\textsubscript{i} said, `Child, have mercy on me' [BM. 143].\\
\textit{řosem-î wat}\\
\gll řosem-î wat\\
\textsc{pn-m.sg.obl} say.\textsc{pst}\\
\glt `Rosam\textsubscript{i} said.' \hfill[BM.144] \label{ex.rosem}
\z 


\ea
Sultan Mahmoud went to the house of the woman\textsubscript{i} [HS. 76]. \\
\textit{jenêç watiş} \\
\gll jen(e)-ê=ç	wat=iş\\
woman\textsc{-f.sg.obl=add}	say\textsc{.pst=3sg:A}\\
\glt `The woman\textsubscript{i} said.' \hfill[HS.77] \label{ex.jene}
\z 

The difference in indexing between (\ref{ex.rosem}) and (\ref{ex.jene}) is reflected in the prosody. Example (\ref{ex.rosem}) with no indexing features a focal A argument, as seen in Figure \ref{fig:rosem}. Example (\ref{ex.jene}) with indexing features predicate focus, as shown by Figure \ref{fig:jene}. Note that there are two pitch accents in Figure \ref{fig:jene}; the first is due to the topicality of the A argument, as marked by the additive particle \textit{=ç}. However, the sentence stress falls on the predicate, exhibiting more intensity than the topic. 
\begin{figure}[!tbh]
  
  \begin{subfigure}[b]{0.5\textwidth}
    \includegraphics[width=\textwidth]{figures/rosem.png}
    \caption{`Rosam said.'}
    \label{fig:rosem}
  \end{subfigure}\begin{subfigure}[b]{0.5\textwidth}
    \includegraphics[width=\textwidth]{figures/jene.png}
    \caption{`The woman said.'}
    \label{fig:jene}
  \end{subfigure}
  \caption{Prosodic focus in predicate focus constructions vs. A-focused constructions}
\end{figure}

Similarly, post-verbal oblique-marked\is{oblique case} A arguments tend to be resumed by the co-indexing clitic pronoun. The clitic indexing here follows from the placement of the A in the post-verbal position as an afterthought.

\ea
\textit{qotê aman. asawekeş bînan qotekey} \\
\gll qot(e)-ê ama=n asaw-eke=\textbf{ş} bîna=n \textbf{qot-ekey} \\
box-\textsc{indf} come.\textsc{pst.ptcp.m=cop.3sg.m:S} mill-\textsc{def=3sg:A} block.\textsc{pst.ptcp.m=cop.3sg.m:O} box-\textsc{def.obl.m} \\
\glt `A box came [floating on the water]. The box blocked the (water) mill.' \\\hfill[MF.75]--[MF.76]
\z

Differential A-indexing\is{differential A indexing} is also attested in subordinate clauses\is{subordinate clauses}. In the following examples, no person agreement clitic occurs with the subject of the relative clause. Note further that the no-indexing pattern with speech act pronouns\is{speech act pronouns}, as also seen in (\ref{ex.DAI-SPA1}), is against the assumption that A-indexing clitic person agreement is in complementary distribution with the overt oblique-marked\is{oblique case} NPs since speech act pronouns\is{speech act pronouns} have lost the case distinction (see \S\ref{sect:perso-pro}). The lack of indexing in subordinate clauses\is{subordinate clauses} follows from the assumption that compared to main clauses, subordinate clauses\is{subordinate clauses} tend to resist change and preserve conservatism. For instance, the change from OV to VO in German\il{German} has not affected subordinate clauses \citep{bybee2001}.

\ea
\textit{î gîre çêş bî min ward?’} \\ 
\gll î gîr=e ç\stackunder[-10pt]{\^{e}}{\`{}}ş bî-∅ min ward-∅ \\ 
 \textsc{dem.prox} hook\textsc{.m=dem} what be\textsc{.pst-3sg.m:S} \textsc{1sg} eat\textsc{.pst.3sg.m:O} \\ 
\glt `What is this situation that I am caught in? [Lit. What is this hook that I ate?]' \hfill[HB.23]
\z

\newpage
\ea
\textit{ane key bê dizî şime kerđen!} \\
\gll ane k\`ey b-ê dizî şime kerđe-n \\
\textsc{dem.dist} how be.\textsc{prs-aug.3sg:S} theft \textsc{2pl} do.\textsc{pst.ptcpl.m=cop.3sg:O} \\
\glt `How could that be considered a theft that you have committed?' \hfill[ED.107]
\z 


\ea
\textit{kuřû şuwaney heke to kuştê ...} \\ 
\gll kuř-û şuwane-î heke to kuşt-ê \\ 
son\textsc{.m-ez.gen} shepherd\textsc{.m-sg.obl} \textsc{rel} \textsc{2sg} kill\textsc{.pst-cond} \\ 
\glt `the shepherd’s son who you [ordered to be] killed [is now in your house and so on.]' \hfill[KŞ.51]
\z

\subsubsection{Differential P indexing\is{differential P indexing}} \label{sect:DOI}
Differential P indexing\is{differential P indexing} means a deviation from the canonical ergative construction whereby the object is not indexed on the verb. This section briefly reviews object indexing in past transitive constructions; see \citet[]{MohammadiradinreviewOindx} for a detailed corpus study. As seen in \S\ref{sect:arginx.pst}, in canonical ergative constructions, verbal affixes are obligatory indexes of direct objects, illustrated by (\ref{ex.oindx}).

\ea
\textit{pase zawlêşa wey kerdê.} \\ 
\gll pase \textbf{zawlê}=şa wey kerd-\textbf{ê} \\ 
 like\_this child\textsc{.pl.dir=3pl:A} raising do\textsc{.pst-3pl:O} \\ 
\glt `They raised children in this way.' \hfill[JE.55] \label{ex.oindx}
\z 

The following examples illustrate the obligatory nature of P indexing, where the \textsc{1sg} index occurs regardless of the presence or absence of the overt P NP. 

\ea
\textit{to minit quɫ kerđa.} \\
\gll to \textbf{min}=it quɫ kerđ-\textbf{a} \\
\textsc{2sg} \textsc{1sg=2sg:A} pierced do.\textsc{pst-1sg:O} \\
\glt `You disabled me.' \hfill[PW.30]
\z 


\ea
\textit{marêwî gesta.} \\
\gll mar-êw-î gest-\textbf{a} \\
snake-\textsc{indf-m.sg.obl} bite.\textsc{pst-1sg:O} \\
\glt `A snake bit me.' \hfill[MP.09] 
\z 

The following excerpt is another illustration of the obligatory nature of P-past indexing.

\ea
\textit{be kune awîşa ardênew. nîyɛneşare.} \\
\gll be kune \textbf{awî}\textsubscript{i}=şa ardê=\textbf{ne}\textsubscript{i}=û nîyɛ=\textbf{ne}\textsubscript{i}=şa=re \\ 
 by clay\_pot\textsc{.m} water\textsc{.f.sg.dir=3pl:A} bring\textsc{.pst.ptcp.f=\textbf{cop.3sg.f:O}}=and put\textsc{.pst.ptcp.f\textbf{=cop.3sg.f:O}=3pl:A=povb} \\ 
\glt `They fetched\textsubscript{i} water\textsubscript{i} in clay pots. They unloaded it\textsubscript{i} [the water].' \hfill[JE.17]
\z 

In clauses with OAV order, the verb tends to agree with the topical P (see \ref{ex.weres}).

\ea \label{ex.weres}
\textit{werêsekê min warđêne.} \\
\gll werês(e)-ekê min warđê=ne \\
rope-\textsc{def.f.sg} \textsc{1sg} eat.\textsc{pst.ptcp.f=cop.3sg.f:O} \\
\glt `[The lion said], `I have eaten the rope.’' \hfill[ÇH.85]
\z

Table \ref{tab:objindexing} summarises the ratio of object indexing per overt and zero objects in the main text corpus \citep{mohammadirad_speking_the_past}. In counting the clauses with a P index, clauses containing speech verbs, e.g., `he said', were dismissed from the count as they could also be employed to fulfil other discourse functions. It can be seen that overall, 88\% of direct objects are indexed on the verb, meaning that the absence of indexing is the marked, unexpected pattern. All zero objects occur with agreeing verbs; in contrast, 15\% of overt Os do not trigger agreement on verbs. Put differently, there are more overt P arguments with non-indexing verbs. Thus, the data provide some support for the complementarity hypothesis, which states that zero arguments are favoured by overt agreement markers and \textit{vice versa} (see \citealt{nichols2019}).\footnote{See \citet[]{MohammadiradinreviewOindx} for a detailed study of object indexing in Tekht Hewramî\il{Hewramî!Tekht} based on a corpus of nearly 36,000 words.} 
\begin{table}[htp]
    
    \begin{tabular}{lcccccc}
\lsptoprule
& n. past tr. clauses & \multicolumn{2}{l}{O is indexed} & \multicolumn{2}{l}{O is not indexed} \\
& & N& \% & N& \% \\
\midrule
Overt object NP & 246 & 208 & 0.85 & 38& 0.15\\
Zero object & 82 & 82 & 100 & -- & -- \\
Total & 328 & 290& 0.88 & 38& 0.12 \\
\lspbottomrule
    \end{tabular}
    \caption{Indexing P-past arguments}
    \label{tab:objindexing}
\end{table}

Now the question is which contexts trigger the use or non-use of agreement suffixes with an object argument. Object indexing is present with topical Os, which are marked with a definite suffix\is{definite suffix}.

\ea
\textit{tomekeş şana zemînekeyne.} \\ 
\gll \textbf{tom-eke}=ş \textbf{şana} zemîn-ekey=ne \\ 
 seed\textsc{.m-def.m.sg.dir=3sg:A} scatter\textsc{.pst.3sg:O} land\textsc{.m-def.m.sg.obl=post} \\ 
\glt `He scattered the seeds in the field.' \hfill[JP.51] \label{ex.oindex11}
\z 


\ea
\textit{ħewt seferê awekêşa mite.} \\
\gll ħewt sefer-ê \textbf{aw(î)-ekê}=şa mit-\textbf{e} \\
seven time-\textsc{pl.dir} water.\textsc{f-def.f.sg=3pl:A} pour\textsc{.pst-3sg.f:O} \\
\glt `They poured the water seven times.' \hfill[MM.29]
\z 


\ea
\textit{maziɫoxekêş arđê.} \\
\gll \textbf{maziɫoxe-(e)kê}=ş arđ-\textbf{ê} \\
prayer\_rug-\textsc{def.pl.dir=3sg:A} take\textsc{.pst-3pl:O} \\
\glt `He took the prayer rugs.' \hfill[ŞE.83] \label{ex.oindex13}
\z

As discussed in \S\ref{sect:lvc-syntax}, in light verb constructions\is{light verb constructions} which do not allow a direct object in their argument structure, the nominal element exhibits some properties of the direct object. For example, it controls agreement on the verb in clauses built on the past stem of the verb. The light verb construction\is{light verb constructions} is \textit{koç kerđey} `to migrate' in (\ref{ex.lvcag1}), containing the masculine\is{masculine} noun \textit{koç} `migration', and \textit{duʕa kerdey} `to pray' in (\ref{ex.lvcag2}), containing the feminine\is{feminine} noun \textit{duʕa} `prayer'. The light verb\is{light verb} agrees in gender\is{gender agreement} and number\is{number agreement} with these nominal elements.

\ea
\textit{koçşa kerden.} \\ 
\gll koç=şa kerde=n \\ 
 migration\textsc{.dir}\textsc{.m=3pl:A} do\textsc{.pst.ptcp.m=cop.3sg.m:O} \\ 
\glt `they returned [to the village].' \hfill[JE.13] \label{ex.lvcag1}
\z 


\ea
\textit{duʕaw xeyrîş kerdêne.} \\ 
\gll duʕa-û xeyr-î=ş kerdê=ne \\ 
 prayer\textsc{.f.sg.dir-ez.gen} goodness\textsc{.m-sg.obl=3sg:A} do\textsc{.pst.ptcp.f=cop.3sg.f:O} \\ 
\glt `They would bless [us].' \hfill[DG.6] \label{ex.lvcag2}
\z 

As seen in Table \ref{tab:objindexing}, the expected P indexing is occasionally absent. A few tendencies can be outlined here (see \citealt[]{MohammadiradinreviewOindx} for a detailed study). In a regular past transitive clause, the verb agrees with the object NP, as seen in (\ref{ex.oindex11})-(\ref{ex.oindex13}). The expected direct object index is absent when there is an additional object in the clause. Put differently, If the O-index slot is filled by the non-core argument, the object cannot be indexed in Hewramî\il{Hewramî}. In (\ref{ex.co-optation1}), the agreement with the plural\is{number agreement} object is missing since the agreement slot for the direct object has been taken over by the index for the indirect object. Following \citet[]{haig_grammaticalization_2018}, this is called ``slot co-optation''. This phenomenon also has parallels cross-linguistically, e.g., in Warlpiri\il{Warlpiri} \citep[251--252]{hale1982warlpiri}. In (\ref{ex.co-optation2}), the direct object is a feminine\is{feminine} noun, which requires \textsc{3sg.f} agreement on the verb, i.e., \textit{-e}. Yet its slot has been taken by the complement of \textit{pey}. 

\ea
\textit{sêŋze danê heserêşa da pene.} \\
\gll sêŋze danê heser(e)-ê=şa da-{\O} pene \\
thirteen \textsc{clf.pl} mule\textsc{.f-pl.dir=3pl:A} give.\textsc{3sg.m:R} to \\
\glt `They gave him thirteen mules.' \hfill[ÇH.69] \label{ex.co-optation1}
\z 


\ea
\textit{sêɫêşa pey kerd.} \\
\gll sêɫê=şa pey kerd-{\O} \\
Halva\textsc{.f} for do.\textsc{pst-3sg.m:R} \\
\glt `They made Halva for him.' \hfill[MM.35] \label{ex.co-optation2}
\z

Second, P-past indexing is absent with plural inanimate Os. In the following examples, the verb has a default \textsc{3sg.m} inflection and does not agree with the plural object\is{number agreement}.

\ea
\textit{penc çemçêşa nîyanre.} \\ 
\gll penc çemç(e)-ê=şa nîya=n=re \\ 
 five spoon\textsc{.m-pl.dir=3pl:A} put\textsc{.pst.ptcp.m=cop.3sg.m:O}\textsc{=povb} \\ 
\glt `They (my family) had set five spoons [on the tablecloth].' \hfill[JE.46]
\z 


\ea
\textit{gir kerêşa kerd.} \\ 
\gll gir ker-ê=şa kerd{-\O} \\
all chore-\textsc{pl.dir=3pl:A} do.\textsc{pst-3sg.m:O} \\
\glt `They did all [their] chores.' \hfill[HB.58] 
\z 

Differential P indexing\is{differential P indexing} is also attested for gender agreement in the text corpus. Where the controller fails to trigger agreement features on the predicate, the \textsc{3sg.m} form appears as the default form on the agreement target, i.e., the verb. In (\ref{ex.mforf}), the direct object is feminine\is{feminine}, yet the verb fails to agree with it and instead appears in the default \textsc{3sg.m} form.

\newpage
\ea
\textit{çaştekêşa kerden be awîrgakê.} \\ 
\gll çaşt(î)-ekê=şa kerde=n be awîrga-(e)kê \\ 
 meal\textsc{.f-def.f.sg=3pl:A} do\textsc{.pst.ptcp.m=cop.3sg.m:O} on hearth\textsc{.f-.f.sg.sg.obl} \\ 
\glt `They made the food on the hearth.' \hfill[JE.39] \label{ex.mforf}
\z

Some head nouns do not carry number agreement\is{number agreement} that is triggered by numeral and measure nouns\is{measure nouns}. In such cases, the verb tends to agree with the singular\is{singular} head noun. 

\ea
\textit{wîs koɫê lokeşa ard.} \\
\gll wîs koɫ(e)-ê loke=şa ard-{\O} \\
twenty load-\textsc{pl.dir} cotton=\textsc{dir.m.sg=3pl:A} bring.\textsc{pst-3sg.m:O} \\
\glt `They brought twenty loads [of cotton].' \hfill[ME.120]
\z 


\ea
\textit{ħewt koɫê xezêneş berđwe law mamojenîş.} \\
\gll ħewt koɫ(e)-ê xezêne=ş berđ{-\O}=we la-û mamojenî=ş \\
seven load-\textsc{pl.dir} treasure=\textsc{3sg:A} take.\textsc{pst-3sg.m:O=copl} to\textsc{-ez} uncle's\_wife=\textsc{3sg:PSR} \\
\glt `He took back seven loads of treasure to his uncle's wife.' \hfill[ED.139]
\z 

Direct objects consisting of a coordinate noun phrase fail to trigger object indexing on the verb if the coordinate nouns are inanimate. In (\ref{ez.doi.plo}), number agreement\is{number agreement} is missing with \textit{liçû lût} `lip(s) and nose(s)'. On the other hand, coordinated direct objects featuring human referents tend to trigger number agreement on the verb, see (\ref{ex.doi.plcoord}). This confirms the typological tendencies laid out in \citet[184--185]{corbett_agreement_2006}.

\ea
\textit{liçû lûtû dêwekaş berđ.} \\
\gll liç=û lût-û dêw-eka=ş berđ-{\O} \\
lip=and nose-\textsc{ez.gen} ogre-\textsc{def.pl.obl=3sg:A} take.\textsc{pst-3sg.m:O} \\
\glt `He took the ogres' lips and noses [to the king].' \hfill[ME.156] \label{ez.doi.plo}
\z 


\ea
\textit{î jenû wêmew î kuřme kuştêbîyɛnê.} \\
\gll î jen(î)-û wê=m=e=w î kuř=m=e kuştê=bî-ɛn-ê \\
\textsc{dem.prox} wife\textsc{.f-ez.gen} \textsc{refl=1sg:PSR=dem=}and \textsc{dem.prox} son\textsc{=1sg:A=dem} kill\textsc{.pst.ptcp.pl=}be\textsc{.pst-cond.aug-3pl:O} \\
\glt `[What if] I had killed my wife and son [by misjudgment]?' \hfill[XŞ.104] \label{ex.doi.plcoord}
\z

\section{Argument flagging\is{argument flagging}}\label{sect:arg_marking}
Argument flagging\is{argument flagging} refers to the flagging of arguments through case morphology or adposition marking. Argument flagging follows accusative alignment\is{accusative alignment} in clauses built on the present stem of the verb and ergative alignment\is{ergative alignment} in clauses built on the past stem of the verb (see Table \ref{tab:argumentflagging}). Core arguments are flagged by fusional case\is{case} suffixes. The split alignment\is{split alignment} is only relevant for third-person nouns and pronouns. Speech act pronouns\is{speech act pronouns} have lost the case distinction. The following examples exhibit accusative alignment\is{accusative alignment} in clauses based on the present stem of the verb: S-prs (\ref{ex.align:s.prs2}) and A-prs (\ref{ex.align:A.prs2}) are marked in the direct case\is{direct case}. In contrast, P-prs (\ref{ex.align:o.prs2}) is marked in the oblique case\is{oblique case}.

\ea
\textit{seʕbê wiɫaxdarê mila.} \\
\gll seʕbê \textbf{wiɫaxdar-ê} mi-l-a \\
morning\textsc{-f.sg.obl} stableman\textsc{-pl.dir} \textsc{ind-}go.\textsc{prs-3pl:S} \\
\glt `In the morning, the horse grooms went.’ \hfill[ŞC.66] \label{ex.align:s.prs2} 
\z 


\ea
\textit{dêwê řas maça.} \\ 
\gll \textbf{dêw-ê} řas m-aç-a \\ 
 ogres\textsc{-pl.dir} truth \textsc{ind-}tell\textsc{.prs}\textsc{-3pl:A} \\ 
\glt `The ogres are telling the truth.'\label{ex.align:A.prs2} \hfill [SK.64] 
\z


\ea
\textit{yewayç bero.} \\ 
\gll \textbf{yew(e)-a}=yç {} ber-o \\ 
 barley\textsc{.f-pl.obl=add} {} take\textsc{.prs.ind-3sg:A} \\ 
\glt `[Out of obligation], he took the barley seeds, too.' \hfill [JP.29] \label{ex.align:o.prs2}
\z  

The following examples illustrate ergative alignment\is{ergative alignment}, limited to TAM categories based on the past stem of the verb. S-past (\ref{ex.align:s.pst2}) and P-past (\ref{ex.align:o.pst2}) are marked in the direct case\is{direct case}. In contrast, A-past (\ref{ex.align:A.pst2}) is marked in the oblique case\is{oblique case}.

\ea
\textit{karewanîyê amêyanê serere.} \\ 
\gll \textbf{karewanî-ê} amêya=nê sere=re. \\ 
 caravan\_people-\textsc{pl.dir} come\textsc{.pst.ptcp.pl=cop.3sg.pl:S} top=\textsc{postp} \\ 
\glt `Some passers-by had stayed there.' \hfill[DB.15] \label{ex.align:s.pst2}
\z 


\ea
\textit{hewarêşa wişkinɛnê.} \\ 
\gll \textbf{hewar-ê}=şa wişkinɛ=nê \\ 
 summer\_habitat\textsc{.m-pl.dir=3pl:A} scour\textsc{.pst.ptcp.pl=cop.3pl:O} \\ 
\glt `They scoured the summer habitats [searching for food, etc.].' \hfill[JE.3] \label{ex.align:o.pst2}
\z


\ea
\textit{ênne paɫewana zorşa kerđen} \\ 
\gll ênne \textbf{paɫewan-a} zor=şa kerđe=n\\
so\_much warrior-\textsc{pl.obl} pressure=\textsc{3pl:A} do\textsc{.pst.ptcp.m=cop.3sg:O} \\ 
\glt `The warriors put much pressure [on the skin]' \hfill [SK.132] \label{ex.align:A.pst2}
\z 

In ditransitive clauses, the R argument tends to be treated differently than the direct object argument in either ditransitive clauses or monotransitive clauses. The alignment\is{alignment} system with ditransitive clauses, where all three arguments are human, follows either ``indirective alignment\is{indirective alignment}" or ``neutral alignment\is{neutral alignment}". In indirective alignment\is{indirective alignment}, P and T are flagged the same, whereas R is treated differently. In neutral alignment\is{neutral alignment}, P, T, and R are flagged the same \citep{haspelmath2005argument}. The following examples illustrate indirective alignment\is{indirective alignment} in present tense constructions. R is flagged by a preposition, whereas T (\ref{ex:T}) and P (\ref{ex:O}) are only flagged by the oblique case.

\ea \label{ex:T}
\textit{dewayekey miđo be zayfekê.} \\ 
\gll \textbf{deway-ekey} mi-đe-o \textbf{be} \textbf{zayfe-(e)kê} \\ 
medicine\textsc{.m-def.m.sg.obl} \textsc{ind-}give\textsc{.prs-3sg:A} to girl\textsc{.f-def.f.sg} \\ 
\glt `He gave the medicine to the girl.’ \hfill[SH.198]
\z


\ea \label{ex:O}
\textit{şîyaweħşî bere.} \\ 
\gll \textbf{şîyaweħş-î} b\'er-e \\ 
 \textsc{pn-m.sg.obl} take.\textsc{imp-2sg:A} \\ 
\glt `Take Siyawahsh.' \hfill[SK.23]
\z 

Neutral alignment\is{neutral alignment} is attested less frequently, and it is characterised by P (\ref{neutral2}), T (\ref{neutral1}), and R (\ref{neutral1}) being marked in the oblique case. However, note that in (\ref{neutral1}) the oblique case is not visible on the feminine definite suffix \textit{-ekê}, yet as definite-marked Ps regularly take oblique marking, it can be assumed that the the T argument in (\ref{neutral1}) is oblique-marked.

\ea \label{neutral2}
\textit{luwe hardeka barewe.} \\ 
\gll l\'u-e \textbf{hard(î)-eka} b-ar-e=we \\ 
 go.\textsc{prs.imp-2sg:S} flour\textsc{.f}\textsc{-def.pl.obl} \textsc{imp-}bring\textsc{.prs-2sg:A}\textsc{=compl} \\ 
\glt `Go [and] bring the flour.' \hfill[RE.19]
\z 


\ea \label{neutral1}
\textit{zayfekê miđewwe îftađekey.} \\
\gll \textbf{zayf(ê)-ekê} mi-đe-w=we \textbf{îftađ(e)-eke}y \\
girl-\textsc{def.f.sg} \textsc{ind-}give.\textsc{prs-1sg:A=compl} messenger-\textsc{def.obl.m} \\
\glt `I will give the girl back to the messenger.' \hfill[KT.166]
\z 

In short, nominal case marking follows the template in Table \ref{tab:argumentflagging}. Direct case and oblique case interact in flagging the core arguments of the verb. 
\begin{table}[htp]
    \begin{tabular}{lll}
    \lsptoprule
    & \textsc{dir} & \textsc{obl} \\
    \midrule
    TAM based on present stem verbs&  S, A & P, T, R\\
    TAM based on past stem verbs&  S, P, T & A, R  \\
    \lspbottomrule
    \end{tabular}
    \caption{Morphological flagging of core arguments}
    \label{tab:argumentflagging}
\end{table}

In reality, not all arguments are case-marked according to the template in Table \ref{tab:argumentflagging}. The rest of this section gives a brief introduction to differential argument flagging\is{differential argument flagging}, by which I am mean the alternation in case marking\is{case marking} on different clausal arguments. 

\subsection{Differential A flagging\is{differential A flagging}} \label{sect:DAM}
The alignment\is{alignment} system licenses case marking\is{case marking} for A arguments. In clauses built on the present stem of the verb, A occurs in the unmarked direct case\is{direct case}, realised by different endings in the singular\is{singular} (depending on the noun class) and -\textit{ê} in the plural\is{plural}. By contrast, in verbal categories derived from the past stem of the verb, A should be, by default, accompanied by the oblique case\is{oblique case} suffixes. The following examples illustrate oblique\is{oblique case} marking on A-past arguments. Case marking\is{case marking} on A interacts with differential A indexing\is{differential A indexing}, which is mainly triggered by information structure (see \S\ref{sect:differential-A-indexing}). 

\ea
\textit{pase herey wat ...} \\ 
\gll pase \textbf{her-e-î} wat \\ 
 like donkey\textsc{.m}-\textsc{2sg}\textsc{-obl}\textsc{.m} say\textsc{.pst} \\ 
\glt `As the donkey said ...' \hfill[HB.54] 
\z 


\ea
\textit{meselen ađîşa hukim kerdênmê.} \\ 
\gll meselen \textbf{ađîşa} hukim kerdê=nmê \\ 
 for\_example \textsc{3pl.obl} rule\textsc{.m} do\textsc{.pst.ptcp.pl}\textsc{=cop}\textsc{.\textsc{1pl:O}} \\ 
\glt `For instance, they ruled over us.' \hfill[BP.10]
\z 

Right-dislocated A NPs in past transitive clauses tend to be oblique-marked\is{oblique case} (\ref{r-disloc1}), though occasionally they may appear in the direct case\is{direct case} (\ref{r-disloc2}).

\ea \label{r-disloc1}
\textit{fermawa cuwanekey be řîşçermekey.} \\ 
\gll fermawa \textbf{cuwan-ekey} be řîşçerme-(e)key \\ 
 say.\textsc{pst} youth-\textsc{def.m.sg.obl} to old\_man\textsc{-def.m.sg.obl}\\ 
\glt `The young man said to the old man.' \hfill[HR.10] 
\z


\ea \label{r-disloc2}
\textit{watiş, `xasa' cuwaneke.} \\
\gll wat=iş xas=a \textbf{cuwan-eke} \\
say.\textsc{3sg:A} well\textsc{=cop.3sg.m:S} youth\textsc{.m-def.m.sg.dir} \\
\glt `The youth said, `alright.’' \hfill[HR.72]
\z 

In reality, not all A arguments in past transitive clauses are oblique-marked\is{oblique case}. A token frequency count of overt As reveals that a quarter of the third-person overt A NPs are not oblique-marked\is{oblique case}; see Table (\ref{tab:overtA}). Table \ref{tab:overtA} excludes overt independent SPA pronouns in A function, whose token frequency is 24. Nor does it include cases where the oblique case is not visible after certain nominal bases or the additive clitic (see \ref{case-section}. Taken together, these make up 39 tokens of A-past NPs, amounting to 38\% of total overt A arguments for which the case marking is not available. As a side note, the total number of transitive clauses derived from past tense stems is 395, of which 103 have overt A arguments, amounting to 26\% of overt A NP arguments in the whole corpus, against 74\% of A arguments being expressed through clitic pronouns, or occasionally dropped. This low percentage of overt A NPs is associated with their high degree of topicality and is a reflection of universal bias against overt lexical NPs \citep{dubois1987}.
\begin{table}[htp]
    
    \begin{tabular}{lll}
\lsptoprule
& Token frequency &  \%  \\
\midrule
Oblique-marked As & 45 & 70\%\\
Direct-marked overt As&  19& 30\%\\
\lspbottomrule
    \end{tabular}
    \caption{Overt third person A arguments in the text corpus}
    \label{tab:overtA}
\end{table}

The counts in Table \ref{tab:overtA}, around a quarter of third-person agents skip oblique marking\is{oblique case}. Some tendencies suggest themselves. The following discussion is based on \citet[]{Mohammadirad2024Berlin}. First, the A argument is not oblique-marked\is{oblique case} when used as the subject in the cleft construction.

\newpage
\ea
\textit{î dêwênê î kinaçêşa bestêne.} \\ 
\gll î dêw-ê=nê î kinaçê=şa bestê=ne \\ 
 \textsc{dem.prox} ogre\textsc{.m}\textsc{-pl.dir}\textsc{=cop}\textsc{.3pl:S} \textsc{dem.prox} girl\textsc{.f.sg=3pl:A} tie\textsc{.pst.ptcp.f=cop.3sg.f:O} \\ 
\glt `It was the ogres who had muted the girl.' \hfill[JP.177]
\z 

The data suggest that information prominence triggers case marking on A-past arguments. The latter operates at two levels: ``local" and ``global" \citep[]{mcgregor2006focal,chappell2019optional}. In local prominence, the presence of case marking on A is associated with the A argument being in focus, especially in contrast with another argument in the immediately preceding clause, or against a presupposition in the mentioned discourse. In global prominence, the presence of case marking on A concerns the expectations about A arguments in larger chunks of discourse (see below). In Hewramî\il{Hewramî}, local prominence is generally associated with oblique case marking\is{oblique case}. Local prominence is associated with the A argument being in narrow focus\is{narrow focus} and contrastive focus\is{contrastive focus}. In (\ref{ex3}), the case marking on \textit{her} `donkey' is triggered by its contrast with \textit{min}.

\ea
\textit{î her-î zûwaniş zana min hîçim nezanan.} \\
\gll î \textbf{her-î} zûwan=iş zana-{\O} min hîç=im ne-zana=n \\ 
 \textsc{dem.prox} donkey\textsc{-sg.obl.m} language\textsc{.m.sg.dir=3sg:A} know\textsc{.pst-3sg.m:O} \textsc{1sg} nothing\textsc{=1sg:A} \textsc{neg-}know\textsc{.pst.ptcp.m=cop.3sg.m:O} \\ 
\glt `The donkey knew the [Sheikh’s] language; I didn’t know a thing!' \\ \hfill[HB.71] \label{ex3}
\z 

Case marking\is{case marking} on an A argument can also be triggered by global prominence. According to \citegen{mcgregor2006focal} ``expected actor principle'', in episodes of discourse with an expected actor, the actor can be left unmarked after its introduction. Any deviation from the expected actor is marked in the ergative case. Similarly, in Hewramî\il{Hewramî}, a topic that is continued in discourse may lose the oblique case\is{oblique case}, appearing instead in the direct case\is{direct case}. In the following excerpt, the established direct-marked\is{direct case} topic of the intransitive clause in (\ref{ex.dam-pro1}) is repeated in the following transitive clause (\ref{ex.dam-pro2}), even though the oblique\is{oblique case} form \textit{ađîşa} is expected (see \citealt{MohammadiradAflagging} {for a detailed discussion}). 

\newpage
\ea
\ea[]{
\textit{tenya ađê luwɛnê.} \\ 
\gll tenya \textbf{ađê} luwɛ=nê \\ 
 only \textsc{3pl.dir} go\textsc{.pst.ptcp.pl=cop.3pl:S} \\ 
\glt `Only they (Baba Khwada, Hama the Invisible, and Little Hama) went [to Iraq].'  \label{ex.dam-pro1}
 }
 \ex[]{
\textit{êtir ađê watenşa, `ême diruwê meyeymê.'} \\ 
\gll êtir \textbf{ađê} wate=n=şa ême diruwê me-de-îmê \\ 
 \textsc{disc.ptcl} \textsc{3pl.dir} say\textsc{.pst.ptcp.m=cop.3sg.m:O=3pl:A} \textsc{1pl} lie\textsc{.f} \textsc{neg.ind-}give\textsc{.prs-1pl:A} \\ 
\glt `They said, ‘We are not going to lie.’' \label{ex.dam-pro2} \hfill[BP.116]--[BP.117]
}
\z 
\z

In terms of information structure, A NPs in the direct case tend not to carry nuclear stress. In other words, they behave like topics and contain given information. In the following examples featuring direct marking\is{direct case} of the A NP, the nuclear stress is on the verb (\ref{ex.nooblA1}), the place adverb (\ref{ex.nooblA2}), and the negation prefix (\ref{ex.nooblA3}).

\ea
\textit{heɫbetene î kuře biřiş dawe.} \\ 
\gll heɫbete=ne î kuř=e biř=iş d\`a=we\,\suppipe{} \\ 
of\_course\textsc{=post} \textsc{dem.prox} boy\textsc{.m=dem} piece\textsc{=3sg:A} give\textsc{.pst}\textsc{=compl} \\ 
\glt `It is obvious that this boy has arrived.' \hfill[KŞ.80] \label{ex.nooblA1}
\z 


\ea
\textit{sibhan aẍe ça aẍegerîş kerden.} \\ 
\gll sibhan aẍe ç\`a aẍegerî=ş kerde=n\,\suppipe{} \\ 
 \textsc{pn} agha\textsc{.m} there governorship\textsc{.m=3sg:A} do\textsc{.pst.ptcp.m=cop.3sg.m:O} \\ 
\glt `Sibhan Agha governed over there.' \hfill[DP.27] \label{ex.nooblA2}
\z 


\ea
\textit{`her ta îse qisêş nekerdênê.'} \\ 
\gll her ta îse qisê=ş n\`e-kerdê=nê\,\suppipe{} \\ 
 donkey until now talk\textsc{.pl.dir}\textsc{=3sg:A} \textsc{neg-}do\textsc{.pst.ptcp.pl=cop.3pl:O} \\ 
\glt `[He said surprisingly], `the donkey hadn’t talked until now!’' \hfill[HB.46] \label{ex.nooblA3}
\z
 
Sometimes, the unmarked A-past argument is processed in a different intonation unit than the rest of the clause. This is the case with the following examples.

\ea
\textit{î kuře her memeş warden.} \\ 
\gll î kuř=e\,\suppipe{} her meme=ş warde=n\,\suppipe{} \\ 
 \textsc{dem.prox} boy\textsc{.m=dem} \textsc{emph} breast\textsc{m.sg.dir=3sg:A} eat\textsc{.pst.ptcp.m=cop.3sg.m:O} \\  
\glt `The boy had kept breast-feeding [from the tree].' \hfill[ZQ.44]
\z 


\ea
\textit{î şexse eger î kinaçêşe weşe kerde ...} \\ 
\gll î şexs=e\,\suppipe{} eger î kinaçê=ş=e weş-e kerd-e\,\suppipe{} \\ 
 \textsc{dem.prox} person\textsc{=dem} if \textsc{dem.prox} girl\textsc{.f.sg=3sg:A}\textsc{=dem} well\textsc{-f} do\textsc{.pst}\textsc{-3sg.f:O} \\ 
\glt `If this person has healed this girl ...' \hfill[ZP.90]
\z 

\subsection{Differential P flagging\is{differential P flagging}} \label{sect:DOM}
Direct objects in clauses derived from present stem verbs feature differential P flagging\is{differential P flagging}. Differential P flagging was recognised early on for Iranian languages in \citegen{Bossong1985} seminal study. Cross-linguistically, differential P flagging is often conditioned by animacy, definiteness, givenness, and person (see \citealt{Witzlack-MakarevichSerzant2018} for a recent overview). P arguments are expected to introduce given referents in discourse and to be low in animacy and definiteness. Any deviation from these features is expected to be marked with an additional flagging. Thus, for instance, in Punjabi\il{Punjabi} (Indic), a P argument that is not definite appears in the bare form, but if it has a definite reading, it is marked in the accusative case (\citealt[]{haspelmath2021role} citing \citealt[]{Bhatia1993}). 

The basic pattern in Tekht Hewramî\il{Hewramî!Tekht} seems to be that direct objects which are proper nouns, definite-marked, and discourse-salient get oblique case marking. By contrast, non-specific generic direct objects lack case marking\is{case marking}. Therefore, in line with cross-linguistic tendencies, P arguments that are high in animacy and definiteness are most likely marked by the oblique case. By way of example, consider the differential marking of the masculine noun \textit{pîya} `man' in the following examples. In both (\ref{mendir}) and (\ref{menobl}), \textit{pîya} is plural and the direct object of a present-tense verb. The direct marking in (\ref{mendir}) occurs because `men' has a non-specific reference, while the oblique marking in (\ref{menobl}) is due to a definite reference of `men'. 

\ea
\textit{pîyɛ bera.}\\
\gll \textbf{pîyɛ} ber-a \\ 
man.\textsc{pl.dir} take.\textsc{prs.ind-3pl:A} \\
\glt `They are taking men [to service].' \hfill[JF.18] \label{mendir}
\z


\ea
\textit{î pîyaya to zanî kamênê.} \\
\gll î \textbf{pîya-ya} to zan-î kamê=nê \\
\textsc{dem.prox} man-\textsc{pl.obl} \textsc{2sg} know.\textsc{prs.ind-2sg} which\textsc{.pl=cop.3pl:S}\\ 
\glt `These men, you know who they are.' \hfill [HM.39] \label{menobl}
\z 

The person of the P argument is a factor in differential P flagging. As stated in \S\ref{sect:SAP}, SAP pronouns have lost the case distinctions, and as such are not marked for oblique case when they appear as P (see \ref{ex.sap-p}). On the other hand, third person pronouns tend to appear in the oblique case when functioning as P; see (\ref{ex.nonsap-p})--(\ref{ex.nonsap-p3}).

\ea \label{ex.sap-p}
\textit{min bere.}\\
\gll min b\'er-e\\
\textsc{1sg} take.\textsc{prs.imp-2sg:A} \\
\glt `Take me.' \hfill[DB.43]
\z 


\ea
\textit{ađî fewtno.} \\ 
\gll \textbf{ađî} fewtn-o \\ 
 \textsc{3sg.obl.m} destroy\textsc{.prs.ind-3sg:A} \\ 
\glt `She destroyed it.' \label{ex.nonsap-p} \hfill[KŞ.78]
\z 


\ea \label{ex.nonsap-p2}
\textit{min çenî îse şime ... ađê mare kerûnew?}\\
\gll min çenî îse şime \textbf{ađê} mare ker-ûne=w\\
\textsc{1sg} how now \textsc{2pl} \textsc{3sg.f.obl} marriage do.\textsc{prs.ind-1sg:A=}and\\
\glt `How can I marry her now?' \hfill[ÇK.32]
\z 


\ea \label{ex.nonsap-p3}
\textit{ađîşa mewêno.}\\ 
\gll \textbf{ađîşa} me-wên-o\\ 
\textsc{3pl.obl} \textsc{neg.ind-}see\textsc{.prs-3sg:A}\\ 
\glt `He did not see them.' \hfill[JP.190]
\z 

Definiteness\is{definiteness} interacts with case marking\is{case marking} on direct objects. As seen in \S\ref{Definiteness}, the definite suffixes\is{definite suffix} \textit{-eke} and \textit{-e} are not used with all nouns with identifiable referents. Rather, once a noun has been identified with a definite status, it no longer needs to be marked by the definite suffix\is{definite suffix}. Definite marked Os tend to get case marking\is{case marking}:

\ea
\textit{milo herey gurc kerowe.} \\ 
\gll mi-l-o \textbf{her-e-î} gurc ker-o=we \\ 
\textsc{ind-}go\textsc{.prs-3sg:S} donkey\textsc{.m}-\textsc{def-m.sg.obl} alert do\textsc{.prs.ind-3sg:A=compl} \\ 
\glt `He went [and] prepared the donkey, [and] set off.' \hfill[HB.15]
\z 


\ea
\textit{yeweka kaɫo.} \\ 
\gll \textbf{yewe-(e)ka} kaɫ-o \\ 
 barley\textsc{.f}\textsc{-def.pl.obl} plant\textsc{.prs.ind-3sg:A} \\ 
\glt `He planted the barley seeds.' \hfill[JP.36]
\z 


\ea
\textit{to luwe zeřekey bere çenû wêt.} \\ 
\gll to l\'u-e \textbf{zeř-ekey} b\'er-e çenû wê=t \\ 
 \textsc{2sg} go.\textsc{prs.imp-2sg:S} money\textsc{.m-def.m.sg.obl} take\textsc{.prs.imp-2sg:A} with\textsc{.ez.gen} \textsc{refl=2sg:PSR} \\ 
\glt `You go, take the money with you.' \hfill[JP.103]
\z 

Direct objects that are proper nouns with identifiable referents generally appear in an oblique case\is{oblique case}; (\ref{ex.pn1})-(\ref{ex.pn2}).

\ea \label{ex.pn1}
\textit{hêyasî bizindê.} \\
\gll \textbf{hêyas-î} bi-zin-dê \\
\textsc{pn-m.sg.obl} \textsc{imp}-take\_out\textsc{.prs-2pl:A} \\
\glt `Throw Heyas out!' \hfill[HS.15]
\z 


\ea \label{ex.pn2}
\textit{řehmanî mijnasû.} \\
\gll \textbf{řehman-î} mi-jnas-û \\
\textsc{pn-m.sg.obl} \textsc{ind-}know.\textsc{prs-1sg:A} \\
\glt `I know Rahman.' \hfill\citep[296]{khan_language_2023}
\z 

On the other hand, P arguments that are common nouns and have non-specific reading tend not to get oblique case marking (\ref{ex.com}).

\ea \label{ex.com}
\textit{řefêqêş berowe pey yaneyşa.}\\
\gll \textbf{řefêq-ê}=ş ber-o=we pey yane-î=şa \\
friend\textsc{.m-pl.dir=3sg:PSR} take.\textsc{prs.ind-3sg:A=compl} to house.\textsc{m.sg.obl=3pl:PSR}\\
\glt `She took her friends to her (lit. their) house.' \hfill[WL.04]
\z 

P arguments which are preceded by demonstrative adjectives generally take oblique marking\is{oblique case}. The presence of case marking\is{case marking} is presumably triggered by the inherent definiteness\is{definiteness} of the direct object in the speech situation.

\ea
\textit{a esbî zînî kere peym.} \\ 
\gll \textbf{a} \textbf{esb-î} zînî k\'er-e pey=m \\ 
 \textsc{dem.dist} horse\textsc{-m.sg.obl} saddle do\textsc{.prs.imp-2sg:A} for\textsc{=1sg:R} \\ 
\glt `Saddle up the horse for me.' \hfill[ŞC.52]
\z 


\ea
\textit{to mişo î birayte şewêne bere.} \\ 
\gll to mişo \textbf{î} \textbf{bira-î=t=e} şew(e)-ê=ne b\'er-e \\ 
 \textsc{2sg} \textsc{aux} \textsc{dem.prox} brother\textsc{.m-sg.obl=2sg:PSR=dem} night\textsc{-f.sg.obl=post} take\textsc{.prs.imp-2sg:A} \\ 
\glt `You should take this brother of yours [there] at night.' \hfill[DG.26]
\z


\ea
\textit{î masî bere.} \\ 
\gll \textbf{î} \textbf{mas-î} b\'er-e \\ 
\textsc{dem.prox} yoghurt\textsc{.m-sg.obl} take\textsc{.prs.imp-2sg:A} \\ 
\glt `Take this yoghurt.' \hfill[JH.40]
\z 

Similarly, nouns whose referents have been mentioned in the previous discourse tend to get oblique-marked\is{oblique case}. In the following examples, case marking\is{case marking} on \textit{yewa} and \textit{pîyay} is triggered by their referents being evoked in previous discourse.

\ea
\textit{werû mecbûrî yewayç bero.} \\ 
\gll wer-û mecbûrî-(î) \textbf{yew(e)-a}=îç {} ber-o \\ 
 out\_of\textsc{-ez.gen} obligation\textsc{-m.sg.obl} barley\textsc{.f-pl.obl=add} {} take\textsc{.prs.ind-3sg:A} \\ 
\glt `Out of obligation, he took the barley seeds, too.' \hfill[JP.29]
\z 


\ea
\textit{hezretû şêxî pîyay kîyano.} \\ 
\gll hezret-û şêx-î \textbf{pîya-î} {} kîyan-o \\ 
 his\_highness\textsc{.m-ez.gen} sheikh\textsc{.m}\textsc{-m.sg.obl} man\textsc{.m-sg.obl} {} send\textsc{.prs.ind-3sg:A} \\ 
\glt `His Highness the Sheikh sent the men [to the donkey keeper].' \hfill[HB.78]
\z 

Similarly, indefinite-marked direct objects with specific referents can get case marking\is{case marking}. In (\ref{ex.dom.spec-indf1}), the direct object has a partitive reading and refers back to a previously mentioned referent. In (\ref{ex.dom.spec-indf2}), the indefinite-marked direct object has been previously mentioned and is possessed. In (\ref{ex.dom.spec-indf3}), the indefinite-marked direct object is known to the speaker. It is notable that all examples with case marking\is{case marking} on direct objects come from the Tekht varieties of Nwên and Silên.\footnote{In vernacular of Hewraman Tekht, where the indefinite suffix\is{indefinite suffix} \textit{-êwe} is now overwhelmingly the indefinite form, the case marking\is{case marking} on indefinite-marked nouns is absent (see \S\ref{Indefiniteness}).}

\ea
\textit{duwê hezarî wezî binyere heta seʕbe dinêwîşa nimazo.} \\
\gll duwê hezar-î wezî bi-nye=re heta seʕbe \textbf{din(ê)-êw-î}=şa nim(e)-az-o \\
two thousand-\textsc{pl.dir} walnut\textsc{pl.dir} \textsc{imp}-put.\textsc{2sg:A=povb} until morning one\textsc{-indf-m.sg.obl=3pl:PSR} \textsc{neg.ind}-leave.\textsc{prs.3sg:A} \\
\glt `If you lay down two thousand walnuts [here], it [rat] doesn't leave [even] one of them (i.e., the walnuts) [intact] until the next morning.' \\ \hfill[PK.37] \label{ex.dom.spec-indf1}
\z 


\ea
\textit{her saɫê meyo kuřêwît bera.} \\
\gll her saɫ(e)-ê m-e-y-o \textbf{kuř-êw-î=t} ber-a \\
each year\textsc{.f-indf} \textsc{ind-}come\textsc{.prs-ep-3sg:S} son\textsc{-indf-m.sg.obl=2sg:PSR} take.\textsc{prs.ind-3pl:A} \\
\glt `[With] each year that comes, they take one of your sons.' \hfill[ÇH.26] \label{ex.dom.spec-indf2}
\z 


\ea
\textit{minîç dewayêwîş şanî miđew.} \\
\gll min=îç \textbf{deway-êw-î}=ş şanî mi-đe-û \\
\textsc{1sg=add} medication-\textsc{indf-m.sg.obl=3sg:R} show \textsc{ind}-give.\textsc{prs-1sg:A} \\
\glt `As for me, I will show him a medication.' \hfill[DB.260] \label{ex.dom.spec-indf3}
\z

By contrast, indefinite-marked nominals with non-specific referents are less likely to get case marking\is{case marking}. In the following examples, the direct objects followed by the reduced indefinite form\is{indefinite suffix} \textit{-ê} and the full form \textit{-êwe} lack case marking\is{case marking}. The lack of grammatical case marking\is{case marking} seems to be triggered by the non-specific referent of the direct object.

\ea
\textit{qaqezê minwîso.} \\ 
\gll \textbf{qaqez-ê} mi-nwîs-o \\ 
 letter\textsc{.m-indf} \textsc{ind-}write\textsc{.prs-3sg:A} \\ 
\glt `He (the sultan) wrote a letter [lit. a paper].' \hfill[JH.68]
\z 


\ea
\textit{çolekêwe gêro minyo baxeɫêş.} \\ 
\gll \textbf{çolek(e)-êwe} gêr-o mi-ny(e)-o baxeɫ-ê=ş \\ 
sparrow\textsc{.f-indf} take\textsc{.prs.ind-3sg:A} \textsc{ind-}put\textsc{.prs-3sg:A} embrace\textsc{-indf=3sg:PSR} \\ 
\glt `He grabbed a sparrow [and] put it on his chest [under his clothing].' \\ \hfill[DP.36]
\z 


\ea
\textit{vatiş, `dey tate dey ba çîwêweş pey bere!’} \\ 
\gll vat=iş dey tate dey ba \textbf{çîw-êwe}=ş pey b\'er-e \\ 
 say\textsc{.pst=3sg:A} \textsc{disc.ptcl} father\textsc{.m} \textsc{disc.ptcl} \textsc{hort} thing\textsc{-indf=3sg:R} for take\textsc{.prs.imp-2sg:A} \\ 
\glt `She said, ‘Father, get him something.’' \hfill[JH.38]
\z 

Similarly, indefinite direct objects modified by an adjective do not get oblique-marked\is{oblique case}:

\ea
\textit{gojdê fire misanaw wera.} \\
\gll gojd-ê fire mi-san-a=w wer-a \\
meat\textsc{-indf} a\-lot \textsc{ind-}buy.\textsc{prs-3pl:A=}and
eat.\textsc{prs.ind-3pl:A} \\
\glt `They bought a lot of meat and ate it.' \hfill[JF.20]
\z 

Likewise, bare\is{bare} direct object nominals with non-specific generic referents, as opposed to individualised senses, do not get case marking\is{case marking}.

\ea
\textit{meɫa mara mareş biřa peyş.} \\ 
\gll \textbf{meɫa} m-ar-a mare=ş biř-a pey=ş \\ 
 mullah\textsc{.m} \textsc{ind-}bring\textsc{.prs-3pl:A} marriage\textsc{=3sg:O} cut\textsc{.prs.ind-3pl:A} for\textsc{=3sg:R} \\ 
\glt `They fetched a Mullah [and] married her (the girl) to him (the shepherd’s son).'\hfill[KŞ.88] \label{ex.dom.unspec-indf1}
\z 


\ea
\textit{toyç nan werî.} \\ 
\gll to=îç \textbf{nan} wer-î \\ 
 \textsc{2sg=add} bread.\textsc{dir.m} eat\textsc{.prs.ind-2sg:A} \\ 
\glt `[and] you will eat [a] meal.' \hfill[HB.41] \label{ex.dom.unspec-indf2}
\z

Lastly, quantified direct object entities are not marked in the oblique case\is{oblique case}, regardless of their information structure. This apparent anomaly seems to be caused by the fact that numerals\is{numerals} and quantifiers\is{quantifiers}, by default, trigger direct case\is{direct case} marking on the nominal heads.

\ea
\textit{beʕzê hêzimê maro mêwe.}\\ 
\gll \textbf{beʕzê} \textbf{hêzm(î)-ê} m-ar-o {} m-ê=we \\ 
 some firewood\textsc{.f-pl.dir} \textsc{ind-}bring\textsc{.prs-3sg:A} {} \textsc{ind-}come\textsc{.prs.3sg:S=compl} \\ 
\glt `He would take some firewood and return [home].' \hfill[ZP.13]
\z 


\ea
\textit{jenêç nîşore duwê zaroɫê wîno.} \\ 
\gll jen(î)-ê=ç nîş-o=re \textbf{duwê} \textbf{zaroɫ(e)-ê} wîn-o \\ 
 woman\textsc{-f.sg.obl}\textsc{=add} sit\textsc{.prs.ind-3sg:S=povb} two child\textsc{-pl.dir} see\textsc{.prs.ind-3sg:A}  \\ 
\glt `The wife gave birth to two babies. [Lit. She sat down [and] saw two babies.]'  \hfill[ZB.24]
\z


\begin{table}[b]
\begin{tabular}{llll}
    \lsptoprule
    Factors & Feature & Case marking \\
    \midrule
    \multirow{2}{*}{Person} & 1st and 2nd pronouns & no \\
%     \cline{2-3}
    & third person pronouns & yes \\
%     \cline{1-3}
    \tablevspace
    \multirow{3}{*}{Animacy} & human\footnote{Case marking for all animacy values is contingent on definite reading of referents, and/or discourse saliency.} & yes \\
%     \cline{2-3}
    & animate (non-humans) & yes \\
%     \cline{2-3}
    & inanimate & yes \\
%     \cline{1-3}
    \tablevspace
    \multirow{2}{*}{Uniqueness} & proper nouns & yes \\
%     \cline{2-3}
    & common nouns & yes/no\footnote{Common nouns can be case-marked if they have definite reference.} \\
%     \cline{1-3}
    \tablevspace
    \multirow{3}{*}{Definiteness} & definite-marked & yes \\
%     \cline{2-3}
    & specific & yes/no\footnote{The presence of case marking on specific P arguments marked by the indefinite suffix \textit{-êw(e)} is subject to dialectal variation (see above).} \\
%     \cline{2-3}
    & non-specific & no \\
%     \cline{1-3}
    \tablevspace
    Quantification & P being quantified & no \\
    \lspbottomrule
\end{tabular}
\caption{Conditions on differential P flagging in Hewramî: a preliminary analysis}
\label{tab:diffp}
\end{table}

\largerpage
In short, Hewramî\il{Hewramî} overtly oblique-marks a great range of objects, whether they are human or non-human objects, as long as they play a salient role in discourse, have been previously evoked, etc. It was also seen that oblique marking\is{oblique case} extends even to indefinite direct objects on the condition that they are specific. This was reported to be limited to the vernaculars of Nwên and Silên. By contrast, oblique-marking\is{oblique case} is absent for nominals with non-specific, generic reference, and for syntactic reasons, when a direct object is quantified. Table \ref{tab:diffp} summarises a preliminary overview of differential P flagging\is{differential P flagging} in Tekht Hewramî\il{Hewramî!Tekht}. As said, the factors conditioning differential P flagging interact with each other. Thus, a human P may not be marked in the oblique case if it has a non-specific reference (see \ref{mendir}).
    
 
\subsection{Differential flagging of non-core arguments} \label{sect:diffobl}
The case\is{case} system licenses oblique case\is{oblique case} marking for third-person nouns and pronouns that function as non-core arguments\is{oblique arguments}, e.g., goals\is{goal}, recipients\is{recipient}, addressees\is{addressee}, comitatives, beneficiaries. Differential flagging is taken to mean that not all non-core arguments\is{oblique arguments} are marked in the oblique case. \citet[]{Mohammadirad2025ICKL} identifies several factors conditioning case marking on the non-core arguments. It should be noted that these factors interact in differential case marking and it is ultimately the combined effect of these factors that is crucial in differential flagging of non-core arguments. The type of adpositional flagging is an important factor triggering differential case marking. On the one hand, flagging by means of prepositions (\ref{ex.goal.prep}) and (less so) circumpositions (\ref{ex.circ1}) tends to trigger oblique case on the adposition complement.

\ea
\textit{luwewe pey yaney!} \\ 
\gll l\'u-e=we \textbf{pey} \textbf{yane-î} \\ 
go.\textsc{prs.imp-2sg:S}\textsc{=post} to house\textsc{.m-sg.obl} \\ 
\glt `[Now] go back home!' \hfill[JH.118] \label{ex.goal.prep}
\z 


\ea \label{ex.circ1}
\textit{î kinaçêw to dermanû derdîş îna la î pîyaywe.} \\ 
\gll î kinaçê-û to derman-û derd-î=ş îna-∅ \textbf{la} \textbf{î} \textbf{pîya-î=we} \\ 
 \textsc{dem.prox} daughter\textsc{.f.sg.dir-ez.gen} \textsc{2sg} treatment\textsc{.m-ez.gen} illness\textsc{.m-sg.obl=3sg:PSR} \textsc{loc.deic.cop-3sg.m:S} with \textsc{dem.prox} man\textsc{.m-sg.obl=post} \\ 
\glt `This daughter of yours, the treatment for her illness lies with this man (i.e., Pir Shaliyar).' \hfill[ZP.33]
\z 

On the other hand, non-core arguments that are postpositional (\ref{ex.postpos1}) or bare (\ref{ex.goal.bare1}) are much less likely to get case-marked.

\newpage
\ea \label{ex.postpos1}
\textit{jenekêm qomyaş venî kelekewe.} \\ 
\gll jen(î)-ekê=m qomya=ş venî \textbf{kel-eke=we} \\ 
 woman\textsc{.f-def.f.sg}\textsc{=1sg:PSR} happen\textsc{.pst.3sg:S}\textsc{=3sg:R} at mountain\textsc{.m-def.m.sg.dir}\textsc{=post} \\ 
\glt `My wife was about to deliver a baby in the mountain.' \hfill[ZQ.14] 
\z 


\ea
\textit{melowe yane.} \\ 
\gll me-l-o=we \textbf{yane} \\ 
 \textsc{neg.ind-}go\textsc{.prs-3sg:S=compl} house\textsc{.m} \\ 
\glt `He didn’t go back home.' \hfill[JH.109] \label{ex.goal.bare1}
\z  

Note that it is not always straightforward whether the case marking on non-core arguments is due to the type of adposition used or another factor. For instance, if the head of the adpositional phrase is a locational noun (see \S\ref{sect:locationalnouns}), the case marking can be triggered by the ezafe marking on the locational noun, which makes the construction look like an adnominal possessive phrase.

\ea
\textit{mêwe dilû hewramanî.} \\ 
\gll m-ê=we \textbf{dil-û} \textbf{hewraman-î} \\ 
 \textsc{ind-}come\textsc{.prs.3sg:S=compl} inside\textsc{-ez.gen} \textsc{pn-m.sg.obl} \\ 
\glt `He returned to Hewraman.' \hfill[JP.141]
\z 

Another factor triggering differential case marking on non-core arguments is whether the adposition complement is an adnominal possessive construction, e.g., \textit{be tate-w min} `to my father'. Here, due to competition between ezafe marking on the head of the NP and case marking on the same slot (see \S\ref{sect:gen-ez} for the interaction of ezafe and case marking), only ezafe marking is viable (see \ref{ex.shuane}), unless the head noun is plural, in which case the two suffixes are compatible (\ref{ex.cem}).

\ea \label{ex.shuane}
\textit{ama la şuwanew gawa.}\\
\gll ama \textbf{la} \textbf{şuwane-û} \textbf{gaw(e)-a}\\
come.\textsc{pst.3sg:S} to shepherd.\textsc{m.sg.dir-ez.gen} cow.\textsc{f-pl.dir} \\
\glt `He came to the cows' shepherds.' \hfill[ÇH.108] 
\z 

\newpage
\ea \label{ex.cem}
\textit{dewayş kerđ çemaw kinaçêw patşay.}\\
\gll deway=ş kerđ \textbf{çem-a-w} \textbf{kinaçê-û} \textbf{patşa-î} \\
medicine.\textsc{m=3sg:A} do.\textsc{pst} eye\textsc{-pl.obl-ez.gen} daughter-\textsc{ez.gen} king-\textsc{m.sg.obl} \\
\glt `He put medicine into the king's daughter's eyes.' \hfill[DB.312]
\z 

Role is another factor conditioning differential case marking on non-core arguments. For instance, a nominal complement of the verb `become' rarely takes an oblique case, regardless of animacy. 

\ea
\textit{bî be patşa.} \\
\gll bî-{\O} be patşa \\
be.\textsc{pst-3sg.m:S} \textsc{adp} king.\textsc{m}\\
\glt `He became a king.' \hfill[DB.161]
\z 

For other roles, there is an animacy component playing a role in differential flagging. For instance, inanimate (\ref{inanim-instr}) and non-human animate instruments (\ref{animal-instr}) are not generally case-marked. On the other hand, human instruments (\ref{ex.hum}) tend to occur in the oblique case.

\ea \label{inanim-instr}
\textit{be kune awîşa ardêne.} \\ 
\gll \textbf{be} \textbf{kune} awî=şa ardê=ne \\ 
 by clay\_pot\textsc{.m} water\textsc{.f.sg.dir=3pl:A} bring\textsc{.pst.ptcp.f=cop.3sg.f:O} \\ 
\glt `They used to fetch water using clay pots.' \hfill[JE.16]
\z 


\ea \label{animal-instr}
\textit{be hesere hêzmîşa ardênê pey zimsanî.} \\ 
\gll \textbf{be} \textbf{hesere} hêzmî=şa ardê=nê pey zimsan-î \\ 
 by mule\textsc{.f} firewood\textsc{.pl.dir=3pl:A} bring\textsc{.pst.ptcp.pl=cop.3pl:O} for winter\textsc{.m-sg.obl} \\ 
\glt `They fetched firewood for the winter on mules.' \hfill[JE.35]
\z 


\ea \label{ex.hum}
\textit{werwe maro be nîrûwekeyş.} \\ 
\gll werwe m-ar-o \textbf{be} \textbf{nîrû-ekey=ş}\\ 
 snow\textsc{.f} \textsc{ind-}bring\textsc{.prs-3sg:A} by force\textsc{.m-def.m.sg.obl=3sg:PSR} \\ 
\glt `He fetched snow [ five or ten loads daily] using his men.' \hfill[DP.34]
\z 

The animacy effect may show itself for other arguments as well. For example, human goals are overwhelmingly case-marked (\ref{ex.hum-goal}). However, non-human goals may sometimes appear in the bare form (\ref{ex.nonh-goal}).

\ea \label{ex.hum-goal}
\textit{milo la ađî.} \\ 
\gll mi-l-o \textbf{la} \textbf{ađî} \\ 
 \textsc{ind-}go\textsc{.prs-3sg:S} to \textsc{3sg.obl.m} \\ 
\glt `He went to him (his uncle).'  \hfill[JP.14]
\z 


\ea \label{ex.nonh-goal}
\textit{ta meřeber amɛ.} \\ 
\gll \textbf{ta} \textbf{meřeber} amɛ \\ 
 until \textsc{pn} come\textsc{.pst.3pl:S} \\ 
\glt `[which means] they came as far as Marabar.' \hfill[BP.114]
\z 

To better illustrate differential case marking\is{differential case marking} of non-core oblique arguments\is{oblique arguments}, I did a frequency count of the goal\is{goal} arguments of verbs of movement, `come' and `go', in the text corpus. The count was limited to nominal and pronominal (i.e., third person) goals; goal arguments of `come' and `go' that are first and second persons were not counted as they do not show case distinctions. Adverbial goals (e.g., He went there) were not counted either. Finally, note that The frequency count shows that around 17\% of goal\is{goal} arguments (21 out of 122) are case-marked. In Table \ref{tab:dgm}, I have classified the differential oblique marking\is{differential oblique marking} based on the type of flagging the goal\is{goal} arguments have, and whether or not they are possessed. 
\begin{table}[htp]
    \begin{tabular}{lrrrrrr}
\lsptoprule
Flagging & \\
\midrule
 & \textbf{N} & \textbf{Obl-marked} & \textbf{\%} & \textbf{Unmarked} & \textbf{\%}\\
Prepositional&22 &19 & 86& 3 &14 \\
Bare\is{bare}& 49 & 2 & 4 & 47 & 96\\
Postpositional & 19& 0& 0& 19& 100\\
Possessed & 32 & 0& 0& 32& 100 \\\midrule
Total& 122 & 21& 17\% &101& 83\% \\
\lspbottomrule
    \end{tabular}
    \caption{Frequencies of oblique-marked goals\is{goal} of `come' and `go', categorised according to flagging}
    \label{tab:dgm}
\end{table}

As can be seen from Table \ref{tab:dgm}, only 17\% of goal\is{goal} arguments of `come' and `go' are oblique-marked\is{oblique case}. Evidently, goals\is{goal} that are flagged by prepositions exhibit stark differences in case marking from postpositionally flagged goals\is{goal}, which are not case marked. Similarly, possessed goals\is{goal} are not case marked across the board. On the other hand, bare\is{bare} goals\is{goal} are not flagged for case marking\is{case marking} across the board.

In short, differential case marking on non-core arguments depends on different factors, which together determine whether or not a non-core argument is marked in the oblique case. The relevant factors were said to be type of adpositional flagging, role, animacy, and whether or not the adposition complement is possessed. See \citet[]{Mohammadiradnon-core} for investigation of the combined effect of these factors on differential case marking of non-core arguments.


























\end{sloppypar}
