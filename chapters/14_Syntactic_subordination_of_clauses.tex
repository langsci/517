\chapter{Syntactic subordination of clauses}

\begin{sloppypar}

\section{Relative clauses}\label{sect:relative_clause}
Relative clauses embedded within NP may appear with a relativiser. There are two relativisers in Tekht Hewramî\il{Hewramî!Tekht}: the more general \textit{ke}, and the attributive ezafe \textit{-î} (see \S\ref{sect:attr.ez}). The latter is limited to subject relativisation in the present tense constructions, and only occurs with pronominal heads.\footnote{Square brackets have been used throughout the book to enclose additional context to help understand the text. In this chapter, square brackets, highlighted in bold, are additionally used to enclose syntactic phrases such as relative clauses and complement clauses.} 

\ea
\textit{aney dewayekey dros kero heqû dewayîş hen.}\\
\gll \textbf{[}ane-\textbf{î} deway-ekey dros ker-o\textbf{]} heq-û deway=ş hen{-\O}\\
\textsc{dem.dist.m.3sg.dir-ez.attr} medicine\textsc{.m-def.m.sg.obl} right do.\textsc{prs-3sg:A} right-\textsc{ez.gen} medicine\textsc{.m.sg.obl=3sg:NC} \textsc{exist-3sg.m:S} \\
\glt `The one who created the medicine has to be paid back.' \hfill[SH.198]
\z 

 \textit{ke} is the general relativiser; it can relativise different clausal arguments, including subjects, objects, indirect objects, non-canonical subjects\is{non-canonical subjects} (\ref{ex.noncanonic}), etc. In all cases, it has the invariable form \textit{ke}. The particle introduces both restrictive and non-restrictive relative clauses. It is used predominantly when the head nominal is definite. In (\ref{ex.definiterel}), the head nominal is definite; in (\ref{ex.indefrel}), it is indefinite.

\ea
\textit{î meʕmûrîye ke ʕêraqo am{ɛ}nê, a zemanû pîr şelîyarî bexşnayşa bexşn{ɛ}nê.} \\ 
\gll î meʕmûr-î=e \textbf{[}ke ʕêraq=o am{ɛ}=nê\textbf{]} a zeman-û pîr şelîyar-î bexşnay=şa bexşn{ɛ}=nê \\ 
\textsc{dem.prox} officer\textsc{.m-sg.obl=dem} \textsc{rel} \textsc{pn=post} come\textsc{.pst.ptcp.pl=cop.3pl:S} \textsc{dem.dist} time\textsc{.m-ez.gen} \textsc{pn} \textsc{pn-m.sg.obl} distribute\textsc{.nmlz=3pl:A} distribute\textsc{.pst.ptcp.pl=cop.3pl:R} \\ 
\glt `The officers who had come from Iraq ... in the time of Pir Shaliyar, people would distribute [food] to them.' \hfill[BP.40] \label{ex.definiterel}
\z 


\ea
\textit{mişo řû we yek neferî kerî ke derwardenû meʕlûma.} \\ 
\gll mişo řû=we yek nefer-î k\'er-î \textbf{[}ke derđwarde=n=û meʕlûm=a\textbf{]} \\ 
 \textsc{aux} face=to one person\textsc{.m-sg.obl} do\textsc{.prs.sbjv-2sg:A} \textsc{rel} qualified\textsc{.m=cop.3sg.m:S}=and obvious\textsc{=cop.3sg.m:S} \\ 
\glt `One should plead with one [ruler] who is qualified and distinct.' \hfill[ŞC.92] \label{ex.indefrel}
\z

The third-person pronominal head of a relative clause can only be expressed by the demonstrative sets in \S\ref{sect:ind-dem-pro}.

\ea
\textit{ane ke berdma şiş mangê menn.} \\ 
\gll \textbf{ane} \textbf{[}ke berd-∅=ma\textbf{]} şiş mang(e)-ê menn-{\O} \\ 
 \textsc{dem.dist.m.3sg.dir} \textsc{rel} take\textsc{.pst-3sg.m:O=1pl:A} six month\textsc{.f-pl.dir} remain\textsc{.pst-3sg.m:S} \\ 
\glt `\textbf{The one} whom we took [with us] lived for six months.' \hfill[ZQ.29]
\z 


\ea
\textit{ey înîşa ke minta gerekna ...} \\ 
\gll ey \textbf{înîşa} \textbf{[}ke min=ta gerek=na\textbf{]} \\ 
 \textsc{voc} \textsc{dem.prox.3pl.obl} \textsc{rel} \textsc{1sg}=\textsc{2pl:NC} necessary\textsc{.m=cop}\textsc{.1sg:S} \\ 
\glt `O those who want me ...' \hfill[KŞ.56] \label{ex.noncanonic}
\z 


The following example illustrates the use of \textit{ke} in a non-restrictive relative clause.

\ea
\textit{herkesîç metawo pêse min ke nelabû êtir ane hîçê derameđêş nîya.} \\ 
\gll herkes=îç me-taw-o pêse min \textbf{[}ke ne-la=b-û\textbf{]} êtir ane hîç-ê derameđ-ê=ş nîy(e)=a \\ 
 anyone\textsc{=add} \textsc{neg.ind-}can\textsc{.prs-3sg:A} like \textsc{1sg} \textsc{rel} \textsc{neg-}go\textsc{.pst.ptcp.m}=be\textsc{.prs-1sg:S} \textsc{disc.ptcl} \textsc{dem.dist.m.3sg.dir} nothing\textsc{-indf} income\textsc{-indf=3sg:NC} \textsc{neg.exist=cop.3sg.m:S} \\ 
\glt `Anyone who is not able [to work as a porter], like me, who has probably not been a porter, well, he has no income.' \hfill[JM.62]
\z 

The more common strategy for relativisation is to use no relativiser in the clause. Asyndetic relative clauses\is{asyndetic relative clauses} can be available for relativisation of any clausal argument in the text corpus, though apparently they are more common with subjects and direct objects. The following examples illustrate the relativisation of subject NP (\ref{ex.sbjrel}) and possessor NP (\ref{poss-rel}) without the relative particle.

\ea \label{ex.sbjrel}
\textit{aney serekeş warđebê luwa řaw hatîre.}\\
\gll \textbf{[}aney sere-(e)ke=ş warđe=b-ê\textbf{]} luwa řa-w hat-î=re\\
\textsc{dem.3sg.m.obl} head-\textsc{def.m.sg.dir=3sg:A}	eat.\textsc{pst.ptcp.m=be.prs-aug.3sg:O} go.\textsc{pst.3sg:S}	road.\textsc{f-ez.gen}	fortune-\textsc{m.sg.obl=povb}\\
\glt `The one who had eaten (the dove’s head) took (lit. went to) the way of good fortune.’ \hfill[DB.109]
\z

\ea
\textit{bû mêmanû a jenê î hêɫeşe hen.}\\
\gll b-û mêman-û a jen(î)-ê \textbf{[}î hêɫe=ş=e hen-{\O}\textbf{]} \\
be\textsc{.prs-1sg:S} guest-\textsc{ez.gen} \textsc{dem.dist} woman-\textsc{sg.f.obl} \textsc{dem.prox} egg\textsc{.m=3sg:NC=deic} \textsc{exist-3sg.m:S}\\ 
\glt `I shall be a guest of the woman who has this egg.' \hfill[DB.43] \label{poss-rel}
\z

 Relatedly, relative clauses modifying generic pronominal heads \textit{herke/herkes} `whoever, anybody who' and \textit{herçê} `whatever', drop the relativiser.

\ea
\textit{herkesî goş darabone, dûr kewtenwe ce, kuçê ce şareke.} \\ 
\gll herkes-î \textbf{[}goş dara=b-one\textbf{]} dûr kewte=n=we ce kuç-ê ce şar-eke \\ 
whoever-\textsc{obl.m} ear\textsc{.m} hold\textsc{.pst.ptcp.m}=be\textsc{.prs-3sg:O} far fall\textsc{.pst.ptcp.m=cop.3sg.m:S}\textsc{=compl} from little\textsc{-indf} from city\textsc{.m-def.m.sg.dir} \\ 
\glt `Anyone who had listened to him had gone away from the city.' \hfill[BP.167]
\z 


\ea
\textit{herçêma hen çane a firmande wêş zano heqû ême kama} \\ 
\gll herçê=ma \textbf{[}hen-∅ ç=a=ne\textbf{]} a firmande wê=ş zan-o heq-û ême kam=a \\ 
whatever\textsc{=1pl:NC} \textsc{exist-3sg.m:S} in=\textsc{dem.dist=post} \textsc{dem.dist} leader\textsc{.m} \textsc{refl=3sg:PSR} know\textsc{.prs.ind-3sg:A} right\textsc{.m-ez.gen} \textsc{1pl} which\textsc{=cop.3sg.m:S} \\ 
\glt `Whatever we possess there, your boss knows which portion [of land] is mine.' \hfill[PM.20]
\z 

The particle \textit{ke} may also function as a subordinator in a subset of factive complement clauses\is{factive content complement clauses} and adverbial temporal clauses\is{adverbial temporal clauses} (see \S\ref{sect:factive_compl}). 

\subsection{Extraposition of relative clauses}\label{sect:rel-extrapos}
Relative clauses may be extraposed from indefinite NPs. The extraposition seems to be common with linking verbs such as the copula or the existential predicate.

\ea
\textit{ħeyçêw nîya sengew min hurbêzno.} \\ 
\gll ħeyç-êw nî=a \textbf{[}senge-û min hur-bêzn-o\textbf{]} \\ 
nothing-\textsc{indf} \textsc{neg.exist=cop.3sg.m:S} weight-\textsc{ez.gen} \textsc{1sg} \textsc{pvb-}put\textsc{.prs-3sg:A} \\ 
\glt `There is nothing which can scale me up.' \hfill[ÇH.53]
\z 


\ea
\textit{pîyawê hen tawo î kinaçê to weşe kerowe.} \\ 
\gll pîya-ê hen-∅ \textbf{[}taw-o î kinaçê to weş-e k\'er-o=we\textbf{]} \\
man\textsc{.m-indf} \textsc{exist-3sg.m:S} can\textsc{.prs.ind-3sg:A} \textsc{dem.prox} girl\textsc{.f} \textsc{2sg} well\textsc{-f} do\textsc{.prs.sbjv-3sg:A=compl} \\ 
\glt `There is a man who can cure your daughter.' \hfill[JP.154]
\z 

Extraposition is also possible when the nominal head is questioned.

\ea
\textit{î gîre çêş bî min ward?} \\ 
\gll \textbf{î} \textbf{gîr=e} çêş bî-∅ \textbf{[}min ward-∅\textbf{]} \\ 
 \textsc{dem.prox} hook\textsc{.m=dem} what be\textsc{.pst-3sg.m:S} \textsc{1sg} eat\textsc{.pst-3sg.m:O} \\ 
\glt `What is this situation that I am caught in? [Lit. What is this hook that I ate?]' \hfill[HB.23]
\z 

\subsection{Resumptive pronouns in relativisation}
To relativise clausal arguments other than subjects, e.g., direct objects, a resumptive clitic pronoun (\ref{ex.resumptive})--(\ref{ex.resumptive2}) or an independent pronoun (\ref{ex.resumptive3}) is used to maintain the reference of the argument.

\ea
\textit{î dega toş vînî çoɫe bîyêne.} \\ 
\gll \textbf{î} \textbf{dega} \textbf{[}to=\textbf{ş} vîn-î\textbf{]} çoɫ-e bîyê=ne \\ 
 \textsc{dem.prox} village{\textsc{.f}} \textsc{2sg=3sg:O} see\textsc{.prs.ind-2sg:A} deserted-\textsc{f} be\textsc{.pst.ptcp.f=cop.3sg.f:S} \\ 
\glt `This village, which you see, was deserted.' \hfill[JE.4] \label{ex.resumptive}
\z 


\ea
\textit {îne şime maçdêş} \\ 
\gll îne şime m-aç-dê=ş \\ 
 \textsc{dem.prox.m.3sg.dir} \textsc{2pl} \textsc{ind-}say\textsc{.prs-2pl:A=3sg:O} \\
\glt `What you are saying [lit. This, which you say].' \hfill[ZP.128] \label{ex.resumptive2}
\z 


\ea
\textit{aneşa zilterû ʕalter bo ađî bera.} \\ 
\gll \textbf{ane}=şa \textbf{[}zil-ter=û ʕal-ter b-o\textbf{]} \textbf{ađî} ber-a \\
\textsc{dem.dist.m.3sg.dir=3pl:PSR} big\textsc{-cmpr}=and good\textsc{-cmpr} be\textsc{.prs.ind-3sg:S} \textsc{3sg.obl.m} take\textsc{.prs.ind-3pl:A} \\
\glt `They took the one who was bigger and healthier; they took him.' \hfill[ZB.40] \label{ex.resumptive3} 
\z 

Similarly, a resumptive pronoun is used for the relativisation of indirect objects. 

\ea
\textit{îne maçmêş pene şalîyare sîyaw lalow kinaçekên.} \\ 
\gll \textbf{îne} \textbf{[}m-aç-mê=\textbf{ş} pene şalîyar-e sîyaw\textbf{]} lalo-û kinaç(ê)-ekê=n \\ 
 \textsc{dem.prox.m.3sg.dir} \textsc{ind-}say\textsc{.prs-1pl:A=3sg:R} to \textsc{pn-ez.cmpd} black maternal\_uncle\textsc{.m-ez.gen} girl\textsc{.f-def.f.sg=cop.3sg.m:S} \\ 
\glt `\textbf{This [person]}, whom we call Shaliyar Siya, was the maternal uncle of the [king’s] daughter.' \hfill[ZP.36]
\z 


\section{Embedded questions\is{embedded questions}}
Embedded questions\is{embedded questions} are framed as dependent clauses introduced by a question particle. The clause is embedded under such verbs as `to know', `to ask', and `to understand'.

\ea
\textit{yo mezano řû we kê kero.} \\ 
\gll yo me-zan-o řû=we kê k\'er-o \\ 
 one\textsc{.m} \textsc{neg.ind-}know\textsc{.prs-3sg:A} face=to who do\textsc{.prs.sbjv-3sg:A} \\ 
\glt `One doesn’t know whom to plead with [when there is a problem].' \\\hfill[ŞC.91]
\z 


\ea
\textit{mezano çêş kero.} \\ 
\gll me-zan-o çêş k\'er-o \\ 
\textsc{neg.ind-}know\textsc{.prs-3sg:A} what do\textsc{.prs.sbjv-3sg:A} \\ 
\glt `He did not know what to do.' \hfill[ZP.111]
\z 

Embedded polar questions\is{embedded polar questions} are generally expressed without any question particle introducing the dependent clause.

\ea
\textit{îmtîhanê teriş kermê bizanmê řasa diron.} \\ 
\gll îmtîhan-ê ter=iş k\'er-mê bi-zan-mê řas=a diro=n \\ 
 test\textsc{.m-indf} another\textsc{=3sg:O} do\textsc{.prs.sbjv-1pl:A} \textsc{sbjv-}know\textsc{.prs-1pl:A} true\textsc{=cop.3sg.m:S} lie\textsc{=cop.3sg.m:S} \\ 
\glt `We will test him again, see if it [his supernatural power] is right [or] it is wrong.' \hfill[JP.64]
\z 

Embedded content questions\is{embedded content question} and embedded polar questions may be preceded by the question particle \textit{daxom}, which indicates the speaker's wonder about the content of the question. 

\ea
\textit{watşa, `daxom kê etik kerđêbo?'} \\
\gll wat=şa daxom \textbf{kê} etik kerđê=b-o \\
say.\textsc{pst=3pl:A} \textsc{q.ptcl} who disgrace do.\textsc{pst.ptcp.f}=be\textsc{.prs.3sg:O} \\
\glt `They said, `We wonder \textbf{who} might have disgraced her?’' \hfill[ED.195]
\z 

\section{Complement clauses\is{subordinate complement clauses}} \label{sect:subcontent}
Complement clauses refer to syntactic constructions where a sentence or predication is an argument of a predicate. A variety of subordinate clauses\is{subordinate clauses} that can be embedded as components of the main clause are brought together in this section under ``complement clauses''. Subordinate complement clauses\is{subordinate complement clauses} may function as a direct object or subject or the main predicate. In all these cases, they may be governed by a complementiser. The complement clause can come in two forms: an entire clause (\ref{ex.clause.sub}) or an infinitive\is{infinitive} (\ref{ex.inf.sub}). The latter occurs only rarely in the text corpus.

\ea
\textit{nimaza bilmê aweyanî.} \\ 
\gll nim(e)-az-a \textbf{[}bi-l-mê aweyanî\textbf{]} \\ 
 \textsc{neg.ind-}let\textsc{.prs-3pl:A} \textsc{sbjv-}go\textsc{.prs-1pl:S} habitat\textsc{.f} \\ 
\glt `We are not allowed to go to the village.' \hfill[DG.35] \label{ex.clause.sub}
\z 


\ea
\textit{kîyanaş pey beẍay wenay.} \\ 
\gll kîyan-a=ş pey beẍa-î \textbf{[}wenay\textbf{]} \\ 
 send\textsc{.prs.ind-3pl:A=3sg:O} to \textsc{pn-m.sg.obl} read\textsc{.inf} \\ 
\glt `They sent him to Baghdad to study.' \hfill[JP.80] \label{ex.inf.sub}
\z 

Subordinate complement clauses\is{subordinate complement clauses} may be subclassified into factive content complement clauses\is{factive content complement clauses} (\S\ref{sect:factive_compl}), non-factive content complement clauses\is{non-factive content complement clauses} (\S\ref{sect:nonfactive}), and purpose clauses\is{purpose clauses} (\S\ref{sect:purpose}), all of which follow the main clause. In the majority of complement clauses, the complement clause may appear next to the main clause without any accompanying complementisers. Purpose clauses may feature complementisers \textit{da} and \textit{ba}.

\subsection{Factive content complement clauses\is{factive content complement clauses}} \label{sect:factive_compl}

Factive content complement clauses\is{factive content complement clauses} appear as complements of factive predicates\is{factive predicates} such as `say', `know', etc. In these clauses, the factual content of the complement clauses is assumed by the predicate. Factual content clauses are generally expressed by asyndetic constructions. The subject of the complement clause does not have the same referent as the subject of the main clause in (\ref{factive1})--(\ref{factive2}), yet the lexical subject is absent since it can be retrieved in the form of an agreement marker on the verb.

\ea
\textit{êtir ađê zana ẍeyb bîyen.} \\ 
\gll êtir ađê zan-a \textbf{[}ẍeyb bîye=n\textbf{]} \\ 
\textsc{disc.ptcl} \textsc{3pl.dir} know\textsc{.prs.ind-3pl:A} disappearance be\textsc{.pst.ptcp.m=cop.3sg.m:S} \\
\glt `Then, they realised that he had disappeared [into the ground].' \hfill[BP.197] \label{factive1}
\z 


\ea
\textit{waçê pêse neçîrî piřa.} \\ 
\gll waç-ê \textbf{[}pêse neçîr-î piřa-∅ piřa-∅\textbf{]} \\ 
 say\textsc{.prs-aug.3sg:A} like hunt\textsc{.m-sg.obl} fly\textsc{.pst-3sg.m:S} fly\textsc{.pst-3sg.m:S} \\ 
\glt `It was said that he jumped like [i.e., as if he was playing] a game.' \hfill[ZB.56] \label{factive2}
\z 

The subordinating particle \textit{ke} may introduce an elaborative appositive clause to a nominal or a parenthetical clause.

\ea
\textit{we \^eđ\^iç ke \^i ş\^ex ʕosmane beferz maça murefeh b\^iyen.} \\
\gll we \^eđ=\^iç \textbf{[}ke \^i ş\^ex ʕosman=e\textbf{]} be-ferz m-aç-a murefeh b\^iye=n. \\
and \textsc{3sg.prox꞊add} \textsc{sbrd} \textsc{dem.prox} \textsc{pn} \textsc{pn=dem} by-assumption \textsc{ind}-say.\textsc{prs-3pl:A} well\_off be.\textsc{pst.ptcp.m꞊cop.3sg.m:S} \\
\glt ‘And he, namely Sheikh Osman, it is supposed that he was well off.’ \\ \hfill\citep[467]{khan_language_2023}
\z 


\ea
\textit{berdaşa ke be heyatim aɫfim nekenen; berdaşa aɫif ken\^e.} \\
\gll berd-a=şa \textbf{[}ke be heyat=im aɫf=im ne-kene=n\textbf{]} berd-a=şa aɫif ken-\^e \\
take.\textsc{pst-1sg:O=3pl:A} \textsc{sbrd} in life=\textsc{1sg:PSR} grass=\textsc{1sg:A} \textsc{neg}-uproot\textsc{.pst.ptcp.m=cop.3sg:O} take.\textsc{pst-1sg:O=3pl:A} grass mow.\textsc{pst-inf} \\
\glt ‘They took me—I have never mowed grass in my life—they took me to mow the grass.’ \hfill\citep[467]{khan_language_2023} 
\z 

\textit{ke} may also function as an adverbial subordinator. In this usage, it introduces a temporal clause. 

\ea
\textit{ke dukandar bênê min luwênê pey kirmaşanî.}\\
\gll \textbf{[}ke dukandar b-ên-ê\textbf{]} min lu-ên-ê pey {} kirmaşan-î \\
\textsc{sbrd} shop\_keeper be.\textsc{prs-aug-1sg:S} \textsc{1sg} go.\textsc{prs-aug-1sg:S} to {} \textsc{pn-m.sg.obl} \\
\glt `When I was a shop owner, I would go to Kermanshah [and bring fruit and such].’ \hfill\citep[468]{khan_language_2023}{}
\z 

\subsection{Non-factive complement clauses}\label{sect:nonfactive}
Non-factive complement clauses\is{non-factive content complement clauses} express activities that are not fulfilled or are potential from the viewpoint of the main predicate. As a result, the complement clause appears in the subjunctive. These clauses are, by default, expressed via the asyndetic strategy, even in (\ref{non-factive1}), where a chain of complement clauses is linked to the main verb via juxtaposition. In (\ref{non-factive2}), the complement clause features equi-deletion of its subject -- which has the same referent as the subject of the main clause.

\ea
\textit{gerekiş bîyen î memliketî dagîr kero pey wêş.} \\ 
\gll gerek=iş bîye=n \textbf{[}î memliket-î dagîr k\'er-o pey wê=ş\textbf{]} \\ 
necessary\textsc{=3sg:NC} be\textsc{.pst.ptcp.m=cop.3sg.m:S} \textsc{dem.prox} country\textsc{.m-sg.obl} occupied do\textsc{.prs.sbjv-3sg:A} for \textsc{refl=3sg:PSR} \\
\glt `He wanted to occupy this region [and control it] for his own interest.' \\ \hfill[DP.5] \label{non-factive2}
\z 


\ea
\textit{îcaze, bizane ađ îcaze miđo ême eçê nehar kermê yam ne.} \\ 
\gll îcaze bi-zan-e \textbf{[}\textbf{[}ađ îcaze mi-đ(e)-o\textbf{]} \textbf{[}ême e=çê nehar k\'er-mê yam ne\textbf{]}\textbf{]} \\ 
 permission \textsc{imp-}know\textsc{.prs-2sg:A} \textsc{3sg.m.dir} permission \textsc{ind-}give\textsc{.prs-3sg:A} \textsc{1pl} in=here lunch\textsc{.m} do\textsc{.prs.sbjv-1pl:A} or no \\ 
\glt `See if he lets us stay here for lunch or not.' \hfill[PM.9] \label{non-factive1}
\z 

\subsection{Purpose clauses\is{purpose clauses}}\label{sect:purpose}
The particle \textit{pîney} (lit. `for this') may express a purpose clause.

\ea
\textit{awdaxşa dênmê pîney dengşa wer bo.} \\
\gll awdax=şa d-ên-mê \textbf{p=îney} deng=şa wer b-o \\
hot\_water=\textsc{3pl:R} give.\textsc{prs-aug-1pl:A} for=\textsc{dem.prox-m.sg.obl} voice꞊\textsc{3pl:PSR} out be.\textsc{prs.sbjv-3sg:S} \\
\glt `We would give them hot water so that their voice be clear [lit. be free].’ \\\hfill\citep[562]{khan_language_2023}
\z 

Alternatively, the hortative particles \textit{ba} and \textit{da} may express complement purpose clauses\is{purpose clauses}. The complement clause appears in the subjunctive. 

\ea
\textit{mişom řosem bilo da bizano î yek nefere kên.} \\
\gll mişom řosem bi-l-o da bi-zan-o î yek nefer=e kê=n \\
\textsc{aux} \textsc{pn} \textsc{sbjv-}go.\textsc{prs-3sg:S} \textsc{sbrd} \textsc{sbjv-}know.\textsc{prs-3sg:A} \textsc{dem.prox} one person\textsc{=dem} who=\textsc{cop.3sg.m:S} \\
\glt `Rosam should go to realise who this person is.' \hfill[BM. 136]
\z 


\ea
\textit{da dey dewrşa gêrdê ba deẍaɫet kerû.} \\ 
\gll da dey dewr=şa g\stackunder[-10pt]{\^{e}}{\'{}}r-dê ba deẍaɫet k\'er-û \\ 
 \textsc{hort} \textsc{disc.ptcl} round\textsc{=3pl:PSR} take\textsc{.prs.imp-2pl:A} \textsc{hort} intervention\textsc{.m} do\textsc{.prs.sbjv-1sg:A} \\ 
\glt `Encircle them so that I can defeat [Sibhan Agha].' \hfill[DP.51]
\z 

Purpose clauses\is{purpose clauses} may be expressed asyndetically, in which case they take the morphological form of an infinitive\is{infinitive}; see (\ref{ex.inf.sub}) above and (\ref{inf.com}) below. The post-verbal infinitive\is{infinitive} in such constructions is limited to a verb of movement and may be alternatively analysed as a metaphorical goal. 

\ea
\textit{milo wenay.} \\
\gll mi-l-o wenay \\
\textsc{ind-}go.\textsc{prs-3sg:S} read.\textsc{inf} \\
\glt `He went to study [Islamic Jurisdiction].' \hfill[XŞ.01] \label{inf.com}
\z 

\subsection{Modifier clauses\is{modifier clauses}}
Clauses expressing wish may be placed following a nominal head and act as a non-restrictive modifier of the head noun.

\ea
\textit{hesûrem -- xwa ʕefweş kere -- waçî} \\
\gll hesûre=m xwa ʕefwe=ş k\'er-e waç-î \\
father\_in\_law=\textsc{1sg:PSR} God pardon=\textsc{3sg:PSR} do\textsc{.prs.imp-2sg:A} say.\textsc{prs-2sg:A} \\
\glt `My father-in-law--God pardon him--whom you talk about.’ \\\hfill\citep[286]{khan_language_2023}
\z 

\section{Adverbial clauses\is{adverbial clauses}}
Adverbial clauses\is{adverbial clauses} here refer to subordinate clauses\is{subordinate clauses} which function as modifiers of a main clause as a whole. In Hewramî\il{Hewramî}, adverbial clauses\is{adverbial clauses} feature grammatical morphemes with lexical content (e.g., \textit{wextê} `when'). In terms of word order\is{word order}, the subordinator precedes the adverbial clause. The adverbial clause generally precedes the main clause.

\ea
\textit{wextê bo be ħewt heşt saɫe xeɫk bero.} \\ 
\gll wext-ê b-o be ħewt heşt saɫe xeɫk ber-o \\ 
 time\textsc{.m-indf} be\textsc{.prs.ind-3sg:S} \textsc{adp} seven eight year\textsc{.f} people\textsc{.m} take\textsc{.prs.ind-3sg:A} \\ 
\glt `When he turned seven [or] eight years old, people would take him [into their houses].' \hfill[KŞ.36]
\z 

Adverbial clauses\is{adverbial clauses} include different types, including temporal, manner, conditional, concessive, etc. In what follows, the morphosyntactic properties of these clauses are discussed. Notably, in many cases, what is considered an adverbial clause in other languages is expressed by a simple juxtaposition of two clauses. 

\subsection{Adverbial temporal clauses}
Adverbial temporal clauses\is{adverbial temporal clauses} are expressed by the temporal lexical morphemes, most of which are connected to the subordinate clause\is{subordinate clause} by the indefinite suffix\is{indefinite suffix} \textit{-ê}, e.g., \textit{wextê} `when', \textit{zemanê} `when'.

\ea
\textit{wextê ʕeb{ɛ} berza kerawe, miđya kes nîya çêrişne.} \\ 
\gll \textbf{wext-ê} ʕeb{ɛ} berz=a ker-a=we mi-đy(e)-a kes nîy(e)=a çêr=iş=ne \\ 
 time\textsc{.m-indf} robe\textsc{.f.sg.obl} high\textsc{=compl} do\textsc{.prs.ind-3pl:A=compl} \textsc{ind-}look\textsc{.prs-3pl:S} person\textsc{.m} \textsc{neg.exist=cop.3sg.m:S} under\textsc{=3sg:R=post} \\ 
\glt `When they lifted the robe, they saw no one was under it.' \hfill[BP.194]
\z 

A subset of subordinators appears in their bare\is{bare} form when connecting to the subordinate clause\is{subordinate clause}. These include \textit{îse} `now that'; \textit{ta} `by the time, until'; \textit{ke} `when' (see \S\ref{sect:subcontent}).

\ea
\textit{a wextîyekey firmanbirđar bîyen ta jenekêş berdêne.} \\ 
\gll a wext-î-ekey firmanbirđar bîye=n \textbf{ta} jen(î)-ekê=ş berdê=ne \\ 
 \textsc{dem.dist} time\textsc{.m-sg.obl-def.m.sg.obl} obedient be\textsc{.pst.ptcp.m=cop.3sg.m:S} until woman\textsc{.f-def.f.sg=3sg:A} take\textsc{.pst.ptcp.f=cop.3sg.f:O} \\ 
\glt `He was at their service until he married the girl.' \hfill[RE.21]
\z 


\ea
\textit{to îse hîçit nîya ême yoma ane hiɫoşet miđo.} \\
\gll to \textbf{îse} hîç=it nîy(e)=a ême yo=ma ane hiɫoşe=t mi-đ(e)-o \\ 
\textsc{2sg} now\_that nothing\textsc{=2sg:NC} \textsc{neg.exist=cop.3sg.m:S} \textsc{1pl} one\textsc{.m=1pl:PSR} \textsc{dem.dist.m.3sg.dir} cracked\_wheat\textsc{.f.sg.f=2sg:R} \textsc{ind-}give\textsc{.prs-3sg:A} \\ 
\glt `Now that you have nothing, one of us will give you cracked wheat.' \\\hfill[JP.227]
\z 

Adverbial temporal clauses\is{adverbial temporal clauses} may be expressed by simply juxtaposing the subordinate clause\is{subordinate clause} to the main clause without any temporal subordinators.

\newpage
\ea
\textit{êşew herkes mêmaniş hen, mêmanekeş şewê witê, sereş biřo.} \\ 
\gll êşew herkes mêman=iş hen-∅ \textbf{mêman-eke=ş} \textbf{şew(e)-ê} \textbf{wit-ê} sere=ş b\'iř-o \\ 
tonight everyone guest\textsc{.m=3sg:NC} \textsc{exist-3sg.m:S} guest\textsc{.m-def.m.sg.dir=3sg:PSR} night\textsc{-f.sg.obl} sleep\textsc{.pst-3pl:S} head\textsc{.m=3sg:O} cut\textsc{.prs.sbjv-3sg:A} \\ 
\glt `Tonight, whoever has guests, \textbf{[when] the guest sleeps at night}, he shall decapitate him.' \hfill[BP.52]
\z 

The corpus data suggest that Hewramî\il{Hewramî} has no equivalent for `before'-clauses. Rather, a prepositional phrase is used.

\ea
\textit{weɫ ce şořišîne, bênêş meseɫen feretir heserêşa bênê.} \\
\gll weɫ ce şořiš-î=ne b-ên-ê=şa meseɫen fire-tir heser(e)-ê-şa b-ên-ê \\
before of revolution-\textsc{obl.m=post} be.\textsc{prs-aug-3pl:S=3pl:NC} for\_instance \textsc{more} mule\textsc{-pl.dir=3pl:NC} be.\textsc{prs-aug-3pl:S} \\
\glt `Before the revolution, people had... for example, people had mostly mules.' 
\z 

\subsection{Hypothetical manner clauses}
Hypothetical manner clauses\is{hypothetical manner clauses} may be expressed by the sentence \textit{pêse waçî} `as if [lit. like you said]'.

\ea
\textit{pêse waçî xway ketê pey kîyasen.} \\ 
\gll pêse w\'aç-î xwa-î ket-ê pey kîyase=n \\ 
as\_if say\textsc{.prs.sbjv-}\textsc{2sg:A} God\textsc{.m-sg.obl} bed\textsc{-indf} to send\textsc{.pst.ptcp.m=cop.3sg.m:R} \\ 
\glt `As if God had sent him a bed.' \hfill[JP.69]
\z 

There are also manner adverbial clauses expressed by the particle \textit{(e)pase} `as' (see \S\ref{sect:mann-adv}). When used in past transitive clauses, the A argument is generally unindexed. The adverbial clause in (\ref{advc}) contextualizes the source of information in the main clause.

\newpage
\ea
\textit{pase herey wat, simerîş kerdê torekekew ađî.} \\ 
\gll \textbf{pase} \textbf{her-e-î} \textbf{wat} simerî=ş kerd-ê toreke-(e)ke=û ađî \\ 
 like donkey\textsc{.m-def-m.sg.obl} say\textsc{.pst} straw\textsc{.pl.dir=3sg:A} do\textsc{.pst-pl} sack\textsc{.m-def.m.sg.dir}=and \textsc{3sg.obl.m} \\ 
\glt `As the donkey said, he (the man) put straw into his (i.e., donkey’s) sack.' \\\hfill[HB.54]
\z 


\ea
\textit{milo weraweriş miđyo epase min wat tomekeş şana zemînekeyne.} \\ 
\gll mi-l-o werawer=iş mi-đy(e)-o \textbf{e=pase} \textbf{min} \textbf{wat} tom-eke=ş şana {} zemîn-ekey=ne \\ 
 \textsc{ind-}go\textsc{.prs-3sg:S} opposite\textsc{=3sg:PSR} \textsc{ind-}look\textsc{.prs-3sg:S} \textsc{emph}=like \textsc{1sg} say\textsc{.pst} seed\textsc{.m-def.m.sg.dir=3sg:A} scatter\textsc{.pst.3sg:O} {} land\textsc{.m-def.m.sg.obl=post} \\ 
\glt `He went to him [and] saw that--as I said--he (Pir Shaliyar) scattered the seeds in the field.' \hfill[JP.51] \label{advc}
\z 

\subsection{Substitutive}
The equivalent to a substitutive subordinate clause\is{substitutive clause} is expressed by a construction consisting of the preposition \textit{cîyay} `instead of' plus the infinitive\is{infinitive} form of the verb.

\ea
\textit{bizekêçiş be lûlekê hurpiřa cîya leweřyayşa.} \\ 
\gll bize-(e)kê=ç=iş be lûle-(e)kê hur-piř-a \textbf{cîya} \textbf{leweřyay=şa} \\ 
 goat\textsc{.f-def.pl.dir=add=3sg:PSR} with flute\textsc{.f-def.pl.dir} \textsc{pvb-}jump\textsc{.prs.ind-3pl:S} instead\_of graze\textsc{.inf=3pl:PSR} \\ 
\glt `With the sound of the flute, the goats danced instead of grazing.' \hfill[JP.37]
\z


\subsection{Conditional clauses}
Conditional clauses\is{conditional clauses} feature a subordinate clause\is{subordinate clause} expressing the condition (protasis) and a main clause expressing the consequence (apodosis). The protasis is generally preceded by the particle \textit{eger} `if'.

\ea
\textit{eger goşiş darayne dûr dûr kewto.} \\ 
\gll eger goş=iş dara-î=ne dûr dûr kewt-∅=o \\ 
 if ear\textsc{.m=3sg:A} hold\textsc{.pst}-\textsc{2sg:R}\textsc{=povb} far far fall\textsc{.pst-3sg.m:S}\textsc{=post} \\ 
\glt `If they [lit. he] listen to you, they will go away.' \hfill[BP.163]
\z 


\ea
\textit{eger neweşîyêma bîyebo, duktur nebîyen.} \\ 
\gll eger neweşî-ê=ma bîye=b-o duktur ne-bîye=n \\ 
 if illness\textsc{.m-indf=1pl:NC} be\textsc{.pst.ptcp.m}=be\textsc{.prs-3sg:S} physician\textsc{.m} \textsc{neg-}be\textsc{.pst.ptcp.m=cop.3sg.m:S} \\ 
\glt `If we happened to be ill, there was no doctor.' \hfill[DG.5]
\z 

The protasis may be partly repeated to provide more emphasis on the condition.

\ea
\textit{dey eger to adiznî, eger to adiznî pey êmew nařihetnî pey ême, luwe g\stackunder[-10pt]{\^{e}}{\'{}}ɫe pî şarre!} \\ 
\gll dey eger to adiz=nî eger to adiz=nî pey ême=û nařihet=nî pey ême lu-e gêɫ-e p=î şar=re \\ 
\textsc{disc.ptcl} if \textsc{2sg} upset\textsc{=cop.2sg:S} if \textsc{2sg} upset\textsc{=cop.2sg:S} for \textsc{1pl}=and sad\textsc{=cop.2sg:S} for \textsc{1pl} go.\textsc{prs.imp-2sg:S} wander\textsc{.prs.imp-2sg:S} at=\textsc{dem.prox} city\textsc{.m=post} \\ 
\glt `If you are upset; if you are worried about us; if you are sad about us, go [and] wander in this town!' \hfill[BP.160]
\z 

Alternatively, the particle \textit{heke} may introduce the protasis of a conditional clause. This was only attested once in the text corpus.

\ea
\textit{kuřû şuwaney heke to kuşt{ɛ}w a wextîyekey hukmit daw înîşa îse îna yaneta înîşa.} \\ 
\gll kuř-û şuwane-î heke to kuşt-ê=û a wext-î-ekey hukm=it da=û înîşa îse îna-∅ yane=ta {} înîşa \\ 
son\textsc{.m-ez.gen} shepherd\textsc{.m-sg.obl} if \textsc{2sg} kill\textsc{.pst-cond}=and \textsc{dem.dist} time\textsc{.m-sg.obl-def.m.sg.obl} rule\textsc{.m=2sg:A} give\textsc{.pst}=and \textsc{dem.prox.3pl.obl} now \textsc{loc.deic.cop-3sg.m:S} house\textsc{.m=2pl:PSR} {} \textsc{dem.prox.3pl.obl} \\ 
\glt `The shepherd’s son who you [ordered to be] killed is now in your house and so on.' \hfill[KŞ.51]
\z 

The conditional clauses\is{conditional clauses} may be expressed without any subordinator in the protasis clause. This asyndetic expression of conditionality has the same function as the syndetic strategy.

\ea
\textit{a wextî to mîsal wat{ɛ}t, `a fiɫane kesem gerekene.’} \\ 
\gll a wext-î to mîsal wat-{ɛ}=t a fiɫan-e kese=m gerek-e=ne \\ 
 \textsc{dem.dist} time\textsc{.m-sg.obl} \textsc{2sg} for\_example say\textsc{.pst-cond.aug.3sg:O=2sg:A} \textsc{dem.dist} such\_and\_such\textsc{-ez.cmpd} person\textsc{.f=1sg:NC} necessary\textsc{-f=cop.3sg.f:S} \\ 
\glt `In earlier times, for instance [if] you said, ‘I want that certain person (i.e., girl).’' \hfill[JE.73]
\z 


\ea
\textit{hez kerî basû a tewana kere.} \\
\gll hez ker-î bas-û a tewen(î)-a ker-e \\
liking do.\textsc{prs.ind-2sg:A} talk-\textsc{ez.gen} \textsc{dem.dist} rock-\textsc{pl.obl} do.\textsc{prs.imp-2sg:A} \\
\glt ‘(If) you like, talk about those rocks.’ \hfill[hearsay]
\z 


\ea
\textit{meberîm yane milû mizgî.} \\ 
\gll me-ber-î=m yane mi-l-û mizgî \\ 
\textsc{neg.ind-}take\textsc{.prs-2sg:A=1sg:O} house\textsc{.m.sg.dir} \textsc{ind-}go\textsc{.prs-1sg:S} mosque\textsc{.m} \\ 
\glt `[If you do not invite me and] won’t take me home, I will go to the mosque.' \hfill[JH.33] 
\z





























































































\end{sloppypar}
