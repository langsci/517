\chapter{Noun phrase}
\begin{sloppypar}
\section{Noun phrase properties}
A noun phrase can consist of a noun, proper or generic (\ref{ex.noun-np}), a noun and optional modifiers (\ref{ex.np-np}), an independent pronoun (\ref{ex.pronoun-np}), a substantivised adjective (\ref{ex.adj-np}), or a verbal noun (\ref{ex.inf-np}).
\ea 
\textit{lawaw aman çawre peře bîye awî.} \\ 
\gll \textbf{lawaw} ama=n \textbf{çawre} peř-e bî-e awî \\ 
flood\textsc{.m} come\textsc{.pst.ptcp.m=cop.3sg.m:S} tent\textsc{.f} full\textsc{-f} be\textsc{.pst-3sg.f:S} water\textsc{.f} \\ 
\glt `The flood had come [and] the tent was filled with water.' \hfill[ZB.20] \label{ex.noun-np}
\z 

\ea 
\textit{î duwe dêwe} \\ 
\gll \textbf{î} \textbf{duwe} \textbf{dêw=e} \\ 
 \textsc{dem.prox} two ogre\textsc{.m=dem} \\ 
\glt `these two ogres' \hfill[JP.178] \label{ex.np-np}
\z 

\ea 
\textit{dey min şûyş kerû pene.} \\ 
\gll dey \textbf{min} şû-î=ş ker-û pene \\ 
 \textsc{disc.ptcl} \textsc{1sg} husband\textsc{.m-sg.obl=3sg:R} do\textsc{.prs.ind-1sg:A} to \\ 
\glt `I will marry him [lit. I will make husband to him].' \hfill[JH.59] \label{ex.pronoun-np}
\z 

\ea 
\textit{cuwanê bê pêsew î girdû cuwana.} \\ 
\gll \textbf{cuwan}-ê b-ê pêse-û î gird-û \textbf{cuwan-a} \\ 
youth\textsc{.m-indf} be\textsc{.prs-aug.3sg:S} like-\textsc{ez.gen} \textsc{dem.prox} all\textsc{-ez.gen} youth\textsc{.m-pl.obl} \\  
\glt `He was a young man like all the young men.' \hfill[ZQ.54] \label{ex.adj-np}
\z 

\ea 
\textit{bizekê cîya leweřyayşa hurpiřa.} \\ 
\gll bize-(e)kê cîya \textbf{leweřyay}=şa hur-piř-a \\ 
 goat\textsc{.f-def.pl.dir} instead\_of graze\textsc{.inf=3pl:PSR} \textsc{pvb-}jump\textsc{.prs.ind-3pl:S} \\ 
\glt `The goats were dancing instead of grazing.' \hfill[JP.55] \label{ex.inf-np}
\z 

A noun phrase may also consist simply of a numeral. In such cases, the bare\is{bare} form of the numeral is used. 
 \ea 
\textit{řowê ce řowa yuwe mê yoyç mê xizmetû şê ʕeladînî.} \\ 
\gll řo-ê ce řo-a \textbf{yuwe} m-ê \textbf{yo}=îç m-ê xizmet-û şê ʕeladîn-î\\ 
 day\textsc{.m-indf} from day\textsc{.m-pl.obl} one\textsc{.f} \textsc{ind-}come\textsc{.prs.3sg:S} one\textsc{.m=add} \textsc{ind-}come\textsc{.prs.3sg:S} service\textsc{.m-ez.gen} sheikh\textsc{.m} \textsc{pn-m.sg.obl} \\ 
\glt `Once [lit. One day of days], a woman and a man [lit. One (\textsc{f}) came, one (\textsc{m})] came into the service of Sheikh Aladin.' \hfill[ZB.1]
\z

\ea 
\textit{duwê yerê mênê pane.} \\ 
\gll \textbf{duwê} \textbf{yerê} m-ê-nê p=a=ne \\ 
two three \textsc{ind-}come\textsc{.prs-3pl:S} at=\textsc{dem.dist=post} \\
\glt `Two or three [men] went [lit. came] there [to Biyare].' \hfill[ŞC.9]
\z 

\section{Modifiers in the noun phrase}
Modifiers in the noun phrase are of four general types: demonstrative determiners\is{demonstrative determiners} (\ref{ex.dem.npmod}), quantifiers\is{quantifiers} (including numerals\is{numerals}) (\ref{ex.quant.npmod}), adjectives (\ref{ex.adj.npmod}), possessors (\ref{ex.poss.npmod}), and relative clauses embedded with NP (\ref{ex.relcl}); see \S\ref{sect:relative_clause} for the syntax of relative clauses). These modifiers differ in their placement within the NP and whether or not they attach to the head noun via a linker (see \S\ref{sect:np-word-order}).
\ea 
\textit{î kuře} \\ 
\gll î kuř=e \\ 
 \textsc{dem.prox} boy\textsc{.m=dem} \\ 
\glt `this boy' \hfill[ZQ.44] \label{ex.dem.npmod}
\z 

\ea 
\textit{beʕzê hêzimê} \\ 
\gll beʕzê hêzm(î)-ê \\ 
 some firewood\textsc{.f-pl.dir} \\ 
\glt `some firewood' \hfill[ZP.13] \label{ex.quant.npmod}
\z 

\ea 
\textit{karî naşerʕî} \\ 
\gll kar-î naşerʕî \\ 
 task\textsc{.m}\textsc{-ez.attr} unlawful \\ 
\glt `unlawful acts' \hfill[BP.62] \label{ex.adj.npmod}
\z 

\ea 
\textit{jenû şuwaney} \\ 
\gll jen(î)-û şuwane-î \\ 
woman\textsc{.f-ez.gen} shepherd\textsc{.m-sg.obl} \\
\glt `the shepherd’s wife' \hfill[KŞ.98] \label{ex.poss.npmod}
\z 

\ea 
\textit{ane ke berdma şiş mangê menn.} \\ 
\gll ane [ke berd-∅=ma] şiş mang(e)-ê menn-{\O} \\ 
 \textsc{dem.dist.m.3sg.dir} \textsc{rel} take\textsc{.pst-3sg.m:O=1pl:A} six month\textsc{.f-pl.dir} remain\textsc{.pst-3sg.m:S} \\  
\glt `The one \textbf{whom we took} [with us] lived for six months.' \hfill[ZQ.29] \label{ex.relcl}
\z 

\section{Linkers in the noun phrase structure} \label{ezafe}
Nominal heads are linked to their modifiers through a head-linking formative, generally referred to as ``ezafe (or izafe)" within Iranian linguistics. Tekht Hewram\^i\il{Hewramî!Tekht} uses two ezafe linkers: \textit{-û} and \textit{-î}. Their use is essentially dependent on the type of modifier that follows the head noun: \textit{-û} is used with locational nouns (\S\ref{sect:locationalnouns}) and genitive nouns (see below), whereas \textit{-î} is used with adjective modifiers. On this basis, \citet[82]{mackenzie_origins_1961}{} calls the former ``genitive ezafe\is{genitive ezafe}'’ and the latter ``epithetic ezafe". I retain MacKenzie's genitive ezafe\is{genitive ezafe} but use `attributive ezafe\is{attributive ezafe}' for his epithetic ezafe. These labels largely predict the use of \textit{-û} and \textit{-î}, but as will be seen below, there are cases where \textit{-î} is used with nominal and pronominal possessors. 
%
\subsection{Genitive ezafe\is{genitive ezafe}}\label{sect:gen-ez}
The presence of genitive ezafe\is{genitive ezafe} is generally present on the head noun, but its presence can be conditioned by the morphophonemic processes resulting from its attachment to some vowel-final bases, the case of the head noun, or (less so) semantics. The genitive ezafe \textit{-û} occurs following consonant-final bases. 
\ea 
\textit{kuřû şuwaney} \\ 
\gll kuř-û şuwane-î \\ 
 son\textsc{.m-ez.gen} shepherd\textsc{.m-sg.obl} \\ 
\glt `shepherd’s son' \hfill[KŞ.51]
\z 

When added to vowel-final bases, vowel hiatus is resolved through different strategies. In nouns marked for the direct case, the genitive ezafe \textit{-û} is realised differently depending on the final vowel of the base noun. It surfaces as a glide [w] following vowel-final masculine nouns (with final stress) and feminine nouns ending in \textit{-ê} or \textit{-á}. 
\noindent
\[
\text{-û} \rightarrow \text{[w]} \, / \, V́ \_ \quad \text{(where V́ is a stressed vowel or \textit{-ê})}
\]

\ea 
\textit{kuřew şuwaney} \\ 
\gll kuř-e-û şuwane-î \\ 
 son\textsc{.m-def-ez.gen} shepherd\textsc{.m-sg.obl}\\ 
\glt `the shepherd’s son' \hfill[KŞ.34]
\z 

\ea
\textit{zemaw patşay} \\
\gll zema-û patşa-î \\
son\_in\_law-\textsc{ez.gen} king-\textsc{m.sg.obl}\\
\glt `the king's son-in-law' \hfill[ED.275]
\z

\ea 
\textit{kinaçêw pađşayç maço} \\ 
\gll kinaçê-û pađşa-î=ç m-aç-o \\ 
 girl\textsc{.f.sg-ez.gen} king\textsc{-m.sg.obl=add} \textsc{ind-}say\textsc{.prs-3sg:A} \\ 
\glt `the king’s daughter said' \hfill[KŞ.40]
\z 


In contrast, the genitive ezafe \textit{-û} triggers the deletion of the final vowel in unmarked feminine nouns ending in unstressed \textit{-e} or \textit{-î}. The rule is summarised below.
\[
V \rightarrow \varnothing \, / \_ \text{-û} \quad \text{(where V = [e, î] and is unstressed)}
\]

\ea 
\textit{jenû ħemey ẍeybî hamîlê bo.} \\ 
\gll \textbf{jen(î)-û} ħeme-î ẍeybî hamîlê b-o\\ 
 wife\textsc{-ez.gen} \textsc{pn-m.sg.obl/ez.attr} invisible pregnant\textsc{.f} be\textsc{.prs.ind-3sg:S} \\ 
\glt `Hama the Invisible’s wife was pregnant.' \hfill[BP.205]
\z 

\ea 
\textit{îne dastanû mina.} \\ 
\gll îne \textbf{dastan(e)-û} min=a \\ 
 \textsc{dem.prox.m.3sg.dir} story\textsc{.f-ez.gen} \textsc{1sg=cop.3sg.m:S} \\ 
\glt `This is my story.' \hfill[JM.12]
\z 

On the other hand, adding the genitive ezafe to a noun -- which, according to the alignment pattern, should appear in the oblique case -- may lead to morphological competition between the genitive ezafe and the oblique case. This occurs primarily with masculine and feminine nouns in the singular oblique. In competition for the post-nominal slot, the genitive ezafe wins, leading to the nominal base being expressed in the unmarked direct case. In (\ref{ex:shuwan}), the preposition complement should appear in the oblique case, hence \textit{şuwane-y} [shepherd\textsc{-m.sg.obl}]. Yet, the oblique case is dropped before the genitive ezafe. Note that this cannot be explained by a phonological rule, as there there is no rule deleting <y> before <û>. In (\ref{ex:derd}), the complement of the preposition should appear in the oblique case, hence \textit{derdê} [benefit.\textsc{f.sg.obl}]; however, the oblique case is not compatible with the ezafe suffix, so instead the unmarked noun is used.
\ea
\textit{ama la şuwanew gawa.}\\
\gll ama \textbf{la} \textbf{şuwane-û} \textbf{gaw(e)-a}\\
come.\textsc{pst.3sg:S} to shepherd.\textsc{m.sg.dir-ez.gen} cow.\textsc{f-pl.obl} \\
\glt `He came to the cowherd.' \hfill[ÇH.108] \label{ex:shuwan}
\z 

\ea
\textit{be derdû min}\\
\gll be derd(e)-û min\\
to benefit\textsc{.f-ez.gen} \textsc{1sg} \\
\glt `to my benefit' \label{ex:derd} \hfill[ÇK.106]
\z 

On the other hand, the plural oblique \textit{-a} and the genitive ezafe are compatible, and the latter follows the former on the head noun.
\ea
\textit{dewayş kerđ çemaw kinaçêw patşay}\\
\gll deway=ş kerđ çem-a-û kinaçê-û patşa-î \\
medicine.\textsc{m=3sg:A} do.\textsc{pst} eye\textsc{-pl.obl-ez.gen} daughter-\textsc{ez.gen} king-\textsc{m.sg.obl} \\
\glt `He put medicine into the king's daughter's eyes.' \hfill[DB.312]
\z 

\ea
\textit{luwe la î dosaw tateyte.}\\
\gll lu-e la î dos-a-û tate-î=t=e \\
go\textsc{.imp-2sg:S} to \textsc{dem.prox} friend-\textsc{pl.obl-ez.gen} father-\textsc{m.sg.obl=2sg:PSR=deic} \\
\glt `Go to your father's friends.' 
\z 

The ezafe linker is sometimes absent. This absence may be semantically motivated, meaning the if possessor gets to be used frequently with a specific possessed noun, a lexicalised reading of the NP may start to take shape, trigger the absence of the genitive ezafe. 
\ea
\textit{mişo zema minîç bo.}\\
\gll mişo zema min=îç b-o\\
\textsc{aux} groom \textsc{1sg=add} be\textsc{.prs-3sg:S}\\
\glt `He should be my son-in-law too.' \hfill[ED.364]
\z 

\ea 
\textit{kinaçê padşay}\\
\gll kinaçê padşa-î\\
daughter king-\textsc{m.sg.obl}\\
\glt `the king's daughter' \hfill[JP.213]
\z

\subsection{Attributive ezafe\is{attributive ezafe}} \label{sect:attr.ez}
The attributive ezafe\is{attributive ezafe} is \textit{-î} (\textit{-y} following some vowel-final nouns). It is dropped following nouns ending in \textit{\E}, \textit{-î}, and \textit{\stackunder[-10pt]{\^{e}}{\'{}}}, but retained elsewhere.
\ea
\textit{ħemey wirdîkɫe} \\
\gll ħeme-î wirdîkɫe \\
\textsc{pn.m-ez.attr} little\textsc{.m} \\
\glt `little Hama' \\
\z 

\ea
\textit{tatey pîrim hen.} \\
\gll tate-î pîr=im hen{-\O} \\
father\textsc{.m-ez.attr} old\textsc{.m=1sg:NC} \textsc{exist-3sg.m:S} \\
\glt `I have an old father.' \hfill[PP.29]
\z 

\TabPositions{2.85cm,6.5cm}
\ea
\textit{diɫ-î bey}\tab`anger [lit. bad heart]’ [DP.38] \\
\textit{ħeme-y wirdîkɫe}\tab`little Muhammad’ [BP.123]  \\
\textit{qudret-î kem}\tab`little power’\\
\textit{pîya-y xas}\tab`good man’\\
\textit{řa-y eweɫ}\tab `first time' [JM.4] \\
\textit{karê xerabê} \tab`bad deeds’ \\
\textit{mîwê ʕalê}\tab`good fruits’\\
\textit{jenî lemepeře}\tab`pregnant woman (lit. full-belly woman)’ 
\z

The formative \textit{-î} is sometimes used with honorary titles, perhaps due to a contact effect from the neighbouring Central Kurdish\il{Kurdish!Central}. 
\TabPositions{2.85cm,7.5cm,8.5cm}
\ea
\textit{şêx-î ʕeladînî}\tab`Sheikh Aladin’\tab[PM.16]  \\
vs.\tab\tab \\
\textit{hezret-û şêxî}\tab `His Highness the Sheikh’\tab [HB.83]  \\
\z

The following examples highlight the use of attributive ezafe\is{attributive ezafe} following the indefinite suffix\is{indefinite suffix} \textit{-êw}. However, the attributive ezafe is dropped after the indefinite suffix\is{indefinite suffix} \textit{-êwe} (see below).
\TabPositions{4cm,8.85cm}
\ea
\textit{duwe qeranî-êw-î çerme}\tab`a white two-kurus coin’ \\
\textit{nan-û ker(e)-êw-î xase}\tab`a good loaf of bread and butter’ \\ 
\z

The following examples, taken from the Hewramî\il{Hewramî} dialects of Silên (\ref{ex.np-indf-ez1}) and Nwên (\ref{ex.np-indf-ez2}), illustrate the compatibility of the indefinite suffix\is{indefinite suffix} \textit{-êw} and the attributive ezafe\is{attributive ezafe}. 
\ea
\textit{cewahêrêwî çêwîş berd.} \\
\gll cewahêr-êw-î çêwî=ş berd-\O \\
treasure\textsc{.m-indf-ez.attr} wooden\textsc{.m=3sg:A} take.\textsc{pst-3sg.m:O} \\
\glt `He took away a wooden treasure.' \hfill[KK.65] \label{ex.np-indf-ez1}
\z 

\ea
\textit{wiɫaxêwî çermema hen.} \\
\gll wiɫax-êw-î çerme=ma hen-\O \\
mule\textsc{.m-indf-ez.attr} white\textsc{.m=1pl:NC} \textsc{exist-3sg.m:S} \\
\glt `We have a white mule.' \hfill[YX.19] \label{ex.np-indf-ez2}
\z 

In the vernacular of Hewraman Tekht, the ezafe is incompatible with nouns marked with the indefinite suffix\is{indefinite suffix}. Here, the two elements of the NP are juxtaposed without any ezafe particle.
\TabPositions{3.25cm,7.5cm,8.5cm}
\ea
\textit{yanêwe çane}\tab`a proper house’\tab[ZP.86]  \\
\textit{kitêbêwe gewre}\tab`a big book’\tab[KŞ.8]  \\
\textit{wiɫatêwe ter}\tab`another country’\tab[JH.6]  \\
\textit{memɫêketê çoɫ} \tab`a deserted region’\tab[ZB.5]  \\
\textit{kinaçê ezemê}\tab`an unmarried daughter’\tab[ZB.8]  \\
\textit{hewrê sîyaw}\tab`a black cloud’\tab[ZB.17]  \\
\textit{jenê xase}\tab`a nice wife’\tab[JH.64]  \\
\textit{kuřê cuwanxas}\tab`a good-looking boy’\tab[KŞ.68]  \\
\z

Finally, the attributive ezafe may be used for subject relativisation -- only with pronominal subjects -- in present tense constructions (see \S\ref{sect:relative_clause}). This construction has been attested in the Nwên vernacular.
\ea \label{ex.aney-rel}
\textit{aney pa milo řaw hatîre bo patşa.} \\
\gll ane-\textbf{î} pa mi-l-o řa-û hat-î=re b-o patşa \\
\textsc{dem.dist.m.3sg.dir-ez.attr} from\_there \textsc{ind-}go.\textsc{prs-3sg:S} way\textsc{.f-ez.gen} fortune-\textsc{m.sg.obl=post} become.\textsc{prs.ind-3sg:S} king \\
\glt `The one who went to the way of good fortune became a king.' [DB.111]\\
\z 

\subsection{Compound ezafe\is{compound ezafe}} \label{sect:cmpdez}
Tekht Hewram\^i\il{Hewramî!Tekht} has another linking element, generally called the ``compound ezafe\is{compound ezafe}'', which it uses in tightly-knit compound NPs which can have definite or indefinite specific readings. The modifier in compound NPs is limited to an adjective. Unlike noun-genitive NPs, the linker \textit{-e} is incompatible with the indefinite\is{indefinite suffix} and definite suffixes\is{definite suffix} in the same slot. These suffixes then appear on the adjective modifier.
\ea
\textit{awîre kuçê} \\
\gll awîr-e kuç-ê \\
fire\textsc{.m-ez.cmpd} small\textsc{-pl.dir} \\
\glt `small fires’ \hfill[JE.37]  \\
\z

\ea 
\textit{giyane gûwînê} \\
\gll giyan-e gûwîn(e)-ê \\
soul\textsc{.m-ez.cmpd} sickly\textsc{.m-indf} \\
\glt `barely alive (lit. sickly soul)’ \hfill[ZQ.19]  \\
\z

\ea 
\textit{şîyawweħşî tûte kuɫêwiş bê.} \\
\gll şîyawweħş-î tût(e)-e kuɫ-êw=iş b-ê \\
\textsc{pn-m.sg.obl} dog\textsc{.m-ez.cmpd} small\textsc{.m-indf=3sg:NC} be.\textsc{prs-aug.3sg:S} \\
\glt `Siyawahsh had a small dog.' \hfill[SK.24]
\z 

\ea 
\textit{zengene wişɫalê} \\ 
\gll zengen(e)-e wişɫal(e)-ê \\ 
 hoe\textsc{.m-ez.cmpd} small\textsc{.m-indf}  \\ 
\glt `a small hoe' \hfill[JP.30]
\z 
\ea
\textit{nane tazêwe }\\
\gll nan-e taz(e)-êwe \\
bread\textsc{.m-ez.cmpd} fresh\textsc{.f-indf} \\
\glt `a fresh [loaf of] bread'
\z 

\ea
\textit{waɫe wiçkiɫekêş} \\
\gll waɫ(ê)-e wiçkiɫ(ê)-ekê꞊ş \\
sister\textsc{.f-ez.cmpd} little\textsc{.f-def.f.sg꞊3sg:PSR} \\
\glt `her younger sister’ \hfill[JH.40]  \\
\z

\newpage
\ea
\textit{kinaçe wiçkiɫekêşa} \\
\gll kinaç(ê)-e wiçkiɫ(ê)-ekê꞊şa \\
daughter\textsc{.f-ez.cmpd} little\textsc{.f-def.f꞊3pl:PSR} \\
\glt `their little daughter’ \hfill[JH.46] 
\z 

In a few cases, the linker has become lexicalised in the structure of some compound NPs. 
\TabPositions{2.25cm,6.5cm,7.5cm}
\ea
\textit{tiş-e henar}\tab`pomegranate molasses’\tab[JP.249]  \\
\textit{qiřoɫ-e dar}\tab`tree hollow’\tab [ZQ.23]  \\
\z

\section{Interaction of nominal case marking and ezafe marking} \label{sect:ez.case}
In \S\ref{sect:gen-ez} we looked at the interaction of case marking\is{case marking} and ezafe marking in genitive constructions with one possessor. This section looks at this interaction in genitive constructions with more than one possessor. In theory, both affixes should be expressed on the genitive; however, in competition for the slot, only one formative remains. 

The data in the text corpus only features genitive constructions with two possessors. Two patterns emerge from these constructions for the combination of case\is{case} and ezafe. First, if the possessor is masculine\is{masculine}, the oblique case\is{oblique case} blocks the genitive ezafe\is{genitive ezafe}.  
\ea \label{ex.poss2ez}
\textit{î gîyane îna qefesû sîney minne} \\ 
\gll î gîyan=e îna-∅ \textbf{qefes-û} \textbf{sîne-\textbf{y}} \textbf{min=ne} \\ 
 \textsc{dem.prox} soul\textsc{.m=dem} \textsc{loc.deic.cop-3sg.m} cage\textsc{.m-ez.gen} chest\textsc{.m-sg.obl} \textsc{1sg:PSR=post} \\ 
\glt `the soul that is in my chest [lit. in the cage of my chest]' \hfill[DP.38]
\z 

In (\ref{ex.poss2ez}), the expected combination for the noun phrase `the cage of my chest' would have been \textit{qefesû sîne-y-û minne} [cage\textsc{.m-ez.gen} chest\textsc{.m-sg.obl-ez.gen} \textsc{1sg=post}]. However, in competition for the slot on the second possessor, only the oblique suffix\is{oblique case} remains, and the ezafe gets deleted. 

The following example suggests that with a feminine\is{feminine} possessor, it is instead the genitive ezafe\is{genitive ezafe} that blocks the expression of oblique case\is{oblique case}. In the following example, one would expect \textit{dega} `village' to appear in the oblique case\is{oblique case} as \textit{degɛ}, hence, \textit{ew des-û degɛ-w hewraman-î=ne} [\textsc{dem.dist} hand\textsc{.m-ez.gen} village\textsc{.f.sg.obl-ez.gen} \textsc{pn-m.sg.obl}\textsc{=post}]. However, the ezafe genitive blocks the expression of the oblique case\is{oblique case} on \textit{dega}. 
\ea 
\textit{a şexse îna ew desû degaw hewramanîne} \\ 
\gll a şexs=e îna-∅ \textbf{ew} \textbf{des-û} \textbf{dega-û} \textbf{hewraman-î}=ne \\ 
\textsc{dem.dist} saint\textsc{=dem} \textsc{loc.deic.cop-3sg.m:S} \textsc{dem.dist} hand\textsc{.m-ez.gen} village\textsc{.f-ez.gen} \textsc{pn-m.sg.obl}\textsc{=post} \\ 
\glt `The saint [whose grave is] on t\textbf{he other side of Hewraman}' \hfill[ZP.6]
\z 

When an adjective modifies a genitive noun with masculine\is{masculine} gender, the attributive ezafe\is{attributive ezafe} \textit{-î} and the singular\is{singular} oblique suffix\is{oblique case} \textit{-î} co-occur in the structure of the NP and merge into one. That is to say, haplology occurs, and one of the identical forms gets lost. This follows from a principle in morphology which disallows the occurrence of two identical morphemes in a row \citep{yip_identity_1998}.
\ea 
\textit{jenû ħemey ẍeybî hamîlê bo.} \\ 
\gll \textbf{jen(î)-û} \textbf{ħeme-î} \textbf{ẍeybî} hamîlê b-o \\ 
wife\textsc{-ez.gen} \textsc{pn}\textsc{-m.sg.obl/.ez.attr} invisible pregnant\textsc{.f} be\textsc{.prs.ind-3sg:S} \\ 
\glt `Hama the Invisible’s wife was pregnant.' \hfill[BP.205]
\z 

In the above example, the attributive \textit{-î} and the oblique case\is{oblique case} \textit{-î} following \textit{ħeme} coalesce into one formative, resulting in a cumulative affix.  

\section{Word order\is{word order} in the noun phrase} \label{sect:np-word-order}
A head noun can take up to five modifiers in the structure of the NP. NP-internal modifiers include demonstrative determiners\is{demonstrative determiners}, quantifiers\is{quantifiers} (including numerals\is{numerals}), adjectives, possessors, and relative clauses. The first two precede the head noun, and the last three follow it. The modifiers are also differentiated in how they attach to the head nouns. Demonstrative determiners\is{demonstrative determiners} and quantifiers\is{quantifiers} precede the noun via simple juxtaposition. In contrast, adjectives and possessors are linked to the noun via an ezafe linker (see \S\ref{ezafe}). Relative clauses are generally not linked to the noun by an ezafe linker (see \ref{ex.aney-rel} for an exception).

As seen in \ref{sect:demonstrative-determiner}, demonstrative determiners\is{demonstrative determiners} are discontinuous; they flank the head noun when it is bare\is{bare} (\ref{dem-bare}) or possessed (\ref{dem-possessed}), but not when it is only oblique-marked\is{oblique case} (\ref{dem-obl}).

\ea \label{dem-bare}
\textit{î kuře} \\ 
\gll \textbf{î} kuř=\textbf{e} \\ 
 \textsc{dem.prox} boy\textsc{.m=dem} \\ 
\glt `this boy' \hfill[ZQ.44] \label{ex.dem.npmod1}
\z 

\ea \label{dem-possessed}
\textit{î kinaçête de pî kuřîme.} \\ 
\gll î kinaçê=t=e d\'e-(e) p=\textbf{î} kuř-î=m=\textbf{e} \\ 
\textsc{dem.prox} daughter\textsc{.f.sg=2sg:PSR=dem} give\textsc{.prs.imp-2sg:A} to=\textsc{dem.prox} son\textsc{.m-sg.obl=1sg:PSR=dem} \\ 
 \\ 
\glt `Give your daughter to my son [in marriage].' \hfill[RE.7]
\z 

\ea \label{dem-obl}
\textit{î kuřî beynne berdê.} \\ 
\gll \textbf{î} kuř-î beyn=ne b\'er-dê \\ 
 \textsc{dem.prox} boy\textsc{.m-sg.obl} between\textsc{=post} take\textsc{.prs.imp-2pl:A} \\ 
\glt `Kill the boy!' \hfill[KŞ.56]
\z 

Quantifiers\is{quantifiers} and numerals\is{numerals} also precede the head noun. Likewise, classifiers\is{classifiers} and measures of units come before the head noun.
\ea 
\textit{şiş mangê} \\ 
\gll şiş mang(e)-ê \\ 
 six month\textsc{.f-pl.dir}\\ 
\glt `six months' \hfill[JM.28]
\z 

\ea 
\textit{çinnê saɫê çêweɫter} \\ 
\gll çinnê saɫ(e)-ê çêweɫter \\ 
some\textsc{.pl} year\textsc{.f-pl.dir} ago  \\ 
\glt `some years ago' \hfill[ZQ.9]
\z 

\ea 
\textit{duwê debê awî} \\
\gll duwê deb(e)-ê awî \\
two bucket-\textsc{pl.dir} water\textsc{.f} \\
\glt `two buckets of water' 
\z

Adjectives and possessors follow the noun they modify. Adjectives are linked to the head noun via the attributive ezafe\is{attributive ezafe} linker \textit{-î}. As discussed in \S\ref{sect:attr.ez}, the ezafe linker appears following certain head nouns but is dropped in most contexts, including after the indefinite suffix\is{indefinite suffix} \textit{-êwe} and the direct plural suffix\is{direct case} \textit{-ê}. 
\ea 
\textit{diɫî beyim çađî nîya.} \\ 
\gll diɫ-î bey=im ç=ađî nîy(e)=a \\ 
heart\textsc{.m-ez.attr} bad\textsc{.m=1sg:NC} from\textsc{=3sg.obl.m} \textsc{neg.exist=cop.3sg.m:S} \\  
\glt `I am not angry with him. [Lit. My bad heart is not at him.]' \hfill[DP.38]
\z 

\ea 
\textit{ħemey wirdîkɫeyç miçkile bo. wirdîkɫe bo.} \\ 
\gll ħeme-î wirdîkɫe=îç miçkile b-o wirdîkɫe b-o  \\ 
 \textsc{pn.m}\textsc{-ez.attr} small\textsc{.m=add} small\textsc{.m} be\textsc{.prs.ind-3sg:S} small\textsc{.m} be\textsc{.prs.ind-3sg:S}  \\ 
\glt `Little Hama was small and young.' \hfill[BP.143]
\z 

\ea 
\textit{kinaçêwe jîreş bê.} \\ 
\gll kinaçê-(e)we jîr-e=ş b-ê  \\ 
 daughter\textsc{.f-indf} intelligent\textsc{-f=}\textsc{3sg:NC} be\textsc{.prs-aug.3sg:S}  \\ 
\glt `He had an intelligent daughter.' \hfill[HR.23]
\z

Multiple adjectives are linked to the head noun via two strategies. The first involves the coordinating conjunction =\textit{û} to link the adjectives:
\ea
\textit{miđyo kuřê noxetû cuwanxasû taze yawa pene.} \\ 
\gll mi-đy(e)-o kuř-ê noxet=û cuwanxas=û taze-yawa-pene \\ 
 \textsc{ind-}look\textsc{.prs-3sg:S} boy\textsc{.m-indf} just\_grown\_beard\textsc{.m}=and gentlemanly\textsc{.m}=and just-arrive\textsc{.pst.ptcp.m}-to \\ 
\glt `She noticed a good-looking young man [who had] just grown a beard and [had] just hit puberty.'  \hfill[KŞ.74]
\z 

The second strategy is to repeat the ezafe linker for each adjective added to the noun phrase:
\ea
\textit{lîbasêwî temîsî aqiɫaneş kerdne jenî.} \\ 
\gll lîbas-êw-î temîs-î aqiɫane=ş kerd{-\O}=ne jenî \\
dress\textsc{.m-indf-ez.attr} clean\textsc{.m-ez.attr} presentable\textsc{.m=3sg:A} do.\textsc{pst-3sg:O=povb} woman \\
\glt `He put a clean, presentable dress on the woman.' \hfill[SH.174]
\z 

Possessors follow the head noun via the genitive ezafe\is{genitive ezafe} linker \textit{-û}. As seen in \ref{sect:ez.case}, in cases where multiple possessors are in a chain, the ezafe interacts with case marking. In (\ref{ex.possessor3}), with the possessor consisting of a coordinated phrase, the oblique occurs on the second coordinate noun. 
\ea 
\textit{jenû şuwaney pêkya.} \\ 
\gll jen(î)-û şuwane-î pêkya \\ 
 woman\textsc{.f-ez.gen} shepherd\textsc{.m-sg.obl} smite\textsc{.pst.pass.3sg:S} \\ 
\glt `They aimed at the shepherd’s wife.' \hfill[KŞ.20]
\z 
 
\ea 
\textit{î menteqew ême kemûkuřîn.} \\ 
\gll î menteqe-û ême kemûkořî=n \\ 
 \textsc{dem.prox} region\textsc{-ez.gen} \textsc{1pl} poverty\textsc{.m}\textsc{=cop.3sg.m:S} \\ 
\glt `Our region is in poverty.' \hfill[JM.19]
\z 

\ea
\textit{desûratû tow î dêwî} \\
\gll desûrat-û to=û î dêw-î \\
affair-\textsc{ez.gen} \textsc{2sg=}and \textsc{dem.prox} ogre-\textsc{obl.m} \\
\glt `the affair of you and the ogre' \hfill[KT.103] \label{ex.possessor3}
\z

When the possessor is modified by an adjective, the case marking appears on the adjective at the right edge of the noun phrase. 
\ea 
\textit{maço wêrega ehmeđe dizey be ehmeđe dizey pîrey.} \\
\gll m-aç-o wêrega ehmeđ-e diz-e-î be \textbf{ehmeđ-e} \textbf{diz-e-î} \textbf{pîr-e-î} \\
 \textsc{ind}.say.\textsc{prs-3sg:A} evening \textsc{pn-ez.cmpd} thief-\textsc{def-m.sg.obl} to \textsc{pn-ez.cmpd} thief-\textsc{def-ez.attr} old\textsc{-def-m.sg.obl} \\
\glt `In the evening, (the young) Ehmed the Thief told the Old Ehmed the Thief.' \hfill[ED.85]
\z 

While the possessor is generally realised NP-internally, it is noticeable that in limited cases, a bound possessor may appear NP-externally. This is especially the case when the head noun is a body part. In the following examples, the \textsc{3sg} bound possessor of \textit{çem} and \textit{diɫ} is realised as a verbal person affix \textit{-\O}.\footnote{Similar external possession\is{external possession} constructions occur in the neighbouring Laki\il{Laki}, where under certain conditions, a \textsc{3sg} bound possessor is realised NP-externally \citep{mohammadirad_epclaki_2024}.}
\ea
\textit{dêwekey heyçêwî çem pene nekewt.} \\
\gll dêw-eke-î heyç-êw-î \textbf{çem} pene ne-kewt-\O \\
ogre-\textsc{def-m.sg.obl} nothing-\textsc{indf-m.sg.obl} \textbf{eye}\textsc{.m} to \textsc{neg-}fall.\textsc{pst-\textbf{3sg.m:S/PSR}} \\
\glt `The ogre, his eyes didn't fall on anything.' \hfill[KT.137]
\z

\ea
\textit{paşaw teqyandosî diɫ şî pene.} \\
\gll paşa-û teqyandos-î diɫ şî-{\O} pene \\
king-\textsc{ez.gen} \textsc{pn-m.sg.obl} heart go.\textsc{pst-3sg.m:S/R} to \\
\glt `The king of Decius became fond of him [lit. his heart went to him].' \\ \hfill[ÇH.49]
\z 

Overall, the basic word order\is{word order} within the NP is DEM NUM N ADJ POSS REL, as suggested by the examples below.  
\ea 
\textit{a duwe kuře ʕalew emîrî} \\ 
\gll a duwe kuř-e ʕal-e-û emîr-î \\ 
\textsc{dem.dist} two son\textsc{-ez.cmpd} good-\textsc{def-ez.gen} \textsc{pn-m.sg.obl}\\ 
\glt `those two good sons of Emir' \label{np.wordorder1}
\z 

\ea 
\textit{a direxte gewrê hencîrê} \\ 
\gll a dirext-e gewr(e)-ê hencîr(î)-ê\\ 
\textsc{dem.dist} tree\textsc{-ez.cmpd} big-\textsc{pl.dir} fig\textsc{-pl.dir}\\ 
\glt `those big fig trees' 
\z 

\section{Agreement in the noun phrase} \label{sect:np-agr}

The only features relevant for agreement within the noun phrase are number and gender. The typical agreeing elements within the noun phrase are adjectives, quantifiers\is{quantifiers}, classifiers\is{classifiers}, and numerals\is{numerals} `two' and `three'. Descriptive adjectives agree in gender\is{gender agreement} (only in the singular\is{singular}) and number\is{number agreement} with the head noun (see \S\ref{adjective-infl}). Examples:
\ea 
\textit{kuřê cuwanxas}\\ 
\gll kuř-ê cuwanxas-{\O} \\ 
 boy\textsc{.m-indf} good\_looking\textsc{-m} \\ 
\glt `a good-looking boy' \hfill[KŞ.68]
\z 

\ea 
\textit{jenê xase} \\
\gll jen(î)-ê xas-e \\
wife.\textsc{f-indf} nice-\textsc{f}\\
\glt `a nice wife’  \hfill[JH.64] \\
\z 

\ea 
\textit{însanê bêʕeqɫê} \\ 
\gll însan-ê bêʕeqɫ-ê \\ 
 human\textsc{.m}\textsc{-dir}\textsc{.pl} silly\textsc{\textsc{-pl}}  \\ 
\glt `silly men' \hfill[ŞC.28]
\z  

 Classifiers\is{classifiers} and measure nouns\is{measure nouns} agree in number\is{number agreement} with a preceding numeral in the structure of the noun phrase, as shown by \textit{dane} (\ref{ex.clf.agreement1}), \textit{ferde} (\ref{ex.clf.agreement2}) and \textit{bine} (\ref{ex.clf.agreement3}) in the following examples.
\ea 
\textit{yerê danê hêɫê maro.} \\ 
\gll yerê dan(e)-ê hêɫ(e)-ê m-ar-o \\ 
 three \textsc{clf}\textsc{-pl.dir} egg\textsc{.m-pl.dir} \textsc{ind-}bring\textsc{.prs-3sg:A} \\ 
\glt `She brought three eggs.' \hfill[JH.81] \label{ex.clf.agreement1}
\z 

\ea 
\textit{duwê ferdê hardî} \\ 
\gll duwê ferd(e)-ê hardî \\ 
 two sack\textsc{-pl.dir} flour \\ 
\glt `two sacks of flour' \hfill[HB.61] \label{ex.clf.agreement2}
\z 

\ea 
\textit{çwar binê darê} \\ 
\gll çwar bin(e)-ê dar-ê \\ 
 four \textsc{clf-pl.dir} tree\textsc{-pl.dir} \\ 
\glt `four trees' \label{ex.clf.agreement3}
\z 

Some head nouns do not carry number agreement\is{number agreement} triggered by the numeral and measure nouns\is{measure nouns}. In such cases, the verb may agree with the singular\is{singular} head noun. 
\ea 
\textit{duwê serê heywan} \\ 
\gll duwê ser-ê heywan \\ 
 two \textsc{clf}\textsc{-pl.dir} animal\textsc{.m} \\ 
\glt `two animals' \label{ex.clf.agreement4}
\z 

\ea
\textit{wîs koɫê lokeşa ard.} \\
\gll wîs koɫ(e)-ê loke=şa ard-{\O} \\
twenty load-\textsc{pl.dir} cotton=\textsc{3pl:A} bring.\textsc{pst-3sg.m:O} \\
\glt `They brought twenty loads of cotton.' \hfill[ME.120]
\z 

Numerals\is{numerals} `two' and `three' have, by default, plural\is{plural} inflection, ending in the direct plural\is{direct case} suffix \textit{-ê}. 
\ea 
\gll yerê kinaçê \\ 
 three daughter\textsc{.pl.dir} \\ 
\glt `three daughters' \hfill[JH.20]
\z 

When combined with demonstratives, and when used as a genitive in a possessive construction, they are used in their bare\is{bare} form, as implied in \textit{duwe-gîyan} `pregnant' (lit. two souls) (see \S\ref{sect:cardinal-numerals}).
 \ea 
\textit{î duwe nefere} \\ 
\gll î duwe nefer=e \\ 
 \textsc{dem.prox} two person\textsc{.m=dem} \\ 
\glt `these two persons' \hfill[BP.150]
\z

\ea 
\textit{dehfew yerey} \\ 
\gll dehfe-û yere-î\\ 
 time\textsc{.f-ez.gen} three\textsc{-m.sg.obl}  \\ 
\glt `the third time' \hfill[KŞ.27]
\z 

When a numeral occurs in a possessive construction, both the possessor and the possessed exhibit number\is{number} inflection.
\ea 
\textit{çil serê zayfê} \\
\textit{çil ser(e)-ê zayfê} \\
forty head-\textsc{pl.dir} girl\textsc{.pl.dir} \\
\glt `forty girl-heads' \hfill[ME.134]
\z


\section{Syntax of definiteness}
\largerpage
The definite suffix\is{definite suffix} \textit{-eke} and its variants are used in Tekht Hewramî (see \S\ref{Definiteness}). However, it is not used with all nouns with identifiable referents. Rather, once a noun has been identified with a definite status, it is no longer necessary to mark it with the definite suffix\is{definite suffix}.\footnote{This is similar to the definiteness\is{definiteness} system of Central Kurdish\il{Kurdish!Central} \citep[see][]{opengin_mukri_2016}{}{}} This means that bare nouns\is{bare noun} can have a definite reading, as discussed in \S\ref{sect: Bare_nouns}. Therefore, the use of the definite marker is different from what we see in languages like English\il{English} and French\il{French}, where the definite article tends to be regularly used with nouns with definite reference. For example, let us consider the coding of \textit{kuř} `son, boy' in text ZQ of the corpus \citep[][]{mohammadirad_speking_the_past}: 

\begin{tabular}{ll}
1st mention:& \textit{kuř-ê} `a boy' \\
2nd mention:& \textit{kuř-eke} (\textsc{def}) \\
3rd mention:& \textit{kuř} (\textsc{def}) \\
4th mention:& \textit{kuř-eke} (\textsc{def}) \\
5th mention:& \textit{î kuř-e} `this boy' (\textsc{def}) \\
6th mention:& \textit{kuř} (\textsc{def}) \\
7th mention:&\textit{kuř-eke} (\textsc{def}) \\
8th mention:& \textit{kuř } (\textsc{def}) \\
\end{tabular}

As can be seen, \textit{-eke} is not regularly used with all the mentions of the referent \textit{kuř} `son, boy'. It tends to recur in discourse whenever the referent's identifiability is at risk. For example, in the third mention, \textit{kuř} is an afterthought and its referent has already been activated in discourse; therefore, it does not need to be definite-marked. 
\ea 
\textit{kuřeke, yoşa ..., weşkewte bê kuř.} \\ 
\gll kuř-eke yo=şa weş kewte b-ê kuř \\ 
 son\textsc{.m-def.m.sg.dir} one\textsc{.m=3pl:PSR} good fall\textsc{.pst.ptcp.m} be\textsc{.prs-aug.3sg:S} son\textsc{.m} \\ 
\glt `The boy, one of them ..., well he was a healthy boy.' \hfill [ZQ.17]
\z 

In the fourth mention, the referent has been absent in the previous discourse for some time. The use of \textit{-eke} helps reactivate the reintroduction of the referent into the discourse. 

The suffix \textit{-eke} has some of the typical functions listed for definite articles cross-linguistically, including marking anaphoric definiteness\is{anaphoric definiteness}, bridging\is{bridging definiteness} definiteness, and contextually unique referents (for cross-linguistic functions of definite articles see \citealt[][]{becker_articles_2021}{}{}). Examples (\ref{ex.def.anaph-1})--(\ref{ex.def.anaph-3}) feature the use of \textit{-eke} in anaphoric\is{anaphoric definiteness} contexts: in these contexts the definite suffix\is{definite suffix} identifies a referent which has been mentioned in the preceding discourse.
\ea
\textit{kuřêş ba. kuřekêş germîyanne nimêniş çene.} \\ 
\gll \textbf{kuř-ê}=ş b-a \textbf{kuř-ekê}=ş germîyan=ne nim(e)-ê-nê=iş çene \\ 
 son\textsc{.m-pl.dir=3sg:NC} be\textsc{.prs.ind-3pl:S} son\textsc{.m-def.pl.dir=3sg:PSR} \textsc{pn=post} \textsc{neg.ind-}come\textsc{.prs-3pl:S=3sg:R} with \\ 
\glt `He had sons. His sons did not accompany him [; they stayed in] Garmiyan.' \hfill [ZB.9]  \label{ex.def.anaph-1}\\
\z

\ea
\textit{zawɫêweş dî ca ʕemrew xwayş kerde. a wextî zawɫeke peya bî.} \\ 
\gll \textbf{zawɫ(e)-êwe=ş} dî-∅ ca ʕemre-û xwa-î=ş kerd-e a wext-î \textbf{zawɫe-(e)ke} peya bî-∅\\ 
 child\textsc{-indf=3sg:A} see\textsc{.pst-3sg.m:O} afterwards order\textsc{.f-ez.gen} God\textsc{.m-sg.obl=3sg:A} do\textsc{.pst-3sg.f:O} \textsc{dem.dist} time\textsc{.m}\textsc{-obl}\textsc{.m} child\textsc{.m-def.m.sg.dir} visible be\textsc{.pst}\textsc{-3sg}\textsc{.m:S}\\ 
\glt `She gave birth to a child before she passed away. Well, the child was born.' \hfill [KŞ.22]--[KŞ.23] \label{ex.def.anaph-2}
\z 
 
\newpage
\ea
\textit{gêɫanê qiřoɫe darêm yoso. berdim nîyamne qiřoɫe dareke.} \\ 
\gll gêɫa-(a)nê \textbf{qiřoɫ-e} \textbf{dar-ê}=m yos-∅=o berd-∅=im nîya-∅=m=ne \textbf{qiřoɫ-e} \textbf{dar-eke} \\ 
wander\textsc{.pst-1sg:S} hollow\textsc{.m-ez.cmpd} tree\textsc{.m-indf=1sg:A} find\textsc{.pst-3sg.m:O=compl} take\textsc{.pst-3sg.m:O=1sg:A} put\textsc{.pst-3sg:O=1sg:A=povb} hollow\textsc{.m-ez.cmpd} tree\textsc{.m-def.m.sg.dir} \\ 
\glt `I wandered around [and] found a tree hollow. I took [him] and put him in the tree hollow.' \hfill [ZQ.23]--[ZQ.24] \label{ex.def.anaph-3}
\z 

\textit{-eke} is also used in bridging\is{bridging definiteness} contexts, where a new nominal referent is taken as having a definite reference due to its association with a previously mentioned referent. In (\ref{ex.def.bridg-2}), \textit{aşpezxane} `kitchen' is definite-marked because it is linked to the preceding referent \textit{yane} `house'.
\ea 
\textit{yewaşê çage mêwe yane. milo wextê miđyo witênê. miđyo witênê girđ. ađîç gêɫo aşpezxanekene.} \\ 
\gll yewaşê ç=age m-ê=we \textbf{yane} mi-l-o wext-ê mi-đy(e)-o witê=nê mi-đy(e)-o witê=nê girđ ađ=îç gêɫ-o \textbf{aşpezxan(e)-eke}=ne\\ 
then in=there \textsc{ind-}come\textsc{.prs.3sg:S=compl} house\textsc{.m.sg.dir} \textsc{ind-}go\textsc{.prs-3sg:S} time\textsc{.m-indf} \textsc{ind-}look\textsc{.prs-3sg:S} sleep\textsc{.pst.ptcp.pl=cop.3pl:S} \textsc{ind-}look\textsc{.prs-3sg:S} sleep\textsc{.pst.ptcp.pl=cop.3pl:S} all \textsc{3sg.m.dir=add} wander\textsc{.prs.ind-3sg:S} kitchen\textsc{.m-def.m.sg.dir=post}  \\ 
\glt `[From there], he (the shepherd’s son) returned home. He realised that they (the people in the king’s palace) were sleeping. He realised that they were all sleeping. He looked into the kitchen.' \hfill [KŞ.60]--[KŞ.63] \label{ex.def.bridg-2}
\z

\largerpage
In (\ref{ex.def.bridg-1}), \textit{yagê} `place' is definite-marked due to its association with a wall built around the area where one of the tale's characters disappears. 
\ea
\textit{qetarleyê tewenîş hesar pey kêşanî. dîyarîş kera yagekê.} \\ 
\gll qetarle-ê tewenî=ş \textbf{hesar} pey kêşa=nî dîyarî=ş ker-a \textbf{yag(ê)-ekê}\\ 
 small\_wall\textsc{-indf} stone\textsc{.f=3sg:A} wall\textsc{.m} for pull\textsc{.pst.ptcp.m=cop.2sg:O} mark\textsc{=3sg:O} do\textsc{.prs.ind-3pl:A} place\textsc{.f-def.f.sg} \\ 
\glt `They made a small wall made of stone for him. They marked the place.' \\\hfill [BP.199]--[BP.200] \label{ex.def.bridg-1}
\z 

The definite suffix\is{definite suffix} \textit{-eke} also marks referents that are unique in a given context. According to \citet[79]{becker_articles_2021}{}: ``Contextually unique referents are mutually and unambiguously identifiable because they are constructed as unique in a larger context based on (general) knowledge shared by the speaker and the hearer. The discourse referent does not necessarily have to be familiar or visible to the discourse participants. It is rather constructed as the only salient referent of its kind in a given context.'' In (\ref{ex.def.unique}), one of the characters in the story returns to a village and asks the people to go to the \textit{aweđanî} `oasis' -- marked by the definite suffix\is{definite suffix} -- close to their village.
\ea
\textit{watim, `beydê dilû aweđanîyekey!'} \\ 
\gll wat=im b-e-îdê dil-û \textbf{aweđanî-ekey} \\ 
 say\textsc{.pst=1sg:A} \textsc{imp-}come\textsc{.prs-2pl:S} inside\textsc{-ez.gen} oasis\textsc{.m-def.m.sg.obl} \\  
\glt `I said, `Come to the oasis!’' \hfill [ZQ.48]  \label{ex.def.unique}\\
\z

In short, it can be said that the definite marker in Hewramî expresses both `anaphoric definites' and `non-anaphoric definites', to use the terminology in \citet[]{dryer2014competing}.
\textit{-eke} and its variants cannot generally be added to a nominal modified by a demonstrative. In the rare cases where this happens, the nominal coded by \textit{-eke} is in contrast with another nominal.  
\ea
\textit{mila ew kuřekey yoyşa bera. aneşa zilterû ʕalter bo ađî bera.} \\ 
\gll mi-l-a \textbf{ew} \textbf{kuř-ekey} yo-î=şa ber-a ane=şa zil-ter=û ʕal-ter b-o ađî ber-a \\ 
 \textsc{ind-}go\textsc{.prs-3pl:S} \textsc{dem.dist} son\textsc{.m-def.m.sg.obl} one\textsc{.m-sg.obl=3pl:PSR} take\textsc{.prs.ind-3pl:A} \textsc{dem.dist.m.3sg.dir=3pl:PSR} big\textsc{-cmpr}=and good\textsc{-cmpr} be\textsc{.prs.ind-3sg:S} \textsc{3sg.obl.m} take\textsc{.prs.ind-3pl:A} \\ 
\glt `They went away [and took] that son. They took one of them (i.e., the boys), the one who was bigger and healthier; they took him.' \hfill [ZB.40]  \\
\z

The definite suffix\is{definite suffix} \textit{-eke} may occur with the adverbial \textit{a wextî} `well, then' (lit. that time). It combines with certain conjunctions, resulting in heavily coded conjunctions such as \textit{egerkete} `if' and \textit{emanekete} `but'. The use of the definite suffix\is{definite suffix} on the conjunctions may suggest that the definite suffix\is{definite suffix} has some discourse management properties outside the nominal reference system. 


\ea 
\textit{a wextîyekey pîr bîyena.} \\ 
\gll \textbf{a} \textbf{wext-î-ekey} pîr bîye=na \\ 
 \textsc{dem.dist} time\textsc{.m-sg.obl-def.m.sg.obl} old be\textsc{.pst.ptcp.m=cop.1sg:S} \\ 
\glt `Well, then, I have aged.' \hfill [JM.10]  \\
\z 

\ea 
\textit{egerketekey serawê fire gewre fire gewre fire gewre bîyen.} \\ 
\gll \textbf{egerket(e)-ekey} seraw-ê fire gewre fire gewre fire gewre bîye=n \\ 
 if\textsc{-def.m.sg.obl} large\_area\textsc{-indf} very big very big very big be\textsc{.pst.ptcp.m=cop.3sg.m:S} \\ 
\glt `Imagine there has been a large area of land [under his control].' \hfill [DG.9]  \\
\z

In contrast to \textit{-eke}, the use of the definite suffix\is{definite suffix} \textit{-e} seems to be limited to principal characters in discourse. For instance, \textit{her} `donkey', as one of the protagonists in text HB of the main text corpus, takes the definite\is{definite suffix} \textit{-e}. The conditions behind the use of \textit{-eke} vs. \textit{-e} remain a topic for future research.
\ea 
\textit{pase herey wat ...} \\ 
\gll pase her-e-î wat \\ 
 like donkey\textsc{.m-def-m.sg.obl} say\textsc{.pst} \\ 
\glt `As the donkey said ...' \hfill [HB.54]
\z 

\section{Syntax of indefiniteness}
This section lays out the use of indefinite suffixes\is{indefinite suffix} (see \S\ref{Indefiniteness} for forms) in different functional contexts. The indefinite suffix\is{indefinite suffix} may express nominals with specific reference to the speaker. In (\ref{ex.indf-funct1}), the indefinite marked noun\is{indefinite suffix} is individuated and has some discourse salience.
\ea 
\textit{maço, `seweqet bû, min birayêwim henû îne waʕzemne.'} \\ 
\gll m-aç-o seweqe=t b-û min \textbf{bira-êw}=im hen-∅=û îne waʕze=m=ne \\ 
\textsc{ind-}say\textsc{.prs-3sg:A} sacrifice\textsc{=2sg:PSR} be\textsc{.prs.ind-3sg:S} \textsc{1sg} brother\textsc{.m-indf=1sg:NC} \textsc{exist-3sg.m:S}=and \textsc{dem.prox.m.3sg.dir} situation\textsc{.f=1sg:NC=\textsc{cop.\textsc{3sg.f}}} \\ 
\glt `He said [to Sheikh Aladin], `May I be your sacrifice! I have a brother; my situation is like this.' \hfill [DG.34] \label{ex.indf-funct1}
\z 

The indefinite suffix\is{indefinite suffix} may attach to specific nominals which do not play a salient role in the discourse but are individuated. 

\newpage
\ea 
\textit{şûnû de neferekeyre her çemçêşa bîyen.} \\ 
\gll şûn-û de nefer-ekey=re her \textbf{çemç(e)-ê}=şa bîye=n \\ 
 after\textsc{-ez.gen} ten person\textsc{.m-def.m.sg.obl=post} only spoon\textsc{.m-indf=3pl:NC} be\textsc{.pst.ptcp.m=cop.3sg.m:S} \\ 
\glt `Then, there was only one spoon for the ten people.' \hfill [JE.28]
\z 

On the other hand, as seen in \S\ref{sect: Bare_nouns}, non-specific indefinite nouns with generic referents are not marked with the indefinite suffix\is{indefinite suffix}. Rather, the bare noun\is{bare noun} is used. 
\ea 
\textit{êtir yo xeleş bîyen.} \\ 
\gll êtir yo xele=ş bîye=n\\ 
 \textsc{disc.ptcl} one\textsc{.m} grain\textsc{.m.sg.dir=3sg:NC} be\textsc{.pst.ptcp.m=cop.3sg.m:S} \\ 
\glt `One [person] had grains [as an agricultural product].' \hfill [JE.30]
\z 

\ea 
\textit{yo bancanîş bîyêne.} \\ 
\gll yo bancanî=ş bîyê=ne \\ 
one\textsc{.m} tomato\textsc{.f.sg=3sg:NC} be\textsc{.pst.ptcp.f=cop.3sg.f:S} \\ 
\glt `One [person] had tomatoes.' \hfill [JE.31]
\z 
 
The indefinite suffix\is{indefinite suffix} appears with temporal adverbials referring to a specific time. 
\ea 
\textit{eça nîşore. şewê waranê waro.} \\ 
\gll e=ça nîş-o=re şew(e)-ê waran-ê {} war-o \\ 
 in=there sit\textsc{.prs.ind-3sg:S=povb} night\textsc{-indf} rain\textsc{-indf} {} rain\textsc{.prs.ind-3sg:S} \\ 
\glt `They settled there (i.e., in that region). One night, it was raining.' \hfill [ZB.10]
\z

The indefinite suffix\is{indefinite suffix} may also appear with a temporal adverbial, which refers to a non-specific time in the past. 
\ea 
\textit{řowê luwan nîştênêre sere gawêşa warden.} \\ 
\gll řo-ê luwa=n nîştê=nê=re sere gaw-ê=şa warde=n \\ 
 day\textsc{.m-indf} go\textsc{.pst.ptcp.m=cop.3sg.m:S} sit\textsc{.pst.ptcp.pl=cop.3pl:S=povb} head\textsc{.m} bull\textsc{.m-indf=3pl:A} eat\textsc{.pst.ptcp.m=cop.3sg.m:O} \\  
\glt `One day, they (the bridegroom’s) family would go (to the bride's family), sit (with the bride's family) and eat a cow.' \hfill [RE.13]
\z 

The plural\is{plural} suffix in the direct case\is{direct case} can be used indefinitely. In (\ref{ex.uns.indf}), \textit{hewar} has indefinite, non-specific reference.
\ea 
\textit{hewarêşa wişkinɛnê.} \\ 
\gll \textbf{hewar-ê}=şa wişkinɛ=nê \\ 
 summer\_habitat\textsc{.m-pl.dir=3pl:A} scour\textsc{.pst.ptcp.pl=cop.3pl:O} \\ 
\glt `They scoured the summer habitats [searching for food etc.].' \hfill [JE.3] \label{ex.uns.indf}
\z 

\section{Quantifiers\is{quantifiers}} \label{sect:quantifiers}
\subsection{girđ `all'}
This quantifier is used with plural\is{plural} and singular\is{singular} nouns and has the sense of `all'. It may be used independently (\ref{ex.gird1}) or attributively, linked to a nominal by the ezafe linker (\ref{ex.gird2})--(\ref{ex.gird3}). 
\ea 
\textit{ħeywanê miđa wer. girđ wizaş weɫê.} \\ 
\gll ħeywan-ê mi-đ(e)-a wer girđ wiz-a=ş weɫê \\ 
 animal\textsc{.m-pl.dir} \textsc{ind-}give\textsc{.prs-3pl:A} out all throw\textsc{.prs.ind-3pl:A=3sg:O} front \\ 
\glt `They herded the animals out. They drove all [of them] forth.' \hfill[ZB.31] \label{ex.gird1}
\z 

\ea 
\textit{girđû şarezûrî êsaɫ meraʕetû ême kero.} \\ 
\gll girđ-û şarezûr-î êsaɫ meraʕet-û ême ker-o \\ 
 all\textsc{-ez.gen} \textsc{pn-m.sg.obl} this\_year care\textsc{.m-ez.gen} \textsc{1pl} do\textsc{.prs.ind-3sg:A} \\ 
\glt `This year, all [the people in] Sharazour care for us.' \hfill[PM.38] \label{ex.gird2}
\z 

\ea 
\textit{a zemane kuweyt ce girđû kêşwereka dewɫetmenter bê.} \\ 
\gll a zeman=e kuweyt ce girđ-û kêşwer-eka dewɫetmen-ter b-ê \\ 
 \textsc{dem.dist} time\textsc{.m=dem} \textsc{pn} from all\textsc{-ez.gen} country\textsc{-def.pl.obl} rich\textsc{-cmpr} be\textsc{.prs-aug.3sg:S} \\ 
\glt `Back then, Kuwait was the richest of all countries.' \hfill[JM.43] \label{ex.gird3}
\z 

\subsection{her `each, every'}
The particle \textit{her} is used with singular\is{singular} nouns in the sense of `each, every' (\ref{ex.her.1}). It may also be used in the sense of `only, just' (\ref{ex.her.2}) and as a phasal aspect marker `still' (\ref{ex.her.3}).
\ea 
\textit{yanew yo terî, her şewê her wêregane î mêmanê mêmanû yoy bîyênê.} \\ 
\gll yane-û yo ter-î her şew(e)-ê her wêrega=ne î mêman-ê mêman-û yo-î bîyê=nê \\ 
 house\textsc{.m-ez.gen} one\textsc{.m} other\textsc{-m.sg.obl} each night\textsc{-f.sg.obl} each evening\textsc{=post} \textsc{dem.prox} guest\textsc{.m-pl.dir} guest\textsc{.m-ez.gen} one\textsc{.m-sg.obl} be\textsc{.pst.ptcp.pl=cop.3pl:S} \\ 
\glt `[Likewise, one tax collector] would be the guest in the house of one [fellow from Hewraman] each evening, each night.' \hfill[BP.43] \label{ex.her.1}
\z

\ea 
\textit{şûnû de neferekeyre her çemçêşa bîyen.} \\ 
\gll şûn-û de nefer-ekey=re her çemç(e)-ê=şa bîye=n \\ 
 after\textsc{-ez.gen} ten person\textsc{.m-def.m.sg.obl=post} only spoon\textsc{.m-indf=3pl:NC} be\textsc{.pst.ptcp.m=cop.3sg.m:S} \\ 
\glt `Then, there was only one spoon for the ten people.' \hfill[JE.28] \label{ex.her.2}
\z 

 \ea 
\textit{çil şewê payîz şîyen ca koçşa kerdeno. her hewarne bîyênê.} \\ 
\gll çil şew(e)-ê payîz şîye=n ca koç=şa kerde=n=o her hewar=ne bîyê=nê\\ 
 forty night\textsc{.f-pl.dir} autumn go\textsc{.pst.ptcp.m=cop.3sg.m:S} afterwards migration\textsc{.m.sg.dir=3pl:A} do\textsc{.pst.ptcp.m=cop.3sg.m:O=compl} still summer\_habitat\textsc{.m=post} be\textsc{.ptcp.pl=cop.3pl:S} \\ 
\glt `They were still at the summer habitat until 40 days after autumn began.' \\ \hfill[JE.9]--[JE.10] \label{ex.her.3}
\z 
 
\textit{her} is frequently used as an emphatic particle, including in clauses with negated verbs.
\ea 
\textit{î kuře her memeş warden; her memeş warden.} \\ 
\gll î kuř=e her meme=ş warde=n her meme=ş warde=n \\ 
 \textsc{dem.prox} boy\textsc{.m=dem} \textsc{emph} breast\textsc{.m.sg.dir=3sg:A} eat\textsc{.pst.ptcp.m=cop.3sg.m:O} \textsc{emph} breast\textsc{.m.sg.dir=3sg:A} eat\textsc{.pst.ptcp.m=cop.3sg.m:O} \\  
\glt `The boy had kept suckling [from the tree]; he had kept suckling.' \\\hfill[ZQ.44]
\z 

\ea 
\textit{xo a wextî wêçiş yaneş her mebo.} \\ 
\gll xo a wext-î wê=ç=iş yane=ş her me-b-o \\ 
 \textsc{dis.ptcl} \textsc{dem.dist} time\textsc{.m-sg.obl} \textsc{refl=add=3sg:PSR} house\textsc{.m=3sg:NC} \textsc{emph} \textsc{neg.ind-}be\textsc{.prs-3sg:S} \\ 
\glt `He had no house at that time.' \hfill[JP.219]
\z 

\subsection{çinnê `some'}
The particle is cognate with Kurdish\il{Kurdish} \textit{çen}. It may be used independently (\ref{ex.çin1}), where it functions as a question particle where the form \textit{çinne} is used. It may also appear attributively, in which case it takes the form \textit{çinnê} `some' containing \textit{çinne} plus the direct plural\is{direct case} suffix \textit{-ê} (\ref{ex.çin2})--(\ref{ex.çin3}).
\ea
\textit{mareyî çinne bê?}\\
\gll mareyî çinne b-ê\\
marriage\_portion how\_much be.\textsc{prs-aug.3sg:S}\\
\glt `How much was the marriage portion?' \label{ex.çin1}
\z 

\ea 
\textit{par ya çinnê saɫê çêweɫter luwaymê pey saraɫû êranî, koy san.} \\ 
\gll par ya çinnê saɫ(e)-ê çêweɫter luwa-îmê pey saraɫ-û êran-î ko-î san \\ 
 last\_year or some\textsc{.pl} year\textsc{.f-pl.dir} ago go\textsc{.pst-1pl:S} to \textsc{pn-ez.gen} \textsc{pn-m.sg.obl} mountain\textsc{-ez.attr} \textsc{pn} \\ 
\glt `Last year or some years ago, we went to the Saral region in Iran, to the San mountain.' \label{ex.çin2}\hfill[ZQ.9]
\z 

\ea 
\textit{ađ maço, `konê î meʕmûrê mine çinnê wextên?'} \\ 
\gll ađ m-aç-o ko=nê î meʕmûr-ê min=e çinnê wext-ê=n \\ 
 \textsc{3sg.m.dir} \textsc{ind-}say\textsc{.prs-3sg:A} where\textsc{=cop.3pl:S} \textsc{dem.prox} officer\textsc{.m-pl.dir} \textsc{1sg=dem} some\textsc{.pl} time\textsc{.m-pl.dir=cop.3sg.m:S} \\  
\glt `He (the agha) says, `Where have my officers been during this time?’' \label{ex.çin3} \hfill[BP.57]
\z 

When preceded by the particle \textit{her}, the singular\is{singular} form \textit{çenne} is used. 

\newpage
\ea 
\textit{her çinne mangê bo. her çinne wextê bo.} \\ 
\gll her çinne mang(e)-ê b-o her çinne wext-ê b-o \\ 
 each some month\textsc{.f-pl.dir} be\textsc{.prs.ind-3sg:S} each some time\textsc{.m-pl.dir} be\textsc{.prs.ind-3sg:S} \\ 
\glt `It took [them] some months. It took them some time.' \hfill[BP.149]
\z 

\subsection{beʕzê `some'}
The particle \textit{beʕzê} is derived from Arabic\il{Arabic} \textit{baʕḍ} `some' and the the direct plural\is{direct case} suffix. 
\ea 
\textit{beʕzê hêzimê maro mêwe}.\\ 
\gll beʕzê hêzm(î)-ê m-ar-o m-ê=we \\ 
 some firewood\textsc{.f-pl.dir} \textsc{ind-}bring\textsc{.prs-3sg:A} \textsc{ind-}come\textsc{.prs.3sg:S=compl} \\ 
\glt `He would take some firewood and return [home].' \hfill[ZP.13]
\z 
\subsection{kuçê `a bit, few'}
The particle is derived from the adjective \textit{kuç} `small' and the indefinite particle \textit{-ê}. It is used in the sense of `a bit, few'. 
\ea 
\textit{kuçêş qupyo.} \\ 
\gll kuç-ê=ş qupy(e)-o \\ 
 little\textsc{-indf=3sg:PSR} misshape\textsc{.prs.ind-3sg:S} \\ 
\glt `It (the prison wall) gave way a bit.' \hfill[BP.171]
\z 

\ea 
\textit{kuçêş çene werû î masî.} \\ 
\gll kuç-ê=ş çene wer-û î mas-î \\ 
 little\textsc{-indf=3sg:R} with eat\textsc{.prs.ind-1sg:A} \textsc{dem.prox} yoghurt\textsc{.m-sg.obl} \\ 
\glt `I shall eat a little of it, [of] this yoghurt.' \hfill[JH.42]
\z 

In the following example, \textit{kuç} has been used as an adjective modifier in the noun phrase.
\ea 
\textit{awîre kuçêşa nîyɛnêre.} \\ 
\gll awîr-e kuç-ê=şa nîyɛ=nê=re \\ 
 fire\textsc{.m-ez.cmpd} little\textsc{-pl.dir=3pl:A} put\textsc{.pst.ptcp.pl=cop.3pl:O=povb}\\  
\glt `[They would] start small fires.' \hfill[JE.37]
\z 

\subsection{ter `other, another'}
The particle \textit{ter} can mean `the other of the two' or `another' or `next, additional'. It behaves like an adjective and can essentially inflect for gender\is{gender} and number\is{number}:
\ea 
\textit{wiɫatêwe terma hen ew dîmew ême.} \\ 
\gll \textbf{wiɫat-êwe} \textbf{ter}=ma hen-∅ \\ 
 country\textsc{.m-indf} other\textsc{=1pl:NC} \textsc{exist-3sg.m:S} \\ 
\glt `There is \textbf{another country} on the other side of us.' \hfill[JH.6]
\z 

\ea
\textit{qisêwe tere kere.} \\
\gll qis(ê)-êwe \textbf{ter-e} ker-e. \\
talk.\textsc{f-indf} another-\textsc{f} do.\textsc{prs.imp-2sg:A} \\
\glt `Say \textbf{another word}!' \hfill[KK.47]
\z 

\ea 
\textit{êtir îse weʕzû hewramanî ta menteqê terê kuçê xastera.} \\ 
\gll êtir îse weʕz(e)-û hewraman-î ta \textbf{menteq(e)-ê} \textbf{ter-ê} kuç-ê xas-ter=a \\ 
 \textsc{disc.ptcl} now situation\textsc{.f-ez.gen} \textsc{pn-m.sg.obl} until region\textsc{-pl.dir} other\textsc{-pl.dir} little\textsc{-indf} good\textsc{-cmpr=cop.3sg.m:S} \\ 
\glt `The situation in Hewraman is a bit better than in \textbf{other regions}.' \hfill[JM.55]
\z

In the following examples, the particle inflects for case\is{case}. 
\ea 
\textit{be beşepîyakew yo terî} \\ 
\gll be beşepîya-(e)ke-û \textbf{yo} \textbf{ter-î} \\ 
 to quota\textsc{.m-def.m.sg.dir-ez.gen} one\textsc{.m} another\textsc{-m.sg.obl}  \\ 
\glt `to the quota of another one (lord)' \hfill[BP.18]
\z 

\ea 
\textit{yanew yo terî} \\ 
\gll yane-û \textbf{yo} \textbf{ter-î}\\ 
 house\textsc{.m-ez.gen} one\textsc{.m} other\textsc{-m.sg.obl} \\ 
\glt `[in] the house of another one' \hfill[BP.43]
\z 

Notably, the particle is on its way to losing nominal and adjectival inflection in such a way that the masculine\is{masculine} form \textit{ter} is used in place of other inflected forms. 
\ea 
\textit{waçê, `dey luwe yanew wêt nîşere ta duwê mangê ter yerê mangê ter.'} \\ 
\gll waç-ê dey lu-e yane-û wê=t n\stackunder[-10pt]{\^{i}}{\'{}}ş-e=re ta \textbf{duwê} \textbf{mang(e)-ê} \textbf{ter} \textbf{yerê} \textbf{mang(e)-ê} \textbf{ter} \\ 
 say\textsc{.prs-aug.3sg:A} \textsc{disc.ptcl} go.\textsc{prs.imp-2sg:S} house\textsc{.m-ez.gen} \textsc{refl=2sg:PSR} sit\textsc{.prs.imp-2sg:S=povb} until two month\textsc{.f-pl.dir} another three month\textsc{.f-pl.dir} another \\  
\glt `They (the family of the girl) would say [to the boy], ‘Go back home [and] wait [lit. sit] for \textbf{the next two or three months}.’' \hfill[JE.82]
\z 

\ea 
\textit{saɫê ter mênêwe.} \\ 
\gll \textbf{saɫ(e)-ê} \textbf{ter} m-ê-nê=we \\ 
year\textsc{.f-indf} other \textsc{ind-}come\textsc{.prs-3pl:S=compl} \\
\glt `\textbf{The following year}, they went back.' \hfill[ZB.46]
\z 

\end{sloppypar}
