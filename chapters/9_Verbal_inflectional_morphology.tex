\chapter{Verbal inflectional morphology and verbal categories}
\section{Verbal inflection: tense-aspect-mood and person} \label{section-vinfmorph}
Present and past stems combine with inflectional person suffixes and modal prefixes to yield several tense-aspect-mood categories. The indicative and subjunctive prefixes are among the TAM prefixes expressing mood. However, the majority of verbs have no overt marking of mood distinction, in which case the difference between indicative and subjunctive verb form is carried out by stress; the verb stem in the subjunctive mood carries stress: \textit{kéro} `that he/she does' vs. \textit{kero} `he/she does' (see \S \ref{verb-stress}). Table \ref{tab:tam-neg_prefix} summarises TAM/negation prefixes:
\begin{table}

    \begin{tabular}{llll}
    \lsptoprule
Tense& \multicolumn{2}{c}{Aspect-mood prefix}&Negative \\
\midrule
            & {indicative}   &  \textit{m-}                & \textit{mé-}, \textit{nimé-} \\
 {Present}  &  {subjunctive} &  \textit{∅}-, \textit{bi-} & \textit{ne-} \\
             &  {imperative} &  \textit{bi-}               &  {\textit{me-}, \textit{ne-}, \textit{nimé-} } \\
\cmidrule(lr){1-1}\cmidrule(lr){2-3}\cmidrule(lr){4-4}
    & perfective, imperfect,                        &  \textit{∅}-&  {\textit{ne-}} \\
Past& perfect\is{perfect}, past                     & \textit{∅}-&  {\textit{ne-}} \\
    & perfect, past conditional\is{past conditional} &\textit{∅}-&  {\textit{ne-}}  \\
\lspbottomrule
    \end{tabular}
    \caption{TAM and negation prefixes}
    \label{tab:tam-neg_prefix}
\end{table}

As can be seen, there are two negation forms for the indicative. The form \textit{mé-} originates from the deletion of \textit{n(í)-} in \textit{n(í)-me} through cluster reduction, with the stress later shifting to \textit{mé-} (see \citealt{karim_demorphologization_nodate}{}). The resultant form \textit{mé-} is thus originally an indicative prefix which happened to carry the stress of the negative form and later remorphologised as the negation marker.

The indicative form \textit{m-} is the result of pretonic shortening of \textit{me-} before stressed syllables: *me-z-ó > \textit{mi-zó } `she gives birth'. The variant \textit{me-} is now morphologised as the negation marker and is evident in the negative form \textit{nime-}.

\subsection{Indicative m(i)-} \label{indicative m-}
The unstressed indicative prefix\textit{ mi-/m-} expresses general present and future tense in verbs derived from the present tense. It is likely related to the Old Iranian *hama-aiwa- `same duration, time’ \citep[26]{windfuhr_dialectology_2009}. The indicative form has a full form \textit{me-} which is reconstructable in the negative of indicative (see above). 

\textit{m-} is, however, only prefixed to a few conjugated verbs. There are thus two classes of present stem verbs regarding the use of the prefix \textit{m-}:\footnote{See \citet[]{karim_imperfective_inreview}{} for the development of indicative and subjunctive prefixes in conservative Hewramî\il{Hewramî} varieties.} Class I consists of the majority of the verbs to which the indicative \textit{m-} is not prefixed; class II verbs take the indicative prefix \textit{m-}. For class II verbs, phonological factors seem to be the main trigger for using the prefix \textit{m-}: verb stems starting with a vowel take the indicative particle. Examples:

\TabPositions{2cm,6cm}
\ea
\textit{m-aç-û}\tab [\textsc{ind}-say.\textsc{prs-1sg:A}]\tab  `I say’ \\
\textit{m-ar-û}\tab \tab `I bring’ \\
\textit{m-az-û}\tab \tab `I let’ \\
\textit{m-êş-û}\tab \tab `I hurt (\textsc{intr})’ \\
\textit{m-êj-û}\tab \tab `I am valued’ \\
\textit{m-êjye-w-re}\tab \tab `I lie down’ \\
\textit{m-ûs-û}\tab \tab `I sleep’ \\
\textit{m-e-w}\tab \tab `I come’ \\
\z

Similarly, the verb stems consisting of only a consonant take the indicative prefix. An exception is the verb \textit{bîyey} `be, become' (\textsc{prs} \textit{b-}; \textsc{pst} \textit{bî-}), which does not take the prefix, seemingly due to assimilation in the bilabial feature.

\ea
\textit{mi-ş-î}\tab [\textsc{ind}-go.\textsc{prs-2sg:S}]\tab  `you go’ \\
\textit{mi-l-î}\tab \tab `you go’\\
\textit{mi-z-o}\tab \tab `she gives birth’ \\
\z

For the rest of the verbs, the syllable structure seems to be the main factor triggering the occurrence of the indicative prefix. Verbs with a CV structure take the \textit{m-} prefix.
\TabPositions{2cm,7cm}
\ea
\textit{mi-ge-w}\tab [\textsc{ind}-copulate.\textsc{prs-1sg:A}] \tab `I copulate’ \\
\textit{mi-đe-w}\tab \tab  `I give’ \\
\z

Similarly, verb stems with CCV structure take the \textit{m-} prefix. The second consonant in the stem of these verbs is a glide. The addition of the \textit{m-} results in the resyllabification of CCV to CVC.CV.
\TabPositions{2cm,6cm}
\ea
\textit{mi-đye-w}\tab[\textsc{ind-}look\textsc{.prs-1sg:S}]\tab`I look’ \\
\textit{mi-nye-w}\tab\tab`I put’ \\
\textit{mi-zye-w}\tab\tab`I arrive’ \\
\z

Verbs with the syllable structure CCVC split in taking the \textit{m-} prefix. The addition of indicative prefix results in the resyllabification of the sequence whereby the verb’s CCVC syllable structure is restructured as CVC.CVC.
\TabPositions{2cm,6cm,9cm}
\ea
\textit{mi-nvîs-û}\tab [\textsc{ind}-write.\textsc{prs-1sg:A}]\tab< cf. \textit{nvîs-}\tab`I write’ \\
\textit{mi-jnew-û} \tab\tab< cf.\textit{ jnew-}\tab`I hear’ \\
\textit{mi-jnas-û} \tab\tab< cf. \textit{jnas-}\tab`I know’ \\
\z

Exceptions include the following:

\ea
\textit{jmar-û}\tab[count.\textsc{prs-1sg:A}]\tab`I count’ \\
\textit{vreş-û}\tab[sell.\textsc{prs-1sg:A}]\tab `I sell'\\
\z

The majority of the stems based on the present tense do not take the indicative prefix. These verbs fall into different morphosemantic and phonological classes. Some generalisations can be made here: 

The indicative prefix is omitted before verbs with CVC structure. The initial consonant is usually a strong consonant. The reason could be that the \textit{m-} prefix undergoes phonetic attrition in consonant clusters, e.g., \textit{*m-ker-o < kero} `he/she does'.

\ea
\textit{piř-o}\tab [fly.\textsc{prs-3sg:S}]\tab`It flies’ \\
\textit{ker-o}\tab\tab`he/she does’ \\
\textit{çin-o}\tab\tab`he/she weaves/plucks’ \\
\textit{saw-o}\tab\tab`he/she rubs’ \\
\textit{şan-o}\tab\tab`he/she scatters/sows’ \\
\textit{xiz-o}\tab\tab`he/she slips’ \\
\textit{xiw-o}\tab\tab`he/she laughs' \\
\textit{doş-o}\tab\tab`he/she milks’ \\
\textit{zan-o}\tab\tab`he/she knows’ \\
\textit{jîw-o}\tab\tab`he/she lives’ \\
\textit{şêl-o}\tab\tab`he/she rubs/presses’ \\
\textit{wêç-o}\tab\tab `he/she sieves' \\
\z

An exception is the verb \textit{sanay} `to buy’, which for some speakers appears with the indicative \textit{m-}:

\ea
\textit{sana}/ \textit{mi-sana}\tab`they buy’ \\
\z

Similarly, verb stems with CVCC structure do not take the \textit{m-} prefix.

\ea
\textit{perm-o}\tab [rely\_on.\textsc{prs-3sg:A}]\tab `he/she relies on' \\
\textit{pers-o}\tab\tab`he/she asks’ \\
\textit{ters-o}\tab\tab`he/she is afraid’ \\
\z

This category also includes verbs with the agentive suffix \textit{-n}:

\ea
\textit{kiřn-o}\tab [scratch.\textsc{prs-3sg:A}] \tab`he/she scratches’ \\
\textit{wuřn-o}\tab\tab`he/she destroys’ \\
\textit{gêɫn-o}\tab\tab`he/she narrates’ \\
\textit{řemn-o}\tab\tab`he/she makes run’ \\
\textit{şokn-o}\tab\tab`he/she shakes (\textsc{tr})’ \\
\z

The indicative prefix does not occur with verbs of sound emission. This includes the verbs that take the agentive suffix \textit{-n} and those that do not.

\ea
\textit{pirxn-o}\tab [snore.\textsc{prs-3sg:A}]\tab`he/she snores’ \\
\textit{qêřn-o}\tab\tab`he/she shouts’ \\
\textit{seř-o}\tab\tab`it brays’ \\
\textit{girew-o}\tab\tab`he/she cries’ \\
\textit{bařî(e)-o꞊we}\tab\tab`it bloats’ \\
\textit{pijm-o}\tab\tab`he/she sneezes’ \\
\z

Unaccusative verbs with added detransitiviser suffix \textit{-ye} do not take the indicative suffix.

\ea
\textit{meřy(e)-o}\tab [break.\textsc{prs-3sg:S}]\tab `it breaks’ \\
\textit{şêwy(e)-o}\tab\tab`he/she gets confused’ \\
\textit{qomy(e)-o}\tab\tab`it happens’ \\
\textit{fařy(e)-o}\tab\tab`it changes’ \\
\textit{fîsy(e)-o}\tab\tab`it overflows’ \\
\z

The same usage is seen with light verb constructions\is{light verb constructions}. The indicative prefix appears only with the light verbs\is{light verb} fulfilling one of the conditions above:

\ea
\textit{tûşiş mê.} \\
\gll tûş꞊iş m-ê \\
accident꞊\textsc{3sg:PSR} \textsc{ind}-come.\textsc{prs.3sg:S} \\
\glt `He ran into him.’ < \textit{tûş amay} `to run into, to get into trouble’ \hfill[JP.132] 
\z 
vs. 

\ea
\textit{tateş fewt kero.} \\ 
\gll tate=ş fewt ker-o \\ 
 father\textsc{.m=3sg:PSR} death\textsc{.m} do\textsc{.prs.ind-3sg:A} \\ 
\glt `His father passed away.' < \textit{fewt kerđey} `to pass away’ \hfill[JP.8]
\z
\subsection{Subjunctive b(i)-} \label{sect:sbjv}
In verbs derived from the present stem, the subjunctive mood is expressed by the prefix \textit{b(i)-}. The prefix \textit{bi-} and its cognates mark the subjunctive and imperative moods in modern West Iranian languages. The subjunctive was without any TAM marker in Middle Iranian, with a different inflection than the indicative mood. Only in the early Iranian period did \textit{b(i)-} come to develop a TAM marker, which also functions as an imperative \citep[]{napiorkowska_convergence_2015}{}. 

The prefix is originally stressed in line with other Iranian languages. However, for the majority of verbs which take the prefix, the stress shifts to the stem, as shown by the phonemic analysis of \textit{bzánmê} `that we know' [RE.51], see Figure \ref{fig:unstressed_sbjv}. The stress shift is blocked before stems starting with a consonant cluster; see the phonemic analysis of \textit{bíjnasû} `that I know' in Figure \ref{fig:stressed_sbjv}.

\begin{figure}
  \begin{subfigure}[b]{0.5\textwidth}
    \includegraphics[width=\textwidth]{figures/text_bizanme_let-us_see.png}
    \caption{Unstressed \textsc{sbjv} `that we know'}
    \label{fig:unstressed_sbjv}
  \end{subfigure}\begin{subfigure}[b]{0.5\textwidth}
    \includegraphics[width=\textwidth]{figures/Text_bijnasu.png}
    \caption{Stressed \textsc{sbjv} `that I know'}
    \label{fig:stressed_sbjv}
  \end{subfigure}
  \caption{The stress pattern prefixed subjunctive stems}
\end{figure}

The subjunctive prefix occurs with a small class of verbs. The limited use of the subjunctive prefix parallels the restricted use of the indicative prefix seen in \S\ref{indicative m-}. There are thus two main classes of verbs regarding the subjunctive prefix: class I verbs (making up around 95\%) not taking the prefix, and class II verbs (around 5\%) taking the prefix. A small set of verb stems may appear with or without the subjunctive prefix (see below). As seen in Figures \ref{fig:unstressed_sbjv} and \ref{fig:stressed_sbjv}, the subjunctive prefix may be stress-bearing depending on the syllable structure of the stem. By contrast, in verb stems featuring no subjunctive prefix, the stress has shifted onto the first vowel of the stem, resulting in the phonemic stress\is{phonemic stress} placement \S\ref{verb-stress}. Historically, the stress shifts onto the stem following the loss of the subjunctive prefix \citep[]{karim_demorphologization_nodate}. Figure \ref{fig:bero1} exhibits the stress placement of the prefix-less subjunctive stem \textit{ber-o} `that he takes' [HB.11].
\begin{figure}

    \includegraphics[width=.6\textwidth]{figures/text_bero_that_he_takes.png}
    \caption{The stress pattern of prefix-less subjunctive stem \textit{ber-} `take'}
    \label{fig:bero1}
\end{figure}

The same morphophonological conditions as the indicative \textit{m-} seem to trigger the use of the subjunctive prefix. Therefore, subjunctive prefixes appear with consonant-only and vowel-initial stems. An exception is the verb \textit{bîyey} `be, become' (\textsc{prs} \textit{b-}; \textsc{pst} \textit{bî-}), which does not take the prefix.
\TabPositions{2cm,6cm,9.5cm}
\ea
\textit{b-\stackunder[-10pt]{\^{u}}{\'{}}s-û}\tab[\textsc{sbjv}-sleep.\textsc{prs-1sg:S}]\tab`that I sleep’\tab[BP.182] \\
\textit{bí-l-o}\tab[\textsc{sbjv}-go.\textsc{prs-3sg:S}]\tab`that he goes’\tab[DG.54] \\
\textit{b-ár-o}\tab[\textsc{sbjv}-bring\textsc{.prs-3sg:A}]\tab`that he brings’\tab[HB.14] \\
\textit{b-\stackunder[-10pt]{\^{e}}{\'{}}-nê}\tab[\textsc{sbjv}-come.\textsc{prs-3pl:S}]\tab`that they come’\tab[JE.78] \\
\textit{b-ó} \tab [be\textsc{.prs.sbjv-3sg:S}] \tab`that she be'\tab [KŞ.17] \\
\z

The following is a sample of verbs which appear with the subjunctive prefix in the text corpus:

\ea
\textit{bí-spar-o}\tab[\textsc{sbjv}-hand\_over\textsc{-3sg:A}]\tab`that he hands over’\tab[ŞC.89] \\
\textit{bí-nye-ûne}\tab[\textsc{sbjv}-set\_up.\textsc{prs-1sg:A}]\tab`that I set up’\tab[JH.13] \\
\textit{bi-sán-a}\tab[\textsc{sbjv}-buy.\textsc{prs-3pl:A}]\tab`that they buy’\tab[BP.58] \\
\textit{bí-zy-a}\tab[\textsc{sbjv}-grow.\textsc{prs-3pl:S}]\tab`that they grow’\tab[JP.27] \\
\z

The following list is a sample of verbs from the text corpus that appear with no subjunctive TAM prefix. 
\TabPositions{2cm,6cm,9.75cm}
\ea
\textit{wér-o}\tab[eat.\textsc{prs.sbjv-3sg:A}]\tab`that they eat’\tab[PM.36] \\
\textit{kér-o}\tab[do\textsc{.prs.sbjv-3sg:A}]\tab`that he does’\tab[ZP.111] \\
\textit{bér-o}\tab[take.\textsc{prs.sbjv-3sg:A}]\tab`that he takes’\tab[RE.52] \\
\textit{kúş-î}\tab[kill.\textsc{prs.sbjv-3sg:A}]\tab`that you kill’\tab[KŞ.98] \\
\textit{sáw-î}\tab[rub.\textsc{prs.sbjv-2sg:A}]\tab`that you rub’\tab[DG.47] \\
\textit{dón꞊mê}\tab[talk\_to.\textsc{prs.sbjv-1pl:A}]\tab`that we talk to’\tab[JE.76] \\
\textit{táw-o}\tab[read.\textsc{prs.sbjv-3sg:A}]\tab`that he/she can’\tab[JM.61] \\
\z


The verb \textit{zanay} `to know’ appears both with and without the subjunctive prefix, though the latter is more common:

\ea
\textit{bi-zan-mê}\tab`that we know’\tab[JP.64] \\
\textit{zan-a}\tab`that they know’\tab[JP.67] \\
\z

In light verb constructions\is{light verb constructions}, the subjunctive prefix is largely absent. The light verb\is{light verb} \textit{day} `give' exhibits variation in taking the subjunctive prefix (see \ref{ex.19e}-\ref{ex.19f}).
\TabPositions{2.5cm,6.5cm}
\ea \label{ex.sbjv}
\ea \textit{dagîr ker-o}\tab`that he occupies'\tab[DP.18] \\
\ex \textit{dawa ker-o}\tab`that he claims'\tab[DP.18] \\
\ex \textit{wey ker-û}\tab `that I take care of'\tab [JP.226] \\
\ex \textit{peket gin-o}\tab`that you get worried' \tab[JP.234] \\
\ex \textit{zerer de-yme}\tab`that we harm'\tab[PM.5] \label{ex.19e}\\
\ex \textit{tefre bi-đ(e)-o}\tab`that he avoids'\tab[KŞ.105] \label{ex.19f}\\
\z
\z  
\subsection{Imperative b-} \label{section.imp}
As stated, the prefix \textit{b-} also expresses the imperative mood. Like marking the subjunctive mood, the imperative \textit{b-} is not regularised. Phonological factors trigger the use of the imperative \textit{b-}:
\TabPositions{2cm,6.5cm,9.5cm}
\ea
\textit{b-ár-e}\tab[\textsc{imp}-bring.\textsc{prs-2sg:A}]\tab`Bring!'\tab[JH.50] \\
\textit{bí-zn-e=re}\tab[\textsc{imp}-take.\textsc{prs-2sg:A=post}]\tab`Take out!'\tab[KŞ.96] \\
\textit{bí-nye}\tab[\textsc{imp}-put.\textsc{prs-2sg:A}]\tab`Put!'\tab [KŞ.96] \\
\textit{bí-nye-ydê}\tab[\textsc{imp}-put.\textsc{prs-2pl:A}]\tab`You (\textsc{pl}) Put!'\tab[BP.186] \\
\z

The verb `give' takes the indicative prefix \textit{m-}, but in the imperative appears with no \textit{b-} prefix, see (\ref{give_imp-z1})--(\ref{give_imp-z2}). This may suggest that the subjunctive/imperative prefix has a more restricted use than the indicative \textit{m-}.

\ea \label{give_imp-z1}
\textit{î kinaçête de pî kuřîme.} \\ 
\gll î kinaçê=t=e dé-(e) p=î kuř-î=m=e \\
 \textsc{dem.prox} daughter\textsc{.f.sg=2sg:PSR=dem} give\textsc{.prs.imp-2sg:A} to=\textsc{dem.prox} son\textsc{.m-sg.obl=1sg:PSR=dem} \\ 
\glt `Give your daughter to my son [in marriage].' \hfill[RE.7]
\z 


\ea \label{give_imp-z2}
\textit{deş!} \\ 
\gll dé-(e)=ş \\
give\textsc{.prs.imp-2sg:A=3sg:O} \\ 
\glt `Give her!' \hfill[RE.55] \\
\z 

As with the subjunctive prefix, the imperative prefix does not appear with most imperative verb forms, including with light verb constructions\is{light verb constructions}.

\TabPositions{24mm,80mm,102mm}
\ea
\textit{çin-e}          \tab [set\_up.\textsc{prs.imp-2sg:A}]    \tab `Set up!'   \hfill [JH.25]{} \\
\textit{g\stackunder[-10pt]{\^{i}}{\'{}}sn-e}\tab [start.\textsc{prs.imp-2sg:A}]  \tab `Start!' \hfill [JH.27] \\
\textit{bér-e}          \tab [take.\textsc{prs.imp-2sg}]         \tab `Take!'     \hfill [HB.62] \\
\textit{kér-e}          \tab [\textsc do\textsc{.prs.imp-2sg:A}] \tab `Do!'       \hfill [RE.20] \\
\textit{táş-e}          \tab [shave.\textsc{prs.imp-2sg:A}]      \tab `Shave!'    \hfill [JH.53] \\
\textit{wáç-e}          \tab [tell.\textsc{prs.imp-2sg:A}]       \tab `Say!'      \hfill [JE.74] \\
\textit{kúj-dê}         \tab [kill.\textsc{prs.imp-2pl:A}]       \tab `You (\textsc{pl}) kill!' \hfill [KŞ.72] \\
\textit{zinî kér-e}     \tab [saddle do\textsc{.prs.imp-2sg:A}]  \tab `Saddle!'   \hfill [ŞC.52] \\
\textit{ser=ma dé-ydê}  \tab [head\textsc{.m}\textsc{=1pl:R} give\textsc{.prs.imp}\textsc{-2pl:A}] \tab `Pay us a visit!' \\
    \hfill [DG.68]\\
\z

\subsection{Negation prefixes with present tense verbs} \label{neg-prs}
\begin{figure}[b]
    \includegraphics[width=.5\textwidth]{figures/text_negimpf_nime.png}
    \caption{The stress pattern of \textit{nimé-}}
    \label{fig:nimestress}
\end{figure}

Tekht H.\il{Hewramî!Tekht} has a number of negative prefixes, listed above in Table \ref{tab:tam-neg_prefix}. The negation prefixes for present indicative\is{present indicative} are \textit{mé-} and \textit{nimé-}. Of these, \textit{mé-} is the general negation form: it expresses the negation of consonant-initial verbs and verbs starting with a high vowel, e.g., \textit{û}. The form \textit{nimé-} occurs only with a handful of vowel-initial verbs, including verbs whose initial vowel is \textit{a}, \textit{e}, and \textit{ê}. The stressed vowel in \textit{nimé-} merges with the vowel-initial verbs. Still, the stress is retrievable in that position, as suggested by the phonemic analysis of \textit{nimáza} `they don't let' in Figure \ref{fig:nimestress}.



\ea  \label{ex.neg.form1}
    \ea
        \ea
            \gll m-az-a\\
                 \textsc{ind}-let.\textsc{prs-3pl:A}\\
            \glt `they let’\\
        \ex
            \gll     nim(e)-az-a\\
                    \textsc{neg.ind}-let.\textsc{prs-3pl:A} \\
            \glt    `they do not let’  \\
        \z
    \ex
        \ea
            \gll    \textit{m-ar-û}\\
                     \textsc{ind}-bring.\textsc{prs-1sg:A}\\
            \glt   `I bring’
        \ex
           \gll      \textit{nim(e)-ar-û} \\
                     \textsc{neg.ind}-bring.\textsc{prs-1sg:A} \\
            \glt     `I do not bring’  \\
        \z
    \ex
        \ea
            \gll \textit{m-e-w}\\
                 \textsc{ind}-come.\textsc{prs-1sg:S}\\
            \glt `I come’
        \ex
            \gll    \textit{nim(e)-e-w} \\
                    \textsc{neg.ind}-come.\textsc{prs-1sg:S} \\
            \glt   `I do not come’  \\
        \z
    \ex
        \ea
            \gll  {m-ûs-û}\\
                     \textsc{ind}-sleep.\textsc{prs-1sg:S}\\
            \glt  `I sleep’
        \ex
            \gll    {me-ws-û} \\
                    \textsc{neg.ind}-sleep.\textsc{prs-1sg:S} \\
            \glt    `I do not sleep’  \\
        \z
    \ex
        \ea
            \gll   {m-êş-o}\\
                    \textsc{ind}-hurt.\textsc{prs-3sg:S}\\
            \glt   `it hurts’
        \ex
            \gll       {nim(e)-êş-o}  \\
                     \textsc{neg.ind}-hurt.\textsc{prs-3sg:S} \\
                    \glt `it does not hurt’  \\
        \ex
            \gll     {me-êş-o} \\
                    \textsc{neg.ind}-hurt\textsc{.prs-3sg:S} \\
                   \glt `it does not hurt’  \\
        \z
    \z
\z








For the rest of the verbs, \textit{me-} is the negator prefix. This includes some verb stems starting with a vowel in their present indicative form (see \ref{ex.neg.form1}d above) and verb stems starting with a consonant, whether they are composed of only one consonant (\ref{ex.neg.form2}a-b) or have a syllabic structure (\ref{ex.neg.form3}a-b).

\ea \label{ex.neg.form2}
 \ea
    \ea
            \gll \textit{mi-l-o}\\
                  \textsc{ind}-go.\textsc{prs-3sg:S}\\
               \glt `he/she goes’
        \ex
            \gll \textit{me-l-o}\\
                 \textsc{neg.ind}-go.\textsc{prs-3sg:S} \\
                  \glt `he/she does not go’
         \z          
    \ex
        \ea
            \gll \textit{mi-ş-o}\\         
                 \textsc{ind}-go.\textsc{prs-3sg:S}\\
                 \glt `it goes’
        \ex
           \gll \textit{me-ş-o}\\
                 \textsc{neg.ind}-go.\textsc{prs-3sg:S} \\ 
                 \glt `it does not go’ \\
        \z 
    \z     
\z    



\ea\label{ex.neg.form3}
\ea
    \ea
            \gll \textit{nîş-û}\\
                  sit.\textsc{prs.ind-1sg:S}\\
               \glt `I sit’
        \ex
            \gll \textit{me-nîş-û}\\
                  \textsc{neg.ind}-sit.\textsc{prs-1sg:S} \\
                  \glt `I do not sit’
         \z          
    \ex
        \ea
            \gll \textit{mi-nye-w=re}\\         
                 \textsc{ind-}put.\textsc{prs.ind-1sg:A=povb}\\
                 \glt `I put down’
        \ex
           \gll \textit{me-nye-w=re}\\
                 \textsc{neg.ind}-put.\textsc{prs.ind-1sg:A=povb} \\ 
                 \glt `I do not put down’  \\
        \z 
    \z     
\z    

In the negation of the present subjunctive\is{present subjunctive}, the subjunctive \textit{b(i)-} is replaced with \textit{né-}. The vowel of the negation prefix merges with the vowel in some vowel-initial stems; see (\ref{ex.neg-sbjv1}).

\ea
\textit{narîşo!} \\
\gll n(e)-ar-î꞊ş꞊o \\
\textsc{neg.sbjv-}bring\textsc{.prs-2sg:A=3sg:O=compl} \\
\glt `May you not bring her back!’ \hfill[ZP.47] \label{ex.neg-sbjv1}
\z 
%

\ea
\textit{newsî!}\\
\gll ne-ûs-î! \\
\textsc{neg.sbjv}-sleep.\textsc{prs-2sg:S} \\
\glt `May you not sleep!’ \label{ex.neg-sbjv2}
\z 
%

\ea
\textit{ême keçê nekero.} \\ 
\gll ême keç-ê ne-ker-o \\ 
\textsc{1pl} crooked\textsc{-pl} \textsc{neg.sbjv-}do\textsc{.prs-3sg:A} \\ 
\glt `He shall not make us crooked.' \hfill[DG.64] 
\z 


The prohibitive is expressed by \textit{me-}, \textit{né-}, and \textit{nimé-} depending on the initial segment of the verb stem. \textit{nimé-} occurs before verbs starting with a low vowel. With verbs starting with a high vowel or a mid vowel, the negative prefix is \textit{mé-}, see (\ref{negation-proh}e-f).
\ea  \label{negation-proh}
    \ea
        \ea
            \gll \textit{m-az-a}\\
                 \textsc{ind}-let.\textsc{prs-3pl:A}\\
            \glt `they let’\\
        \ex
            \gll \textit{nim(e)-áz-e}   \\
                  \textsc{proh}-let.\textsc{prs-2sg:A}  \\
            \glt  `do not let!’     \\
        \z
    \ex
        \ea
            \gll \textit{m-ar-î}   \\
                  \textsc{ind}-bring.\textsc{prs-1sg:A}   \\
            \glt   `You bring’
        \ex
           \gll \textit{nim(e)-ár-e} \\
                 \textsc{proh}-bring.\textsc{prs-2sg:A} \\
            \glt  `do not bring’ \\
        \z
    \ex
        \ea
            \gll \textit{m-e-y}\\
                 \textsc{ind}-come.\textsc{prs-2sg:S}\\
            \glt `You come’
        \ex
            \gll \textit{nim(e)-ó}   \\
                 \textsc{proh}-come.\textsc{prs.2sg:S}   \\
            \glt  `do not come!’  \\
        \z
    \ex
        \ea
            \gll \textit{m-e-ydê} \\
                  \textsc{ind}-come.\textsc{prs-2pl:S} \\
            \glt  `you (\textsc{pl}) come’
        \ex
            \gll \textit{nim(e)-e-ydê}    \\
                  \textsc{proh}-come.\textsc{prs.2pl:S} \\
            \glt `do not come!’ \\
        \z
    \ex
        \ea
            \gll \textit{m-ûs-î}  \\
                 \textsc{ind}-sleep.\textsc{prs-1sg:S}   \\
            \glt `you sleep’  
        \ex
            \gll \textit{mé-ws-e}       \\
                 \textsc{proh}-sleep.\textsc{prs-2sg:S}    \\
             \glt `do not sleep’ \\
        \z
    \ex
        \ea
            \gll \textit{m-êjîye-y=re}    \\
                 \textsc{ind}-lie\_down.\textsc{prs-2sg:S=povb}   \\
                 \glt `you lie down’  \\
        \ex
            \gll \textit{mé-êjîy(e)-e=re}       \\
                 \textsc{proh}-lie\_down.\textsc{prs-2sg:S=povb}    \\
                    \glt `do not lie down’ \\
        \z
    \z
\z



With verbs starting with a consonant, the negative prefix \textit{mé-} is used:

\ea
\textit{mele pey deg{ɛ}!} \\ 
\gll me-l-e pey deg{ɛ} \\ 
 \textsc{proh-}go\textsc{.prs}-\textsc{2sg:S} to village\textsc{.f-obl} \\ 
\glt `Don’t go to the village! \\
\z 
 
The use of \textit{né-} is more restricted. It may occur before the stems starting with \textit{m}, e.g., (\ref{ni-proh1}).

\ea
\textit{nemendêwe eçê!} \\ 
\gll ne-men-dê=we e=çê \\ 
 \textsc{proh-}remain\textsc{.prs}-\textsc{2pl}\textsc{=compl} in=here \\ 
\glt `Don’t stay here!' \hfill[PM.22] \label{ni-proh1}\\
\z 

\textit{nîye-} expresses the negation of enclitic copula (see \S \ref{negative copula}).

\ea
\textit{jenew heçkesî nîyene.} \\
\gll jene-û heçkes-î nîye=ne \\
wife.\textsc{def-ez.gen} no\_one-obl.\textsc{m } \textsc{neg.exist=cop.3sg.f:S} \\
\glt `She is no one's wife.' \hfill[SH.200]
\z 

\subsection{Summary of TAM and negation forms in verbs with present time reference} 
The verb forms with present time reference fall broadly into four classes. In all verb classes, the negation of indicative is identical to the prohibitive, as opposed to the negation of the subjunctive. 

Class 1 features the majority of verbs, as exemplified by the verb \textit{berđey} `take'. Here, indicative, subjunctive, and imperative verb forms are prefix-less. The verbs beginning with \textit{m} in this class exceptionally have the prohibitive prefix \textit{ne-}, see (\ref{ni-proh1}).

Class 2 is specific to verbs with a C(V) structure, except for \textit{bîyey} `be, become' (\textsc{prs} \textit{b-}; \textsc{pst} \textit{bî-}), which belongs to class 1. The verb forms in this class regularly take the TAM prefixes, except occasionally, the imperative prefix is dropped; see (\ref{give_imp-z1})--(\ref{give_imp-z2}). 

Class 3 is limited to low-vowel-initial verbs. Here, the negatives of indicative and prohibitive are \textit{nime-}, unlike the verbs in classes 1 and 2.

Class 4 is limited to high back-vowel and mid-vowel-initial verbs. Like the verbs in class 3, the verb forms in this class feature vowel coalescence of the TAM prefixes with the stem. However, unlike in class 3, the verb forms in this class have their negation forms in \textit{mé-}. An exception is the verb \textit{êşay} `to hurt', for which the negative of the indicative can be expressed by either \textit{mé-} or \textit{nimé-} (see \S\ref{neg-prs}), hence belonging to both class 3 and class 4.

\begin{table}[hbt]

    \begin{tabular}{rllccc}
    \lsptoprule
&& & \textsc{ind} & \textsc{sbjv} & \textsc{imp/proh} \\
\midrule
\multirow{2}{*}{1} & \multirow{2}{*}{\textit{ber-} `take'} & \textsc{aff} & \textit{ber-\stackunder[-10pt]{\^{i}}{\'{}}} & \textit{bér-\^i} & \textit{bér-e} \\
&& \textsc{neg} & \textit{mé-ber-\^i} & \textit{né-ber-\^i} & \textit{mé-ber-e} \\
\midrule
\multirow{2}{*}{2} & \multirow{2}{*}{\textit{de-} `give'} & \textsc{aff} & \textit{mi-đe-\stackunder[-10pt]{\^{i}}{\'{}}} & \textit{bi-đé-\^i} & \textit{(bi)-đ(e)-é} \\
&& \textsc{neg} & \textit{mé-đe-\^i} & \textit{né-đe-\^i} & \textit{mé-đ(e)-e} \\
\midrule
\multirow{2}{*}{3} & \multirow{2}{*}{\textit{az-} `let'} & \textsc{aff} & \textit{m-az-\stackunder[-10pt]{\^{i}}{\'{}}} & \textit{b-áz-\^i} & \textit{b-áz-e} \\
&& \textsc{neg} & \textit{nim(e)-áz-\^i} & \textit{n-áz-\^i} & \textit{nim(e)-áz-e} \\
\midrule
\multirow{2}{*}{4} & \multirow{2}{*}{\textit{ûs-} `sleep'} & \textsc{aff} & \textit{m-\^us-\stackunder[-10pt]{\^{i}}{\'{}}} & \textit{b-\stackunder[-10pt]{\^{u}}{\'{}}s-\^i} & \textit{b-\stackunder[-10pt]{\^{u}}{\'{}}s-e} \\
&& \textsc{neg} & \textit{mé-ws-\^i} & \textit{né-ws-\^i}  & \textit{mé-ws-e} \\
\lspbottomrule
    \end{tabular}
    \caption{Verb classes with present time reference, inflected in \textsc{2sg} }
    \label{tab:text_me1}
\end{table}

\subsection{TAM and negation forms in verbs with past time reference}
The verb forms with a past time reference are characterised by the absence of a specific TAM prefix. the negator \textit{ne-} negates verbs with past time reference, as shown in the following examples. 


\ea Past perfective\is{past perfective}\\
    \ea
            \gll \textit{wat=iş}\\
                  say.\textsc{pst=3sg:A}\\
               \glt `he/she said’
        \ex
            \gll \textit{ne-wat=iş}\\
                  \textsc{neg}-say.\textsc{pst=3sg:A} \\
                  \glt `he/she did not say’
         \z   
\z

\ea Imperfect / past irrealis\\
\ea
\gll \textit{waç-ên-î}\\ 
say.\textsc{prs-aug-2sg:A}\\
\glt `you were saying’/ `you would say'
\ex 
\gll \textit{ne-waç-ên-î} \\
\textsc{neg}-say.\textsc{prs-aug-2sg:A} \\
\glt `you were not saying'/ `you would not say'
\z
\z 

\ea Present perfect\\
\ea 
\gll \textit{wate=n=iş}\\ 
say.\textsc{pst.ptcp.m=cop.3sg.m:O=3sg:A}\\
\glt `he/she has said (it)’
\ex 
\gll \textit{ne-wate=n=iş} \\
\textsc{neg}-say.\textsc{pst.ptcp.m=cop.3sg.m:O=3sg:A} \\
\glt `he/she has not said it’ \\
\z 
\z 

\ea Past perfect\is{past perfect}\\
\ea 
\gll \textit{wate=b-ê=ş}\\
say.\textsc{pst.ptcp.m}=be.\textsc{prs-aug.3sg:O=3sg:A}\\
\glt `he/she had said (it)’\\
\ex 
\gll \textit{ne-wate=b-ê=ş}\\
\textsc{neg}-say.\textsc{pst.ptcp.m}=be.\textsc{prs-aug.3sg:O=3sg:A}\\
\glt `he/she has said it’
\z 
\z 
 
\ea Past conditional\is{past conditional}\\
\ea 
\gll \textit{wat-{ɛ}=t}\\
say.\textsc{pst-cond=2sg:A}\\
\glt `(If) you said!'
\ex 
\gll \textit{ne-wat-{ɛ}=t}\\
\textsc{neg}-say.\textsc{pst-cond=2sg:A}\\
\glt `(If) you didn't say!'
\z 
\z 

 
\ea Past perfect conditional\\
\ea 
\gll \textit{wat-e=bî-{ɛ}=t}\\
say.\textsc{pst.ptcp.m}=be.\textsc{pst-cond=2sg:A}\\
\glt `(If) you had said!'
\ex 
\gll \textit{ne-wate=bî-{ɛ}=t}\\
\textsc{neg}-say.\textsc{pst.ptcp.m}=be.\textsc{pst-cond=2sg:A}\\
\glt `(If) you hadn't said!'
\z 
\z 

\subsection{Bound person/number markers}\label{bound PMs}
\begin{sloppypar}
Tekht H.\il{Hewramî!Tekht} has different paradigms of verbal person/number agreement/anaphora suffixes (and gender agreement\is{gender agreement} in \textsc{3sg} in some paradigms) in its morphosyntax, used with verbal and non-verbal predicates. These paradigms include (i) copula person markers (PMs), (ii) verbal person/number affixes, and (iii) clitic pronouns. Verbal person/number affixes are further divided into two paradigms: one used with verbs derived from the present stem and the other used with verbs derived from the past stem. Table \ref{tab:bound PMs} exhibits the paradigms of person marking. 
 \end{sloppypar}
\begin{table}
    \begin{tabular}{llllll}
    \lsptoprule
& Verbal & Verbal & Copula& Clitic & Imperative \\
& PMs \textsc{prs.}& PMs \textsc{pst.}&PMs & pronouns & suffixes \\
\midrule
\textsc{1sg} & \textit{-û, -ûne}& \textit{-a, -anê}& \textit{=na}& \textit{=m}& \\ \midrule
\textsc{2sg} & \textit{-î, -îne}& \textit{-î}& \textit{=nî}& \textit{=t} & \textit{-e} \\\midrule
\textsc{3sg.m}& \multirow{2}{*}{\textit{-o, -one}}& \textit{-∅}& \textit{=n}, \textit{=a} &\multirow{2}{*}{\textit{=ş}}& \\
\textsc{3sg.f}& & \textit{-e}& \textit{=ne} & & \\\midrule
\textsc{1pl}& \textit{-mê, -îmê} & \textit{-îmê, -mê} & \textit{=nmê }& \textit{=ma} & \textit{-mê, -imdê}\\\midrule
\textsc{2pl}& \textit{-de, -îdê} & \textit{-îdê, -dê}& \textit{=ndê} & \textit{=ta} & \textit{-dê} \\\midrule
\textsc{3pl}& \textit{-a, -an(ê)} & \textit{-ê} & \textit{=nê}& \textit{=şa} &\\
\lspbottomrule
    \end{tabular}
    \caption{Paradigms of person/number and gender marking}
    \label{tab:bound PMs}
\end{table}

To begin with the verbal person/number affixes, the paradigm of present tense verbal affixes is used with TAM categories derived from the present stem of the verbs. The verbal person/number affixes used in this set are stress-taking. They are used to express the subject of present stem verbs. The indexing of the subject argument is obligatory in these contexts, meaning that the endings are agreement markers. The variant forms in the singular\is{singular}, i.e., \textit{-ûne}, \textit{-îne} and \textit{-one}, are phonologically- and morphologically-conditioned variants of \textit{-û}, \textit{-î}, and \textit{-o}, respectively. The heavier forms are used when the endings are followed by the coordinate particle \textit{=û}; contrast (\ref{-u}) and (\ref{-une}).

\ea
\textit{kuçêş çene werû î masî.} \\ 
\gll kuç-ê=ş çene wer-\textbf{û} î mas-î \\ 
 little\textsc{-indf=3sg:R} from eat\textsc{.prs.ind-1sg:A} \textsc{dem.prox} yoghurt\textsc{.m-sg.obl} \\ 
\glt `I shall eat a little of it, [of] this yoghurt.' \hfill[JH.42] \label{-u}
\z 


\ea
\textit{mareş biřûnew.} \\ 
\gll mare=ş biř-\textbf{ûne}=û\\ 
 marriage\textsc{=3sg:O} cut\textsc{.prs.ind-1sg:A}=and \\ 
\glt `I shall marry her [to Pir Shaliyar], and so on.’ \hfill[JP.225] \label{-une}
\z 

The heavy forms tend to occur with verbs of Goal of motion, such as `come' and `go',\footnote{The vowel-only person forms may equally be used with verbs of motion. The choice of the heavier forms may express some discourse property related to the subject.} and occasionally with verbs in the irrealis perfect\is{irrealis perfect}.

\ea
\textit{milone yane.} \\ 
\gll mi-l-\textbf{one} yane \\ 
 \textsc{ind-}go\textsc{.prs-3sg:S} house\textsc{.m.sg.dir} \\ 
\glt `He (the sultan) went inside.' \hfill[JH.76]
\z 


\ea
\textit{herkesî goş darabone} \\ 
\gll herkes-î goş dara=b-\textbf{one} \\ 
 whoever-\textsc{obl.m} ear\textsc{.m} hold\textsc{.pst.ptcp.m}=be\textsc{.prs-3sg:O} \\ 
\glt `Anyone who had listened to him, [had gone away from the city].' \\\hfill[BP.167]
\z 

The variant forms in \textsc{1pl} and \textsc{2pl} are phonologically conditioned allomorphs of \textit{-mê} and \textit{-dê}, respectively. \textit{-îmê} and \textit{-îdê} are used following consonant-only stems (before which there is a word-boundary), e.g., \textit{b-îmê} `that we are', and following stems ending in a vowel, e.g., \textit{mi-đe-ymê} `we give'. \textit{-mê} and \textit{-dê} are used elsewhere: \textit{bi-l-mê} `let us go'; \textit{ker-mê} `that we do'.

The verbal person/number affixes in the past tense are historically the result of the univerbation of the copula base with lexical verbs, hence their close similarity with the copula paradigm. The paradigm of past verbal person/number affixes has not totally lost its clitic origin, as it is not stress-bearing. The verbal affixes in this paradigm are obligatory indices of intransitive subjects and (to a large extent) direct objects (see \S\ref{sect:DOI}). In addition, they are used pronominally to express oblique arguments\is{oblique arguments} such as recipients\is{recipient}, addressees\is{addressee}, possessors, and beneficiaries (see \S\ref{sect:arginx.pst}). 

The variant forms for \textsc{1pl} and \textsc{2pl}, i.e., \textit{-îmê} and \textit{-îdê}, exhibit morphologically conditioned allomorphy\is{morphologically conditioned allomorphy} with \textit{-mê} and \textit{-dê}. The former are used in the structure of past perfective\is{past perfective}, see (\ref{pfv1pl}); the latter are used in the formation of past perfect\is{past perfect} (\ref{prf1pl}):

\ea
\gll wit-\textbf{îmê} \\
sleep.\textsc{pst-1pl:S} \\
\glt `We slept.' \label{pfv1pl}\\
\z 


\ea
\gll \textit{witê=b-ên-\textbf{mê}} \\
sleep\textsc{.pst.ptcp.pl}=be.\textsc{prs-aug-1pl:S} \\
\glt `We had slept.' \label{prf1pl}\\
\z

The copula endings consist of the stem \textit{n} and set 2 verbal person/number suffixes. The stem has zero inflection in the \textsc{3sg.m}. The variant form \textit{-a} is likely to result from the omission of the \textit{n}. Copula endings are obligatory indices of intransitive subjects in non-verbal predicates. 

Clitic pronouns are reflexes of Old Iranian genitive/dative and accusative person clitic sets \citep[]{korn_western_2009}{}. As discussed in \S\ref{sect:clitic-function}, they assume both phrasal (e.g., possessor in an attributive possessive NP, adposition complements) and clausal functions (e.g., indexing past transitive subjects). Their use may be optional or near-obligatory, depending on the function they have. For instance, while they may alternate with free pronouns in indexing possessors, clitic pronouns have near-obligatory use in indexing past transitive subjects (\S\ref{sect:differential-A-indexing}). 

The multifunctionality of clitic pronouns and verbal affix PMs results in a tense-based role-reference inversion\is{role-reference inversion} of bound person markers in transitive clauses. Consider the expression of A and O in (\ref{ex.inversion1}). The same person value may express different clausal arguments according to tense.
\ea \label{ex.inversion1}
\ea
\gll ber\textbf{-î=m} \\
take.\textsc{prs.ind-2sg:A=1sg:O}\\
\glt `You take me.'
\ex
\gll berđ\textbf{-î=m} \\
take.\textsc{pst-2sg:O=1sg:A} \\
\glt `I took you.'
\z
\z

The tense-based role-reference inversion\is{role-reference inversion} of bound person marker in transitive clauses may extend to non-core arguments as well. In (\ref{ex.inversion2}), the same set of person markers expresses A and R triggered by tense.
\ea \label{ex.inversion2}
\ea
\gll m-aç\textbf{-î=m}  pey\\
\textsc{ind}-say.\textsc{prs-2sg:A=1sg:R} to\\
\glt `You say (it) to me.'
\ex
\gll wat\textbf{-î=m} pey\\
say.\textsc{pst-2sg:R=1sg:A}  to\\
\glt `I said (it) to you.'
\z
\z

Imperative suffixes are, for the most part, limited to the second person. As can be seen, the \textsc{2sg} has a different form than the corresponding forms in other paradigms. It is, however, noticeable that the imperative person form \textit{-imdê} (attested in the Tekht varieties of Silên and Nwên) is interpreted as containing `the speaker and more than one addressee\is{addressee}' is differentiated from \textit{-mê} which is interpreted as meaning `the speaker and one addressee'. 

\ea
\textit{pîyakeyç maço tatew kinaçɫa, `da roɫe gîyan bilimdê pey hêzma.'} \\
\gll pîya-(e)ke=îç m-aç-o tate-û kinaçɫa, da roɫe gîyan bi-l-imdê pey hêzm(î)-a \\
man-\textsc{def=add} \textsc{ind}-say.\textsc{prs-3sg:A} father-\textsc{ez.gen} girl.\textsc{dim.pl.obl} \textsc{hort} child dear \textsc{imp}-go.\textsc{prs-1sg+2pl:S} for wood-\textsc{pl.obl} \\
\glt `The man, the girls' father said, `Dear child(ren), let's go for [collecting] woods.’' \hfill[ÇK.46]
\z


\ea
\textit{maço, `dey bilmê şerʕene!'} \\
\gll m-aç-o dey bi-l-mê şerʕe=ne! \\
\textsc{ind}-say.\textsc{prs-3sg:A} \textsc{hort} \textsc{sbjv-}go.\textsc{prs-1pl:S} sharia\_law\textsc{-post}\\
\glt `He (Moses) said [to the snake], `Let us (you and me) go to Sharia law.’' \\\hfill[MR.21]
\z

\section{The copula}\label{sect:copula-morph}
\begin{sloppypar}
As remarked in \S\ref{bound PMs}, the copula paradigm consists of the stem \textit{n} and the inflectional endings from the set 2 verbal person/number suffixes. The copula forms are used for the predication of non-verbal elements, such as nouns, adjectives and prepositional phrases. They are also used in the inflection of certain perfect tenses (see below). The paradigm of the present copula is exhibited in Table \ref{tab:cop.prs}.
\end{sloppypar}
\begin{table}

    \begin{tabular}{llllll}
    \lsptoprule
&Copular endings   \\
\midrule
\textsc{1sg} & \textit{=na} \\
\textsc{2sg} & \textit{=nî}
\textsc{3sg.m}& \textit{=n-∅}, \textit{=a-∅} \\
\textsc{3sg.f}& \textit{=ne} \\
\textsc{1pl}& \textit{=nmê } \\
\textsc{2pl}& \textit{=ndê} \\
\textsc{3pl}& \textit{=nê} \\ 
\lspbottomrule
    \end{tabular}
    \caption{Paradigm of present copula}
    \label{tab:cop.prs}
\end{table}

In the \textsc{3sg}, \textit{=n} is used following vowel-final non-verbal words (see \ref{ex.xas}), and \textit{=a} is used following consonant-final words (see \ref{ex.gewre}). The following examples illustrate the inflection of copula in ascriptive\is{ascriptive} clauses. Recall that adjectives inflect for gender\is{gender} (only in the singular\is{singular}) and number\is{number} (see \S\ref{adjective-infl}).

\TabPositions{1.5cm,4cm,7.5cm}
\ea \textit{xas} `well’\label{ex.xas}\\
\textsc{1sg.m}\tab\textit{xas꞊na}\tab[well.\textsc{m꞊cop.1sg:S}]\tab`I (\textsc{m}) am well’ \\
\textsc{1sg.f}\tab\textit{xase=na}\tab[well.\textsc{f=cop.1sg:S}]\tab `I (\textsc{f}) am well' \\
\textsc{2sg.m}\tab\textit{xas꞊nî}\tab[well.\textsc{m꞊cop.2sg:S}]\tab `you (\textsc{m}) are well'\\
\textsc{2sg.f}\tab\textit{xase=nî}\tab[well.\textsc{f=cop.2sg:S}]\tab `you (\textsc{f}) are well'\\
\textsc{3sg.m}\tab\textit{xas꞊a}\tab[well.\textsc{m=cop.3sg:S}]\tab `he is well'\\
\textsc{3sg.f}\tab\textit{xase꞊ne}\tab[well.\textsc{f=cop.3sg:S}]\tab `she is well'\\
\textsc{1pl}\tab\textit{xasê=nmê}\tab[well.\textsc{pl=1pl:S}]\tab `we are well' \\
\textsc{2Pl}\tab\textit{xasê꞊ndê}\tab[well.\textsc{pl=2pl:S}]\tab `you are well' \\
\textsc{3Pl}\tab\textit{xasê꞊nê}\tab[well.\textsc{pl=3pl:S}]\tab `they are well' \\
\z 

\TabPositions{1.5cm,4cm,7.5cm}
\ea \textit{gewre} `old (of age)’\label{ex.gewre}\\
\textsc{1sg.m}\tab\textit{gewre꞊na}\tab[old.\textsc{m꞊cop.1sg:S}]\tab`I (\textsc{m}) am old’ \\
\textsc{1sg.f}\tab\textit{gewrê=na}\tab[old.\textsc{f=cop.1sg:S}]\tab `I (\textsc{f}) am old' \\
\textsc{2sg.m}\tab\textit{gewre꞊nî}\tab[old.\textsc{m꞊cop.2sg:S}]\tab `you (\textsc{m}) are old'\\
\textsc{2sg.f}\tab\textit{gewrê=nî}\tab[old.\textsc{f=cop.2sg:S}]\tab `you (\textsc{f}) are old'\\
\textsc{3sg.m}\tab\textit{gewre꞊n}\tab[old.\textsc{m=cop.3sg:S}]\tab `he is old'\\
\textsc{3sg.f}\tab\textit{gewrê꞊ne}\tab[old.\textsc{f=cop.3sg:S}]\tab `she is old'\\
\textsc{1pl}\tab\textit{gewrê=nmê}\tab[old.\textsc{pl=1pl:S}]\tab `we are old' \\
\textsc{2Pl}\tab\textit{gewrê꞊ndê}\tab[old.\textsc{pl=2pl:S}]\tab `you are old' \\
\textsc{3Pl}\tab\textit{gewrê꞊nê}\tab[old.\textsc{pl=3pl:S}]\tab `they are old' \\
\z 

\subsection{Negative copula} \label{negative copula}
The negation of the present copula is expressed by adding\textit{ nîye-} to the copula endings (Table \ref{tab:cop-neg}). The negative form of the copula can be considered a truncation of the negation \textit{nî-} and the existential copula, e.g., \textit{nîye=na} <* \textit{nî-he=na} [\textsc{neg-exist=cop.1sg:S}] `I am not'. 
\begin{table}
\begin{tabular}{llllll}
   \lsptoprule
 &Negative copula \\
\midrule
\textsc{1sg} & \textit{nîye=na} \\
\textsc{2sg} & \textit{nîye=nî}
\textsc{3sg.m}& \textit{nîya-∅} \\
\textsc{3sg.f}& \textit{nîye=ne} \\
\textsc{1pl}& \textit{nîye=nmê } \\
\textsc{2pl}& \textit{nîye=ndê} \\
\textsc{3pl}& \textit{nîye=nê} \\  
\lspbottomrule
\end{tabular}
\caption{The negative copula paradigm}
\label{tab:cop-neg}
\end{table}
    
The following examples illustrate the use of negative copula in the text corpus.

\ea
\textit{ħîçê nîya.} \\ 
\gll ħîç-ê nîy(e)=a \\ 
 nothing\textsc{-indf} \textsc{neg.exist=cop.3sg.m:S} \\  
\glt `There is nothing [left].' \hfill[HB.4]
\z

\ea
\textit{hermane be desû min nîyene.} \\ 
\gll hermane be des-û min nîye=ne\\ 
work\textsc{.f} to hand\textsc{.m-ez.gen} \textsc{1sg} \textsc{neg.exist=cop.3sg.f:S} \\ 
\glt `The task [of ruling] is not in my hands.' \hfill[ŞC.86] \\
\z 

Proof for the analysis of \textit{nîye-} as consisting of the negation \textit{nî-} and the existential copula comes from the use of the \textit{nîye-} as the negated predicate in predicative possessive constructions\is{predicative possessive constructions}, which are formed based on the existential copula (see \S\ref{poss-pred}).

\ea
\textit{zeraʕetma nîyaw bencanîma nîyenew xîyarma nîyaw genmîma nîyene.} \\ 
\gll zeraʕet=ma nîy(e)=a=û bencanî=ma nîye=ne=û xîyar=ma nîy(e)=a=û genmî=ma nîye=ne\\ 
agriculture\textsc{.m=1pl:NC} \textsc{neg.exist=cop.3sg.m:S}=and tomato\textsc{.f=1pl:NC} \textsc{neg.exist=cop.3sg.f:S}=and cucumber\textsc{.m=1pl:NC} \textsc{neg.exist=cop.3sg.m:S}=and wheat\textsc{.f=1pl:NC} \textsc{neg.exist=cop.3sg.f:S} \\ 
\glt `We don’t have much agriculture; we don’t have cucumbers; we don’t have tomatoes; we don’t have wheat.' \hfill[PM.37]
\z 

\subsection{Past copula}
There are two paradigms of the past copula; see Table \ref{tab:cop.pst}. One set is formed by combining the past stem of the verb `to be' with verbal person/number affixes from the past tense. The second set uses the present stem of the verb `to be', to which the augment \textit{-ên} is added, and the resulting form is inflected by the same set of past tense verbal person/number suffixes. In the current state of the language, set 2 occurs at a higher rate than set 1. In the literature, the second set has been referred to as the ``imperfect copula", e.g., \citet[551]{mahmoudveysi_hawrami_2018}, primarily because the augment \textit{-ên} which is used in the structure of past imperfective is used in the structure of this past copula as well. 
\begin{table}

    \begin{tabular}{lll}
\lsptoprule
& Set 1& Set 2 \\
\midrule
\textsc{1sg} & \textit{bî-a} & \textit{b-ên-ê}\\ 
\textsc{2sg} & \textit{b(î)-î} & \textit{b-ênî} \\
\textsc{3sg.m}& \textit{bî-∅}& \textit{b-ê}<*\textit{b-ên} \\ 
\textsc{3sg.f}& \textit{bî-e} & \textit{b-ê}<*\textit{b-ên} \\ 
\textsc{1pl}& \textit{bî-mê} & \textit{b-ên-mê}\\ 
\textsc{2pl}& \textit{bî-dê} & \textit{b-ên-dê} \\ 
\textsc{3pl}& \textit{bî-ê\footnotemark}& \textit{b-ên-ê} \\
\lspbottomrule
    \end{tabular}
    \caption{Past copula paradigms}
    \label{tab:cop.pst}
\end{table}\footnotetext{In one case, the form \textit{bî-a} is attested, e.g., \textit{gumê bîya} [ŞC.39] `they got lost', which would indicate syncretism\is{syncretism} between \textsc{1sg} and \textsc{3pl}. However, it is possible that the narrator actually meant `I was lost', which would discard the syncretism\is{syncretism} scenario.}

In the current state of the dialect, set 2 past copula functions as the basic copula. Examples:

\ea
\textit{ême firê bênmê.} \\ 
\gll ême fire-ê b-ên-mê \\ 
 \textsc{1pl} a\_lot\textsc{-pl} be.\textsc{prs-aug-1pl:S} \\  
\glt `We were a large number.' \hfill[BP.110]
\z 


\ea
\textit{min zaroɫe bênê.} \\ 
\gll min zaroɫe b-ên-ê \\ 
 \textsc{1sg} child be\textsc{.prs-aug-1sg:S} \\  
\glt `I was a child.' \hfill[JH.1]
\z 

Set 1 is used in the inflection of light verb constructions\is{light verb constructions} which are based on the light verb\is{light verb} `to be'.

\ea
\textit{luwanê nizîkû qiřoɫekey bîyawe.} \\ 
\gll luwa-anê nizîk-û qiřoɫ-ekey bî-a=we \\ 
 go\textsc{.pst-1sg:S} close\textsc{-ez.gen} hollow\textsc{.m-def.m.sg.obl} be\textsc{.pst-1sg:S=compl} \\ 
\glt `I went [and] got closer to the tree hollow.' \hfill[ZQ.36]
\z 


\ea
\textit{kuřêş peya bî.} \\ 
\gll kuř-ê=ş peya bî-∅ \\ 
 son\textsc{.m-indf=3sg:PSR} visible be\textsc{.pst-3sg.m:S} \\ 
\glt `A boy was born to him.' \hfill[BP.4]
\z

Less frequently, set 1 is used as the past copula, including in cleft constructions (\ref{cleft-cop}).

\ea
\textit{î gîre çêş bî min ward?} \\ 
\gll î gîr=e çêş bî-∅ min ward-∅ \\ 
 \textsc{dem.prox} hook\textsc{.m=dem} what be\textsc{.pst-3sg.m:S} \textsc{1sg} eat\textsc{.pst-3sg.m:O} \\ 
\glt `What is this situation that I am caught in?’ \hfill[HB.23] \label{cleft-cop}
\z 


\ea
\textit{min çen bê ʕeqiɫ bîya!} \\ 
\gll min çen bê-ʕeqiɫ bî-a \\ 
 \textsc{1sg} so\_much without-wisdom be\textsc{.pst-1sg:S} \\ 
\glt `[Look] how stupid I was!’ \hfill[HB.19]
\z 


\ea
\textit{maço, `îne her hêɫe bî’} \\ 
\gll m-aç-o îne her hêɫe bî-∅ \\ 
 \textsc{ind-}say\textsc{.prs-3sg:A} \textsc{dem.prox.m.3sg.dir} just egg\textsc{.m} be\textsc{.pst-3sg.m:S} \\  
\glt `He said, `This one was just an egg.’' \hfill[JH.95]
\z 

\subsection{Existential copula} \label{sect:existential-copula}
The existential copula is formed by adding inflectional suffixes from the past tense to existential base \textit{hen} <* \textit{he + =n}.
\begin{table}

    \begin{tabular}{lll}
    \lsptoprule
& Gloss & Lexical gloss \\
\midrule
\textit{hen-a}& [\textsc{exist-1sg}]& `I am' \\
\textit{hen-î} && `you are' \\
\textit{hen-∅}&& `he is' \\
\textit{hen-e}&& `she is' \\
\textit{hen-mê} && `we are' \\
\textit{hen-dê} && `you are' \\
\textit{hen-ê} && `they are' \\
\lspbottomrule
    \end{tabular}
    \caption{The existential copula paradigm}
    \label{tab:exist_cop}
\end{table}

Examples from the text corpus:

\ea
\textit{pîyawê hen tawo î kinaçê to weşe kerowe.} \\ 
\gll pîya-ê hen-∅ taw-o î kinaçê to weş-e kér-o=we \\
man\textsc{.m-indf} \textsc{exist-3sg.m:S} can\textsc{.prs.ind-3sg:A} \textsc{dem.prox} girl\textsc{.f} \textsc{2sg} well\textsc{-f} do\textsc{.prs.sbjv-3sg:A=compl} \\ 
\glt `There was a man in the Hewraman region who could cure his daughter.' \\\hfill[JP.154] 
\z 


\ea
\textit{ne nan henû ne hardî henê.} \\ 
\gll ne nan hen-∅=û ne hardî hen-ê \\ 
 no bread\textsc{.m.sg.dir} \textsc{exist-3sg.m:S}=and no flour\textsc{.f.pl.dir} \textsc{exist-3pl:S} \\ 
\glt `There is no bread. There is no flour.' \hfill[HB.3]
\z 

 The existential copula can be used nominally, in which case it may appear as the head noun in a nominal phrase.

\ea
\textit{ey henû nîyenû min} \\
\gll ey hen=û nîyen-û min \\
\textsc{voc} being=and non\_being-\textsc{ez.gen} \textsc{1sg} \\
\glt `O my relatives [lit. O my being and non-being]' \hfill[KŞ.56] 
\z 
\subsubsection{Possessive function of existential copula} \label{poss-pred}
The existential copula is used as the predicate in predicative possessive constructions\is{predicative possessive constructions}, where it agrees in gender\is{gender agreement} and number\is{number agreement} with the possessed noun. The possessor is indexed by clitic pronouns.\footnote{See \citet[]{mohammadirad_predicative_2020}{} for an overview of predicative possessive constructions\is{predicative possessive constructions} in Western Iranian languages.}

\ea
\gll hen-∅=im \\
\textsc{exist-3sg.m:S=1sg:NC} \\
\glt `I have it (\textsc{m}).'
\z

\ea
\gll hen-e=m \\
\textsc{exist-3sg.f:S=1sg:NC} \\
\glt `I have it (\textsc{f}).'
\z

\ea
\gll hen-ê=m \\
\textsc{exist-pl:S=1sg:NC} \\
\glt `I have them.'
\z 

If the possessed noun proceeds the existential stem, the possessor-indexing clitic moves onto the possessed NP.

\ea
\textit{min birayêwem hen.} \\
\gll min bira-êwe=m hen-∅ \\
 \textsc{1sg} brother\textsc{.m}\textsc{-indf}\textsc{=1sg:NC} \textsc{exist}\textsc{-3sg}\textsc{.m:S} \\
\glt `I have a brother.' \hfill[DG.34] 
\z 


\ea
\textit{tifengiş henû îneş henû aneş hen!} \\ 
\gll tifeng=iş hen-∅=û îne=ş hen-∅=û ane=ş hen-∅ \\ 
 gun\textsc{=3sg:NC} \textsc{exist-3sg.m:S}=and \textsc{dem.prox.m.3sg.dir=3sg:NC} \textsc{exist-3sg.m:S}=and \textsc{dem.dist.m.3sg.dir=3sg:NC} \textsc{exist-3sg.m:S} \\ 
\glt `He has a gun, he has this, and he has that!' \hfill[ŞC.12]
\z 


\ea
\textit{xway deseɫatê çanêş henê.} \\ 
\gll xwa-î deseɫat-ê çan(e)-ê=ş hen-ê \\ 
 God\textsc{.m}\textsc{-obl}\textsc{.m} power\textsc{.m-pl.dir} such\textsc{-pl}\textsc{=3sg:NC} \textsc{exist}\textsc{-3pl:S} \\ 
\glt `God has such powers [to exert].' \hfill[ZQ.55] 
\z 

The negation of predicate possession is expressed by the negative copula \textit{nîye-}:

\ea
\textit{warîyatma nîya.} \\ 
\gll warîyat=ma nîy(e)=a \\ 
income\textsc{.m=1pl:NC} \textsc{neg.exist=cop.3sg.m:S} \\ 
\glt `We have no income.' \hfill[JM.60]
\z 

\subsubsection{Locational function of the existential copula}
The existential copula stem may function as the predicate in locational copula clauses, in which case it is inflected with the same set of endings presented in Table \ref{tab:exist_cop}.

\ea
\textit{hena yane.} \\ 
\gll hen-a yane \\ 
 \textsc{exist-1sg:S} home\textsc{.m} \\ 
\glt `I am at home.' \label{loc.cop}
\z 

The construction in (\ref{loc.cop}) seems to be a recent innovation in Tekht Hawramî\il{Hewramî!Tekht}. It is absent in the entire corpus, and I only came across it in the speech of a few young speakers. The more common structure for expressing locational copula is to use the deictic particle \textit{îna} (see \S\ref{loc-cop}).

\subsection{Locational copula}\label{loc-cop}
The locational copula is expressed by the locational deictic particle \textit{îna-} to which inflectional suffixes from the past tense are added; see Table \ref{tab:loc_cop}.
\begin{table}

    \begin{tabular}{lll}
    \lsptoprule
& Gloss & Lexical gloss \\
\midrule
\textit{îna-(a)nê}& [\textsc{loc.deic.cop-1sg}]& `I am at/in' \\
\textit{îna-y} && `you are at/in' \\
\textit{îna-∅}&& `he is at/in' \\
\textit{îna-(e)}&& `she is at/in' \\
\textit{îna-ymê} && `we are at/in' \\
\textit{îna-ydê} && `you are at/in' \\
\textit{în(a)-ê} && `they are at/in' \\
\lspbottomrule
    \end{tabular}
    \caption{The locational copula paradigm}
    \label{tab:loc_cop}
\end{table}

The following examples illustrate a sample of the locational copula constructions containing the deictic copula \textit{îna} found in the text corpus:

\ea
\textit{înanê çêge.} \\ 
\gll îna-(a)nê çêge \\ 
 \textsc{loc.deic.cop-1sg:S} here \\ 
\glt `I am here.' \hfill[JP.136]
\z 


\ea
\textit{înaymê fiɫane yagê.} \\ 
\gll îna-îmê fiɫan-e yagê \\ 
 \textsc{loc.deic.cop-1pl:S} such\_and\_such\textsc{-ez.cmpd} place\textsc{.f} \\ 
\glt `We are at such-and-such a place.' \hfill[PM.12]
\z 


\ea
\textit{înê weɫêne.} \\ 
\gll în(a)-ê weɫê=ne \\ 
 \textsc{loc.deic.cop-3pl:S} front\textsc{=post} \\
\glt `(They) were in front.' \hfill[BP.105] 
\z

The locational deictic base can also be \textit{ana}. This was attested in the vernaculars of Nwên and Silên.

\ea
\textit{zeře çermekeş ana ça yawyano.} \\
\gll zeř-e çerme-(e)ke=ş ana{-\O} ça yawya=n=o \\
coin-\textsc{ez.cmpd} white-\textsc{def.m.sg.dir=3sg:PSR} \textsc{loc.deic.cop-3sg.m:S} there be\_spread.\textsc{pst.ptcp.m=cop.3sg.m:S=compl} \\
\glt `Her white coins were there, spread out over [the ground].' \hfill[PK.27]
\z

The negation of the locational copula is expressed by the negative copula, which is inflected by the copula paradigm. 

\ea
\textit{min nîyena yanene.}\\
\gll min nîye=na yane-ne \\
 \textsc{1sg} \textsc{neg.exist=cop.1sg:S} house=\textsc{post} \\
\glt `I am not home.' 
\z

\subsection{b- copula} 
The present stem of the verb `to be', i.e., \textit{b-}, is additionally used as a copula verb and existential stem. Unlike the copula paradigm seen in Table \ref{tab:cop.prs}, the \textit{b-} copula is inflected with set 1 verbal person/number suffixes, e.g., \textit{b-û} `that I will be'. The \textit{b-} form of the copula has both realis and irrealis functions,regardless of which, it can indicate the inchoative\is{inchoative} verb `become', expressing `change of state':
%

\ea
\textit{wextê bo be ħewt heşt saɫe xeɫk bero.} \\ 
\gll wext-ê \textbf{b-o} be ħewt heşt saɫe xeɫk ber-o \\ 
 time\textsc{.m-indf} be\textsc{.prs.ind-3sg:S} \textsc{adp} seven eight year\textsc{.f} people\textsc{.m} take\textsc{.prs.ind-3sg:A} \\ 
\glt `When he turned seven [or] eight years old, people would take him [into their houses].' \hfill[KŞ.36] 
\z 


\ea
\textit{ba bîmêwe yo.} \\ 
\gll ba \textbf{b-\stackunder[-10pt]{\^{i}}{\'{}}mê=we} yo \\ 
 \textsc{hort} be\textsc{.prs.sbjv-1pl:S=compl} one\textsc{.m} \\ 
\glt `Let us be (lit. become) one.' \hfill[BP.55]
\z 


\ea
\textit{ađ bilo bo mêmanşa.} \\ 
\gll ađ bí-l-o \textbf{b-o} mêman=şa \\
\textsc{3sg.m.dir} \textsc{sbjv-}go\textsc{.prs-3sg:S} become\textsc{.prs.sbjv-3sg:S} guest\textsc{.m=3pl:PSR} \\ 
\glt `He (the king) would go [and] become their guest.' \hfill[JH.64]
\z 

\subsubsection{Realis uses of b- copula} 
The \textit{b}-copula can have several realis functions. In (\ref{ex.b-copula1})--(\ref{ex.b-copula2}), it is used as an existential verb. The realis form typically sets the scene at the beginning of the tales.

\ea
\textit{dastanû î hewramanû êmeyçe duwê padş{ɛ} eçê ba.} \\ 
\gll dastan(e)-û î hewraman-û ême=îç=e duwê padşa-ê e=çê \textbf{b-a} \\ 
 story\textsc{.f-ez.gen} \textsc{dem.prox} \textsc{pn-ez.gen} \textsc{1pl=add=dem} two king\textsc{.m.pl.dir} in=here be\textsc{.prs.ind-3pl:S} \\ 
\glt `As for the tales of this Hewraman region of ours, there used to be two kings here.' \hfill[DP.1] \label{ex.b-copula1}
\z 


\ea
\textit{pîr şelîyar hetîm bo.} \\ 
\gll pîr şelîyar hetîm \textbf{b-o} \\ 
 \textsc{pn} \textsc{pn} orphan be\textsc{.prs.ind-3sg:S} \\ 
\glt `Pir Shaliyar was an orphan.' \hfill[JP.10] \label{ex.b-copula2}
\z    

The \textit{b-} copula may also function as the predicate in predicative possessive constructions\is{predicative possessive constructions}.

\ea
\textit{padşaw mîsrî kinaçêş bo felece bo.} \\ 
\gll padşa-û mîsr-î kinaç(ê)-ê=ş \textbf{b-o} felec-e b-o \\ 
 king\textsc{.m-ez.gen} \textsc{pn-m.sg.obl} daughter\textsc{.f-indf=3sg:NC} be\textsc{.prs.ind-3sg:S} disabled\textsc{-f} be\textsc{.prs.ind-3sg:S} \\ 
\glt `The king of Egypt had a daughter [who] was disabled.' \hfill[JP.146] 
\z 


\ea
\textit{yerê kinaçêş ba.} \\ 
\gll yerê kinaçê=ş \textbf{b-a} \\ 
 three daughter\textsc{.pl.dir=3sg:NC} be\textsc{.prs.ind-3pl:S} \\ 
\glt `He had three daughters.' \hfill[JH.20]
\z 


\ea
\textit{kuřêş ba.} \\ 
\gll kuř-ê=ş \textbf{b-a} \\ 
son\textsc{.m-pl.dir=3sg:NC} be\textsc{.prs.ind-3pl:S} \\
\glt `He had sons.' \hfill[ZB.9] 
\z 

\subsubsection{Irrealis uses of b- copula} 
The \textit{b-}copula also expresses irrealis functions. In (\ref{ex.b-copula3}), it expresses speaker-oriented modality\is{speaker-oriented modality} in the main clause, the category which in terms of \citet[179]{bybee_evolution_1994}{} expresses directives such as commands, demands, warnings, exhortations, and recommendations, imposed on the addressee\is{addressee}.

\ea
\textit{mubarekû sahêbîş bo!} \\ 
\gll mubarek-û sahêb-î=ş \textbf{b-o} \\ 
 blessed\textsc{-ez.gen} owner\textsc{.m-sg.obl=3sg:PSR} be\textsc{.prs.sbjv-3sg:S} \\  
\glt `May she be happy with her owner (i.e., father)!' \hfill[ZP.106] \label{ex.b-copula3}
\z 

The \textit{b-}form can be used in subordinate clauses\is{subordinate clauses}:

\ea
\textit{pêse îsey nebîyen girđçê ʕal bo weʕze hîn bo.} \\ 
\gll pêse îse-î ne-bîye=n girđçê ʕal \textbf{b-o} weʕze hîn \textbf{b-o} \\ 
like now\textsc{-m.sg.obl} \textsc{neg-}be\textsc{.pst.ptcp.m=cop.3sg.m:S} everything good be\textsc{.prs.sbjv-3sg:S} situation\textsc{.f} \textsc{fill} be\textsc{.prs.sbjv-3sg:S} \\ 
\glt `It wasn’t like nowadays, when everything is fine, and the situation is thingummy.' \hfill[JE.72] 
\z 

\section{TAM categories built on the present stem}
A number of tense-aspect-mood constructions are built from the present stem of the verb. In this section, I enumerate the formal makeup of these TAM categories and their functions. The TAM categories constructed on the present stem of the verb exhibit accusative alignment\is{accusative alignment}, carried out by inflectional person suffixes agreeing with the transitive subject (A) and intransitive subject (S). 

\subsection{Present indicative\is{present indicative}}
The present indicative\is{present indicative} is built in two ways depending on the verb class. For the majority of verbs, the structure of the present indicative\is{present indicative} is stem + inflectional person affixes. This is represented by the inflection of \textit{kerđey} `do' in Table \ref{tab:prs.ind}.\footnote{Recall from \S\ref{indicative m-} that the indicative \textit{m-} is missing with these verbs.} For the rest of the verbs, making up around 5\% of total verbs, the present indicative\is{present indicative} is made by attaching the indicative prefix \textit{m-} to the present stem of the verb, followed by inflectional person suffixes. The verbs in this class are represented by \textit{arđey} `bring' and \textit{luway} `go' in Table \ref{tab:prs.ind}.

\begin{table}

    \begin{tabular}{llll}
\lsptoprule
 & \textit{kerđey} `do' & \textit{luway} `go' & \textit{arđey} `bring'   \\
 \midrule
\textsc{1sg} & \textit{ker-û}&\textit{mi-l-û} &  \textit{m-ar-û}  \\
\textsc{2sg} & \textit{ker-î}& \textit{mi-l-î} & \textit{m-ar-î}  \\
\textsc{3sg} & \textit{ker-o}& \textit{mi-l-o} & \textit{m-ar-o}  \\
\textsc{1pl} & \textit{ker-mê} &\textit{mi-l-mê} &\textit{m-ar-mê} \\
\textsc{2pl} & \textit{ker-dê} &\textit{mi-l-dê} &\textit{m-ar-dê}\\
\textsc{3pl} &\textit{ker-a} &\textit{mi-l-a }&\textit{m-ar-a} \\  
    \lspbottomrule
    \end{tabular}
    \caption{The present indicative\is{present indicative}--sample paradigms} 
    \label{tab:prs.ind}
\end{table}

The negation of the present indicative\is{present indicative} is expressed by \textit{me-} for the majority of verbs and \textit{nime-} for a subset of vowel-initial verbs (see \S \ref{neg-prs} for details). The present indicative\is{present indicative} paradigms seen in Table \ref{tab:prs.ind} are negated as follows (Table \ref{tab:neg.prs.ind}).
%
\begin{table}

    \begin{tabular}{llll}
\lsptoprule
 & \textit{kerđey} `do' & \textit{luway} `go' & \textit{arđey} `bring'   \\
 \midrule 
\textsc{1sg} & \textit{me-ker-û} &\textit{me-l-û} &\textit{nim(e)-ar-û}  \\
\textsc{2sg} & \textit{me-ker-î}&\textit{me-l-î}&\textit{nim(e)-ar-î}   \\
\textsc{3sg} & \textit{me-ker-o}&\textit{me-l-o}&\textit{nim(e)-ar-o}   \\
\textsc{1pl} & \textit{me-ker-mê}&\textit{me-l-mê} &\textit{nim(e)-ar-mê}  \\
\textsc{2pl} & \textit{me-ker-dê}&\textit{me-l-dê}&\textit{nim(e)-ar-dê} \\
\textsc{3pl} &\textit{me-ker-a}&\textit{me-l-a} &\textit{nim(e)-ar-a} \\ 
  \lspbottomrule
    \end{tabular}
    \caption{The negation of the present indicative\is{present indicative}--sample paradigms}
    \label{tab:neg.prs.ind}
\end{table}


\subsubsection{Narrative present\is{narrative present}} \label{sect:narrativeprs}
The present indicative\is{present indicative} expresses habitual\is{habitual} actions with repeated eventualities at unspecified points in time. In (\ref{ex.prsind1}), the present indicative expresses sequential perfective events in the narrative.
 %

\ea
\textit{milonû wer werwe maro be nîrûwekeyş de koɫê penj koɫê gi řowê.} \\ 
\gll mi-l-on=û wer werwe m-ar-o be nîrû-ekey=ş de koɫ-ê penj koɫ-ê gi řo-ê \\ 
\textsc{ind-}go\textsc{.prs-3sg:S}=and out snow\textsc{.f} \textsc{ind-}bring\textsc{.prs-3sg:A} by force\textsc{.m-def.m.sg.obl=3sg:PSR} ten load/shoulder\textsc{.m-pl.dir} five load/shoulder\textsc{.m-pl.dir} each day\textsc{.m-pl.dir} \\ 
\glt `He (i.e., Jamsher Shah) fetched snow, five or ten loads daily, using his men.' \hfill[DP.34] \label{ex.prsind1}
\z 

This habitual\is{habitual} function of the present indicative\is{present indicative} allows its extension to the narrative present, which is the use of the present tense form to refer to past events. In this function, the present narrative alternates with the past tense to make a past tense event more vivid \citep[]{schiffrin_tense_1981}{}. In the following excerpt, the shift to the present tense adds an element of surprise to a past tense event.\footnote{See \citet[]{Noorlander2022-narrative}{} for a similar function of narrative present in the Kurdish\il{Kurdish} and Neo-Aramaic\il{Neo-Aramaic} dialects of the region.}
%

\ea
\textit{luwanê nizîkû qiřoɫekey bîyawe. miđyew sîwînîş mê qiřoɫekey. nizîk bîyawe temaşem kerd. wextê miđyew kuřeke bîyen peřû qiřoɫêwe, qiřoɫekey.} \\ 
\gll \textbf{luwa-(a)nê} nizîk-û qiřoɫ-ekey \textbf{bî-a=we} \textbf{mi-đye-û} sîwînî=ş \textbf{m-ê} qiřoɫ-ekey nizîk \textbf{bî-a=we} \textbf{temaşe=m} \textit{kerd} wext-ê \textbf{mi-đye-w} kuř-eke \textbf{bîye=n} peř-û qiřoɫ-êwe qiřoɫ-ekey\\ 
go\textsc{.pst-1sg:S} close\textsc{-ez.gen} hollow\textsc{.m-def.m.sg.obl} be\textsc{.pst-1sg:S=compl} \textsc{ind-}look\textsc{.prs-1sg:S} voice\textsc{=3sg:PSR} \textsc{ind-}come\textsc{.prs.3sg:S} hollow\textsc{.m-def.m.sg.obl} close be\textsc{.pst-1sg:S=compl} looking\textsc{=1sg:A} do\textsc{.pst} time\textsc{.m-indf} \textsc{ind-}look\textsc{.prs-1sg:S} son\textsc{.m-def.m.sg.dir} be\textsc{.pst.ptcp.m=cop.3sg.m:S} full\textsc{-ez.gen} hollow\textsc{.m-indf} hollow\textsc{.m-def.m.sg.obl} \\
\glt `I \textbf{went} [and] \textbf{got closer} to the tree hollow. I \textbf{noticed} [lit. notice] a voice \textbf{came} (lit. comes) from the hollow. I \textbf{got closer} and \textbf{looked} inside. I \textbf{noticed} that the boy [had grown up so much that he] had filled a tree hollow, i.e., the tree hollow.' \hfill[ZQ.36--ZQ.38]
\z 

\subsubsection{Performative\is{performative}} 
The present indicative\is{present indicative} may also express a perfective present. Here, the verb expresses a bounded perfective event. 
 \ea 
\textit{fermawo, `weɫa çûn ana bêheyanî kerût penû ʕaɫemî.'} \\ 
\gll fermaw-o weɫa çûn an(e)=a bêheya=nî ker-û=t pen-û ʕaɫem-î \\ 
 say\textsc{.prs.ind-3sg:A} indeed as \textsc{dem.dist.m.3sg.dir=deic} shameless\textsc{=cop.2sg:S} do\textsc{.prs.ind-1sg:A=2sg:O} advice\textsc{.m-ez.gen} world\textsc{-m.sg.obl} \\  
\glt `He (Pir Shaliyar) said, `Indeed, since you are so shameless, I (hereby) expose your impropriety to the world. [Lit. I (hereby) make you the news of the world.]’' \hfill[JP.198]
\z 


\ea
\textit{maço, `dey min şûyş kerû pene.'} \\ 
\gll m-aç-o dey min şû-î=ş ker-û pene \\ 
 \textsc{ind-}say\textsc{.prs-3sg:A} \textsc{disc.ptcl} \textsc{1sg} husband\textsc{.m-sg.obl=3sg:R} do\textsc{.prs.ind-1sg:A} to \\ 
\glt `She said, `I will marry him.’' \hfill[JH.59]
\z 
\subsubsection{Future} 
The present indicative\is{present indicative} is frequently used to express a future time reference. This use of the future in the narrative is sequential to what precedes.

\ea
\textit{ême milmê şeşik. milmê goşew şarî, awayekey. çowe neferêwe mê, pêwaymawe.} \\ 
\gll ême mi-l-mê şeşik mi-l-mê goşe-û şar-î awaî-ekey ç=o=we nefer-êwe m-ê pêway=ma=we \\ 
\textsc{1pl} \textsc{ind-}go\textsc{.prs-1pl:S} \textsc{pn} \textsc{ind-}go\textsc{.prs-1pl:S} corner\textsc{-ez.gen} city\textsc{.m-sg.obl} village\textsc{.m-def.m.sg.obl} in\textsc{=dem.dist=post} person\textsc{.m-indf} \textsc{ind-}come\textsc{.prs.3sg:S} welcoming\textsc{.m=1pl:PSR=post} \\
\glt `We will go to Shashk. We will go to the city suburb, to the village. There, a person will welcome us.' \hfill[HB.32--HB.34]
\z 
 
\subsection{Present subjunctive\is{present subjunctive}}\label{sec:subj-morph}
\begin{table}[b]
 \begin{tabular}{lllll}
  \lsptoprule
 & \textit{kerđey} `do' & \textit{luway} `go'& \textit{arđey} `bring' \\
 \midrule
\textsc{1sg} & \textit{kér-û} &\textit{bí-l-û} &\textit{b-ár-û} \\
\textsc{2sg} & \textit{kér-î} &\textit{bí-l-î} &\textit{b-ár-î} \\
\textsc{3sg} & \textit{kér-o} &\textit{bí-l-o}  &\textit{b-ár-o}\\
\textsc{1pl} & \textit{kér-mê} &\textit{bí-l-mê} &\textit{b-ár-mê} \\
\textsc{2pl} & \textit{kér-dê} &\textit{bí-l-dê}&\textit{b-ár-dê}\\
\textsc{3pl} &\textit{kér-a} &\textit{bí-l-a } &\textit{b-ár-a}\\
    \lspbottomrule
 \end{tabular}
    \caption{The present subjunctive\is{present subjunctive}--sample paradigms}
    \label{tab:prs.sbjv}
\end{table}

Like in the present indicative\is{present indicative}, the vast majority of verbs in the present subjunctive\is{present subjunctive} lack any expression of the subjunctive prefix; see the conjugation of \textit{kerđey} `do' in Table \ref{tab:prs.sbjv}. As discussed in \S\ref{section-vinfmorph}, for this class of verbs, the stress on the stem distinguishes the subjunctive from the indicative, thus \textit{kéro} `that he/she does' vs. \textit{keró} `he/she does'. For the rest of the verbs, the present subjunctive\is{present subjunctive} is built by attaching the subjunctive prefix \textit{b(i)-} to the present stem of the verb, followed by inflectional person affixes. With these verbs, the subjunctive prefix may or may not be stressed, depending on the syllable structure of the verb stem (see \S\ref{sect:sbjv} for details).


The negation of present subjunctive\is{present subjunctive} verb forms is expressed by \textit{ne-}, reduced as \textit{n-}, before vowel-initial verbs (see Table \ref{tab:neg.prs.sbjv}). 

\begin{table}[t]
    \begin{tabular}{llll}
\lsptoprule
 & \textit{kerđey} `do' & \textit{luway} `go' & \textit{arđey} `bring'\\
 \midrule
\textsc{1sg} & \textit{né-ker-û} &\textit{né-l-û}  &\textit{n-ár-û}\\
\textsc{2sg}& \textit{né-ker-î} &\textit{né-l-î}  &\textit{n-ár-î} \\
\textsc{3sg} & \textit{né-ker-o} &\textit{né-l-o}  &\textit{n-ár-o}\\
\textsc{1pl} & \textit{né-ker-mê} &\textit{né-l-mê}  &\textit{n-ár-mê}\\
\textsc{2pl} & \textit{né-ker-dê} &\textit{né-l-dê}&\textit{n-ár-dê}\\
\textsc{3pl} &\textit{né-ker-a} &\textit{né-l-a } &\textit{n-ár-a}\\
    \lspbottomrule
    \end{tabular}
    \caption{The negation of the present subjunctive\is{present subjunctive}--sample paradigms}
    \label{tab:neg.prs.sbjv}
\end{table}

The present subjunctive\is{present subjunctive} verb forms are, by default, used in subordinate clauses\is{subordinate clauses}. This includes, for example, the protasis of general conditions (\ref{ex.prssbjv1}), dependent purpose clauses (\ref{ex.prssbjv2}), and prospective aspect (\ref{ex.prssbjv3}).


\ea
\textit{dey çêşim miđey min kuřlekeyt weş kerûwe?} \\ 
\gll dey çêş=im mi-đe-î min kuřle-(e)key=t weş \textbf{kér-û=we} \\
\textsc{disc.ptcl} what=\textsc{1sg:R} \textsc{ind-}give.\textsc{prs-2sg:A} \textsc{1sg} son\textsc{.dim-def.m.sg.obl=2sg:PSR} well do\textsc{.prs.sbjv-1sg:A=compl} \\
\glt `What will you give me if I bring back your little son to life [lit. heal]?' \\\hfill[ÇK.109] \label{ex.prssbjv1}
\z 


\ea
\textit{am{ɛ}nmê xizmetû to îcazema biđey.} \\ 
\gll am{ɛ}=nmê xizmet-û to îcaze=ma \textbf{bi-đé-î} \\
 come\textsc{.pst.ptcp.pl=cop.1pl:S} service\textsc{.m-ez.gen} \textsc{2sg} permission\textsc{=1pl:R} give\textsc{.prs.sbjv-2sg:A} \\ 
\glt `I (lit. We) have come to your service so that you might permit us.' \\\hfill[PM.15] \label{ex.prssbjv2}
\z 

\ea
\textit{tejnayene epê wext bê soçmê nařehetîyene.} \\ 
\gll tejnay=ene epê wext b-ê \textbf{sóç-mê} nařehetî=ene \\
thirst\textsc{.inf=post} in.\textsc{dem.prox} time\textsc{.m} be\textsc{.prs-aug.3sg:S} burn\textsc{.prs.sbjv-1pl:S} sadness\textsc{=post} \\ 
\glt `We were about [lit. it was time] to burn from thirst and sorrow [from leaving behind one of the sons].' \hfill[ZQ.25] \label{ex.prssbjv3}
\z  

The subjunctive verb form is used following deontic particles \textit{ba}, and \textit{meger}, used in interactional discourse expressing a wish, asking for permission, or giving advice.

\ea
\textit{watiş, `pilekanê çine ba bilmê!'} \\ 
\gll wat=iş pilekan(î)-ê çín-e ba bi-l-mê \\
 say\textsc{.pst=3sg:A} stair\textsc{.f-indf} set\_up\textsc{.prs.imp-2sg:A} \textsc{hort} \textsc{sbjv-}go\textsc{.prs-1pl:S} \\ 
\glt `He said, `Set up a stairway for us to move!’' \hfill[JH.25]
\z 


\ea
\textit{ʕerziş kerđ, `qurban ba dastanêt pey gêɫnûwe, dastane.'} \\ 
\gll ʕerz=iş kerđ qurban \textbf{ba} dastan(e)-ê=t pey g\stackunder[-10pt]{\^{e}}{\'{}}ɫn-û=we dastane \\ 
 petition\textsc{.m=3sg:A} do\textsc{.pst} sir.\textsc{voc} \textsc{hort} story\textsc{.f-indf=2sg:R} to narrate\textsc{.prs.sbjv-1sg:A=compl} story\textsc{.f} \\  
\glt `He said, `Sir, let me tell you a story, the story!’' \hfill[ZQ.7]
\z 


\ea
\textit{meger berî kîyanîş pey hewramanî.} \\ 
\gll \textbf{meger} bér-î k\stackunder[-10pt]{\^{i}}{\'{}}yan-î=ş pey hewraman-î \\
 if\_only take\textsc{.prs.sbjv-2sg:A} send\textsc{.prs.sbjv-2sg:A=3sg:O} to \textsc{pn-m.sg.obl} \\  
\glt `Maybe you could take her to Hewraman.' \hfill[ZP.34]
\z 

The subjunctive verb form is used in some factive content clauses and non-factive complement clauses.
%

\ea
\textit{metawo bilo aweyanî.} \\ 
\gll me-taw-o bi-l-o aweyanî \\ 
\textsc{neg.ind-}can\textsc{.prs-3sg:A} \textsc{sbjv-}go\textsc{.prs-3sg:S} habitat\textsc{\textsc{.m}} \\ 
\glt `He wasn’t allowed to go to the village.' \hfill[DG.54]
\z 

\ea
\textit{îcaze, bizane ađ îcaze miđo ême eçê nehar kermê yam ne.} \\ 
\gll îcaze bi-zán-e ađ îcaze mi-đ(e)-o ême e=çê nehar kér-mê yam ne \\
 permission \textsc{imp-}know\textsc{.prs-2sg:A} \textsc{3sg.m.dir} permission \textsc{ind-}give\textsc{.prs-3sg:A} \textsc{1pl} in=here lunch\textsc{.m} do\textsc{.prs.sbjv-1pl:A} or no \\ 
\glt `See if he lets us stay here for lunch or not.' \hfill[PM.9]
\z 

In some cases, a subjunctive verb form occurs in a main clause. A case in point is the use of subjunctive mood following the particle \textit{da} in a construction expressing speaker-oriented modality\is{speaker-oriented modality}. In the following example, the context is self-hortative, i.e., the speaker encourages herself to action. 
%

\newpage
\ea
\textit{maço, `da bizanû çêşşa ser ama.'} \\
\gll m-aç-o da bi-zan-û çêş=şa ser ama \\
\textsc{ind-}say.\textsc{prs-3sg:A} \textsc{hort} \textsc{sbjv-}know.\textsc{prs-1sg:A} what=\textsc{3pl:R} to come.\textsc{pst.3sg:S} \\
\glt `She said, `I shall see what happened to them.’'  \hfill[SH.124]
\z

The subjunctive may be used in indefinite relative clauses.

\ea
\textit{ne kesê hen be weş dađiş berî la.} \\ 
\gll ne kes-ê hen-∅ be weş dađ=iş bér-î la \\
nor person\textsc{.m-indf} \textsc{exist-3sg.m:S} by good cry\textsc{.m=3sg:R} take\textsc{.prs.sbjv-2sg:A} to \\ 
\glt `Nor is there a person with whom one can take counsel in peace.' \hfill[ZB.60]
\z 

\subsection{Imperative}\label{sect:imperative}
\largerpage
For most verbs, the imperative verb forms are constructed by adding the 2nd person inflectional person suffixes to the present stem of the verb (see \S \ref{section.imp}). Additionally, a small subset of verbs takes the subjunctive/imperative prefix; see Table \ref{tab:prs.imp} \textit{bi-} for sample paradigms. 
%
\begin{table}

    \begin{tabular}{llll}
\lsptoprule
 & \textit{kerđey} `do' & \textit{luway} `go'  & \textit{arđey} `bring'\\
 \midrule
\textsc{2sg} & \textit{ker-e} &\textit{lu-e, bi-l-e} &\textit{b-ar-e} \\
\textsc{2pl}& \textit{ker-dê}  &\textit{lo-dê}&\textit{b-ar-dê}\\
    \lspbottomrule
    \end{tabular}
    \caption{The imperative--sample paradigms}
    \label{tab:prs.imp}
\end{table}

The negation of the imperative is carried out by \textit{me-} for verbs which in their imperative form do not take any prefixes, \textit{ne}- for stems starting with \textit{m}, and \textit{nime-} for a subset of vowel-initial verbs (see \S\ref{neg-prs} for details). Table \ref{tab:neg.prs.imp} lists sample paradigms of prohibitive verbs.

\begin{table}

    \begin{tabular}{llll}
\lsptoprule
 & \textit{kerđey} `do' & \textit{luway} `go'  & \textit{arđey} `bring'\\
 \midrule
\textsc{2sg} & \textit{mé-ker-e} &\textit{mé-l-e}  &\textit{nim(e)-ár-e}\\
\textsc{2pl} & \textit{mé-ker-dê} &\textit{mé-l-dê}&\textit{nim(e)-ár-dê}\\
    \lspbottomrule
    \end{tabular}
    \caption{The negation of the imperative--sample paradigms}
    \label{tab:neg.prs.imp}
\end{table}

The imperative mood is used to command an action or express a request. When used negatively, it prohibits an action from being undertaken.

\ea
\textit{bereşo!} \\ 
\gll bér-e=ş=o \\
take\textsc{.prs.imp-2sg:A=3sg:O=compl} \\  
\glt `Take her back!' \hfill[ZP.106]
\z 


\ea
\textit{watiş, `fermawdê beydê mêmanû minindê.'} \\ 
\gll wat=iş fermaw-dê b-e-îdê mêman-û min=indê \\ 
 say\textsc{.pst=3sg:A} say\textsc{.prs.imp-2pl:A} \textsc{imp-}come\textsc{.prs-2pl:S} guest\textsc{.m-ez.gen} \textsc{1sg=cop.2pl:S} \\  
\glt `He said, `Please come! You are my guests.’' \hfill[HB.51]
\z 

 \ea 
\textit{îneyşa mekerdê zînan.} \\ 
\gll îney=şa me-ker-dê zînan \\ 
 \textsc{\textsc{dem.prox}.obl.m.3sg=3pl:PSR} \textsc{proh-}do\textsc{.prs-2pl:A} prison\textsc{.m} \\ 
\glt `Do not put this one in prison.' \hfill[BP.133]
\z 

An imperative verb form may be given added immediacy by combining with the particle \textit{da}.
\largerpage

\ea
\textit{fermawo, `da lodê bizandê fiɫane kes çî nama?'} \\ 
\gll fermaw-o da lo-dê bi-zan-dê fiɫan-e kes çî n(e)-ama-∅ \\ 
 say\textsc{.prs.ind-3sg:A} \textsc{hort} go.\textsc{prs.imp-2pl:S} \textsc{imp-}know\textsc{.prs-2pl:A} such\_and\_such\textsc{-ez.cmpd} person\textsc{.m} why \textsc{neg-}come\textsc{.pst-3sg.m:S} \\ 
\glt `He (Baba Khwada) said, `Go [and] see (lit. know) why such-and-such person hasn’t come [to the mosque]?’' \hfill[BP.77]
\z 

The particle \textit{da} may be combined with the discourse particle \textit{dey} for an even greater degree of immediacy. 
%

\ea
\textit{da dey hurbêze î mexlûqî girdiş gêɫeş pone.} \\ 
\gll da dey hur-b-êz-e î mexlûq-î gird=iş g\stackunder[-10pt]{\^{e}}{\'{}}ɫ-e=ş pone \\ 
\textsc{hort} \textsc{disc.ptcl} \textsc{pvb-imp-}rise\textsc{.prs-2sg:S} \textsc{dem.prox} people\textsc{-m.sg.obl} all\textsc{=3sg:PSR} wander\textsc{.prs.imp-2sg:S=3sg:R} at \\ 
\glt `Come on, get up and search among all these people.' \hfill[HB.90]
\z 

\subsection{Present progressive\is{present progressive}}
The present progressive\is{present progressive} can be expressed in several ways. It can be expressed by the inflected form of the verb identical to the present indicative\is{present indicative}. The development seems to be an extension from the progressive sense to the habitual\is{habitual} sense \citep[]{deo_semantic_2015}{}. 
%

\ea
\textit{min îse î nîştenare îsrahetîç kerû.} \\ 
\gll min îse î nîşte=na=re îsrahet=îç ker-û \\ 
 \textsc{1sg} now \textsc{dem.prox} sit\textsc{.pst.ptcp.m=cop.1sg:S=povb} rest\textsc{.m=add} do\textsc{.prs.ind-1sg:A} \\  
\glt `Now I am seated, and I am resting.' \hfill[PM.45]
\z 


\ea
\textit{înê î qisê çêşene? çêş maçdê? maça çêş?} \\ 
\gll înê î qisê çêşe=ne çêş m-aç-dê m-aç-a çêş \\ 
 \textsc{dem.prox.f.3sg.dir} \textsc{dem.prox} talk\textsc{.f.sg} what\textsc{.f=cop.3sg.f:S} what \textsc{ind-}say\textsc{.prs-2pl:A} \textsc{ind-}say\textsc{.prs-3pl:A} what \\ 
\glt `What is this talk? What are you saying? What are they saying?' \hfill[JP.223]
\z 

The present progressive\is{present progressive} may be expressed by a reduplicated progressive construction\is{reduplicated progressive construction} consisting of the inflected form of the verb preceded by a form containing the present form of the verb and the suffix \textit{-ay}, resembling the infinitive\is{infinitive} suffix. The copy takes the same inflectional prefix as the inflected verb, suggesting it is on its way to being grammaticalised as a verbal form.\footnote{A similar construction consisting of the infinitive\is{infinitive} and the inflected form of the verb exists in the neighbouring Jewish Neo-Aramaic dialect of Sanandaj, e.g., \textit{şatoê şatêna} `I am drinking' \citep[275]{khan_jewish_2009}{}. This has been taken as a case of matching between Hewramî\il{Hewramî} and Neo-Aramaic\il{Neo-Aramaic} in the contact setting in Sanandaj (see \citealt{khan_language_2023}{} for details).}
%
\begin{table}
    \begin{tabular}{lllll}
\lsptoprule
 & \textit{arđey} `bring' & & \textit{warđey} `eat' \\
 \midrule
\textsc{1sg} &\textit{m-ar-ay m-ar-û} & `I am bringing' & \textit{wer-ay wer-û}& `I am eating'\\
\textsc{2sg} &\textit{m-ar-ay m-ar-î} & & \textit{wer-ay wer-î} \\
\textsc{3sg} &\textit{m-ar-ay m-ar-o} & & \textit{wer-ay wer-o} \\
\textsc{1pl} &\textit{m-ar-ay m-ar-mê} & & \textit{wer-ay wer-mê} \\
\textsc{2pl} &\textit{m-ar-ay m-ar-dê} & & \textit{wer-ay wer-dê}\\
\textsc{3pl} &\textit{m-ar-ay m-ar-a} & &\textit{wer-ay wer-a} \\  
    \lspbottomrule
    \end{tabular}  
\caption{The reduplicated present progressive\is{reduplicated present progressive}--sample paradigms}
\label{tab:prs.progress}
\end{table}


\ea
\textit{weray werû.} \\ 
\gll wer-ay wer-û \\ 
 eat\textsc{.prs-nmlz} eat.\textsc{prs-1sg} \\ 
\glt `I am eating.'
\z


\ea
\textit{yewaşê dewrû dimaw ʕîşay kero, şewe kero waray waro mê ew peřû deg{ɛ}.} \\ 
\gll yewaşê dewr-û dima-û ʕîşa-î ker-o şewe ker-o \textbf{war-ay} \textbf{war-o} m-ê ew peř-û deg{ɛ} \\ 
 then around\textsc{-ez.gen} afterwards\textsc{-ez.gen} evening\_prayers\textsc{.m-sg.obl} do\textsc{.prs.ind-3sg:A} night\textsc{.f} do\textsc{.prs.ind-3sg:A} rain\textsc{.prs-nmlz} rain\textsc{.prs.ind-3sg:S} \textsc{ind-}come\textsc{.prs.3sg:S} \textsc{dem.dist} side\textsc{.m-ez.gen} village\textsc{.f.sg.obl} \\ 
\glt `Then, it was either during the evening prayers or during the night that he arrived at the other side of the village while it was raining.' \hfill[KŞ.59] \label{redup-progr1}
\z 

The reduplicated progressive construction\is{reduplicated progressive construction} may also be used in negation and interrogative clauses, exhibiting a grammaticalised progressive form. The reduplicated construction in negation clauses conveys extra emphasis.

\ea
\textit{meweray mewerû!} \\ 
\gll me-wer-ay me-wer-û \\ 
 \textsc{neg.ind}-eat\textsc{-nmlz} \textsc{neg.ind}-eat\textsc{.prs-1sg:A} \\ 
\glt `I am not eating!'
\z

\ea
\textit{milay milî?} \\ 
\gll mi-l-ay mi-l-î\\ 
 \textsc{ind-}go\textsc{.prs-nmlz} \textsc{ind-}go\textsc{.prs-2sg:S} \\ 
\glt `Are you going?'
\z


\ea
\textit{taway tawû barşa kerû? taway tawû çêş kerû? }\\
\gll taw-ay taw-û bar=şa ker-û taw-ay taw-û çêş ker-û \\
can\textsc{.prs-nmlz} can.\textsc{prs-1sg} load=\textsc{3pl:O} do.\textsc{prs.ind-1sg:A} can\textsc{.prs-nmlz} can.\textsc{prs-1sg} what do.\textsc{prs.ind-1sg:A} \\
\glt `Am I able to load them? What am I able to do?' \hfill[SH.127]
\z 


\ea
\textit{mewînay mewînmê?} \\ 
\gll me-wîn-ay me-wîn-mê\\ 
 \textsc{neg.ind-}see\textsc{.prs-adv} \textsc{neg.ind-}see\textsc{.prs-1pl:A} \\ 
\glt `Don't we really see?' 
\z

The construction may also be used for future reference in certain contexts.

\ea
\textit{êşew gêɫay gêɫa pey pîyaya.} \\ 
\gll êşew gêɫ-ay gêɫ-a pey pîya-ya \\ 
 tonight search\textsc{.prs-nmlz} search\textsc{.prs.ind-3pl:A} for man\textsc{-pl.obl} \\ 
\glt `Tonight, they will be looking for men [who can serve them].' \hfill[JL.34]
\z


\ea
\textit{min gunakê may mê milimre.} \\ 
\gll min guna-(e)kê m-ay m-ê mil=im=re \\ 
\textsc{1sg} sin\textsc{.f-def.f.sg} \textsc{ind-}come\textsc{.prs.nmlz} \textsc{ind-}come\textsc{.prs.3sg:S} shoulder\textsc{.m=1sg:PSR=post} \\  
\glt `I won’t shoulder the burden [of injuring them]. [Lit. the sin will come to me.]' \hfill[DG.67]
\z 


\ea
\textit{řama dûrene. milay milmêwe.} \\ 
\gll řa=ma dûr-e=ne mi-l-ay mi-l-mê=we \\ 
road\textsc{.f=1pl:PSR} far\textsc{-f=cop.3sg.f:S} \textsc{ind-}go\textsc{.prs-nmlz} \textsc{ind-}go\textsc{.prs-1pl:S=compl} \\  
\glt `We have a long way [to go]. We should be going.' \hfill[BP.191]
\z 

Another strategy to express the present progressive\is{present progressive} is through a noun phrase consisting of the nominal form \textit{xerîk} `busy' combined with the infinitive\is{infinitive} form of the verb or another nominal. 


\ea
\textit{î meʕmûrê xerîkû genekarînê.} \\ 
\gll î meʕmûr-ê \textbf{xerîk-û} \textbf{genekarî=nê} \\ 
 \textsc{dem.prox} officer\textsc{.m-pl.dir} busy\textsc{-ez.gen} debauchery\textsc{.m}\textsc{=cop}\textsc{.3pl:S} \\ 
\glt `The officers were busy engaging in debauchery.' \hfill[BP.45]
\z

Similarly, the present progressive\is{present progressive} may be expressed by \textit{xerîk bîyey} `be busy' combined with the inflected form of the verb. 

\ea
\textit{mîyo jenekêş xerîkene kebab kero.} \\
\gll mi-đy(e)-o jen(î)-ekê=ş \textbf{xerîk-e=ne} \textbf{kebab} \textbf{ker-o} \\
\textsc{ind-}notice\textsc{.prs-3sg:S} wife-\textsc{def.f.sg=3sg:PSR} busy\textsc{-f=cop.3sg:A} kebab do.\textsc{prs.ind-3sg:A} \\
\glt `He noticed his wife was busy making kebab.' \hfill[KK.28]
\z 

A more innovative construction for expressing the present progressive\is{present progressive} is to combine the adjective \textit{xerîk} with the reduplicated progressive\is{reduplicated progressive construction}.

\ea
\textit{xerîkna weray werû.} \\
\gll xerîk=na wer-ay wer-û \\
busy\textsc{.m=cop.1sg:S} eat.\textsc{prs-nmlz} eat.\textsc{prs-1sg:A} \\
\glt `I am eating.'
\z 

The present progressive\is{present progressive} expresses, by default, an event in progress.

\ea
\textit{yanew kê milî?} \\ 
\gll yane-û kê mi-l-î \\ 
 house\textsc{.m}=and who \textsc{ind-}go\textsc{.prs-2sg:S} \\ 
\glt `Whose house are you going to?' \hfill[JH.17]
\z 

The reduplicate progressive construction reinforces the progressive sense of the verb, as seen in (\ref{redup-progr1}) above, repeated here for convenience.

\ea
\textit{yewaşê dewrû dimaw ʕîşay kero, şewe kero waray waro mê ew peřû deg{ɛ}.} \\ 
\gll yewaşê dewr-û dima-û ʕîşa-î ker-o şewe ker-o \textbf{war-ay} \textbf{war-o} m-ê ew peř-û deg{ɛ} \\ 
 then around\textsc{-ez.gen} afterwards\textsc{-ez.gen} evening\_prayers\textsc{.m-sg.obl} do\textsc{.prs.ind-3sg:A} night\textsc{.f} do\textsc{.prs.ind-3sg:A} rain\textsc{.prs-nmlz} rain\textsc{.prs.ind-3sg:S} \textsc{ind-}come\textsc{.prs.3sg:S} \textsc{dem.dist} side\textsc{.m-ez.gen} village\textsc{.f.sg.obl} \\
\glt `Then, it was either during the evening prayers or during the night that he arrived at the other side of the village while it was raining.' \hfill[KŞ.59]
\z 

\subsection{Past progressive\is{past progressive}}
Like present progressive\is{present progressive}, past progressive\is{past progressive} may be expressed through several strategies. In what appears to be the default pattern, a reduplicated construction\is{reduplicated past progressive construction} may express the past progressive\is{past progressive}. The latter consists of the inflected verb featuring the present stem of the verb, followed by the augment \textit{-ên} and set 2 inflectional person suffixes combined with the double consisting of the present stem of the verb followed by the nominaliser \textit{-ay}, see Table \ref{tab:pst.progress} for sample paradigms. The augment \textit{-ên} may be considered a past converter suffix, giving the verb form a past time reference. Gender\is{gender} distinction is neutralised in the \textsc{3sg} following the past converter \textit{-ên}.

\begin{table}

    \begin{tabular}{llll}
\lsptoprule
 & \textit{kerđey} `do' & \textit{luway} `go'  & \textit{arđey} `bring'\\
 \midrule
\textsc{1sg} & \textit{keray ker-ên-a}&\textit{luway lu-ên-a}   &\textit{aray ar-ên-a}\\
\textsc{2sg} & \textit{keray ker-ên-î} &\textit{luway lu-ên-î}  &\textit{aray ar-ên-î}\\
\textsc{3sg}  & \textit{keray ker-ê}&\textit{luway lu-ê} &\textit{aray ar-ê} \\
\textsc{1pl} & \textit{keray ker-ên-mê} &\textit{luway lu-ên-mê} &\textit{aray ar-ên-mê} \\
\textsc{2pl} & \textit{keray ker-ên-dê} &\textit{luway lu-ên-dê}&\textit{aray ar-ên-dê}\\
\textsc{3pl} &\textit{keray ker-ên-ê}  &\textit{luway lu-ên-ê }&\textit{aray ar-ên-ê}\\  
    \lspbottomrule
    \end{tabular}
    \caption{The past progressive\is{past progressive}--sample paradigms}
    \label{tab:pst.progress}
\end{table}

The following examples exhibit the parsing of reduplicated past progressive\is{reduplicated past progressive construction}:

\ea
\textit{waçay waçênî} \\
\gll waç-ay waç-ên-î \\
say\textsc{.prs-nmlz} say\textsc{.prs-aug-2sg:A}\\
\glt `You were saying.'
\z 

\ea
\textit{aray arênmê.} \\
\gll ar-ay ar-ên-mê\\
bring\textsc{.prs-nmlz} bring\textsc{.prs-aug-1pl:A}\\
\glt `You were bringing.'
\z 

The past progressive\is{past progressive} may also be expressed by the inflected form of the verb alone, consisting of the present stem of the verb followed by augment and set 2 inflectional affixes. This is the same verb form as the habitual past\is{habitual past} (see \S\ref{sec: past.habitual}).
%

\ea
\textit{ce hewramanne karî naşerʕî kerênê.} \\ 
\gll ce hewraman=ne kar-î naşerʕî \textbf{ker-ên-ê} \\ 
 at \textsc{pn=post} task\textsc{.m-ez.attr} unlawful do\textsc{.prs-aug-3pl:A} \\ 
\glt `They were committing unlawful acts in Hewraman.’ \hfill[BP.62]
\z 

The past progressive\is{past progressive} may alternatively be expressed by \textit{xerîk bîyey} `be busy' combined with the inflected form of the verb.
%

\ea
\textit{xerîk bêna waçênat pene.} \\
\gll xerîk b-ên-a waç-ên-a=t pene\\
busy.\textsc{m} be\textsc{.prs-aug-1sg:S} say\textsc{.prs-aug-1sg:A=2sg:R} to\\
\glt `I (\textsc{m}) was telling you.'
\z 


\ea
\textit{xerîke bêna waçênat pene.} \\
\gll xerîk-e b-ên-a waç-ên-a=t pene\\
busy\textsc{-f} be\textsc{.prs-aug-1sg:S} say\textsc{.prs-aug-1sg:A=2sg:R} to\\
\glt `I (\textsc{f}) was telling you.'
\z 

A more innovative construction for the expression of past progressive\is{past progressive} is to combine \textit{xerîk bîyey} `be busy' with the reduplicated past progressive\is{reduplicated past progressive construction}.

\ea
\textit{xerîk bêna waçayt waçêna pene.} \\
\gll xerîk b-ên-a waç-ay=t waç-ên-a pene\\
busy.\textsc{m} be\textsc{prs-aug-1sg} say\textsc{.prs-nmlz=2sg} say\textsc{.prs-aug-1sg} to\\
\glt `I (m) was telling you.'
\z 

Another strategy for expressing past progressive\is{past progressive} is to have a nominal as the complement of \textit{xerîk} `busy' in a copular clause.

\ea
\textit{xerîkû genekarî bênê.} \\ 
\gll xerîk-û genekarî b-ên-ê \\ 
 busy\textsc{-ez.gen} debauchery\textsc{.m} be\textsc{.prs-aug-3pl:S} \\ 
\glt `They were busy engaging in debauchery.' \hfill[BP.61]
\z 
 
The past progressive\is{past progressive} expresses an event that was going on for a while in the past. In this use, it may be preceded by a verb with a past time reference.

\ea
\textit{ađê nam{ɛ}w; xerîkû genekarî bênê.} \\ 
\gll ađê n(e)-am{ɛ}=û xerîk-û genekarî b-ên-ê\\ 
 \textsc{3pl.dir} \textsc{neg-}come\textsc{.pst.3pl:S}=and busy\textsc{-ez.gen} debauchery\textsc{.m} be\textsc{.prs-aug-3pl:S} \\ 
\glt `[However] they didn’t come back [to you]. They engaged in debauchery.' \hfill[BP.125]
\z 

\subsection{Habitual past} \label{sec: past.habitual}
The habitual past\is{habitual past} is expressed by the present stem of the verb followed by the augment \textit{-ên} and set 2 inflectional person suffixes; see Table \ref{tab:pst.habitual} for sample paradigms.
%
\begin{table}

    \begin{tabular}{llll}
\lsptoprule
  & \textit{kerđey} `do' & \textit{luway} `go'  & \textit{arđey} `bring'\\
 \midrule
\textsc{1sg} & \textit{ker-ên-a} &\textit{lu-ên-a} &\textit{ar-ên-a} \\
\textsc{2sg}& \textit{ker-ên-î}  &\textit{lu-ên-î}  &\textit{ar-ên-î}\\
\textsc{3sg}  & \textit{ker-ê}&\textit{lu-ê} &\textit{ar-ê} \\
\textsc{1pl} & \textit{ker-ên-mê} &\textit{lu-ên-mê} &\textit{ar-ên-mê} \\
\textsc{2pl} & \textit{ker-ên-dê} &\textit{lu-ên-dê}&\textit{ar-ên-dê}\\
\textsc{3pl} &\textit{ker-ên-ê} &\textit{lu-ên-ê } &\textit{ar-ên-ê}\\  
    \lspbottomrule
    \end{tabular}
    \caption{The habitual past\is{habitual past}--sample paradigms}
    \label{tab:pst.habitual}
\end{table}

Examples of the habitual past\is{habitual past} in the text corpus are presented below.

\ea \label{ex.habitual1}
\textit{řowê sî penj timenê kar kerênmê.} \\ 
\gll řo-ê sî penj timen-ê kar ker-ên-mê \\ 
 day\textsc{.m-indf} thirty five \textsc{pn-pl.dir} task\textsc{.m} do\textsc{.prs-aug-1pl:A} \\  
\glt `We used to work for a daily salary of thirty-five tomans.' \hfill[JM.46]
\z 

The negation of habitual past\is{habitual past} is expressed by \textit{ne-}, see Table \ref{tab:neg.pst.habitual}.

\begin{table}
    \begin{tabular}{llll}
\lsptoprule
   & \textit{kerđey} `do' & \textit{luway} `go' & \textit{arđey} `bring'\\
 \midrule
\textsc{1sg}  & \textit{ne-ker-ên-a} &\textit{ne-lu-ên-a} &\textit{n(e)-ar-ên-a}\\
\textsc{2sg}   & \textit{ne-ker-ên-î} &\textit{ne-lu-ên-î}&\textit{n(e)-ar-ên-î}\\
\textsc{3sg}  & \textit{ne-ker-ê}  &\textit{ne-lu-ê}&\textit{n(e)-ar-ê}\\
\textsc{1pl}  & \textit{ne-ker-ên-mê}  &\textit{ne-lu-ên-mê}&\textit{n(e)-ar-ên-mê}\\
\textsc{2pl}  & \textit{ne-ker-ên-dê}&\textit{ne-lu-ên-dê}&\textit{n(e)-ar-ên-dê}\\
\textsc{3pl} &\textit{ne-ker-ên-ê}  &\textit{ne-lu-ên-ê }&\textit{n(e)-ar-ên-ê}\\  
    \lspbottomrule
    \end{tabular}
    \caption{The negation of habitual past\is{habitual past}--sample paradigms}
    \label{tab:neg.pst.habitual}
\end{table}

The habitual past\is{habitual past} expresses habitual\is{habitual} events in the past, see (\ref{ex.habitual1}) above, including in the apodosis of conditional clauses\is{conditional clauses} (\ref{ex.hab-past2}).

\ea
\textit{êtir ênêwe eger řezaşa bîyê, kinaçê dênê.} \\ 
\gll êtir ênê=we eger řeza=şa bî-ê kinaçê \textbf{d(e)-ên-ê} \\ 
 \textsc{disc.ptcl} come\textsc{.prs.aug.3pl:S=compl} if satisfaction\textsc{.f=3pl:NC} be\textsc{.pst.cond.aug.3sg:S} girl\textsc{.f} give\textsc{.prs-aug-3pl:A} \\ 
\glt `They (the relatives) would come [to the girl's family] and if they agreed to it, they would give the girl in marriage.' \hfill[JE.84] \label{ex.hab-past2}
\z 
 
\subsection{Irrealis past\is{irrealis past}}
The irrealis past\is{irrealis past} is constructed in the same way as the habitual past\is{habitual past}.

\ea
\textit{her waçênî beçkêwen gîyaniş çenîn her ane!} \\ 
\gll her waç-ên-î beçk(e)-êwe=n gîyan=iş çenî=n her ane \\ 
 just say\textsc{.prs-aug-2sg:A} baby\textsc{-indf=cop.3sg.m:S} soul\textsc{.m=3sg:R} in\textsc{=cop.3sg.m:S} just that\_much \\ 
\glt `One would say it was a child who was just alive, just that!' \hfill[ZQ.20]
\z

The irrealis past\is{irrealis past} is used to describe hypothetical situations in the past. In the following examples, the highlighted verbs refer to a hypothetical situation where one would go and ask for a girl's hand.

\ea
\textit{luwênî law î kuřî waçênî, `da luwe waçe eđêş bizane î kinaçêşe miđom pene ya ne!’} \\ 
\gll \textbf{lu-ên-î} la-û î kuř-î \textbf{waç-ên-î} da lu-e wáç-e eđê=ş bi-zan-e î kinaçê=ş=e mi-đ(e)-o=m pene ya ne \\
 go\textsc{.prs-aug-2sg:S} to\textsc{-ez.gen} \textsc{dem.prox} boy\textsc{.m-sg.obl} say\textsc{.prs-aug-2sg:A} \textsc{hort} go.\textsc{prs.imp-2sg:S} say\textsc{.prs.imp-2sg:A} mother\textsc{.f.sg.obl=3sg:PSR} \textsc{imp-}know\textsc{.prs-2sg:A} \textsc{dem.prox} daughter\textsc{.f.sg=3sg:PSR=dem} \textsc{ind-}give\textsc{.prs-3sg:A=1sg:R} to or no \\ 
\glt `[Let us say you said, `I want that certain person (i.e., girl).'] You \textbf{would go} to this boy [seated next to the narrator] and \textbf{would say}, `Go [and] tell her mother [about me], see (lit. know) if she gives me her daughter or not!’' \hfill[JE.74]
\z 

\subsection{Summary of TAM categories derived from the present stem}
Table \ref{tab:tam-prs} summarises the verbal forms derived from the present stem for the verbs `sleep' (\textsc{intr}) and `do' (\textsc{tr}) inflected in the first person. 
\begin{table}
\resizebox{.94\textwidth}{!}{%
\begin{tabular}{lll}
\lsptoprule
TAM category & Inflection & Gloss \\
\midrule
Present subjunctive\is{present subjunctive} & \textit{b-ûs-û}& [\textsc{sbjv}-sleep.\textsc{prs-1sg:S}] \\
Imperative & \textit{b-ûs-e}& [\textsc{imp}-sleep.\textsc{prs-2sg:S}] \\
Present indicative\is{present indicative}& \textit{m-ûs-û}& [\textsc{ind}-sleep.\textsc{prs-1sg:S:S}] \\
Present progressive\is{present progressive}& \textit{m-ûs-ay m-ûs-û}&[\textsc{ind}-sleep.\textsc{prs-nmlz} \textsc{ind}-sleep.\textsc{prs-1sg:S}] \\
Past progressive\is{past progressive}& \textit{wis-ay wis-ên-a}&[sleep.\textsc{prs-nmlz} go.\textsc{prs-aug-1sg:S}] \\
Habitual past\is{habitual past} & \textit{wis-ên-a}&[sleep.\textsc{prs-aug-1sg:S}] \\
Irrealis past\is{irrealis past}& \textit{wis-ên-a}&[sleep.\textsc{prs-aug-1sg:S}] \\
\\
Present subjunctive\is{present subjunctive} & \textit{kér-û}& [do.\textsc{prs.sbjv-1sg:A}] \\
Imperative & \textit{kér-e}&[do.\textsc{prs.imp-2sg:A}] \\
Present indicative\is{present indicative}& \textit{ker-\stackunder[-10pt]{\^{u}}{\'{}}}& [do.\textsc{prs.ind-1sg:A}] \\
Present progressive\is{present progressive}& \textit{ker-ay ker-û}&[do.\textsc{prs-nmlz} do.\textsc{prs.ind-1sg:A}] \\
Past progressive\is{past progressive}& \textit{ker-ay ker-ên-a}&[do.\textsc{prs-nmlz} do.\textsc{prs-aug-1sg:A}] \\
Habitual past\is{habitual past} & \textit{ker-ên-a}&[do.\textsc{prs-aug-1sg:A}] \\
Irrealis past\is{irrealis past}& \textit{ker-ên-a}&[do.\textsc{prs-aug-1sg:A}] \\
\lspbottomrule
\end{tabular}}
    \caption{TAM categories derived from the present stem--summary}
    \label{tab:tam-prs}
\end{table}


\section{TAM categories derived from past stem}
A number of tense-aspect-mood constructions are built from the past stem of the verb. In this section, I enumerate the formal makeup of these TAM distinctions and their functions. The TAM categories built on the past stem of the verb all exhibit ergative alignment\is{ergative alignment} on the verb, carried out by inflectional person/number suffixes agreeing with the intransitive subject (S) and transitive object (O). The transitive subject is indexed by clitic pronouns, though under some circumstances, it is not indexed at all (see \S\ref{sect:differential-A-indexing} for details). 

\subsection{Past perfective\is{past perfective}/preterite}
The past perfective\is{past perfective} is constructed using the past stem of the verb plus appropriate bound person markers. With intransitive verbs, the relevant person indices are inflectional suffixes. With transitive stems, the relevant endings are clitic pronouns; see Table \ref{tab:pst.perfective}. 

\begin{table}
    \begin{tabular}{llllll}
\lsptoprule
 & \textit{merđey} & &\textit{arđey}& \\
 \midrule
\textsc{1sg} &\textit{mérđ-a} & `I died' & \textit{árđ-∅=im}& `I brought it (\textsc{m}).'\\
\textsc{2sg} &\textit{mérđ-î} & &\textit{árđ-∅=it}& `You brought it (\textsc{m}).'\\
\textsc{3sg.m} &\textit{mérđ-∅} & &\textit{árđ-∅=iş} & `He/she brought it (\textsc{m}).'\\
\textsc{3sg.f} &\textit{mérđ-e}  & &\textit{árđ-e=iş} & `He/she brought it (\textsc{f}).'\\
\textsc{1pl} &\textit{mérđ-îmê} & &\textit{árđ-∅=ma}& `We brought it (\textsc{m}).'\\
\textsc{2pl} &\textit{mérđ-îdê} & &\textit{árđ-∅=ta}&`You brought it (\textsc{m}).'\\
\textsc{3pl} &\textit{mérđ-ê} & &\textit{árđ-∅=şa}&`They brought it (\textsc{m}).'\\
    \lspbottomrule
    \end{tabular}
    \caption{The past perfective\is{past perfective}/preterite--sample paradigms}
    \label{tab:pst.perfective}
\end{table}

As seen in Table \ref{tab:pst.perfective}, S and A are indexed by different paradigms of person endings. Transitive objects (O) are indexed the same as S. When combined with the clitic pronoun indexing the A argument, the ordering is V-O=A, irrespective of the person of the O; see Table \ref{tab:pastpfv-ard}. 
%

\begin{table}
\fittable{\begin{tabular}{lllll}
\lsptoprule
O suffix & \textsc{3pl} A & Gloss & \\\midrule
\textsc{1sg} &\textit{arđ-a=şa} & [bring\textsc{.pst-1sg:O=3pl:A}]& `they brought me' \\
\textsc{2sg} &\textit{arđ-î=şa} & [bring\textsc{.pst-2sg:O=3pl:A}]& `they brought you' \\
\textsc{3sg.m} &\textit{arđ-∅=şa} & [bring\textsc{.pst-3sg.m:O=3pl:A}]& `they brought him' \\
\textsc{3sg.f} &\textit{arđ-e=şa} & [bring\textsc{.pst-3sg.f:O=3pl:A}]& `they brought her' \\
\textsc{1pl} &\textit{arđ-îmê=şa} & [bring\textsc{.pst-1pl:O=3pl:A}]& `they brought us' \\
\textsc{2pl} &\textit{arđ-îdê=şa} & [bring\textsc{.pst-2pl:O=3pl:A}]& `they brought you' \\
\textsc{3pl} &\textit{arđ-ê=şa} & [bring\textsc{.pst-3pl:O=3pl:A}]& `they brought them' \\ 
\lspbottomrule
\end{tabular}}
\caption{Past perfective--the inflection of \textit{arđey} `bring'}
    \label{tab:pastpfv-ard}
\end{table}


The negation of past perfective\is{past perfective} is expressed by \textit{ne-}:

\ea
\textit{maço, `nezanam.'} \\ 
\gll m-aç-o ne-zana=m \\ 
 \textsc{ind-}say\textsc{.prs-3sg} \textsc{neg-}know\textsc{.pst}\textsc{=1sg:A} \\ 
\glt `He (the man) said, `I didn’t understand [his point].’' \hfill[JH.26]
\z 

The past perfective\is{past perfective} is used to express specific time-bound events (i.e., completed events) at a particular time in the past. 
%

\newpage
\ea
\textit{duwê beçkêş dîyê, zarowê.} \\ 
\gll duwê beçk(e)-ê=ş dî-ê zaro-ê \\ 
 two baby\textsc{-pl.dir=3sg:A} see\textsc{.pst-3pl:O} child\textsc{-pl.dir} \\ 
\glt `She gave birth to two babies.' \hfill[ZQ.15]
\z 
 
The past perfective\is{past perfective} may be used to refer to sequential time-bound events in a narrative:
%

\ea
\textit{gêɫanê qiřoɫe darêm yoso. berdim nîyamne qiřoɫe dareke.} \\ 
\gll gêɫa-(a)nê qiřoɫ-e dar-ê=m yos=o \textbf{berd-{\O}=im} \textbf{nîya-{\O}=m=ne} qiřoɫ-e dar-eke \\ 
 wander\textsc{.pst-1sg:S} hollow\textsc{.m-ez.cmpd} tree\textsc{.m-indf}{=\textsc1sg} find\textsc{.pst=compl} take\textsc{.pst-3sg.m:O=1sg:A} put\textsc{.pst-3sg:O=1sg:A=povb} hollow\textsc{.m-ez.cmpd} tree\textsc{.m-def.m.sg.dir} \\
\glt `I wandered around [and] found a tree hollow. I took [him] and put him in the tree hollow.' \hfill[ZQ.23--ZQ.24]
\z 
 
 The past perfective\is{past perfective} may express an action that has a starting and end point in the past but lasted for a long period. In (\ref{ex.pst-perf1}), `raising children' would have lasted several years.

\ea
\textit{pase zawlêşa wey kerdê.} \\ 
\gll pase zawlê=şa wey kerd-ê \\ 
 like\_this child\textsc{.pl.dir=3pl:A} raising do\textsc{.pst-3pl:O} \\ 
\glt `They raised children in this way.' \hfill[JE.55] \label{ex.pst-perf1}
\z 

The extended period of time may overlap with other events in the surrounding discourse. In (\ref{ex.pst-perf2}), the adverbial phrase `when I got married' sets the frame for all the events relating to the period after marriage.

\ea
\textit{wextê jenîm arde, yewaşê yanem nebê. cîya bîyanê. ca zəmsan bê. cîya bîya. luwanê hîçim nebê. çenû jenî luwaymê yanema gêrt kirahe.} \\
\gll wext-ê jenî꞊m ard-e yewaşê yane=m ne-b-ê cîya bî-anê. ca zəmsan b-ê. cîya bî-a luwa-(a)nê hîç=im ne-b-ê çen(î)-û jenî luwa-îmê yane꞊ma gêrt-∅ kirahe \\
when-\textsc{indf} woman꞊\textsc{1sg:A} bring.\textsc{pst-3sg.f:O} well house꞊\textsc{1sg:NC} \textsc{neg-}be.\textsc{prs-aug.3sg:S} separate be.\textsc{pst-1sg:S} \textsc{disc.ptcl} winter be\textsc{-prs.aug.3sg:S} separate be.\textsc{pst-1sg:S} go.\textsc{pst-1sg:S} nothing꞊\textsc{1sg:NC} \textsc{neg-}be.\textsc{prs-aug.3sg:S} with woman go.\textsc{pst-1pl:S} house\textsc{dir.m=1pl:A} take.\textsc{pst-3sg.m:O} rent \\
\glt `When I got married (I took a wife), well, I didn’t have a house. I left my father's house (lit. I became separate). It was winter. I left the family of my father, and I went away. I did not have anything. Together with my wife, we rented a house.’ \hfill\citep[309, glossing and transcription modified]{khan_language_2023}{} \label{ex.pst-perf2}
\z 

The perfective is also used for the expression of a time-bound event in the immediate past, corresponding to the English\il{English} perfect of recent past/hot news perfect. In the following example, the narrator witnesses a guest coming through the door and asks whether the recording should continue.

\ea
\textit{dey aneyç ama mêmana çêş kermê?}\\
\gll dey ane=yç \textbf{ama-\O} mêman=a çêş kér-mê\\
\textsc{disc.ptcl} \textsc{dem.dist.3sg.dir.m=add} come.\textsc{pst-3sg:S} guest=\textsc{cop.3sg:S} what do.\textsc{prs.sbjv-1sg:A}\\
\glt `Oh, he has arrived. He is guest. What should we do? [Should we keep recording?]' \hfill[HR.59]
\z 

The perfective may express a completed action with a projected future sense. In this usage, the perfective occurs in a subordinate clause\is{subordinate clause}, whether syndetic (\ref{ex.pst-perf3}) or asyndetic (\ref{ex.pst-perf4}). 
%

\ea
\textit{eger goşiş darayne dûr dûr kewto.} \\ 
\gll eger goş=iş dara-î=ne dûr \textbf{dûr} \textbf{kewt{-\O}=o} \\ 
 if ear\textsc{.m=3sg:A} hold\textsc{.pst-2sg:R=povb} far far fall\textsc{.pst-3sg.m:S=compl} \\ 
\glt `If they [lit. he] listen to you, they will go away.' \hfill[BP.163] \label{ex.pst-perf3}
\z 


\ea
\textit{dey êşew herkes mêmaniş hen, êşew herkes mêmaniş hen, mêmanekeş şewê witê, sereş biřo.} \\ 
\gll dey êşew herkes mêman=iş hen-∅ êşew herkes mêman=iş hen-∅ \textbf{mêman-eke=ş} \textbf{şew(e)-ê} \textbf{wit-ê} sere=ş bíř-o \\
 \textsc{disc.ptcl} tonight everyone guest\textsc{.m=3sg:NC} \textsc{exist-3sg.m:S} tonight everyone guest\textsc{.m=3sg:NC} \textsc{exist-3sg.m:S} guest\textsc{.m-def.m.sg.dir=3sg:PSR} night\textsc{-f.sg.obl} sleep\textsc{.pst-3pl:S:S} head\textsc{.m=3sg:O} cut\textsc{.prs.sbjv-3sg:A} \\ 
\glt `Tonight, whoever has guests, \textbf{[when] the guest sleeps at night}, he shall decapitate him.' \hfill[BP.52] \label{ex.pst-perf4}
\z 

\subsection{Past conditional}
The past conditional\is{past conditional} is formed by attaching the conditional affix \textit{-εn} to the past stem of the verb, followed by set 2 inflectional person suffixes. \citet[34]{mackenzie_dialect_1966}{} argues that \textit{-εn} is presumably derived from the conditional infix \textit{-a} plus the augment \textit{-ên}, hence \textit{-εn} <*\textit{-a} + \textit{-ên}.

\begin{table}
\fittable{\begin{tabular}{llll}
\lsptoprule
  S& Gloss& \\\midrule
\textsc{1sg}& \textit{yáw(a)-εn-ê}& [arrive\textsc{.pst-cond.aug-1sg:S}]& `(if) I had arrived'\\
\textsc{2sg}& \textit{yáw(a)-εn-î}& [arrive\textsc{.pst-cond.aug-2sg:S}]& `(if) you had arrived'\\
\textsc{3sg}& \textit{yáw(a)-ε}& [arrive\textsc{.pst-cond.aug.3sg:S}]& `(if) he/she had arrived'\\
\textsc{1pl}& \textit{yáw(a)-εn-mê}& [arrive\textsc{.pst-cond.aug-1pl:S}]& `(if) we had arrived'\\
\textsc{2pl}& \textit{yáw(a)-εn-dê}& [arrive\textsc{.pst-cond.aug-2pl:S}]& `(if) you had arrived'\\
\textsc{2pl}& \textit{yáw(a)-εn-ê}& [arrive\textsc{.pst-cond.aug-3pl:S}]& `(if) they had arrived' \\
\lspbottomrule
\end{tabular}}
\caption{Past conditional--the inflection of `arrive'}
    \label{tab:pastcint}
\end{table}

\begin{table}
\fittable{\begin{tabular}{lllll}
\lsptoprule
O suffix & \textsc{3pl} A & Gloss & \\\midrule
\textsc{1sg} &\textit{arđ-εn-ê=şa}& [bring\textsc{.pst-cond.aug-1sg:O=3pl:A}]& `(if) they had brought me' \\
\textsc{2sg} &\textit{arđ-εn-î=şa}& [bring\textsc{.pst-cond.aug-2sg:O=3pl:A}]& `(if) they had brought you' \\
\textsc{3sg} &\textit{arđ-ε=şa}& [bring\textsc{.pst-cond.aug.3sg:O=3pl:A}]& `(if) they had brought her/him' \\
\textsc{1pl} &\textit{arđ-εn-mê=şa}& [bring\textsc{.pst-cond.aug-1pl:O=3pl:A}]& `(if) they had brought us' \\
\textsc{2pl} &\textit{arđ-εn-dê=şa}& [bring\textsc{.pst-cond.aug-2pl:O=3pl:A}]& `(if) they had brought you' \\
\textsc{3pl} &\textit{arđ-εn-ê=şa}& [bring\textsc{.pst-cond.aug-3pl:O=3pl:A}]& `(if) they had brought them' \\
\lspbottomrule
\end{tabular}}
\caption{Past conditional--the inflection of \textit{arđey} `bring'}
    \label{tab:pastctr}
\end{table}

The negation of the past conditional\is{past conditional} is expressed by \textit{ne-}:

\newpage
\ea
\textit{eger řezaşa nebî{ɛ} neđênê.} \\ 
\gll eger řeza=şa ne-bî-{ɛ} ne-đ(e)-ên-ê \\ 
 if satisfaction\textsc{.f=3pl:NC} \textsc{neg-}be\textsc{.pst.cond.aug.3sg:S} \textsc{neg-}give\textsc{.prs-aug-3pl:A} \\ 
\glt `If they didn’t agree, they wouldn’t give her.' \hfill[JE.85]
\z


\ea
\textit{eger minit çene bîy{ɛ}nê} \\
\gll eger min=it çene bî-{ɛ}n-ê \\
if \textsc{1sg=2sg:R} with be.\textsc{pst.cond.aug-1sg:S} \\
\glt `If you had me with you ..' \hfill[PW.88]
\z 

The past conditional\is{past conditional} expresses hypothetical situations in the past.

\ea
\textit{a wextî to mîsal wat{ɛ}t, `a fiɫane kesem gerekene.’} \\ 
\gll a wext-î to mîsal \textbf{wat-{ɛ}=t} a fiɫan-e kese=m gerek-e=ne \\ 
 \textsc{dem.dist} time\textsc{.m-sg.obl} \textsc{2sg} for\_example say\textsc{.pst-cond.aug=2sg:A} \textsc{dem.dist} such\_and\_such\textsc{-ez.cmpd} person\textsc{.f=1sg:NC} necessary\textsc{-f=cop.3sg.f:S} \\ 
\glt `In earlier times, \textbf{let us say (lit. for instance.) you said}, `I want that certain person (i.e., girl).’' \hfill[JE.73]
\z 

\largerpage
\ea
\textit{eger kinaçekêşa don{ɛ} wat{ɛ}ş, `erê, kerû,’} \\ 
\gll eger kinaç(ê)-ekê=şa \textbf{don{ɛ}} \textbf{wat-{ɛ}=ş} erê ker-û \\ 
 if girl\textsc{.f-def.f.sg=3pl:A} talk\_to\textsc{.pst.cond.aug.3sg:R} say\textsc{.pst.cond.aug=3sg:A} yes do\textsc{.prs.ind-1sg:A} \\ 
\glt `After they \textbf{had talked to the girl} [and] \textbf{she had said}, `Yes, I will [marry him],’' \hfill[JE.77]
\z 

\subsection{Perfect}
The perfect\is{perfect} is expressed by combining the resultative participle with the copula PMs. The participle agrees in gender\is{gender agreement} and number\is{number agreement} with the underlying S and O. In Table \ref{tab:perf}, the participle forms for the verb `to die' is \textit{merde} [die.\textsc{pst.ptcp.m.sg}], \textit{merdê} [die.\textsc{pst.ptcp.f.sg}], and \textit{merdê} [die\textsc{.pst.ptcp.pl}]. Similarly, the copula PMs agree with the O and S argument in person. The clitic pronouns express the A argument. 

\begin{table}

    \begin{tabular}{ll@{\qquad}ll@{}l}
      \lsptoprule
 S/O& \textit{merđey} `die'& \textit{arđey} `bring'\\
 \midrule
\textsc{1sg.m} &\textit{merđe=na}& \textit{arde=na=ş}& [bring.\textsc{pst.ptcp.m=cop.3sg:O=3sg:A}]\\
\textsc{1sg.f} &\textit{merđê=na} & \textit{ardê=na=ş}& [bring.\textsc{pst.ptcp.f=cop.3sg:O=3sg:A}]\\
\textsc{2sg.m} &\textit{merđe=nî} & \textit{arde=nî=ş}&\\
\textsc{2sg.f} &\textit{merđê=nî} & \textit{ardê=nî=ş}\\
\textsc{3sg.m} &\textit{merđe=n} & \textit{arde=n=iş}\\
\textsc{3sg.f} &\textit{merđê=ne} & \textit{ardê=ne=ş}\\
\textsc{1pl} &\textit{merđê=nmê} & \textit{ardê=nmê=ş}\\
\textsc{2pl} &\textit{merđê=ndê}& \textit{ardê=ndê=ş} \\
\textsc{3pl} &\textit{merđê=nê}& \textit{ardê=nê=ş} \\  
    \lspbottomrule
     \end{tabular}
    \caption{The perfect\is{perfect}--sample paradigms}
    \label{tab:perf}   
\end{table}

The following examples parse the perfect\is{perfect} verb forms. In (\ref{ex.perf1}), both the participle and the copula agree with the S argument. In (\ref{ex.perf3})--(\ref{ex.perf4}) they agree with the O argument.

\ea
\textit{bizêw menêne cîyay sayqekewe.} \\ 
\gll biz(e)-êw \textbf{menê=ne} cîyay sayqe-(e)ke=we \\ 
 goat\textsc{.f-indf} remain\textsc{.pst}\textsc{.ptcp.f=cop.3sg.f:S} behind lightning\textsc{.m-def.m.sg.dir=post} \\ 
\glt `a goat was [accidentally] left behind [healthy] from the lightning [that caused the flood]' \hfill[ZB.42] \label{ex.perf1}
\z 


\ea
\textit{î dêwênê î kinaçêşa bestêne.} \\ 
\gll î dêw-ê=nê î \textbf{kinaçê}=şa \textbf{bestê=ne} \\ 
 \textsc{dem.prox} ogre\textsc{.m-pl.dir=cop.3pl:S} \textsc{dem.prox} girl\textsc{.f.sg=3pl:A} tie\textsc{.pst.ptcp.f=cop.3sg.f:O} \\ 
\glt `It was the ogres who had muted the girl.' \hfill[JP.177]  \label{ex.perf3}
\z 


\ea
\textit{hewarêşa wişkin{ɛ}nê.} \\ 
\gll hewar-ê=şa wişkin{ɛ}=nê \\ 
 summer\_habitat\textsc{.m-pl.dir=3pl:A} scour\textsc{.pst.ptcp.pl=cop.3pl:O} \\ 
\glt `They scoured the summer habitats [searching for food etc.].' \hfill[JE.3]  \label{ex.perf4}
\z 

The negation of the perfect\is{perfect} is expressed by \textit{ne-}:


\ea
\textit{dizîm nekerđenû hîzîm nekerdenû. girđkar nebîyena.} \\ 
\gll dizî=m ne-kerđe=n=û hîzî=m ne-kerde=n=û girđkar {} ne-bîye=na \\ 
 theft\textsc{.m.sg=1sg:A} \textsc{neg-}do\textsc{.pst.ptcp.m=cop.3sg.m:O}=and adultery\textsc{.m.sg=1sg:A} \textsc{neg-}do\textsc{.pst.ptcp.m=cop.3sg.m:O}=and know\_\_it\_all {} \textsc{neg-}be\textsc{.pst.ptcp.m=cop.1sg:S} \\  
\glt `I have not committed theft or adultery. I was not a know-it-all.' \hfill[JM.14]
\z 

The perfect\is{perfect} refers to a situation that has come about as a result of an action in the past.

\ea
\textit{padşay kerdena wekêɫ.} \\ 
\gll padşa-î kerde=na wekêɫ \\ a
 king\textsc{.m-sg.obl} do\textsc{.pst.ptcp.m=cop.1sg:O} advocate\textsc{.m} \\ 
\glt `The king has put me in charge.' \hfill[ZP.99]
\z 

One salient function of the perfect\is{perfect} is its use to refer to habitual\is{habitual} actions in the far past. The perfect\is{perfect} may be used here to refer to imperfective habitual activities and perfective events alike. This function of perfect\is{perfect} may be considered ``narrative perfect\is{narrative perfect}''. An entire narrative may be built on this function of the perfect\is{perfect}.
%

\ea
\textit{dey çêgeyç qeyîm ʕaɫifşa kennenû. ħeywanşa wey kerdenû. hewarêşa wişkin{ɛ}nê. î dega toş vînî çoɫe bîyêne. ħîç nebîyen.} \\ 
\gll dey çêge=îç qeyîm ʕaɫif=şa \textbf{kenne=n}=û ħeywan=şa \textbf{wey} \textbf{kerde=n}=û hewar-ê=şa \textbf{wişkin{ɛ}}=nê î dega to=ş vîn-î çoɫ-e \textbf{bîyê=ne} ħîç \textbf{ne-bîye=n}\\ 
\textsc{disc.ptcl} here\textsc{=add} old\_time grass\textsc{.m.sg.dir=3pl:A} mow\textsc{.pst.ptcp.m=cop.3sg.m:O}=and animal\textsc{.m.sg.dir=3pl:A} raising do\textsc{.pst.ptcp.m=cop.3sg.m:O}=and summer\_habitat\textsc{.m-pl.dir=3pl:A} scour\textsc{.pst.ptcp.pl=cop.3pl:O} \textsc{dem.prox} village{\textsc{.f}} \textsc{2sg=3sg:O} see\textsc{.prs.ind-2sg:A} deserted-\textsc{f} be\textsc{.pst.ptcp.f=cop.3sg.f:S} nothing \textsc{neg-}be\textsc{.pst.ptcp.m=cop.3sg.m:S} \\
\glt `In the past, they (people) \textbf{mowed} grass. They \textbf{raised} animals. They \textbf{scoured} the summer habitats [searching for food etc.]. This village, which you see, \textbf{was} deserted. There \textbf{was} nothing [here].' \hfill[JE.1--JE.5]
\z 

The perfect\is{perfect} may additionally be used to express an event that the speaker has not witnessed, but is hearsay. This is the evidentiality\is{evidentiality} function of the perfect\is{perfect}, which is also found in neighbouring languages such as Persian\il{Persian}, Turkish\il{Turkish}, Armenian\il{Armenian} \citep[]{lazard_grammaticalization_2001}{}, and in Neo-Aramaic\il{Neo-Aramaic} \citep[]{khan_perfect_2020}{}.

\ea
\textit{heta min jinyenim pîyewe ama xizmetû şê ʕeladînî.} \\ 
\gll heta min \textbf{jinye=n=im} pîye-(ê)we ama xizmet-û şê ʕeladîn-î \\ 
 even \textsc{1sg} hear\textsc{.pst.ptcp.m=cop.3sg.m:O=1sg:A} man\textsc{.m-indf} come\textsc{.pst.3sg:S} service\textsc{.m-ez.gen} sheikh\textsc{.m} \textsc{pn-m.sg.obl} \\ 
\glt `I have even heard that a man came to the service of Sheikh Aladin.' \\\hfill[ZQ.2]
\z 

Another context of the evidential\is{evidential} function of the perfect\is{perfect} is its use in inferential contexts. In (\ref{ex.perf5}), the context is one in which a mute girl starts to speak. The narrator infers from this evidence that the reason the girl was mute was because the ogres had muted her. 

\ea
\textit{î dêwênê î kinaçêşa bestêne.} \\ 
\gll î dêw-ê=nê î kinaçê=şa \textbf{bestê=ne} \\ 
 \textsc{dem.prox} ogre\textsc{.m-pl.dir=cop.3pl:S} \textsc{dem.prox} girl\textsc{.f.sg=3pl:A} tie\textsc{.pst.ptcp.f=cop.3sg.f:O} \\  
\glt `It was the ogres who had muted the girl.' \hfill[JP.177] \label{ex.perf5}
\z 

The perfect\is{perfect} may refer to legendary events the speaker has only learned about from reports. This is another instance of the evidential\is{evidential} function of the perfect\is{perfect} since the speaker has not witnessed the event himself.

\ea
\textit{ca padşakey waten be lalowe, ew lalowe kinaçekê.} \\ 
\gll ca padşa-(e)ke-î \textbf{wate=n} be lalo-e ew lalo=e kinaç(ê)-ekê \\ 
 afterwards king\textsc{-def.m.sg.obl} say\textsc{.pst.ptcp.m=cop.3sg.m:O} to maternal\_uncle\textsc{.m-def} \textsc{dem.dist} maternal\_uncle\textsc{.m=dem} daughter\textsc{.f-def.f.sg} \\ 
\glt `Oh, the king had said to the uncle, to [his] daughter’s uncle.' \hfill[ZP.43]
\z 

Related to expressing evidentiality\is{evidentiality}, the perfect\is{perfect} may be used to express mirativity\is{mirativity}, i.e., ``marking statements based on inference and statements based on direct experience for which the speaker had no psychological preparation'' \citep[35--36]{delancy_mirativity_1997}{}. In (\ref{ex.perf6}), the perfect\is{perfect} marks the speaker's unprepared mind and his surprise that he has just witnessed the donkey starting to talk.

\ea
\textit{`her ta îse qisêş nekerdênê.'} \\ 
\gll her ta îse qisê=ş ne-kerdê=nê \\ 
 donkey until now talk\textsc{.pl.dir=3sg:A} \textsc{neg-}do\textsc{.pst.ptcp.pl=cop.3pl:O} \\  
\glt `[He said surprisingly], `the donkey hadn’t talked until now!’' \hfill[HB.46] \label{ex.perf6}
\z 
 
\subsection{Perfect progressive\is{perfect progressive}}
The perfect progressive\is{perfect progressive} is built by a reduplicated construction\is{reduplicated construction} consisting of an inflected verbal form in the perfect\is{perfect} preceded by a double comprising the past stem and the suffix \textit{-î}.

\begin{table}

    \begin{tabular}{lll}
\lsptoprule
 S & \textit{kewtey} `to fall'& \\
 \midrule
\textsc{1sg.m} &\textit{kewt-î kewte=na}& `I (\textsc{m}) have been/had been falling'\\
\textsc{1sg.f} &\textit{kewt-î kewtê=na} & `I (\textsc{f}) have been/had been falling'\\
\textsc{2sg.m} &\textit{kewt-î kewte=nî} \\
\textsc{2sg.f} &\textit{kewt-î kewtê=nî} \\
\textsc{3sg.m} &\textit{kewt-î kewte=n} \\
\textsc{3sg.f} &\textit{kewt-î kewtê=ne} \\
\textsc{1pl} &\textit{kewt-î kewtê=nmê} \\
\textsc{2pl} &\textit{kewt-î kewtê=ndê} \\
\textsc{3pl} &\textit{kewt-î kewtê=nê} \\  
    \lspbottomrule
    \end{tabular}
    \caption{The perfect progressive\is{perfect progressive}--a sample paradigm}
    \label{tab:perf.progress}
\end{table}
The negation of the perfect progressive\is{perfect progressive} is expressed by \textit{ne-}.

\ea
\textit{newatîş newaten.} \\ 
\gll ne-wat-î=ş ne-wate=n \\ 
 \textsc{neg-}say.\textsc{pst-nmlz=3sg:A} \textsc{neg-}say.\textsc{pst.ptcp.m=cop.3sg:O} \\ 
\glt `He has not been saying [what was not to be told].' \hfill[hearsay]
\z 

The following example illustrates the use of perfect progressive\is{perfect progressive} in an interrogative clause.

\ea
\textit{maço, `çî amay amêndê?'} \\ 
\gll m-aç-o çî amay amê=ndê \\ 
 \textsc{ind-}say\textsc{.prs-3sg:A} why come\textsc{.nmlz} come\textsc{.pst.ptcp.pl=cop.2pl:S} \\ 
\glt `He said, ‘Why have you come [here]?’' \hfill[ŞC.35]
\z 


The perfect progressive\is{perfect progressive} may refer to habitual\is{habitual} actions in the far past for which the speaker has learned only from reports. This use of perfect progressive\is{perfect progressive} occurs primarily in folktales, exhibiting the evidential\is{evidential} function of perfect progressive\is{perfect progressive}.

\ea
\textit{î kabr{ɛ}çe ce ʕêraqo am{ɛ}nê, wêreganew nîmeřonew seʕbne bexşnayşa bexşn{ɛ}nêwe.} \\ 
\gll î kabr{ɛ}=ç=e ce ʕêraq=o am{ɛ}=nê wêrega=ne=û nîmeřo=ne=û seʕb=ne \textbf{bexşnay=şa} \textbf{bexşn{ɛ}=nê}=we \\ 
 \textsc{dem.prox} man\textsc{.pl.dir=add=dem} from \textsc{pn=post} come\textsc{.pst.ptcp.pl=cop.3pl:S} evening\textsc{=post}=and noon\textsc{=post}=and morning\textsc{=post} distribute\textsc{.nmlz=3pl:A} distribute\textsc{.pst.ptcp.pl=cop.3pl:R=compl} \\ 
\glt `They (people) would donate [food] to the fellows (the tax collectors) who had come from Iraq, in the evenings, mornings, and at noon.'\\
\hfill[BP.38]
\z 

Relatedly, the perfect progressive\is{perfect progressive} may express far past events that the speaker has not witnessed, but the events have some personal significance for the speaker. This might be called the ``experiential perfect progressive\is{experiential perfect progressive}''. The events in question were occurring continually but were completed at some point. In the following excerpt, the speaker talks about his life a long time ago when he was away from home and did not know what was happening to his children during his absence.

\ea
\textit{min eçagene karîger bîyena. şiş mangê xeberêm nezanan. bizanû kewtî kewtênê warđîşa warden dizîş dizîyen.} \\ 
\gll min e=çagene karîger bîye=na şiş mang(e)-ê xeber-ê=m ne-zana=n bi-zan-û \textbf{kewt-î} \textbf{kewtê=nê} \textbf{warđ-î=şa} \textbf{warde=n} \textbf{diz(î)-î=ş} \textbf{dizîye=n}\\ 
\textsc{1sg} in=there labourer\textsc{.m} be\textsc{.pst.ptcp.m=cop.1sg:S} six month\textsc{.f-pl.dir} news\textsc{.m-indf=1sg:A} \textsc{neg-}know\textsc{.pst.ptcp.m=cop.3sg.m:PSR} \textsc{sbjv-}know\textsc{.prs-1sg:A} fall\textsc{.pst-nmlz} fall\textsc{.pst.ptcp.pl=cop.3pl:S} eat\textsc{.pst-nmlz=3pl:A} eat\textsc{.pst.ptcp.m=cop.3sg.m:O} steal\textsc{.pst-nmlz=3sg:A} steal\textsc{.pst.ptcp.m=cop.3sg.m:O} \\
\glt `I was a worker there. I was unaware of them (lit. I didn’t know their news.) (my children) for six months. I [was not around to] witness [if] they (the children) had fallen, [if] they had eaten, or stolen [something].' \\
\hfill[JM.28--JM.29]
 \z 
 
\subsection{Irrealis perfect\is{irrealis perfect}}
The irrealis perfect\is{irrealis perfect} is constructed by combining the participle with the subjunctive form of the verb `to be'. With intransitive verb forms, the participle agrees in gender\is{gender agreement} and number\is{number agreement} with the S, and the verb `to be' agrees in person with the S (see Table \ref{tab:irrprf-fall}).

\begin{table}
\fittable{\begin{tabular}{llll}
\lsptoprule
  S& & Gloss& \\\midrule
\textsc{1sg.m} &\textit{kewte=b-û} & [fall\textsc{.pst.ptcp.m}=be.\textsc{prs-1sg:S}]& `I (\textsc{m}) may have fallen' \\
\textsc{1sg.f} &\textit{kewtê=b-û} & [fall\textsc{.pst.ptcp.f}=be\textsc{.prs-1sg:S}]& `I (\textsc{f}) may have fallen' \\
\textsc{2sg.m} &\textit{kewte=b-î} & & `you (\textsc{m}) may have fallen' \\
\textsc{2sg.f} &\textit{kewtê=b-î} & & `you (\textsc{f}) may have fallen' \\
\textsc{3sg.m} &\textit{kewte=b-o} & & `he may have fallen' \\
\textsc{3sg.f} &\textit{kewtê=b-o} & & `she may have fallen' \\
\textsc{1pl} &\textit{kewtê=b-îmê} & & `we may have fallen' \\
\textsc{2pl} &\textit{kewtê=b-îdê} & & `you may have fallen' \\
\textsc{3pl} &\textit{kewtê=b-a} & & `they may have fallen' \\
\lspbottomrule
\end{tabular}}
\caption{Irrealis perfect--the inflection of `fall'}
    \label{tab:irrprf-fall}
\end{table}

In the transitive irrealis perfect\is{irrealis perfect}, number\is{number agreement} and gender agreement\is{gender agreement} with O is carried out by the participle and person agreement with O is carried out by set 1 inflectional suffixes on the verb `to be'. On the other hand, the A argument is expressed by the clitic pronouns (see Table \ref{tab:irrprf-see}). The negator \textit{ne-} marks the negation of irrealis perfect\is{irrealis perfect}.

\begin{table}
\fittable{\begin{tabular}{llll}
\lsptoprule
O& \textsc{3pl} A& Gloss& \\\midrule
\textsc{1sg.m} &\textit{dîye=b-û=şa} & [see\textsc{.pst.ptcp.m}=be.\textsc{prs-1sg:O=3pl:A}]& `they may have seen me (\textsc{m})' \\
\textsc{1sg.f} &\textit{dîyê=b-û=şa} & [see\textsc{.pst.ptcp.f}=be.\textsc{prs-1sg:O=3pl:A}]& `they may have seen me (\textsc{f})' \\
\textsc{2sg.m} &\textit{dîye=b-î=şa} & & `they may have seen you (\textsc{m})' \\
\textsc{2sg.f} &\textit{dîyê=b-î=şa} & & `they may have seen you (\textsc{f})' \\
\textsc{3sg.m} &\textit{dîye=b-o=şa} & & `they may have seen him' \\
\textsc{3sg.f} &\textit{dîyê=b-o=şa} & & `they may have seen her' \\
\textsc{1pl} &\textit{dîyê=b-îmê=şa} & & `they may have seen us' \\
\textsc{2pl} &\textit{dîyê=b-îdê=şa} & & `they may have seen you' \\
\textsc{2pl} &\textit{dîyê=b-a=şa} & & `they may have seen them' \\
\lspbottomrule
\end{tabular}}
\caption{Irrealis perfect--the inflection of `see'}
    \label{tab:irrprf-see}
\end{table}

 
%
\newpage
\ea
\textit{herkesîç metawo pêse min ke nelabû êtir ane hîçê derameđêş nîya.} \\ 
\gll herkes=îç me-taw-o pêse min ke \textbf{ne-la=b-û} êtir ane hîç-ê derameđ-ê=ş nîy(e)=a \\ 
 anyone\textsc{=add} \textsc{neg.ind-}can\textsc{.prs-3sg:A} like \textsc{1sg} \textsc{rel} \textsc{neg-}go\textsc{.pst.ptcp.m}=be\textsc{.prs-1sg:S} \textsc{disc.ptcl} \textsc{dem.dist.m.3sg.dir} nothing\textsc{-indf} income\textsc{-indf=3sg:NC} \textsc{neg.exist=cop.3sg.m:S} \\ 
\glt `Anyone who is not able [to work as a porter], like me, who has probably not been a porter, well, he has no income.' \hfill[JM.62]
\z 

The irrealis form of the perfect expresses epistemic modality\is{epistemic modality}, meaning that the speaker is not totally committed to the truth of the action of a verb with past time reference.

\ea
\textit{maço, `ce maseket keç nekerđebo řaře waɫê!’} \\ 
\gll m-aç-o ce mas-eke=t keç \textbf{ne-kerđe=b-o} řa=ře waɫê \\ 
 \textsc{ind-}say\textsc{.prs-3sg:A} from yoghurt\textsc{.m-def.m.sg.dir=2sg:A} crooked \textsc{neg-}do\textsc{.pst.ptcp.m}=be\textsc{.prs-3sg:O} road\textsc{.f=post} sister\textsc{.f} \\ 
\glt `She (the older sister) had said [to her younger sister], `Sister, could [it be that] the [quantity of] yoghurt was reduced?’' \hfill[JH.48]
\z 

The irrealis perfect\is{irrealis perfect} may express a hypothetical situation in the far past (\ref{ex.irr-perf1}). In this usage, the irrealis perfect\is{irrealis perfect} can occur in the protasis of hypothetical conditional clauses\is{conditional clauses} (\ref{ex.irr-perf2}).
\newpage
\ea
\textit{herkam girew{ɛ}ba bînîyenim koɫîmre.} \\ 
\gll herkam \textbf{girew{ɛ}=b-a} bînîye=n=im koɫî=m=re \\ 
 whoever cry\textsc{.pst.ptcp.pl}=be\textsc{.prs-3pl:S} tie\textsc{.pst.ptcp.m=cop.3sg.m:O=1sg:A} shoulder\textsc{.f=1sg:PSR=post} \\ 
\glt `Each [of my kids] who \textbf{might have cried}, I would put on my shoulders.' \\\hfill[JE.64] \label{ex.irr-perf1}
\z 


\ea
\textit{ême eger zemanê ya qeymîyêma ya zemanû wêma jenîma ardêbo ...} \\ 
\gll ême eger zeman-ê ya qeymî-ê=ma ya zeman-û wê=ma jenî=ma \textbf{ardê=b-o} \\ 
 \textsc{1pl} if time\textsc{.m-indf} either the\_elderly\textsc{-indf=1pl:PSR} or time\textsc{.m-ez.gen} \textsc{refl=1pl:PSR} woman\textsc{.f.sg.dir=1pl:A} bring\textsc{.pst.ptcp.f}=be\textsc{.prs-3sg:O} \\ 
\glt `Once, in the time of our elders or in our time, if one of us married (lit. brought a wife) ...' \hfill[RE.2] \label{ex.irr-perf2}
\z 
\subsection{Conditional perfect}
The conditional perfect\is{conditional perfect} is built by the participle form of the verb followed by the past conditional\is{past conditional} form of the verb `to be', which consists of the past stem of the verb `be', followed by \textit{-{ɛ}n} (which may be parsed as a merger of the conditional affix \textit{-a}, and the augment \textit{-ên}), and set 2 verbal person/number affixes. With intransitive verbs, both the participle and the person endings agree with the S; see Table \ref{tab:condprf-sleep}. 

\begin{table}
\fittable{\begin{tabular}{llll}
\lsptoprule
  S& & Gloss& \\\midrule
\textsc{1sg.m} &\textit{wite=bî-{ɛ}n-ê} & [sleep\textsc{.pst.ptcp.m}=be.\textsc{pst-cond.aug-1sg:S}]& `If I (\textsc{m}) had slept' \\
\textsc{1sg.f} &\textit{witê=bî-{ɛ}n-ê} & [sleep\textsc{.pst.ptcp.f}=be.\textsc{pst-cond.aug-1sg:S}]& `If I (\textsc{f}) had slept' \\
\textsc{2sg.m} &\textit{wite=bî-{ɛ}n-î} & & `if you (\textsc{m}) had slept' \\
\textsc{2sg.f} &\textit{witê=bî-{ɛ}n-î} & & `if you (\textsc{f}) had slept' \\
\textsc{3sg.m} &\textit{wite=bî-{ɛ}} & & `if he had slept' \\
\textsc{3sg.f} &\textit{witê=bî-{ɛ}} & & `if she had slept' \\
\textsc{1pl} &\textit{witê=bî-{ɛ}n-mê} & & `if we had slept' \\
\textsc{2pl} &\textit{witê=bî-{ɛ}n-dê} & & `if you had slept' \\
\textsc{3pl} &\textit{witê=bî-{ɛ}n-ê} & & `if they had slept' \\
\lspbottomrule
\end{tabular}}
\caption{Conditional perfect--the inflection of `sleep'}
    \label{tab:condprf-sleep}
\end{table}

With transitive verbs, both the participle and the auxiliary verb `to be' agree with the O; see Table \ref{tab:condprf-see}.

\begin{table}
\fittable{\begin{tabular}{llll}
\lsptoprule
O& \textsc{3pl} A& Gloss& \\\midrule
\textsc{1sg.m} &\textit{dîye=bî-{ɛ}n-ê=şa} & [see\textsc{.pst.ptcp.m}=be.\textsc{pst-cond.aug-1sg:O=3pl:A}]& `if they had seen me (\textsc{m})' \\
\textsc{1sg.f} &\textit{dîyê=bî-{ɛ}n-ê=şa} & [see\textsc{.pst.ptcp.f}=be.\textsc{pst-cond.aug-1sg:O=3pl:A}]& `if they had seen me (\textsc{f})' \\
\textsc{2sg.m} &\textit{dîye=bî-{ɛ}n-î=şa} & & `if they had seen you (\textsc{m})' \\
\textsc{2sg.f} &\textit{dîyê=bî-{ɛ}n-î=şa} & & `if they had seen you (\textsc{f})' \\
\textsc{3sg.m} &\textit{dîye=bî-{ɛ}=şa} & & `if they had seen him' \\
\textsc{3sg.f} &\textit{dîyê=bî-{ɛ}=şa} & & `if they had seen her' \\
\textsc{1pl} &\textit{dîyê=bî-{ɛ}n-mê=şa} & & `if they had seen us' \\
\textsc{2pl} &\textit{dîyê=bî-{ɛ}n-dê=şa} & & `if they had seen you' \\
\textsc{2pl} &\textit{dîyê=bî-{ɛ}n-ê=şa} & & `if they had seen them' \\
\lspbottomrule
\end{tabular}}
\caption{Conditional perfect--the inflection of `see'}
    \label{tab:condprf-see}
\end{table}

The negation of conditional perfect\is{conditional perfect} is expressed by \textit{ne-}:

\ea
\textit{nedîyêbîy{ɛ}nmêşa.} \\ 
\gll ne-dîyê=bî-{ɛ}n-mê=şa \\ 
 \textsc{neg-}see\textsc{.pst.ptcp.pl}=be.\textsc{pst-cond.aug-1pl:O=3pl:A} \\ 
\glt `If they had not seen us.' 
\z 

The conditional perfect\is{conditional perfect} expresses counterfactual conditionals, i.e., events which did not or could not happen. 

\ea
\textit{î jenû wêmew î kuřme kuştêbîy{ɛ}nê êtir îse min çêşim kerd{ɛ}?}\\
\gll î jen(î)-û wê=m=e=w î kuř=m=e \textbf{kuştê=bî-{ɛ}n-ê} êtir îse min çêş=im kerd-{ɛ} \\
\textsc{dem.prox} wife-\textsc{ez.gen} \textsc{refl=1sg:PSR=dem=}and \textsc{dem.prox} son=\textsc{1sg:A=dem} kill.\textsc{pst.ptcp.pl=}be\textsc{.pst-cond.aug-3pl:O} well now \textsc{1sg} what=\textsc{1sg:A} do.\textsc{pst-cond.aug} \\
\glt `Had I killed my wife and my son, what would I have done now?'\\ \hfill[XŞ.104]
\z 

\subsection{Past perfect\is{past perfect}}
The past perfect\is{past perfect} is constructed by the past participle followed by the augmented form of the verb `to be'. Recall that the augment is a past-converter suffix. With intransitive verbs, both the participle and `be' agree with the S; see Table \ref{tab:ppf-sleep}.

\begin{table}
\fittable{\begin{tabular}{llll}
\lsptoprule
  S& & Gloss& \\\midrule
\textsc{1sg.m} &\textit{wite=b-ên-ê} & [sleep\textsc{.pst.ptcp.m}=be.\textsc{prs-aug-1sg:S}]& `I (\textsc{m}) had slept' \\
\textsc{1sg.f} &\textit{witê=b-ên-ê} & [sleep\textsc{.pst.ptcp.f}=be.\textsc{prs-aug-1sg:S}]& `I (\textsc{f}) had slept' \\
\textsc{2sg.m} &\textit{wite=b-ên-î} &[sleep\textsc{.pst.ptcp.m}=be.\textsc{prs-aug-2sg:S}] & `you (\textsc{m}) had slept' \\
\textsc{2sg.f} &\textit{witê=b-ên-î} &[sleep\textsc{.pst.ptcp.f}=be.\textsc{prs-aug-2sg:S}] & `you (\textsc{f}) had slept' \\
\textsc{3sg.m} &\textit{wite=b-ê} & [sleep\textsc{.pst.ptcp.m}=be.\textsc{prs-aug.3sg:S}]& `he had slept' \\
\textsc{3sg.f} &\textit{witê=b-ê} &[sleep\textsc{.pst.ptcp.f}=be.\textsc{prs-aug.3sg:S}] & `she had slept' \\
\textsc{1pl} &\textit{witê=b-ên-mê} &[sleep\textsc{.pst.ptcp.pl}=be.\textsc{prs-aug-1pl:S}] & `we had slept' \\
\textsc{2pl} &\textit{witê=b-ên-dê} &[sleep\textsc{.pst.ptcp.pl}=be.\textsc{prs-aug-2pl:S}] & `you had slept' \\
\textsc{3pl} &\textit{witê=b-ên-ê} &[sleep\textsc{.pst.ptcp.pl}=be.\textsc{prs-aug-3pl:S}] & `they had slept' \\
\lspbottomrule
\end{tabular}}
\caption{Past perfect--the inflection of `sleep'}
    \label{tab:ppf-sleep}
\end{table}

With transitive verbs, the participle and the auxiliary `be' agree with the O; see Table \ref{tab:ppf-see}.

\begin{table}
\fittable{\begin{tabular}{llll}
\lsptoprule
O& \textsc{3pl A} & Gloss& \\\midrule
\textsc{1sg.m} &\textit{dîye=b-ên-ê=şa} & [see\textsc{.pst.ptcp.m}=be.\textsc{prs-aug-1sg:O=3pl:A}]& `they had seen me (\textsc{m})' \\
\textsc{1sg.f} &\textit{dîyê=b-ên-ê=şa} & [see\textsc{.pst.ptcp.f}=be.\textsc{prs-aug-1sg:O=3pl:A}]& `they had seen me (\textsc{f})' \\
\textsc{2sg.m} &\textit{dîye=b-ên-î=şa} &[see\textsc{.pst.ptcp.m}=be.\textsc{prs-aug-2sg:O=3pl:A}] & `they had seen you (\textsc{m})' \\
\textsc{2sg.f} &\textit{dîyê=b-ên-î=şa} &[see\textsc{.pst.ptcp.f}=be.\textsc{prs-aug-2sg:O=3pl:A}] & `they had seen you (\textsc{f})' \\
\textsc{3sg.m} &\textit{dîye=b-ê=şa} &[see\textsc{.pst.ptcp.m}=be.\textsc{prs-aug.3sg:O=3pl:A}] & `they had seen him' \\
\textsc{3sg.f} &\textit{dîyê=b-ê=şa} &[see\textsc{.pst.ptcp.f}=be.\textsc{prs-aug.3sg:O=3pl:A}] & `they had seen her' \\
\textsc{1pl} &\textit{dîyê=b-ên-mê=şa} &[see\textsc{.pst.ptcp.pl}=be.\textsc{prs-aug-1pl:O=3pl:A}] & `they had seen us' \\
\textsc{2pl} &\textit{dîyê=b-ên-dê=şa} &[see\textsc{.pst.ptcp.pl}=be.\textsc{prs-aug-2pl:O=3pl:A}] & `they had seen you' \\
\textsc{2pl} &\textit{dîyê=b-ên-ê=şa} &[see\textsc{.pst.ptcp.pl}=be.\textsc{prs-aug-3pl:O=3pl:A}] & `they had seen them' \\
\lspbottomrule
\end{tabular}}
\caption{Past perfect--the inflection of `see'}
    \label{tab:ppf-see}
\end{table}

The negation of the past perfect\is{past perfect} is expressed by \textit{ne-}:

\ea
\textit{çaweɫ ta cawe nelabê pane pêwyê tewenekê.} \\ 
\gll çaweɫ ta cawe \textbf{ne-la=b-ê} p=ane pêwy(e)-ê tewen(î)-ekê \\ 
 in\_the\_past until road \textsc{neg-}go\textsc{.pst.ptcp.m}=be\textsc{.prs-aug.3sg:S} at=\textsc{dem.dist.m.3sg.dir} be\_visible\textsc{.prs-aug.3sg:S} stone\textsc{.f-def.f.sg}\\  
\glt `In the past, when no road was constructed there [lit. The road had not gone there.], the stone was visible.' \hfill[ZP.54]
\z 

 The past perfect\is{past perfect} may express states held in the past that are the result of actions in a remoter past.

\ea
\textit{a wextî milarewe maça, `wiɫa î meʕmûrû şime kîyasêbênêta pey îne, înema ane heke şime gerekbê d{ɛ}ma pene.'} \\ 
\gll a wext-î mi-l-a=re=we m-aç-a wiɫa \textbf{î} \textbf{meʕmûr-û} \textbf{şime} \textbf{kîyasê=b-ên-ê=ta} pey îne îne=ma ane heke şime gerek b-ê d{ɛ}=ma pene \\ 
 \textsc{dem.dist} time\textsc{.m-sg.obl} \textsc{ind-}go.\textsc{prs-3pl:S=povb=compl} \textsc{ind-}say\textsc{.prs-3pl:A} by\_God \textsc{dem.prox} officer\textsc{.m-ez.gen} \textsc{2pl} send\textsc{.pst.ptcp.pl}=be\textsc{.prs-aug-3pl:O=2pl:A} for \textsc{dem.prox.m.3sg.dir} \textsc{dem.prox.m.3sg.dir=1pl:A} \textsc{dem.dist.m.3sg.dir} if \textsc{2pl} necessary be\textsc{.prs-aug.3sg:S} give\textsc{.pst.3pl:O=1pl:R} to \\ 
\glt `Then, they went [to the agha and] said, `Indeed, \textbf{the officers whom you had sent} to us, whatever [taxation] you had asked for, we gave them.’' \\\hfill[BP.124]
\z  

 The past perfect\is{past perfect} may express events that the speaker has not witnessed himself but has only learned about through reports. In (\ref{ex.pst-perf}), the narrator discusses how it was reported to him that he should go to military service. 
 \ea
 \textit{zemanew şay, minû hesenî taze min sinhim şaŋzene bê. îne çayxane bê. îne girđiş çayxane bê duweṣe yereṣe nafarêş luw{ɛ}nê. watebêşa ... `fiɫanû fiɫan Baqîyû ħesen yoşa gêrmê bilo sarwazî.'} \\
 \gll zemane-û şa-î min=û hesen-î taze min sinh=im şangze=ne b-ê îne çayxane b-ê îne girđ=iş çayxane b-ê duweṣe yereṣe nafar-ê=ş luw{ɛ}=nê \textbf{wate=b-ê=şa} fiɫan=û fiɫan baqî=û ħesen yo=şa gêl-mê bi-l-o sarwazî \\
 period-\textsc{ez.gen} \textsc{pn-m.sg.obl} \textsc{1sg}=and \textsc{pn-m.sg.obl} just \textsc{1sg} age\textsc{=1sg:PSR} sixteen=\textsc{post} be.\textsc{prs-aug.3sg:S} \textsc{dem.prox} tea\-house be.\textsc{prs-aug.3sg:S} \textsc{dem.prox} all\textsc{=3sg:PSR} tea\-house be.\textsc{prs-aug.3sg:S} 200 300 person\textsc{-pl.dir=3sg:S} go.\textsc{pst.ptcp.pl=cop.3pl:S} say.\textsc{pst.ptcp.m}=be.\textsc{prs-aug.3sg:O=3pl:A} such\-and\-such=and such\-and\-such \textsc{pn}=and \textsc{pn} one=\textsc{.m=3pl:PSR} grab.\textsc{prs.ind-1pl:A} \textsc{sbjv-}go.\textsc{prs-1pl:S} military\_service \\
 \glt `In the period of the Shah, Hasan and I . . . I had just turned sixteen. Here, there were a lot of teahouses where 200 or 300 people would gather. \textbf{They said (according to what was reported to me)}, `Baqî or Hasan, we will send one of them to go to the military service.’' \citep[313, glossing and transcription modified]{khan_language_2023}{} \label{ex.pst-perf}
 \z

 \subsection{Perfect pluperfect\is{perfect pluperfect}}
 The perfect pluperfect\is{perfect pluperfect} is made of the participle form of the verb followed by the perfect form of the auxiliary `to be'. This TAM category seems outdated as it was only occasionally attested in the tales narrated by the narrator from Nwên. The perfect pluperfect\is{perfect pluperfect} has not been listed as a TAM category in Luhon \citep{mackenzie_dialect_1966}, reflecting its rare use. With intransitive verbs, both the participle form of the main verb, the participle form of the auxiliary `be', and the copula agree with the S argument of the verb, the first two in gender\is{gender agreement} and number\is{number agreement}, the latter in person\is{person agreement}. Table \ref{tab:perfppf-sleep} exhibits the perfect pluperfect\is{perfect pluperfect} of the verb `to sleep'.

\begin{table}
\fittable{\begin{tabular}{llll}
\lsptoprule
  S& & Gloss& \\\midrule
\textsc{1sg.m} &\textit{wite=bîye=na} & [sleep\textsc{.pst.ptcp.m}=be\textsc{.pst.ptcp.m=cop.1sg:S}] \\
\textsc{1sg.f} &\textit{witê=bîyê=na} & [sleep\textsc{.pst.ptcp.f}=be\textsc{.pst.ptcp.f=cop.1sg:S}] \\
\textsc{2sg.m} &\textit{wite=bîye=nî} &[sleep\textsc{.pst.ptcp.m}=be\textsc{.pst.ptcp.m=cop.2sg:S}] \\
\textsc{2sg.f} &\textit{witê=bîyê=nî} &[sleep\textsc{.pst.ptcp.f}=be\textsc{.pst.ptcp.f=cop.2sg:S}] \\
\textsc{3sg.m} &\textit{wite=bîye=n} & [sleep\textsc{.pst.ptcp.m}=be\textsc{.pst.ptcp.m=cop.3sg:S}]\\
\textsc{3sg.f} &\textit{witê=bîyê=ne} &[sleep\textsc{.pst.ptcp.f}=be\textsc{.pst.ptcp.f=cop.3sg:S}] \\
\textsc{1pl} &\textit{witê=bîyê=nmê} &[sleep\textsc{.pst.ptcp.pl}=be\textsc{.pst.ptcp.pl=cop.1pl:S}]\\
\textsc{2pl} &\textit{witê=bîyê=ndê} &[sleep\textsc{.pst.ptcp.pl}=be\textsc{.pst.ptcp.pl=cop.2pl:S}] \\
\textsc{3pl} &\textit{witê=bîyê=nê} &[sleep\textsc{.pst.ptcp.pl}=be\textsc{.pst.ptcp.pl=cop.3pl:S}] \\
\lspbottomrule
\end{tabular}}
\caption{Perfect pluperfect--the inflection of `sleep'}
    \label{tab:perfppf-sleep}
\end{table}

\newpage
With transitive verbs, the participles in both the main verb and the auxiliary agree in gender\is{gender agreement} and number\is{number agreement} with the O argument. On the other hand, cumulative person/number copula endings agree in person\is{person agreement}/number with the O argument. The paradigm in Table \ref{tab:perfppf-bring} features the O argument appearing in different persons in combination with the \textsc{3pl} A argument, hence `they have had brought me (\textsc{m})', `they have had brought me (\textsc{f})', etc.

\begin{table}
\fittable{\begin{tabular}{llll}
\lsptoprule
O& \textsc{3pl A} & Gloss& \\ \midrule 
\textsc{1sg.m} &\textit{arđe=bîye=na=şa} & [bring\textsc{.pst.ptcp.m}=be\textsc{.pst.ptcp.m=cop.1sg:O=3pl:A}] \\
\textsc{1sg.f} &\textit{arđê=bîyê=na=şa} & [bring\textsc{.pst.ptcp.f}=be\textsc{.pst.ptcp.f=cop.1sg:O=3pl:A}] \\
\textsc{2sg.m} &\textit{arđe=bîye=nî=şa} &[bring\textsc{.pst.ptcp.m}=be\textsc{.pst.ptcp.m=cop.2sg:O=3pl:A}] \\
\textsc{2sg.f} &\textit{arđê=bîyê=nî=şa} &[bring\textsc{.pst.ptcp.f}=be\textsc{.pst.ptcp.f=cop.2sg:O=3pl:A}] \\
\textsc{3sg.m} &\textit{arđe=bîye=n=şa} & [bring\textsc{.pst.ptcp.m}=be\textsc{.pst.ptcp.m=cop.3sg:O=3pl:A}]\\
\textsc{3sg.f} &\textit{arđê=bîyê=ne=şa} &[bring\textsc{.pst.ptcp.f}=be\textsc{.pst.ptcp.f=cop.3sg:O=3pl:A}]  \\
\textsc{1pl} &\textit{arđê=bîyê=nmê=şa} &[bring\textsc{.pst.ptcp.pl}=be\textsc{.pst.ptcp.pl=cop.1pl:O=3pl:A}]\\
\textsc{2pl} &\textit{arđê=bîyê=ndê=şa} &[bring\textsc{.pst.ptcp.pl}=be\textsc{.pst.ptcp.pl=cop.2pl:O=3pl:A}] \\
\textsc{3pl} &\textit{arđê=bîyê=nê=şa} &[bring\textsc{.pst.ptcp.pl}=be\textsc{.pst.ptcp.pl=cop.3pl:O=3pl:A}] \\
\lspbottomrule
\end{tabular}}
\caption{Perfect pluperfect--the inflection of `bring'}
    \label{tab:perfppf-bring}
\end{table}

The negation of the perfect pluperfect\is{perfect pluperfect} is expressed by \textit{ne-}.

\ea
\textit{newitebîyena.} \\
\gll ne-wite=bîye=na \\
\textsc{neg-}sleep\textsc{.pst.ptcp.m}=be\textsc{.pst.ptcp.m=cop.1sg:S} \\
\glt `I have had not slept.' (Pseudo-English\il{English} translation) 
\z 

The perfect pluperfect\is{perfect pluperfect} seems to occur only in tales. It may express actions which have started in the past but continue to impact the present state of affairs. 

\ea
\textit{gîr wardêbîyênê ana ça matiɫê sergerdanê.}\\
\gll gîr wardê=bîyê=nê ana-{\O} ça matiɫ-ê sergerdan-ê \\
hook eat.\textsc{pst.ptcp.pl=}be\textsc{.pst.ptcp.pl=cop.3pl:S} \textsc{loc.deic.cop-3sg.m:S} there waiting-\textsc{pl} wandering\textsc{-pl}\\
\glt `They were stuck there; they are there, waiting and wandering [not knowing what to do].' \hfill[KT.54]
\z 

\subsection{Summary of TAM categories derived from the past stem}
Table \ref{tab:tam-pst} summarises the verbal forms derived from the past stem for the verbs `sleep' and `do' inflected in the first person. 

\begin{table}
   \resizebox{.94\textwidth}{!}{%
\begin{tabular}{lll}
\lsptoprule
TAM category & Inflection & Gloss \\
\midrule
Past perfective\is{past perfective}& \textit{wit-a}& [sleep.\textsc{pst-1sg:S}] \\
Past conditional\is{past conditional}& \textit{wit-εn-ê}&[sleep.\textsc{pst.cond.aug-1sg:S}] \\
Perfect\is{perfect} & \textit{wite=na}&[sleep.\textsc{pst.ptcp.m=cop.1sg:S}] \\
Perfect progressive\is{perfect progressive}& \textit{wit-î wite=na}&[sleep.\textsc{pst-nmlz} sleep.\textsc{pst.ptcp.m=cop.1sg:S}] \\
Irrealis perfect\is{irrealis perfect}& \textit{wite=b-û}&[sleep.\textsc{pst.ptcp.m}=be\textsc{.prs-1sg:S}] \\
Conditional perfect\is{conditional perfect}& \textit{wite=bî-{ɛ}n-ê}&[sleep\textsc{.pst.ptcp.m}=be.\textsc{pst-cond.aug-1sg:S}] \\
Past perfect\is{past perfect}& \textit{wite=b-ên-ê}&[sleep.\textsc{pst.ptcp.m}=be\textsc{-aug-1sg:S}] \\
Perfect pluperfect& \textit{wite=bîye=na}&[sleep.\textsc{pst.ptcp.m}=be.\textsc{pst.ptcp.m=cop.1sg:S}]\\
\\
Past perfective\is{past perfective}& \textit{kerđ-∅=im}& [do.\textsc{pst-3sg.m:O=1sg:A}] \\
Past conditional\is{past conditional}& \textit{kerđ-ε=m}&[do.\textsc{pst-cond.aug.3sg:O=1sg:A}] \\
Perfect\is{perfect} & \textit{kerđe=n=im}&[do.\textsc{pst.ptcp.m=cop.3sg.m:O=1sg:A}] \\
Perfect progressive\is{perfect progressive}& \textit{kerđ-î kerđe=n=im}&[do.\textsc{pst-nmlz} do.\textsc{pst.ptcp.m=cop.3sg.m:O=1sg:A}] \\
Irrealis perfect\is{irrealis perfect}& \textit{kerđe=b-o=m}&[do.\textsc{pst.ptcp.m}=be\textsc{.prs-3sg:O=1sg:A}] \\
Conditional perfect\is{conditional perfect}& \textit{kerđe=bî-ε=m}&[do\textsc{.pst.ptcp.m}=be.\textsc{pst-cond.aug.3sg:O=1sg:A}] \\
Past perfect\is{past perfect}& \textit{kerđe=b-ê=m}&[do.\textsc{pst.ptcp.m}=be.\textsc{aug.3sg:O=1sg:A}] \\
Perfect pluperfect\is{perfect pluperfect}& \textit{kerđe=bîye=n=im}&[do.\textsc{pst.ptcp.m}=be.\textsc{pst.ptcp.m=cop.3sg:O=1sg:A}]\\
\lspbottomrule
\end{tabular}}
    \caption{TAM categories derived from the past stem--summary}
    \label{tab:tam-pst}
\end{table}















































