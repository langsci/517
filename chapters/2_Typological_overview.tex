\chapter{Typological overview} \label{ch.typology}
This chapter lays out the major typological features of Tekht Hewram\^i\il{Hewramî!Tekht}. This should provide readers with a first-hand touch on the language, though the coverage of features remains concise and eclectic. Defining grammatical properties of Hewramî includes a primary phonological gender assignment system, split-ergative alignment, two-term case system, differential argument flagging, differential argument indexing, disharmonic SOV order, phonemic stress placement, and a complex deictic system. 
\begin{sloppypar}

\section{Phonology}
The consonant inventory includes 29 consonant phonemes, four of which are peripheral phonemes occurring in a few loanwords\is{loanwords} or limited in their distribution within the syllable or word. These are represented in parentheses in Table \ref{tab:Cchart_typological-overview}. 
\begin{table}
\caption{Consonant phonemes}
\label{tab:Cchart_typological-overview}\small
\fittable{\begin{tabular}{@{}llllllllll@{}}
		\lsptoprule
&{Labial} & {Lab.-dent.} & {Alv.} & {Postalv.} & {Pal.} & {Vel.} & {Uv.} & {Phar.} & {Glott.} \\
\midrule

{Stop} & p b &  & t  d &  &  & k  ɡ& q &  & (ʔ) \\
{Affricate} &  &  &  & t͡ʃ  d͡ʒ &  &  &  &  & \\
{Nasal} &\phantom{0 }m &  &\phantom{0 }n &  &  &\phantom{0 }(ŋ)  &  &  & \\
{Fricative} &  & f  & s  z & ʃ      ʒ &  & x    (γ) &   & ħ  ʕ & h \\
{Tap} &  &  &\phantom{0 }ɾ &  &  &  &  &  & \\
{Trill} &  &  &\phantom{0 }r &  &  &  &  &  & \\
{Lateral} &  &  &\phantom{0 }l (ɫ)& &  &  &  &  & \\
{Approx.} &\phantom{0 }w &  &         &  &\phantom{0 }j &  & & & \\
\lspbottomrule
	\end{tabular}
	}
\end{table}%

The vowel inventory consists of nine phonemic vowels: four front vowels <î> /i/, <ê> /e/, <ɛ> /ɛ/, <e> [ɛ${\sim}$æ]; three back vowels <û> /u/, <o> /o/, <a> /ɑ/; and two central vowels <i> /ɨ/ and <u> /ʊ/. Of these, /i, e, ɛ, u, ɑ, o/ and [ɛ${\sim}$æ] are long vowels, while /ɨ/ and /ʊ/ are short. The vowel phonemes are distinguished by height and backness, as well as by their phonetic realisation (see \S\ref{phonetic-realisation}). 
\begin{figure}
    \includegraphics[width=.60\textwidth]{figures/vowel_qualityn.png}
    \caption{Hewramî\il{Hewramî} vowel inventory}
    \label{Hewrami-vowels}
\end{figure}

A syllable in Hewramî\il{Hewramî} consists minimally of a vowel and maximally of a vowel flanked by two consonants, yielding (C)(C)V(C)(C). Consonant clusters in the syllable structure are broken up by the epenthetic vowels \textit{i} and \textit{u}, depending on the quality of the adjacent consonants.

Hewramî\il{Hewramî} is a language with predictable stress placement. Masculine\is{masculine} nouns follow what may be considered the general rule of word-final stress placement, whereas feminine\is{feminine} nouns generally have penultimate stress (except nouns ending in \stackunder[-10pt]{\^{e}}{\'{}}). Similarly, in past tense verbal categories, the stress is penultimate. 
Hewramî\il{Hewramî} is a language with phonemic stress\is{phonemic stress} placement: for most verbs, stress is the only mechanism to distinguish between subjunctive and indicative moods in present tense verbs. Figure \ref{fig:bero_ind-typ} and \ref{fig:bero_sbjv-typ} exhibit different stress patterns associated with the verbs \textit{ber\'o} `he/she takes' and \textit{b\'ero} `that he/she takes'. For each verb, the higher intensity peak highlights the stressed vowel.
\begin{figure}[htb]
  \begin{subfigure}[b]{0.5\textwidth}
    \includegraphics[width=\textwidth]{figures/text_bero_he_takes.png}
    \caption{Indicative `he/she takes'}
    \label{fig:bero_ind-typ}
  \end{subfigure}\begin{subfigure}[b]{0.5\textwidth}
    \includegraphics[width=\textwidth]{figures/text_bero_that_he_takes.png}
    \caption{Subjunctive `that he/she takes'}
    \label{fig:bero_sbjv-typ}
  \end{subfigure}
  \caption{The stress position for the verb `he takes'}
\end{figure}

The most important morphophonemic processes are metathesis\is{metathesis} and assimilation\is{assimilation}. Many disyllabic loans from Arabic containing a pharyngeal consonant in the coda of the second syllable are subject to metathesis\is{metathesis}, illustrated in (\ref{ex.metathesis.1}). The metathesis\is{metathesis} occurs following the restriction against the rise of sonority across syllable boundaries \citep{gouskova_falling_2001}.
\TabPositions{2cm,5cm,7.5cm}
\ea \label{ex.metathesis.1}
\textit{seʕbe}\tab  [ˈsæʕ.bɛ]\tab  cf. Ar.\il{Arabic} \textit{sabāħ}\tab  `morning’ \\
\textit{weʕze}\tab  [ˈwæʕ.zɛ]\tab  cf. Ar.\il{Arabic} \textit{wazʕ}\tab  `situation’ \\
\textit{cuʕme}\tab  [d͡ʒʊħ.ˈmɛ]\tab  cf. Ar.\il{Arabic} \textit{jomʕa}\tab  `Friday’ \\
\z

The assimilation\is{assimilation} processes, both progressive\is{progressive assimilation} and regressive\is{regressive assimilation}, have the effect of creating geminate consonants. Total progressive assimilation\is{progressive assimilation} is seen in \textit{çinne} where /d/ fully assimilates to the preceding nasal sound. On the other hand, \textit{kulle} features total regressive assimilation with /ř/ in \textit{kuř} assimilating to the following lateral phoneme.
\ea
\textit{çinne}\tab  `how much’\tab  < *\textit{çinde}\tab  cf. CK.\il{Kurdish!Central} \textit{çende} \\
\textit{kulle}\tab  `small boy’\tab  < *\textit{kuřle}\\
\z

\section{Morphology}
Hewramî\il{Hewramî} morphology is largely concatenative, though some fusional patterns are attested. Nouns are morphologically marked for case\is{case}, number\is{number}, and gender\is{gender}. These categories are expressed by fusional inflectional affixes on the noun. All three represent a two-way distinction: singular\is{singular} and plural\is{plural} for number\is{number}, masculine\is{masculine} and feminine\is{feminine} for gender\is{gender} (only in the singular\is{singular}), and \textsc{direct}\is{direct case} and \textsc{oblique}\is{oblique case} for case\is{case}. Direct\is{direct case} and oblique\is{oblique case} are Iranian-internal terms roughly equivalent to `nominative' and `non-nominative' cases, respectively. 

 There is one underlying nominal inflectional class in Tekht Hewramî\il{Hewramî!Tekht}. The inflection of a noun is predictable from the phonological shape of the base and from its gender. The underlying fusional suffixes are included in Table \ref{tab:nom-infl-under1}.

\begin{table}
    \begin{tabular}{lllllll} 
    \lsptoprule
& \textsc{sg.dir}& \textsc{sg.obl}& \textsc{pl.dir} & \textsc{pl.obl} \\
\midrule
\textsc{m}& {-\O}& \textit{-î} & \multirow{2}{*}{\textit{-ê}}& \multirow{2}{*}{\textit{-a}} \\
\textsc{f}& {-\O}& \textit{-ê} & & \\
\lspbottomrule
\end{tabular}
    \caption{Nominal inflectional suffixes--underlying forms}
    \label{tab:nom-infl-under1}
\end{table}

The oblique case\is{oblique case} is a continuation of the functions expressed by old non-nominative case endings. It expresses, among other things, direct objects of present tense verbs, transitive subjects of past tense verbs, possessors, and complements of prepositions. The direct case\is{direct case} expresses intransitive subjects, transitive subjects of verbs derived from the present stem, and direct objects of past tense verbs (see \S\ref{case-section}). 

Nouns are overtly marked for the category of gender\is{gender}. Hewramî\il{Hewramî} has two genders, masculine\is{masculine} and feminine\is{feminine}. Gender\is{gender} is assigned to nouns primarily based on the ending the nouns take. Nouns which in their citation form end in a consonant or stressed \textit{-é}, \textit{-\stackunder[-10pt]{\^{i}}{\'{}}}, \textit{-\stackunder[-10pt]{\^{u}}{\'{}}}, and \textit{-ó} are masculine\is{masculine}. In addition, a subset of nouns ending in stressed \textit{-á }are masculine\is{masculine}. On the other hand, nouns ending in \textit{-ê} (whether stressed or not) and those ending in unstressed \textit{-e} and \textit{-î}, are feminine\is{feminine}. The class of feminine\is{feminine} nouns also includes a subset in stressed \textit{-á}. In other words, the language has a phonological gender assignment system (according to the typology in \citealt{corbett_gender_1991}). It is notable that semantic and morphological factors also have a role in gender assignment\is{gender assignment} (see \S\ref{gendersection}). 

Case distinctions\is{case distinction} are lost in the speech act pronouns\is{speech act pronouns}. Third-person pronouns mark case\is{case} and gender\is{gender} distinctions in the singular\is{singular} and case distinctions\is{case distinction} in the plural\is{plural} (see Table \ref{tab:case-pronouns}). 

\begin{table}
    \begin{tabular}{llcl}
    \lsptoprule
 &  \textsc{dir}& \textsc{obl}\\
\midrule
\textsc{1sg}& \multicolumn{2}{c}{\textit{min}} \\
\textsc{2sg}& \multicolumn{2}{c}{\textit{to}} \\
\textsc{3sg.m}&\textit{ađ}&\textit{ađî} \\
\textsc{3sg.f}& \textit{ađe}& \textit{ađê}\\
\textsc{1pl}& \multicolumn{2}{c}{\textit{ême}} \\
\textsc{2pl}& \multicolumn{2}{c}{\textit{êşme}} \\
\textsc{3pl}& \textit{ađê}&\textit{ ađîşa} \\
\lspbottomrule
    \end{tabular}
    \caption{Case distinctions\is{case distinction} in pronouns}
    \label{tab:case-pronouns}
\end{table}

Gender\is{gender} is a morphosyntactic category in Hewramî\il{Hewramî} since it is involved in agreement. At the level of noun phrases, agreement in gender\is{gender agreement} can be found on (i) adjectives and (ii) definite suffixes\is{definite suffix}. At the clause level, agreement targets for the category of gender\is{gender agreement} are predicative adjectives, \textsc{3sg} copula markers (\textsc{3sg.m} \textit{꞊n/ ꞊a}; \textsc{3sg.f} \textit{꞊ne}), and \textsc{3sg} inflectional person suffixes in past stem verbs (\textsc{3sg.m} \textit{-∅}; \textsc{3sg.f} \textit{-e}) (see \S\ref{sect:gend-agreement}). In (\ref{agr-adj}), the predicative adjective, the participle, and the \textsc{3sg} copula agree with the \is{gender agreement} topicalised NP in gender:

\newpage
\ea
\textit{î dega toş vînî çoɫe bîyêne.} \\ 
\gll î dega to=ş vîn-î çoɫ-\textbf{e} \textbf{bîyê=ne} \\ 
 \textsc{dem.prox} village{\textsc{.f}} \textsc{2sg=3sg:O} see\textsc{.prs.ind-2sg:A} deserted-\textsc{f} be\textsc{.pst.ptcp.f=cop.3sg.f:S} \\ 
\glt `This village, which you see, was deserted.' \hfill [JE.4] \label{agr-adj}
\z

Similarly, number\is{number} is a morphosyntactic category in Hewramî\il{Hewramî}. Typical agreeing elements for the category of number\is{number agreement} are adjectives and some classifiers\is{classifier} (see \S\ref{Number-section}). 
\ea
\textit{yerê danê hêɫê} \\ 
\gll yerê dan(e)-ê hêɫ(e)-ê \\ 
 three \textsc{clf}\textsc{.pl} egg\textsc{.m}\textsc{-pl.dir} \\ 
\glt `three eggs' \hfill [JH.81] 
\z

Adjectives feature gender\is{gender agreement} and number agreement\is{number agreement} with head nouns when used attributively and predicatively. When substantivised, adjectives inflect for case\is{case} as well. In light verb constructions\is{light verb constructions}, the adjective complement of a light verb\is{light verb} carries number\is{number agreement} and gender agreement\is{gender agreement} with S and O. In (\ref{ex.lv-agr-1}), the predicate is \textit{neweş kewtey} `fall ill'. In (\ref{ex.lv-agr-2}), the predicate is \textit{zamdar kerđey} `injure'. In both predicates, the adjective complement exhibits agreement with clausal arguments.
\ea 
\textit{kinaçêw padşaw misrî neweşe gino.} \\ 
\gll kinaçê-û padşa-û misr-î neweş-\textbf{e} gin-o \\ 
daughter\textsc{.f.dir-ez.gen} king\textsc{.m-ez.gen} \textsc{pn-m.sg.obl} ill\textsc{-f} fall\textsc{.prs.ind-3sg:S} \\  
\glt `The king of Egypt’s daughter fell sick.' \hfill [ZP.25] \label{ex.lv-agr-1}
\z 

\ea
\textit{ême zamdarê nekero} \\
\gll ême zamdar-\textbf{ê} ne-ker-o \\
\textsc{1pl} wounded\textsc{-pl} \textsc{neg.sbjv-}do\textsc{.prs-3sg:A} \\ 
\glt `He should not injure us.’ \hfill [DG.64] \label{ex.lv-agr-2}
\z 

As a reflection of it being spoken in a high mountainous area, Hewramî\il{Hewramî} features a complicated demonstrative system. In addition to the third person forms in Table \ref{tab:case-pronouns}, which are unmarked anaphoric pronouns, the language has two sets of anaphoric demonstratives\is{anaphoric demonstratives}, distinguished based on distance, and two sets of demonstrative pronouns. 

\begin{table}
    \begin{tabular}{lllll}
 \lsptoprule
 \multicolumn{3}{c}{Proximal} & \multicolumn{2}{c}{Distal} \\
& \textsc{dir}& \textsc{obl} & \textsc{dir}& \textsc{obl} \\ 
\midrule
\textsc{m}&\textit{îđ} &\textit{îđî} &\textit{ew} &\textit{ewî}\\
\textsc{f}& \textit{îđ(e)}& \textit{îđê}& \textit{ewe}& \textit{ewê} \\
\textsc{pl}& \textit{îđê}& \textit{îđîşa} & \textit{ewê, ewêşa}& \textit{ewîşa} \\
 \lspbottomrule
    \end{tabular}
    \caption{Anaphoric demonstratives}
    \label{tab:anaphoricpronouns}
\end{table}

\begin{table}
    \begin{tabular}{lllll}
 \lsptoprule
\multicolumn{3}{c}{Proximal} & \multicolumn{2}{c}{Distal} \\
& \textsc{dir}& \textsc{obl} & \textsc{dir}& \textsc{obl} \\ 
\midrule
\textsc{m}&\textit{îne} &\textit{îney} &\textit{ane, ûne} &\textit{aney} \\
\textsc{f}& \textit{îne, înî}& \textit{înê}& \textit{anê}& \textit{ane} \\
\textsc{pl}& \textit{înê, înî}& \textit{îna, înîşa}& \textit{anê}& \textit{ana, anîşa} \\
 \lspbottomrule
    \end{tabular}
    \caption{Demonstrative pronouns}
    \label{tab:demonstative pronouns}
\end{table}

The different sets of third-person pronouns express diverse, sometimes overlapping, functions. Unmarked anaphoric pronouns track participants that are established as topics. Anaphoric demonstratives\is{anaphoric demonstratives} may reactivate a referent that has occurred with some distance in the previous discourse. They are also used to establish new discourse topics (see \S\ref{sect:3perspronouns}). The choice between unmarked personal pronouns and anaphoric demonstratives\is{anaphoric demonstratives} is evident in the following excerpt. The former tracks an already-established topic. The latter expresses a topic shift and new information.
\ea 
\textit{mila ew kuřekey yoyşa bera. aneşa zilterû ʕalter bo ađî bera. ewî minya qiřoɫû darêwe.} \\ 
\gll mi-l-a ew kuř-ekey yo-î=şa ber-a ane=şa zil-ter=û ʕal-ter b-o \textbf{ađî} ber-a \textbf{ewî} mi-ny(e)-a qiřoɫ-û dar-êwe \\ 
\textsc{ind-}go\textsc{.prs-3pl:S} \textsc{dem.dist} son\textsc{.m-def.m.sg.obl} one\textsc{.m-sg.obl=3pl:PSR} take\textsc{.prs.ind-3pl:A} \textsc{dem.dist.m.3sg.dir=3pl:PSR} big\textsc{-cmpr}=and good\textsc{-cmpr} be\textsc{.prs.ind-3sg:S} \textsc{3sg.obl.m} take\textsc{.prs.ind-3pl:A} \textsc{3sg.obl.m} \textsc{ind-}put\textsc{.prs-3pl:A} hollow\textsc{.m-ez.gen} tree\textsc{.m-indf}\\ 
\glt `They went away [and took] that son. They took one of them (i.e., of the boys), the one who was bigger and healthier; they took \textbf{him}. They left \textbf{him (i.e., the other one)} in the hole in the tree.' \hfill [ZB.40]--[ZB.41]
\z 

The demonstrative pronouns can have exophoric use, anaphoric use, discourse presentative use, and emphatic use (see \S\ref{sect:ind-dem-pro}). Local adverbial demonstratives\is{local adverbial demonstratives} distinguish between visible from the deictic centre and invisible from the deictic centre (see \S\ref{sect:loc-adv-dem}). 

Nouns are marked for definiteness\is{definiteness} by way of the suffixes \textit{-eke} and \textit{-e}. The former inflects for gender\is{gender}, case\is{case}, and number\is{number}. Unlike known definiteness\is{definiteness} systems, the definite suffix\is{definite suffix} is not used with all nouns with identifiable referents. Rather, once a noun has been identified with a definite status, it is no longer necessary to mark it by the definite suffix. This means that bare nouns\is{bare noun} can have a definite reading. In the following excerpt, \textit{kinaçê} has a definite reference by virtue of its appearance with the demonstrative \textit{î}. In the continuation of the discourse, the same referent occurs in its bare\is{bare} form. 
\ea
\ea[]{ 
\textit{be mezebû wêşa î kinaçeşa pey nîkah kerew. narîşo!} \\ 
\gll be mezeb-û wê=şa \textbf{î} \textbf{kinaç(ê)=e}=şa pey nîkah k\'er-e=û n(e)-ar-î=ş=o \\ 
by religion\textsc{-ez.gen} \textsc{refl=3pl:PSR} \textsc{dem.prox} girl\textsc{.f=dem=3pl:R} to marriage\textsc{.m} do\textsc{.prs.imp-2sg:A}{=and} \textsc{neg.sbjv-}bring\textsc{.prs-2sg:A=3sg:O=compl} \\  
\glt `Marry the girl to him according to [the customs of] their religion. May you not bring her back!' \hfill [JP.165]
}
 
\ex[]{ 
\textit{lalo gino gelû kinaçê.} \\ 
\gll lalo gin-o gel-û \textbf{kinaçê} \\ 
maternal\_uncle\textsc{.m} fall\textsc{.prs.ind-3sg:S} with\textsc{-ez.gen} girl\textsc{.f} \\ 
\glt `The uncle set off on the road with the girl.' \hfill [JP.166]
}
\z 
\z 

The simple verb has two verbal forms divided into \textsc{present} and \textsc{past} stems. The two-stem system is divided into two tense-based categories roughly equivalent to present and past tenses. The most productive tool for forming new verbs is light verb constructions\is{light verb constructions} consisting of a light verb\is{light verb} and a non-verbal element. \\
Verbs inflect for the morphological and morphosyntactic features of number\is{number}, person, gender\is{gender} (only in \textsc{3sg}, in verbs derived from past stem), tense, mood, and aspect. Verbal categories are built by present and past stems combined with inflectional person suffixes and modal prefixes. Table \ref{tab:my_label} exemplifies the inflection of the suppletive verb \textit{witey} `sleep' in \textsc{1sg} across different TAM forms. 
\begin{table}
    \fittable{%
    \begin{tabular}{lll}
\lsptoprule
Present indicative\is{present indicative}& \textit{m-ûs-û}& [\textsc{ind}-sleep.\textsc{prs-1sg:S}] \\
Present subjunctive\is{present subjunctive}& \textit{b-ûs-û}&[\textsc{sbjv}-sleep.\textsc{prs-1sg:S}] \\
Present progressive\is{present progressive}& \textit{m-ûs-ay m-ûs-û}&[\textsc{ind}-sleep.\textsc{prs-nmlz} \textsc{ind}-go.\textsc{prs-1sg:S}] \\
Past progressive\is{past progressive}& \textit{wis-ay wis-ên-a}&[sleep.\textsc{prs-nmlz} go.\textsc{prs-aug-1sg:S}] \\
Habitual past\is{habitual past} & \textit{wis-ên-a}&[sleep.\textsc{prs-aug-1sg:S}] \\
Irrealis past\is{irrealis past}& \textit{wis-ên-a}&[sleep.\textsc{prs-aug-1sg:S}] \\
\\
Past perfective\is{past perfective}& \textit{wit-a}& [sleep.\textsc{pst-1sg:S}] \\
Past conditional& \textit{wit-εn-ê}&[sleep.\textsc{pst-cond.aug-1sg:S}] \\
Perfect\is{perfect} & \textit{wite=na}&[sleep.\textsc{pst.ptcp.m=cop.1sg:S}] \\
Perfect progressive\is{perfect progressive}& \textit{wit-î wite=na}&[sleep.\textsc{pst-nmlz} sleep.\textsc{pst.ptcp.m=cop.1sg:S}] \\
Irrealis perfect\is{irrealis perfect}& \textit{wite=b-û}&[sleep.\textsc{pst.ptcp.m}=be\textsc{.prs-1sg:S}] \\
Conditional perfect\is{conditional perfect}& \textit{wite=bî-{ɛ}n-ê}&[sleep\textsc{.pst.ptcp}=be.\textsc{pst-cond.aug-1sg:S}] \\
Past perfect\is{past perfect}& \textit{wite=b-ên-ê}&[sleep.\textsc{pst.ptcp.m}=be\textsc{-aug-1sg:S}] \\
Perfect pluperfect\is{perfect pluperfect}& \textit{wite=bîye=na}&[sleep.\textsc{pst.ptcp.m}=be.\textsc{pst.ptcp.m=cop.1sg:S}]\\
\lspbottomrule
    \end{tabular}}
    \caption{The inflection of \textit{witey} `sleep' in \textsc{1sg} across different TAM categories}
    \label{tab:my_label}
\end{table}

The verb forms with present-time reference fall broadly into four classes. In all verb classes, the negation of the indicative is identical to the prohibitive, as opposed to the negation of the subjunctive. Class 1 features the majority of verbs, as exemplified by the verb \textit{berđey} `take'. Here, indicative, subjunctive, and imperative verb forms are prefix-less. The verbs beginning with \textit{m} in this class exceptionally have the prohibitive prefix \textit{ne-}. Class 2 is specific to verbs with a C(V) structure, with the exception of \textit{bîyey} `be, become' (\textsc{prs} \textit{b-}; \textsc{pst} \textit{bî-}), which belongs to class 1. The verb forms in this class regularly take the TAM prefixes, except the imperative prefix is occasionally dropped. Class 3 is limited to low-vowel-initial verbs, with the negative prefixes for both the indicative and prohibitive being \textit{nime-}, unlike the verbs in classes 1 and 2. Class 4 is limited to high back-vowel and mid-vowel-initial verbs. Like the verbs in class 3, the verb forms in this class feature vowel coalescence of the TAM prefixes with the stem. However, unlike in class 3, the verb forms in class 4 use the negation forms \textit{m\'e-}. An exception is the verb \textit{êşay} `to hurt', for which the negative of the indicative can be expressed by either \textit{m\'e-} or \textit{nim\'e-} (see \S\ref{section-vinfmorph}).

\begin{table}[hbt!]

    \begin{tabular}{rllccc}
        \lsptoprule
 
&& & \textsc{ind} & \textsc{sbjv} & \textsc{imp/proh} \\
   \midrule
\multirow{2}{*}{1} & \multirow{2}{*}{\textit{ber-} `take'} & \textsc{aff} & \textit{ber-\stackunder[-10pt]{\^{i}}{\'{}}} & \textit{b\'er-\^i} & \textit{b\'er-e} \\
&& \textsc{neg} & \textit{m\'e-ber-\^i} & \textit{n\'e-ber-\^i} & \textit{m\'e-ber-e} \\
\midrule
\multirow{2}{*}{2} & \multirow{2}{*}{\textit{de-} `give'} & \textsc{aff} & \textit{mi-đe-\stackunder[-10pt]{\^{i}}{\'{}}} & \textit{bi-đ\'e-\^i} & \textit{(bi)-đ(e)-\'e} \\
&& \textsc{neg} & \textit{m\'e-đe-\^i} & \textit{n\'e-đe-\^i} & \textit{m\'e-đ(e)-e} \\
\midrule
\multirow{2}{*}{3} & \multirow{2}{*}{\textit{az-} `let'} & \textsc{aff} & \textit{m-az-\stackunder[-10pt]{\^{i}}{\'{}}} & \textit{b-\'az-\^i} & \textit{b-\'az-e} \\
&& \textsc{neg} & \textit{nim(e)-\'az-\^i} & \textit{n-\'az-\^i} & \textit{nim(e)-\'az-e} \\
\midrule
\multirow{2}{*}{4} & \multirow{2}{*}{\textit{ûs-} `sleep'} & \textsc{aff} & \textit{m-\^us-\stackunder[-10pt]{\^{i}}{\'{}}} & \textit{b-\stackunder[-10pt]{\^{u}}{\'{}}s-\^i} & \textit{b-\stackunder[-10pt]{\^{u}}{\'{}}s-e} \\
&& \textsc{neg} & \textit{m\'e-ws-\^i} & \textit{n\'e-ws-\^i}  & \textit{m\'e-ws-e} \\
    \lspbottomrule
    \end{tabular}
    \caption{Verb classes with present time reference, inflected in \textsc{2sg} }
    \label{tab:text_me}
\end{table}
Hewramî\il{Hewramî} has a mixed adpositional typology, which reflects its structure being affected by both OV languages like Turkish\il{Turkish}, and VO languages like Arabic\il{Arabic}, and Aramaic\il{Aramaic} \citep{stilo_circumpositions_2009}. Additionally, adpositions exhibit applicative-like properties when taking pronominal arguments (see \S\ref{sect:absolute_prep}). Example:
\ea 
\textit{xway ketê pey kîyasen.} \\ 
\gll xwa-î ket-ê \textbf{pey} kîyase=\textbf{n} \\ 
God\textsc{.m-sg.obl} bed\textsc{-indf} to send\textsc{.pst.ptcp.m=cop.3sg.m:R} \\ 
\glt `[As if] God had sent him a bed.' \hfill [JP.69]
\z

\section{Syntax}
The basic order of modifiers within the NP is DEM NUM N ADJ POSS, exemplified in (\ref{ex.np.structure}). Hewramî\il{Hewramî} uses two different head linkers in the structure of the NP, dubbed ezafe/ezafeh in Iranian linguistics, depending on the category of the modifier: \textit{-î} marks attributive ezafe\is{attributive ezafe}, whereas \textit{-û} marks genitive ezafe\is{genitive ezafe}. Additionally, an \textit{ezafe compound} \textit{-e} is used in the language with tightly-knit compound NPS, e.g. \textit{nan-e taz(e)-êwe} [bread\textsc{.m-ez.cmpd} fresh-\textsc{indf}] `a fresh [loaf of] bread'.
\ea 
\textit{a duwe ku\v{r}e ʕalew emîrî} \\ 
\gll a duwe ku\v{r}-e ʕal-e-û emîr-î \\ 
\textsc{dem.dist} two son.\textsc{ez.cmpd} good-\textsc{def-ez.gen} \textsc{pn-m.sg.obl}\\ 
\glt `those two good sons of Emir' \label{ex.np.structure}
\z 

Case marking\is{case marking} and ezafe marking interact in the structure of the noun phrase. When two possessors follow the head noun, only one formative is retained. In the following example, the expected construction would be \textit{qefesû sîne-y-û minne} [cage\textsc{.m-ez.gen} chest\textsc{.m-sg.obl-ez.gen} \textsc{1sg=post}]. However, in competition for the slot on the first possessor, only the oblique\is{oblique case} suffix remains, and the ezafe gets deleted.
\ea 
\textit{qefesû sîney minne} \\ 
\gll qefes-û sîne-\textbf{y} min=ne \\ 
 cage\textsc{.m-ez.gen} chest\textsc{.m-sg.obl} \textsc{1sg=post} \\ 
\glt `in my chest [lit. in the cage of my chest]' \hfill [DP.38]
\z 

\subsection{Word order\is{word order}}
Hewramî\il{Hewramî} has a default SOV order. This ordering is characterised by the subject not carrying the nuclear stress. The immediate pre-verbal slot is associated with the basic place of the focus\is{focus} in the clause, illustrated in (\ref{ex.sov1a}). Occasionally, the subject constituent comes between the verb and its direct object (\ref{ex.sov1typ}). This typically occurs when the subject constituent is in focus\is{focus}. The focality of the A argument in past constructions can trigger the absence of indexing of the A argument on the verb (see \S\ref{sect:damtyp}).
\ea 
\textit{çêrhur zaroɫeke şot wero.} \\ 
\gll çêr=hur zaroɫe-(e)ke şot wer-o \\ 
under\textsc{=post} child\textsc{-def.m.sg.dir} milk\textsc{.m} eat\textsc{.prs.ind-3sg:A} \\ 
\glt `The baby drank [its] milk from below.' \hfill [ZB.45]  \label{ex.sov1a}
\z

\ea
\textit{heywane awê berde.} \\ 
\gll heywane aw\stackunder[-10pt]{\^{e}}{\`{}} berd-e \\ 
 animal\textsc{.f.sg.dir} water\textsc{f.sg.obl} take\textsc{.pst-3sg.f:O} \\ 
\glt `The flood [lit. water] took away the animals.' \hfill[ZB.21] \label{ex.sov1typ}
\z

Post-verbal objects are rare. If they occur at all, they are limited to nominals with definite reference evoked in the previous discourse. They seem to be limited to certain clause types, e.g., interrogatives (\ref{ex.VO1a}) and imperatives (\ref{ex.VO2a}). 
\ea 
\textit{maça, `şanat tomeke?’} \\ 
\gll m-aç-a şana=t tom-eke \\ 
\textsc{ind-}say\textsc{.prs-3pl:A} scatter\textsc{.pst.3sg:O=2sg:A} seed\textsc{-def.m.sg.dir}\\ 
\glt `They would say, ‘Did you plant the seeds?’' \hfill [JP.39] \label{ex.VO1a}
\z 

\ea 
\textit{mekojdê a kabray!} \\
\gll me-koj-dê a kabra-î \\
\textsc{proh-}kill.\textsc{prs-2pl:A} \textsc{dem.dist} fellow\textsc{-m.sg.obl} \\
\glt `Do not kill that man!' \hfill [SH.268] \label{ex.VO2a}
\z 

Despite having OV order, Hewramî\il{Hewramî} exhibits several head-initial configurations, including Noun-Adjective, Possessed-Possessor, Matrix clauses-complement clause, Verb-Goal, and Verb-Recipient\is{recipient}, running against the predictions of head-directionality hypothesis \citep[]{dryer_greenbergian_1992}{}. 

As for non-core arguments\is{oblique arguments}, goals of verbs of motion (\ref{ex.goal2a}), recipients (\ref{ex.rec1typ}), and addressees (\ref{ex.add1typ}) are overwhelmingly realised in the post-verbal position, representing \citegen{hawkins_ordering_2008} SOVX type, where X is the non-core argument. Though note that \citet{hawkins_ordering_2008} uses the notation `X' to refer to all kinds of non-core arguments, including also comitatives, instrumentals, place, etc. These latter arguments tend to be realised pre-predicatively in Hewramî.
\ea 
\textit{êtir milo law ađî.} \\ 
\gll êtir mi-l-o \textbf{la-û} \textbf{ađî} \\ 
\textsc{disc.ptcl} \textsc{ind-}go\textsc{.prs-3sg:S} to\textsc{-ez.gen} \textsc{3sg.obl.m} \\ 
\glt `Anyhow, he went to him.' \hfill [JP.13] \label{ex.goal2a}
\z 

\ea 
\textit{nanîç miđa to.} \\ 
\gll nan=îç mi-đ(e)-a \textbf{to} \\ 
bread\textsc{.m=add} \textsc{ind-}give\textsc{.prs-3pl:A} \textsc{2sg} \\ 
\glt `They will give you a meal.' \hfill [HB.40] \label{ex.rec1typ}
\z 

\ea 
\textit{maço be xanî.} \\ 
\gll m-aç-o \textbf{be} \textbf{xan-î} \\ 
 \textsc{ind-}say\textsc{.prs}\textsc{-3sg:A} to chief\textsc{.m-sg.obl} \\ 
\glt `He said to the chief.' \hfill [KŞ.97] \label{ex.add1typ}
\z 

\subsection{Alignment\is{alignment}} \label{sect:alignment}
Hewramî\il{Hewramî} features a tense-based split ergative system both in verbal argument indexing and argument case marking\is{case marking}. The alignment\is{alignment} system is nominative-accusative for verbs derived from the present tense and ergative-absolutive for verbs derived from the past stem, though note that ergative alignment is most consistently evident in the pattern of argument indexing. This system of tense-sensitive alignment\is{tense-sensitive alignment} depends not on the transitivity of the clause in a semantic sense but on the lexical transitivity of individual verbs. That is, a verb is lexically specified as either transitive or intransitive, and that will determine how it inflects in TAM constructions based on the past stem of the verb. Whether a verb has an overt object or not is irrelevant.

In terms of verbal argument indexing, Hewramî uses verbal affix person markers to index A and S in verbs derived from the present stem, while O is expressed via person clitics. The indexing of S and A is obligatory, whereas the indexing of O is contingent on the absence of the coreferent nominal.
\ea 
\textit{milo.} \\ 
\gll mi-l-\textbf{o} \\ 
\textsc{ind-}go\textsc{.prs-\textbf{3sg:S}} \\ 
\glt `It (a tortoise) went.' \hfill [DG.61]
\z 

\ea 
\textit{beroş.} \\ 
\gll ber-\textbf{o=ş} \\ 
 take\textsc{.prs.ind-\textbf{3sg:A=3sg:O}} \\ 
\glt `He took him.' \hfill [DG.17]
\z 

In verbs derived from the past stem, O and S are indexed by verbal affixes (or copula person endings in perfect tenses), whereas clitic pronouns index the A argument. Only indexing of S is obligatory. O-indexing (see \S\ref{sect:DOI}) and A-indexing (see \S\ref{sect:differential-A-indexing}) are nearly obliqgatory. 
\ea 
\textit{amaymê.} \\ 
\gll ama-\textbf{îmê} \\ 
 come\textsc{.pst-\textbf{1pl:S}} \\ 
\glt `We came.' 
\z 

\ea 
\textit{berđîmêşa.} \\ 
\gll berđ-\textbf{îmê=şa} \\ 
take\textsc{.pst-\textbf{1pl:O=3pl:A}} \\ 
\glt `They took us.'
\z 

In terms of case marking\is{case marking}, in clauses where the verb is derived from the present stem, S (\ref{ex.align:s.prs}) and A (\ref{ex.align:A.prs}) are marked in the direct case\is{direct case}, whereas O (\ref{ex.align:o.prs}) is marked in the oblique case\is{oblique case}. 
\ea
\textit{seʕbê wiɫaxdarê mila.} \\
\gll seʕbê \textbf{wiɫaxdar-ê} mi-l-a \\
morning\textsc{.f.sg.obl} stableman-\textsc{pl.dir} \textsc{ind}-go.\textsc{prs-3pl:S} \\
\glt `In the morning, the horse grooms went.’ \hfill [ŞC.66]  \label{ex.align:s.prs} 
\z 

\ea 
\textit{îse dêwê xeber zana} \\ 
\gll îse \textbf{dêw-ê} xeber z\'an-a \\ 
 now ogre-\textsc{pl.dir} news know\textsc{.prs.sbjv-3pl:A} \\ 
\glt `If the ogres find out [about you] now' \hfill [SK.100] \label{ex.align:A.prs} 
\z 

\ea 
\textit{yewayç bero.} \\ 
\gll \textbf{yew(e)-a}=îç {} ber-o \\ 
 barley\textsc{.f-pl.obl=add} {} take\textsc{.prs.ind-3sg:A} \\ 
\glt `He took the barley seeds, too.' \hfill [JP.29] \label{ex.align:o.prs} 
\z   

In clauses with verbs derived from the past stem, S (\ref{ex.align:s.pst}) and O (\ref{ex.align:o.pst}) are marked in the direct case\is{direct case}, whereas A (\ref{ex.align:A.pst}) is marked in the oblique case\is{oblique case}.
\ea 
\textit{karewanîyê amêyanê serere.} \\ 
\gll \textbf{karewanî-ê} amêya=nê sere=re. \\ 
 caravan\_people-\textsc{pl.dir} come\textsc{.pst.ptcp.pl=cop.3sg.pl:S} top=\textsc{postp} \\  
\glt `Some passers-by had stayed there.' \hfill[DB.15]\label{ex.align:s.pst}
\z 

\ea
\textit{hewarêşa wişkin{ɛ}nê.} \\ 
\gll \textbf{hewar-ê}=şa wişkin{ɛ}=nê \\ 
 summer\_habitat\textsc{.m-pl.dir=3pl:A} scour\textsc{.pst.ptcp.pl=cop.3pl:O} \\ 
\glt `They scoured the summer habitats [searching for food etc.].' \hfill [JE.3] \label{ex.align:o.pst}
\z

\ea 
\textit{ênne paɫewana zorşa kerđen} \\ 
\gll ênne \textbf{paɫewan-a} zor=şa kerđe=n\\
so\_much warrior-\textsc{pl.obl} pressure=\textsc{3pl:A} do\textsc{.pst.ptcp.m=cop.3sg:O} \\ 
\glt `The warriors put much pressure [on the skin]' \hfill [SK.132] \label{ex.align:A.pst}
\z 
\subsection{Grammatical relations}
\subsubsection{Differential A indexing\is{differential subject indexing}} \label{sect:damtyp}
Hewramî\il{Hewramî} features tense-sensitive indexing of the transitive subject A argument: in clauses based on the present stem of the verb, verbal person/number suffixes index the A argument, whereas in clauses derived from the past stem of the verb, the historical clitic pronouns index the A argument. Unlike the verbal affixes, the clitic pronouns indexing the A-past argument are mobile. Furthermore, while indexation of A-\textsc{prs} through inflectional suffixes is obligatory, the person clitics indexing A-past arguments are sometimes missing. They may be, at least partially, in complementary distribution with an overt oblique case-marked\is{oblique case} A argument. In (\ref{ex.damintro1a}), the oblique-marked\is{oblique case} A \textit{pad\c{s}a} is the sole way to express the A argument. In (\ref{ex.damintro1b}), the \textsc{3sg} clitic \textit{=i\c{s}}, resumes the coreferent absent A argument.   
\ea
\ea[]{
\textit{min taze padşay kerdena wekêɫ.} \\ 
\gll min taze \textbf{padşa-î} kerde=na wekêɫ \\ 
\textsc{1sg} anyway king\textsc{.m-sg.obl} do\textsc{.pst.ptcp.m=cop.1sg:O} advocate\textsc{.m} \\ 
\glt `I—the king has put me in charge.' \hfill[ZP.107] \label{ex.damintro1a}
}
 
\ex[]{
\textit{watenîçiş, mişyo neberûşo.} \\ 
\gll wate=n=îç=\textbf{iş} mişyo ne-ber-û=ş=o \\ 
say\textsc{.pst.ptcp.m=cop.3sg.m:O=add=3sg:A} \textsc{aux} \textsc{neg.sbjv-}take\textsc{.prs-1sg:A=3sg:O=compl} \\ 
\glt `He (the king) has said [to me], “You shall not take her back.’'' \label{ex.damintro1b} \\ \hfill[ZP.108] 
}
\z
\z 

This optional indexing of A-past arguments is triggered by the A argument displaying properties related to focus\is{focus} \citep{MohammadiradinreviewAindx}. A focused A argument in Hewramî\il{Hewramî} can be in non-contrastive or constrastive focus. The non-contrastive focus is further divided into wh-focus\is{wh-focus focus} and completive focus\is{completive focus} (see \S\ref{sect:arginx.pst}). By way of example, in (\ref{ex.dai1}), the A argument has nuclear focus, and the A-indexing person clitic \textit{=ş} is missing:
\ea
\textit{î zeře çermeme kê berden eçêge?} \\
\gll î zeře çerme=m=e \textbf{k\stackunder[-10pt]{\^{e}}{\`{}}} berde=n eçêge \\
\textsc{dem.prox} money-\textsc{ez.cmpd} white\textsc{=1sg:PSR=deic} who take.\textsc{pst.ptcp.m=cop.3sg.m:O} here \\
\glt `\textbf{Who} has taken my white money [that is now] here?' \hfill [PK.29] \label{ex.dai1}
\z 

When the A argument is in completive focus\is{completive focus}, it is not indexed via mobile person clitics. In (\ref{ex.completive-focus1}), in response to the wh-question, the focused \textsc{2sg} A argument is not indexed. 
\ea 
\textit{maço `miɫk ehmeđ! î dijmenême kê kuştênê? to kuştênê!?'} \\
\gll m-aç-o miɫk ehmeđ î dijmen-ê-m=e \textbf{k\stackunder[-10pt]{\^{e}}{\`{}}} kuştê=nê \textbf{t\`o} kuştê=nê \\
\textsc{ind}-say.\textsc{prs-3sg:A} \textsc{pn} \textsc{pn} \textsc{dem.prox} enemy\textsc{-pl.dir=1sg=deic} who kill\textsc{.pst.ptcp.pl=cop.3pl:O} \textsc{2sg} kill\textsc{.pst.ptcp.pl=cop.3pl:O} \\
\glt `He (the king) said, Oh Milk Ahmad! \textbf{Who} has killed my enemies? \textbf{You} have killed them!?' \hfill [ME.150] \label{ex.completive-focus1}
\z 

By contrast, when the oblique-marked\is{oblique case} A NP is in the topic position and not focused, the person clitic resumes it. This explains the co-occurrence of the subject indexing clitic and the oblique-marked A argument in (\ref{ex.daiindex1}). Similarly, oblique-marked\is{oblique case} A argument in the post-verbal position co-occurs with the co-indexing person clitic (\ref{ex.daiindex2}). The clitic indexing here follows from the fact that the topical A argument is placed in the non-focal post-verbal position as an afterthought. 
\ea 
\textit{î pîyay tawaş î kinaçêşe kerde be qerarê weşeş kerdewe...} \\ 
\gll î pîya-î \textbf{taw\`a}=ş î kinaçê=ş=e kerd-e be qerar-ê weş-e=ş kerd-e=we\\ 
 \textsc{dem.prox} man\textsc{.m-sg.obl} can\textsc{.pst=3sg:A} \textsc{dem.prox} girl\textsc{.f.sg=3sg:A=dem} do\textsc{.pst-3sg.f:O} to settlement\textsc{-indf} well\textsc{-f=3sg:A} do\textsc{.pst-3sg.f:O=compl} \\  
\glt `[And if] the man has been \textbf{able} to cure the girl...' \hfill [ZP.45] \label{ex.daiindex1}
\z 

\ea 
\textit{qotê aman. asawekeş bînan qotekey.} \\
\gll qot(e)-ê ama=n asaw-eke=\textbf{ş} bîna=n \textbf{qot-ekey} \\
box-\textsc{indf} come.\textsc{pst.ptcp.m=cop.3sg.m:S} mill\textsc{.m-def=3sg} block.\textsc{pst.ptcp.m=cop.3sg.m:O} box-\textsc{def.obl.m} \\
\glt `A box came [floating on the water]. The box blocked the (water) mill.'\label{ex.daiindex2} \\\hfill [MF.75]--[MF.76] 
\z

\subsubsection{Differential P indexing}
As seen in \S\ref{sect:alignment}, P arguments are indexed by clitic pronouns in present tense constructions but via inflectional person/number suffixes in past transitive constructions. Here, differential P indexing\is{differential object indexing} means a deviation from the canonical ergative construction whereby the object is not indexed on the verb. Verbal affixes are obligatory indexes of direct objects in canonical ergative constructions, illustrated by the following example. 
\ea
\textit{to minit quɫ kerđaw.} \\
\gll to \textbf{min}=it quɫ kerđ-\textbf{a}=û \\
\textsc{2sg} \textsc{1sg=2sg:A} pierced do.\textsc{pst-1sg:O=}and \\
\glt `You disabled me.' \hfill [PW.30]
\z 

In clauses with OAV order, the verb tends to agree with the topical O. 
\ea 
\textit{werêsekê min warđêne.} \\
\gll werês(e)-ekê min warđê=ne \\
rope-\textsc{def.f.sg} \textsc{1sg} eat.\textsc{pst.ptcp.f=cop.3sg.f:O} \\
\glt `[The lion said], `I have eaten the rope.’' \hfill [ÇH.85]
\z

The expected P indexing is sometimes absent with inanimate Ps that are plural\is{plural}. In the following examples, the verb has a default \textsc{3sg.m} inflection and does not agree with the plural\is{plural} object. 
\ea 
\textit{penc çemçêşa nîyanre.} \\ 
\gll penc çemç(e)-ê=şa nîya=n=re \\ 
five spoon\textsc{.m-pl.dir=3pl:A} put\textsc{.pst.ptcp.m=cop.3sg.m:O=povb} \\  
\glt `They (my family) had set five spoons [on the tablecloth].' \hfill [JE.46]
\z 

\ea 
\textit{ewêş nîya biraw wêş.}\\
\gll ewê=ş nîya bira-û wê=ş\\
\textsc{3pl.dir=3sg:A} put\textsc{.pst.3sg.m} brother\textsc{.gen.ez} \textsc{reflx=3sg:PSR}\\
\glt `He made them (the ogres) his brothers.' \hfill[ME.99]
\z 

O-past indexing may also be absent due to affix co-optation by a higher-ranked argument in terms of animacy. In the following example, the expected \textsc{3pl} O-agreement suffix is absent on the verb because its slot has been taken over by the \textsc{3sg} suffix indexing the oblique argument\is{oblique argument}.  
\ea
\textit{kîyast sênze danê heserê\c{s}a da pene.} \\
\gll kîyast sênze danê heser(e)-ê=\c{s}a da-{\O} pene \\
 send\textsc{.pst} thirteen \textsc{clf.pl} mule\textsc{.f-pl.dir=3pl:A} give.\textsc{pst-3sg:R} to \\
\glt `He [the king] sent [his men]. They [his men] gave him (Imam Ali) thirteen mules.' \hfill [ÇH.69]  
\z 

\subsubsection{Differential A flagging\is{differential A flagging}}
The alignment\is{alignment} system licenses case marking\is{case marking} for A arguments. In TAM constructions derived from the present stem of the verb, A is marked in the direct case\is{direct case}, realised as a zero suffix in the singular\is{singular} and -\textit{ê} in the plural\is{plural}. By contrast, in verbal categories derived from the past stem of the verb, the A argument should be, by default, accompanied by the oblique case\is{oblique case} suffixes. The split alignment\is{split alignment} is only relevant for third-person nouns and pronouns. Speech act pronouns\is{speech act pronouns} have lost the case distinction\is{case distinction} (see \S\ref{tab:case-pronouns}). The following example illustrates the oblique case\is{oblique case} marking on the A-past argument.
\ea 
\textit{cafir sanî fermawan, `lodê!'} \\ 
\gll cafir san-î fermawa=n lo-dê \\ 
 \textsc{pn} \textsc{pn-m.sg.obl} say\textsc{.pst.ptcp.m=cop.3sg.m:O} go.\textsc{prs.imp-2pl:S} \\ 
\glt `Jafir San said, `Go [and bring him]!’' \hfill [ŞC.36]
\z 

In reality, not all A-past arguments are oblique-marked\is{oblique case}. A token frequency count of overt As reveals that a quarter of the third person As are not oblique-marked\is{oblique case} (see \S\ref{sect:DAM}). The data suggest that information prominence triggers oblique case marking\is{oblique case} on A-past arguments. The latter operates at two levels: ``local'' and ``global'' \citep[]{chappell2019optional}. In Hewramî\il{Hewramî}, the former is generally associated with oblique\is{oblique case} marking and the latter with direct case\is{direct case} marking. ``Local'' prominence is associated with the A argument being in narrow focus\is{narrow focus} and contrastive focus\is{contrastive focus}. In (\ref{ex2}), the case marking\is{case marking} on \textit{her} `donkey' is triggered by its contrast with \textit{min}.
\ea 
\textit{î her-î zûwaniş zana min hîçim nezanan.} \\
\gll î \textbf{her-î} zûwan=iş zana-{\O} min hîç=im ne-zana=n \\ 
 \textsc{dem.prox} donkey\textsc{-sg.obl.m} language\textsc{.m.sg.dir=3sg:A} know\textsc{.pst-3sg.m:O} \textsc{1sg} nothing\textsc{=1sg:A} \textsc{neg-}know\textsc{.pst.ptcp.m=cop.3sg.m:O} \\ 
\glt `The donkey knew the [Sheikh’s] language; I didn’t know a thing!' \label{ex2} \\ \hfill [HB.71] 
\z 

Case marking\is{case marking} on the A argument can also be triggered by global prominence. According to \citegen{mcgregor2006focal} ``expected actor principle'', in episodes of discourse with an expected actor, the actor can be left unmarked after its introduction. Any deviation from the expected actor is marked in the ergative case. One of the factors conditioning differential subject marking\is{differential subject marking} seems to be topic continuity. In the following excerpt, the established direct-marked topic of the intransitive clause in (\ref{ex.dam-pro1a}), is repeated with the transitive clause in (\ref{ex.dam-pro1b}), even though the oblique\is{oblique case} form \textit{ađîşa} is expected with the latter clause. 
\ea
\ea[]{ 
\textit{tenya ađê luw{ɛ}nê.} \\ 
\gll tenya \textbf{ađê}\textsubscript{i} luw{ɛ}=nê \\ 
 only \textsc{3pl.dir} go\textsc{.pst.ptcp.pl=cop.3pl:S} \\  
\glt `Only they\textsubscript{i} (Baba Khwada, Hama the Invisible, and Little Hama) went [to Iraq].'  \label{ex.dam-pro1a}
 }
 \ex[]{
\textit{êtir ađê watenşa, `ême diruwê meyeymê.} \\ 
\gll êtir \textbf{ađê}\textsubscript{i} wate=n=şa ême diruwê me-de-îmê \\ 
 \textsc{disc.ptcl} \textsc{3pl.dir} say\textsc{.pst.ptcp.m=cop.3sg.m:O=3pl:A} \textsc{1pl} lie\textsc{.f} \textsc{neg.ind-}give\textsc{.prs-1pl:A} \\ 
\glt `They\textsubscript{i} said, ‘We are not going to lie." \label{ex.dam-pro1b} \hfill [BP.116]--[BP.117]
}
\z 
\z 

In terms of information structure, non-oblique marked overt A-past arguments tend not to carry the nuclear stress. In other words, they behave like topics and contain given information (see \S\ref{sect:DAM} for discussion).

\subsubsection{Differential P flagging}
Direct objects in clauses based on present stem verbs exhibit differential P flagging\is{differential P flagging}. The basic pattern is that direct objects which are definite-marked (\ref{ex.def-obl}), proper nouns (\ref{ex.human-obl}), inanimate direct objects with definite reading (\ref{ex.def-obl-inanim}), and direct objects modified by demonstrative pronouns (\ref{ex.def-obl-dem}) are marked in the oblique case\is{oblique case}. 
\ea 
\textit{zeřekey bere.} \\ 
\gll \textbf{zeř-ekey} b\'er-e \\ 
 money\textsc{.m-def.m.sg.obl} take\textsc{.prs.imp-2sg:A} \\  
\glt `Take the money.' \hfill [JP.104] \label{ex.def-obl}
\z 

\ea
\textit{hêyasî bizindê!} \\
\gll \textbf{hêyas-î} bi-zin-dê \\
\textsc{pn-m.sg.obl} \textsc{imp}-take\_out\textsc{.prs-2pl:A} \\
\glt `Throw Heyas out!' \hfill [HS.15]  \label{ex.human-obl}
\z 

\ea 
\textit{muxteserû kelamî, yaney yozawe.} \\ 
\gll muxteser-û kelam-î \textbf{yane-î} yoz-a=we \\ 
 summary\textsc{.m-ez.gen} speech\textsc{-m.sg.obl} house\textsc{.m-sg.obl} find\textsc{.prs.ind-3pl:A=compl} \\ 
\glt `To cut a long story short [lit. gist of speech], they found the house.' \label{ex.def-obl-inanim} \\\hfill[JP.205] 
\z 

\ea 
\textit{a esbî zînî kere peym.} \\ 
\gll \textbf{a} \textbf{esb-î} zînî k\'er-e pey=m \\ 
 \textsc{dem.dist} horse\textsc{-m.sg.obl} saddle do\textsc{.prs.imp-2sg:A} for\textsc{=1sg:R} \\  
\glt `Saddle up the horse for me.' \hfill [ŞC.52] \label{ex.def-obl-dem}
\z 

Similarly, nouns whose referents have been mentioned in the previous discourse tend to carry case marking\is{case marking}. In the following example, the case marking\is{case marking} on \textit{yewe} is triggered by its referent being evoked in previous discourse. 
\ea 
\textit{werû mecbûrî yewayç bero.} \\ 
\gll wer-û mecbûrî-(î) \textbf{yew(e)-a}=îç {} ber-o \\ 
 out\_of\textsc{-ez.gen} obligation\textsc{-m.sg.obl} barley\textsc{.f-pl.obl=add} {} take\textsc{.prs.ind-3sg:A} \\ 
\glt `Out of obligation, he took the barley seeds, too.' \hfill [JP.29]
\z 

Similarly, in principle, indefinite-marked direct objects with specific reference can get oblique case marking\is{oblique}. In (\ref{ex.dom.spec-indf2a}), the indefinite-marked direct object has been previously mentioned and is possessed. 
\ea 
\textit{her saɫê meyo kuřêwît bera.} \\
\gll her saɫ(e)-ê m-e-y-o \textbf{kuř-êw-î=t} ber-a \\
each year\textsc{.f-indf} \textsc{ind-}come\textsc{.prs-ep-3sg:S} son\textsc{-indf-m.sg.obl=2sg:PSR} take.\textsc{prs.ind-3pl:A} \\
\glt `[With] each year that comes, they take one of your sons.' \hfill [ÇH.26] \label{ex.dom.spec-indf2a}
\z 

On the other hand, indefinite-marked direct objects (\ref{ex.indf-marked}), indefinite-inspecific direct objects (\ref{ex.indf-inspecific-not-marked}), and generic nominals (\ref{ex.generic-not-marked})--(\ref{ex.indf-not-marked}), are not oblique-marked.
\ea
\textit{çolekêwe gêro minyo baxeɫêş.} \\ 
\gll \textbf{çolek(e)-êwe} gêr-o mi-ny(e)-o baxeɫ-ê=ş \\ 
sparrow\textsc{.f-indf} take\textsc{.prs.ind-3sg:A} \textsc{ind-}put\textsc{.prs-3sg:A} embrace\textsc{-indf=3sg:PSR} \\ 
\glt `[He] grabbed a sparrow [and] put it on his chest [under his clothing].' \label{ex.indf-marked} \\ \hfill [DP.36] 
\z 

\ea 
\textit{ađ dastanê zano.} \\
\gll ađ \textbf{dastane-ê} zan-o \\
\textsc{3sg.m.dir} tale-\textsc{pl.dir} know.\textsc{prs.ind-3sg:A} \\
\glt `He knows tales.’  \label{ex.indf-inspecific-not-marked}
\z 

\ea 
\textit{toyç nan werî.} \\ 
\gll to=îç \textbf{nan} wer-î \\ 
 \textsc{2sg}\textsc{=add} bread.\textsc{dir.m} eat\textsc{.prs.ind-2sg:A} \\ 
\glt `You will eat your meal.' \hfill [HB.41]   \label{ex.generic-not-marked}
\z

\ea 
\textit{meɫa mara mareş biřa peyş.} \\ 
\gll \textbf{meɫa} m-ar-a mare=ş biř-a pey=ş \\ 
 mullah\textsc{.m} \textsc{ind-}bring\textsc{.prs-3pl:A} marriage\textsc{=3sg:O} cut\textsc{.prs.ind-3pl:A} for\textsc{=3sg:R} \\ 
\glt `They fetched a Mullah [and] married her (the girl) to him (the shepherd’s son).' \hfill [KŞ.88] \label{ex.indf-not-marked}
\z 

There are also morphosyntactic constraints on differential P flagging. Quantified direct object entities are not oblique-marked\is{oblique case}, regardless of the information prominence. This apparent anomaly seems to be caused by the fact that numerals\is{numerals} and quantifiers\is{quantifiers}, by default, trigger direct case\is{direct case} marking on the nominal heads. 
\ea
\textit{jenêç nîşore duwê zaroɫê wîno.} \\ 
\gll jen(î)-ê=ç nîş-o=re \textbf{duwê} \textbf{zaroɫ(e)-ê} wîn-o \\ 
 woman\textsc{.f.sg.obl=add} sit\textsc{.prs.ind-3sg:S=povb} two child\textsc{-pl.dir} see\textsc{.prs.ind-3sg:A} \\ 
\glt `The wife gave birth to two babies. [Lit. She sat down [and] saw two babies.]'  \hfill [ZB.24]
\z

\subsubsection{Differential case marking on non-core arguments}
Hewramî\il{Hewramî} also features differential case marking on non-core arguments. This means that not all non-core arguments are marked in the same case\is{case}. The relevant arguments are recipients\is{recipient}, goals, beneficiaries, comitatives, etc. Differential case marking on non-core arguments depends partly on the type of adpositions used to flag non-core arguments\is{oblique arguments}. The oblique case marking tends to be absent when the non-core argument is flagged by a postposition (\ref{dobl1}) or bare (\ref{ex.bare11}). 
\ea 
\textit{jenekêm qomyaş venî kelekewe.} \\ 
\gll jen(î)-ekê=m qomya=ş venî \textbf{kel-eke=we} \\ 
 woman\textsc{.f-def.f.sg=1sg:PSR} happen\textsc{.pst.3sg:S=3sg:R} at mountain\textsc{-def.m.sg.dir=post} \\  
\glt `My wife was about to deliver a baby in the mountain.' \hfill [ZQ.14] \label{dobl1} 
\z 

\ea \label{ex.bare11}
\textit{melowe yane.} \\ 
\gll me-l-o=we \textbf{yane} \\ 
 \textsc{neg.ind-}go\textsc{.prs-3sg:S=compl} house\textsc{.m} \\ 
\glt `He didn't go back home.' \hfill[JH.109] \label{ex.goal.bare}
\z 

On the other hand, except for preposition \textit{ta} `until' (\ref{dobl2}), other prepositions can, in principle, trigger oblique marking on the non-core arguments (\ref{dobl3}).
\ea
\textit{ta meřeber am{ɛ}.} \\ 
\gll ta \textbf{meřeber} am{ɛ} \\ 
 until \textsc{pn} come\textsc{.pst.3pl:S} \\ 
\glt `They came as far as Marabar.' \hfill [BP.114] \label{dobl2} 
\z 

\ea 
\textit{luwewe pey yaney!’} \\ 
\gll lu-e=we pey \textbf{yane-î} \\ 
 go.\textsc{prs.imp-2sg:S=compl} to house\textsc{.m-sg.obl} \\ 
\glt `[Now] go back home!' \hfill [JH.118] \label{dobl3}
\z 

In \S\ref{sect:diffobl}, I outline other factors that are important in differential case marking on non-core arguments, including animacy, role, etc.
\end{sloppypar}
