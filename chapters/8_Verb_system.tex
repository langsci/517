\chapter{Verb system: stems and major derivational processes}
\begin{sloppypar}
Verbs inflect for the morphological and morphosyntactic features of number\is{number}, person, gender\is{gender} (only in \textsc{3sg} in verbs derived from past stem), tense, mood, and aspect. On a par with related Iranian languages, the verb has a two-stem system divided into two tense-based categories roughly equivalent to present and past tenses. 

\section{The two verb stems}
The verb has two basic kinds of verbal stems, divided into \textsc{present} and \textsc{past} stems. The present stem is descended from the Old Iranian present stem and features accusative alignment\is{accusative alignment}. The past stem is descended from the resultative participle and entails ergative alignment\is{ergative alignment}. 

More broadly, verbal morphological stems are organised into the following categories: present, imperative, past, resultative participle, and infinitive\is{infinitive}. The last three are formed based on the past stem. The imperative is based on the present stem (see \S\ref{imperative-stem}).
\TabPositions{3.85cm}
\ea
\textit{berđey} `to take’ \\
Present stem \tab  \textit{ber}- \\
Imperative stem \tab  \textit{ber-}\\
Past stem \tab  \textit{berđ}- \\
Resultative participle \tab \textit{berđe} (\textsc{m}); \textit{berđê} (\textsc{f/pl}) \\
Infinitive\tab  \textit{berđ-ey} \\
\z
 
The division into present and past generally equates to using the stems in different TAM formations. Thus, for example, the present stem is used in the formation of present progressive. An exception is the past progressive and past habitual, which are based on the present stem. The following list summarises the TAM categories that are built on the present stem.

\newpage
\begin{itemize}
    \item present/future indicative\is{present indicative}
    \item present subjunctive\is{present subjunctive}
    \item imperative
    \item present progressive\is{present progressive} 
    \item past progressive\is{past progressive}
    \item past habitual\is{habitual past}
    \item irrealis past\is{irrealis past}
\end{itemize}

The past stem is the basis for the formation of the following TAM categories:
\begin{itemize}
    \item past perfective\is{past perfective} (preterite)
    \item past conditional\is{past conditional}
    \item perfect
    \item perfect progressive\is{perfect progressive}
    \item irrealis perfect\is{irrealis perfect}
    \item conditional perfect\is{conditional perfect}
    \item past perfect\is{past perfect}
    \item perfect pluperfect\is{perfect pluperfect}
   
\end{itemize}

\subsection{Stem types} 
There are different classes of stems in Hewramî\is{Hewramî} that are motivated by the relation between present and past stems. In each class, the present stem is considered the unmarked stem, to which a segment is added, resulting in the past stem. In presenting the examples of stem pairs, I follow the tradition in Iranian linguistics of taking the infinitive\is{infinitive} as the citation form. The infinitive\is{infinitive} is formed by adding the suffix \textit{-ey} to the past stem (see \S\ref{sect:infinitive}). This motivates presenting the past stem closer to the citation form since it bears more similarity to the infinitive\is{infinitive} than the present stem. Thus, in presenting stem types, the order is past stem first, present stem second.

The adoption of the present stem as the basic stem is motivated by two reasons: first, the past stem can be derived from the present stem by the addition of a past formation affix or stem modification (see Table \ref{tab:stem-c}). Second, the present stem is the basis for deriving passive (\ref{ex:pass1}) and most causative stems (\ref{ex:caus1}).\footnote{See \citet{Ahmadi_formal_2025} for a similar treatment of stem derivation in Central Kurdish\il{Kurdish!Central}.}

\TabPositions{1.5cm,3.25cm,5cm,7.30cm,9.15cm}
\ea \label{ex:pass1}
    \tab Transitive stem \tab \tab Passive stem\\
     Gloss \tab  \textsc{pst} stem \tab  \textsc{prs} stem \tab  Gloss \tab  \textsc{pst} stem \tab  \textsc{prs} stem \\
     `sell' \tab  \textit{wiret-}\tab  \textit{wireş-} \tab  `to be sold' \tab  \textit{wireşye-} \tab  \textit{wireşya-} \\
\z

\ea \label{ex:caus1}
     \tab Intransitive stem \tab \tab Causative stem\\
     Gloss \tab  \textsc{pst} stem \tab  \textsc{prs} stem \tab  Gloss \tab  \textsc{pst} stem \tab  \textsc{prs} stem \\
     `sleep' \tab  \textit{wit-}\tab  \textit{ûs-} \tab  `put to sleep' \tab  \textit{ûsna-} \tab  \textit{ûsn-} \\
     `burn' \tab  \textit{sot-} \tab  \textit{soç-}\tab  `burn' \tab  \textit{soçna-}\tab  \textit{soçn-} \\
\z

The following stem classes can be identified in Tekht H.\il{Hewramî!Tekht}, summarised in Table \ref{tab:stem-c}. As can be seen, in most cases, the shape of the past stem can be guessed from the present stem, e.g., by adding a dental preterite /t/. 
\begin{table}
\caption{Correspondence between present and past stems in Hewramî}
\fittable{%
\begin{tabular}{clllll}
\lsptoprule
Type & \centering \textsc{prs} stem ending & \centering \textsc{pst} stem ending &\textsc{prs} & \textsc{pst}& Gloss \\
\midrule
1 & -C & \textit{-a} & \textit{piř-} & \textit{piřa-} & `fly'\\
2 & \textit{-r}, \textit{-ş}, \textit{-n}, \textit{-s} & added dental stop \textit{-d} or \textit{-t} & \textit{kuş-} & \textit{kuşt-}& `kill' \\
3 & \textit{-z}, \textit{-j} & \textit{-st} (for \textit{-z}), \textit{-şt} (for \textit{-j}) & \textit{az-} & \textit{ast-} & `let'\\
4 & -C& \textit{-î} & \textit{biř-} & \textit{birî-} & `cut'\\
5 & \textit{-ye} & \textit{-ya} & \textit{dîye-} & \textit{dîya-} & `look' \\
6 & \textit{-ç} & \textit{-t} & \textit{soç} & \textit{sot} & `burn' \\
7 & -- & suppletive allomorphy & \textit{gin-} & \textit{ket-} & `fall'\\
  &     &                        & \textit{şor-} & \textit{şit-} &`wash'\\
\lspbottomrule
\end{tabular}%}
}
\label{tab:stem-c}
\end{table}

In what follows, we delve into each of these stem types. Admittedly, the classification is not without its flaws. For instance, there is no formal difference between stems that take \textit{a} and \textit{î} in types 1 and type 4, respectively, except that the latter is used with limited number of stems. The presentation is inspired by \citegen{mackenzie_dialect_1966} description of stem relations in Luhon H.\il{Hewramî!Luhon} and \citegen{suleymanov_grammar_2020} description of stem variation in Şirvan Tat\il{Tat}.

\subsubsection{Type 1}
The past stem in this class is characterised by an additional \textit{-a}, typically when the present stem ends in a consonant. The stems used in this class can be either transitive or intransitive. \\
\\
Intransitive:
\TabPositions{2cm,5.5cm,7.5cm}
\ea
Infinitive\tab  \tab \textsc{pst}\tab  \textsc{prs} \\
\textit{tersay}\tab `be afraid’\tab \textit{tersa-}\tab \textit{ters-} \\
\textit{pijmay}\tab `sneeze’\tab \textit{pijma}-\tab \textit{pijm}- \\
\textit{piřay}\tab `fly’\tab \textit{piřa-}\tab \textit{piř-} \\
\textit{bexşay}\tab `forgive’\tab \textit{bexşa-}\tab \textit{bexş-} \\
\textit{waray}\tab `rain’\tab \textit{wara-}\tab \textit{war-} \\
\textit{topay}\tab `die (animal)'\tab \textit{topa-}\tab \textit{top-} \\
\textit{cimay}\tab `move’\tab \textit{cima-}\tab \textit{cim-} \\
\textit{řemay}\tab `run’\tab \textit{řema-}\tab \textit{řem-} \\
\z

Transitive:
\TabPositions{2cm,5.5cm,7.5cm}
\ea
\textit{mařay}\tab `break’\tab \textit{mařa}-\tab \textit{mař}- \\
\textit{jinasay}\tab `know (somebody)’\tab \textit{jinasa}-\tab \textit{jinas}- \\
\textit{zanay}\tab `know (something)’\tab \textit{zana}-\tab \textit{zan}- \\
\textit{fařay}\tab `change’\tab \textit{fařa-}\tab \textit{fař-} \\
\textit{persay}\tab `ask’\tab \textit{persa-}\tab \textit{pers-} \\
\textit{şanay}\tab `scatter, sow’\tab \textit{şana}-\tab \textit{şan}- \\
\textit{kaɫay}\tab  `plough' \tab  \textit{kaɫa-}\tab  \textit{kaɫ-} \\
\textit{sanay} \tab `buy’\tab \textit{sana}-\tab \textit{san}- \\
\textit{biřfanay}\tab `snatch’\tab \textit{biřfana}-\tab \textit{biřfan}- \\
\z

In Luhon H.\il{Hewramî!Luhon}, the past stem for `buy’ (\textit{esay}) and `snatch’ (\textit{eřfay)} is short of the final \textit{-n} of the present stem \citep[28]{mackenzie_dialect_1966}{}, which is an exception to the general rule of the past stem having an additional segment compared to the present stem, thus \textit{esay} `buy’: \textit{esa-} (\textsc{pst}); \textit{esan-} (\textsc{prs}); \textit{eřfay} `snatch’: \textit{eřfa-} (\textsc{pst}); \textit{eřfan-} (\textsc{pst}). The Tekht Hewramî\il{Hewramî!Tekht} pairs in (\ref{ex.sanay-text}), repeated for convenience, show that the relation between present and past stems is regularised for these verb stems.

\ea \label{ex.sanay-text}
\textit{sanay} \tab  `buy' \tab  \textit{sana-} \tab  \textit{san-} \\
\textit{biřfanay}\tab  ‘snatch’ \tab  \textit{biřfana-}\tab  \textit{biřfan-}\\
\z

In addition, the causative\is{causative} counterparts of some of the intransitive verbs seen above show the same relationship between present and past stems (see \S\ref{sect:causative} for causative morphology). In (\ref{ex.caus7}), the causative stem is formed using \textit{-n} for the present stem and \textit{-na} for the past stem, added to the present base of the verb.

\ea\label{ex.caus7}
\textit{tersnay}\tab `scare’\tab \textit{tersna-}\tab \textit{tersn-} \\
\textit{bexşnay}\tab `distribute’\tab \textit{bexşna-}\tab \textit{bexşn-} \\
\textit{topnay}\tab `kill (animal)’\tab \textit{topna-}\tab \textit{topn-} \\
\textit{cimnay}\tab `move’\tab \textit{cimna-}\tab \textit{cimn-} \\
\textit{řemnay}\tab `make run’\tab \textit{řemna-}\tab \textit{řemn-} \\
\z

\subsubsection{Type 2}
Stem pairs in type 2 are characterised by the presence of a dental stop \textit{-d} or \textit{-t} in the past stem. This is most common when the present stem ends in \textit{-r}, \textit{-ş}, \textit{-n}, and \textit{-s}. The dental stop assimilates in voicing to the past stem's final consonant.

\ea
\textit{kuştey}\tab `kill’\tab \textit{kuşt-}\tab \textit{kuş-} \\
\textit{kêştey}\tab `weigh, pull’\tab \textit{kêşt-}\tab \textit{kêş-} \\
\textit{kerdey}\tab  `do'\tab  \textit{kerd-} \tab  \textit{ker-} \\
\textit{taştey}\tab `shave, cut’\tab \textit{taşt-}\tab \textit{taş-} \\
\textit{doştey}\tab `milk’\tab \textit{doşt-}\tab \textit{doş-} \\
\textit{berđey}\tab `take’\tab \textit{berđ-}\tab \textit{ber-} \\
\textit{arđey}\tab `bring’\tab \textit{arđ-}\tab \textit{ar-} \\
\textit{gêrtey}\tab `grab’\tab \textit{gêrt}-\tab \textit{gêr}- \\
\z

With the present stem ending in \textit{-n} and \textit{-s}, the dental preterite undergoes total progressive assimilation, resulting in the loss of the dental preterite in the past stem.
\TabPositions{2cm,4.5cm,8.5cm}
\ea 
\textit{nivîsey}\tab `write’\tab \textit{nvîs(s)-} < \textit{nvîst}-\tab \textit{nvîs-} \\
\textit{kenney}\tab `dug’\tab \textit{ken(n)-} < \textit{kend-}\tab \textit{ken-} \\
\textit{řêsey}\tab `spin’\tab \textit{řês(s)-} < \textit{řêst-}\tab \textit{řês-} \\
\textit{wistey-(re)} \tab  `throw’ \tab  \textit{wis(s)=re} < \textit{wist-re}\tab  \textit{wiz=re} \\ 
\z

Exceptions:
\TabPositions{2cm,4.5cm,6.5cm}
\ea
\textit{wiretey}\tab `sell’\tab \textit{wiret-}\tab \textit{wireş-} \\
\textit{kîyast}\tab  `send' \tab  \textit{kîyas-} \tab  \textit{kîyan-} \\
\z

\subsubsection{Type 3}
The past stem ending in this class \textit{-st} and \textit{-şt} corresponds to the present stem ending in \textit{-z} and \textit{-j}, respectively. As it seems, the stem pairs in this class are similar to those in class 2, the difference being that the voiced sibilants in the past stem assimilate in voicing to the unvoiced dental stop \textit{-t}.
\TabPositions{2cm,5.5cm,7.5cm}
\ea
\textit{miştey}\tab `suck up’\tab \textit{mişt-}\tab \textit{mij-} \\
\textit{brêştey}\tab `roast’\tab \textit{brêşt-}\tab \textit{brêj-} \\
\textit{astey}\tab `let’\tab \textit{ast-}\tab \textit{az-} \\
\textit{gestey}\tab `bite’\tab \textit{gest-}\tab \textit{gez-} \\
\textit{wirastey}\tab `sew’\tab \textit{wirast-}\tab \textit{wiraz-} \\
\textit{wastey} \tab  `request, beg’\tab  \textit{wast-} \tab  \textit{waz-} \\
\textit{westey-re} \tab  `get off’\tab \textit{west=re} \tab  \textit{wez=re} \\
\textit{yostey=we}\tab  `find’\tab  \textit{yos(s)=we} \tab  \textit{yoz=we} 
\z

\subsubsection{Type 4}
The stem pairs in this subclass are characterised by an additional vocalic \textit{-î} in the past stem. The vocalic \textit{-î} on the past stem is probably descended from a secondary suffix \textit{-îd}, the d of which dropped due to the historical post-vocalic deletion of /d/ \citep[]{mohammadirad_lenition_nodate}{}.

\ea
\textit{biřyey}\tab `cut’\tab \textit{biřî}\tab \textit{biř}- \\
\textit{dizyey}\tab `steal’\tab \textit{dizî}-\tab \textit{diz}- \\
\textit{seřyey}\tab `wipe’ (\textsc{tr})\tab \textit{seřî-}\tab \textit{seř}- \\
\textit{çinyey}\tab `pick, pluck, weave’\tab \textit{çinî}-\tab \textit{çin}- \\
\textit{řinyey}\tab  `scratch' (\textsc{tr})\tab  \textit{řinî-} \tab \textit{řin-}\\
\textit{şîyey} \tab  `go’ (\textsc{intr})\tab  \textit{şî-} \tab  \textit{ş-} \\
\textit{misyey}\tab `learn’\tab \textit{misî-}\tab \textit{mis}- \\
\z

\subsubsection{Type 5}
The stem pairs in this subclass show a relationship between the present stem ending in \textit{-ye} and the past stem ending in \textit{-ya}. This subclass typically comprises unaccusative verbs.
\TabPositions{2cm,5.85cm,7.85cm}
\ea
\textit{fařyay}\tab `change' (\textsc{intr})\tab \textit{fařya}-\tab \textit{fařye}- \\
\textit{toryay}\tab `get offended’\tab \textit{torya}-\tab \textit{torye}- \\
\textit{dîyay}\tab `look’\tab \textit{dîya}-\tab \textit{dîye}- \\
\textit{pijgyay}\tab `scatter' (\textsc{intr})\tab \textit{pijgya}-\tab \textit{pijgye}- \\
\textit{xirabyay}\tab `worsen’\tab \textit{xirabya}-\tab \textit{xirabye}- \\
\textit{piřokyay}\tab `be exhausted'\tab \textit{piřokya}-\tab \textit{piřokye}- \\
\textit{temamyay}\tab  `finish'\tab \textit{temamya}-\tab \textit{temamye}- \\ 
\textit{bezyay}\tab  `be vanquished' (\textsc{intr}) \tab  \textit{bezya-} \tab \textit{bezye-} \\
\z

This subclass also contains present stems ending in \textit{-e}:
\TabPositions{2cm,5.5cm,7.5cm}
\ea
\textit{day}\tab `give’\tab \textit{da}-\tab \textit{de}- \\
\textit{gay}\tab `copulate’\tab \textit{ga}-\tab \textit{ge-} \\
\z

\subsubsection{Type 6}
The stem pairs in this subclass have present stems ending in a voiceless post-alveolar affricate and past stems with the dental marker.

\ea
\textit{sotey}\tab `burn’ (\textsc{intr})\tab \textit{sot-}\tab \textit{soç}- \\
\textit{wetey}\tab `doff’\tab \textit{wet-}\tab \textit{weç-} \\
\textit{wêtey}\tab `sift’ \tab \textit{wêt-}\tab \textit{wêç}- \\
\textit{watey}\tab `say, tell’\tab \textit{wat-}\tab \textit{waç-, aç-} \\
\textit{petey}\tab `bake’\tab \textit{pet-}\tab \textit{peç-} \\
\textit{pêtey}\tab `fold (grass)’\tab \textit{pêt-}\tab \textit{pêç-} \\
\textit{patey}\tab `cut (hair)’\tab \textit{pat-}\tab \textit{paç-} \\
\z

The present stem of \textit{watey} `say, tell’ features stem allomorphy\is{stem allomorphy}. Depending on the morphological context, it can be realised as either \textit{waç-} or \textit{aç-}. The variant \textit{waç-} occurs in more morphological contexts than \textit{aç-} and may be considered the inherited form, given its similarity to the past stem \textit{wat}. The variant \textit{aç-} is limited to occur with present indicative\is{present indicative} verbs. \textit{waç} occasionally appears in the present indicative\is{present indicative}, e.g., \textit{waçmê} `we say' [JH.23].

\begin{table}[htp]
    \centering
    \begin{tabular}{llll}
\lsptoprule
TAM categories & Stem& Example & Gloss \\
\midrule
Present indicative\is{present indicative} & \textit{aç-} & \textit{m-aç-û }& `I say' \\
Present subjunctive\is{present subjunctive}&  \textit{waç-} & \textit{waç-û }& `that I say'  \\
Imperative&  \textit{waç-} & \textit{waç-e}& `Say!' \\
Past habitual & \textit{waç-}& \textit{waç-ên-a}& `I used to say' \\ 
\lspbottomrule
    \end{tabular}
    \caption{Morphological allomorphy in the present stem of the verb \textit{watey} `to say'}
    \label{tab:stem.allomorphy.say}
\end{table}

\subsubsection{Type 7}
The stem pairs in this subclass feature suppletive allomorphy\is{suppletive allomorphy} with two groups of verbs. In the first group, illustrated in (\ref{ex.supp-allomorphy}), suppletive allomorphy is completely unpredictable. Thus, no regular relationship can be established between stems.
\TabPositions{2.5cm,5cm,7.5cm}
\ea \label{ex.supp-allomorphy}
\textit{kewtey}\tab `fall’\tab \textit{kewt-}\tab \textit{gin-} \\
\textit{witey}\tab `sleep’\tab \textit{wit-}\tab \textit{ûs-} \\ 
\textit{dîyey, wînay}\tab `see’\tab \textit{dî-, wîna-}\tab \textit{wîn-} \\
\textit{şitey}\tab `wash’\tab \textit{şit-}\tab \textit{şor-} \\
\textit{amay}\tab `come’\tab \textit{ama-}\tab \textit{e-} \\
\z

The second group features segmentally similar stems. The stem alternation has been subject to different morphophonological processes, making it hard to establish a regular relationship between the two stems.\footnote{The past stem for verb \textit{kîyastey} `to send’ (pst. \textit{kîyast-}, prs. \textit{kîyan-}) is derived by the addition of the past marking <st>, following an initial deletion of \textit{-n}.}
\TabPositions{2.75cm,5.25cm,7.55cm}
\ea
\textit{ejyay}\tab `guess’\tab \textit{ejya}-\tab \textit{ejo}- \\
\textit{mitey}\tab `spill’ (\textsc{tr})\tab \textit{mit}-\tab \textit{mij-} \\
\textit{merđey}\tab `die’\tab \textit{merđ}-\tab \textit{mir}- \\
\textit{wardey}, \textit{werdey}\tab `eat’\tab \textit{ward-}, \textit{werd}-\tab \textit{wer-} \\
\textit{kîyastey} \tab `send’\tab \textit{kîyast}-\tab \textit{kîyan-} \\
\textit{luway}\tab `go’\tab \textit{luwa-, la-}\tab \textit{l-, lu-} \\
\z

The present stem of \textit{luway} `go’ exhibits stem allomorphy\is{stem allomorphy}.\footnote{There are two verbs for `to go’: \textit{luway}, and \textit{şîyey}. The latter is limited certain expressions: For instance, \textit{ce hoş şîyey} `to forget', [lit. `go out of mind']; \textit{diɫ şîyey} `be fond of' [lit. `heart to go'].} Depending on morphological context, it can be realised as either \textit{l-} or \textit{lu-}. The variant \textit{lu-} seems to be the old stem based on its relation to the past stem \textit{luwa}. \textit{l-} is now used entirely in the present indicative\is{present indicative} and present subjunctive\is{present subjunctive} paradigms. However, occasionally the variant \textit{lu-} can be attested, e.g., \textit{mi-lw-a} `they go' [ZB.23], where /u/ changes to the glide /w/. In the imperative, the choice between the two allomorphs is triggered by the presence of the imperative \textit{bi-} before the stem; see Table \ref{tab:stem.allomorphy.Go}.
\begin{table}[htp]
    \begin{tabular}{lllll}
\lsptoprule
TAM categories & Stem& Example & Gloss \\
\midrule
Present indicative\is{present indicative} & \textit{l-} & \textit{mi-l-û }& `I go' \\
Present subjunctive\is{present subjunctive}&  \textit{l-} & \textit{bi-l-û }& `that I go'  \\
Imperative& \textit{l-}/ \textit{lu-} & \textit{lu-e} [DG.57], \textit{bi-l-e} [DG.56]& `Go!' \\
& & \textit{bi-lu-e} & \\
Past habitual\is{habitua pastl} & \textit{lu-}& \textit{lu-ên-mê}& `We used to go' \\ 
\lspbottomrule
    \end{tabular}
    \caption{Morphological allomorphy in the present stem of the verb \textit{luway} `to go'}
    \label{tab:stem.allomorphy.Go}
\end{table}

Similarly, the past stem of \textit{luway} exhibits stem allomorphy\is{stem allomorphy}. The relevant allomorphs are \textit{luwa-} (with epenthetic /w/) and \textit{la-}. The former occurs in affirmative TAM verbal categories, the latter in the negation of TAM verbal categories. The /ɛ/ in the past conditional and conditional perfect forms is the result of the merger between the stem-final vowel \textit{-a} and the initial vowel for the augment \textit{-ên}. 
\begin{table}[htp]
    \centering
    \begin{tabular}{lll}
\lsptoprule
TAM categories & Affirmative & Negative \\
\midrule
Past perfective\is{past perfective}& \textit{luwa-(a)nê}& \textit{ne-la-(a)nê} \\
Past conditional\is{past conditional} & \textit{luwɛn-ê }& \textit{ne-lɛn-ê} \\
Perfect&  \textit{luwa=na} & \textit{ne-la=na} \\
Irrealis perfect\is{irrealis perfect}& \textit{luwa=b-û}& \textit{ne-la=b-û} \\
Conditional perfect\is{conditional perfect} & \textit{luwa=bîɛn-ê}& \textit{ne-la=bîɛn-ê} \\
Past perfect\is{past perfect}& \textit{luwa=b-ên-ê} & \textit{ne-la=b-ên-ê} \\
\lspbottomrule
    \end{tabular}
    \caption{Morphological allomorphy in the past stem of the verb \textit{luway} `to go' in \textsc{1sg}}
    \label{tab:stem.allomorphy.pst.Go}
\end{table}

Similarly, the present stem of \textit{witey} exhibits morphological allomorphy triggered by the TAM category. The stem is \textit{ûs-} in TAM categories built on the present stem of the verb that are preceded by a TAM prefix. On the other hand, in TAM categories built on the present stem where there is no pre-verbal material, the variant \textit{wis-} is used (see Table \ref{tab:stem.allomorphy.sleep}). 
\begin{table}[htp]
    \begin{tabular}{lllll}
\lsptoprule
TAM categories & Stem& Example & Gloss \\
\midrule
Present indicative\is{present indicative} & \textit{ûs-} & \textit{m-ûs-û }& `I sleep' \\
Present subjunctive\is{present subjunctive}&  \textit{ûs-} & \textit{b-ûs-û }& `that I sleep'  \\
Imperative& \textit{ûs-} & \textit{b-ûs-e} & `Go!' \\
Habitual past\is{habitua pastl} & \textit{wis-} & \textit{wis-ên-a} & `I used to sleep' \\ 
Irrealis past & \textit{wis-} & \textit{wis-ên-a} & `I would sleep' \\
\lspbottomrule
    \end{tabular}
    \caption{Morphological allomorphy in the present stem of the verb \textit{witey} `to sleep'}
    \label{tab:stem.allomorphy.sleep}
\end{table}

\subsection{Variation in stem ending}
With some verbs, speakers vary as to which past stem marker to use. This variation is particularly significant in type 2 verbs with a dental marker. While older speakers tend to use the dental marker to build the past stem, younger speakers tend to generalise the past tense marker \textit{-a}, presumably the most productive past stem marking suffix, of type 1 verbs to build the past stem. This suggests that the ending \textit{-a} is the basis for analogical change.
\TabPositions{2cm,5.5cm,8.5cm}
\ea
\textit{taştey}\tab `shave, cut’\tab \textit{taşt-}/ \textit{taşa}-\tab \textit{taş}- \\
\textit{kêştey}\tab `weigh, \textit{}pull’\tab \textit{kêşt-}/ \textit{kêşa-}\tab \textit{kêş}- \\
\textit{lêstey}\tab `lick’\tab \textit{lêst-}/ \textit{lêsa}-\tab \textit{lês-} \\
\textit{poştey}\tab `put on’\tab \textit{poşt-}/ \textit{poşa}-\tab \textit{poş}- \\
\textit{menney}\tab  `remain’\tab  \textit{menn-}/ \textit{mena-} \tab \textit{men-}
\z

A relevant example is the variation in the past stem of the verb \textit{dîyey} `to see'. The basic pattern for the verb is to have a suppletive past stem: prs. \textit{wîn-} vs. pst. \textit{dî}. Some speakers regularise this by extending the past stem formative \textit{-a} to the present stem, resulting in the past stem \textit{wîna-}.

\subsection{Denominal and deadjectival verbs\is{denominal verbs}} \label{sect:denominal-verbs}
In addition to regular verb stems derived from common Iranian verbal forms, Tekht H.\il{Hewramî!Tekht} makes use of a relatively common process of stem formation in which the detransitiviser suffix \textit{-ya} (\textsc{pst})/ \textit{-ye} (see \S\ref{passive}) attaches to some nouns and adjectives and derives new verb stems.
\TabPositions{1.90cm,5.05cm,6.80cm,8.45cm}
\ea
Infinitive\tab  \tab  \textsc{pst}\tab  \textsc{prs}  \\
\textit{xirabyay}\tab `get worse’\tab \textit{xirabya}-\tab \textit{xirabye}- \tab   cf. \textit{xerab} `bad’\\
\textit{temamyay}\tab `finish’ \tab \textit{temamya}-\tab \textit{temamye}- \tab  cf. \textit{temam} `end’\\
\textit{temyay}\tab `be sad’\tab \textit{temyay}-\tab \textit{temye}- \tab  cf. \textit{tem} `fog’\\
\textit{ẍeɫetyay}\tab `be deceived’\tab \textit{ẍeɫetya}-\tab \textit{ẍeɫetye}-\tab  cf. \textit{ẍeɫet} `false’ \\
\textit{teɫefyay}\tab `vanish, fade’\tab \textit{teɫefya}-\tab \textit{teɫefye}- \tab  cf. \textit{teɫef} `waste’\\
\textit{gijyay}\tab `fight’\tab \textit{gijya}-\tab \textit{gijye}- \tab  cf. \textit{gij} `stature’\\
\textit{xiciɫyay} \tab  `be entertained' \tab  \textit{xiciɫya-} \tab  \textit{xiciɫye-} \tab  cf. Ar.\il{Arabic} \textit{xajal} `shame'\\
\textit{dêwyay} \tab  `become angry' \tab  \textit{dêwya-} \tab  \textit{dêwye-}\tab  cf. \textit{dêw} `ogre' \\
\textit{xerepyay}\tab  `decline (mentally)'\tab  \textit{xerepya-}\tab  
\textit{xerepye-}\tab  cf. Ar.\il{Arabic} \textit{xarf} `senile' \\
\z

The transitive counterpart of these denominal verbs\is{denominal verbs} is expressed by replacing the detransitiviser suffix with the causative\is{causative} suffix \textit{-n} (\textsc{prs}); \textit{-na} (\textsc{pst}) (see \S\ref{sect:causative}).

\subsection{Imperative stem} \label{imperative-stem}
As remarked, the imperative is formed based on the present stem of the verb. For the formation of the imperative verb forms, most verbs just use the stressed bare stem and the relevant person suffixes. A limited number of verbs additionally take the imperative/subjunctive prefix, mostly due to phonological conditions (see \S\ref{sec:subj-morph} and \S\ref{sect:imperative} for explanations).

\ea
\gll b-ûs-e \\
\textsc{imp}-sleep.\textsc{prs-2sg} \\
\glt `Sleep!’ \\
\z


\ea
\gll bi-řem-e \\
\textsc{imp}-run.\textsc{prs-2sg} \\
\glt `Run!’ \\
\z

The imperative stem for the frequent verbs \textit{amay} `come’, \textit{luway} `go’, and \textit{watey} `say' exhibits some suppletive morphology concerning the present stem:
\TabPositions{1.75cm,4.5cm,5.70cm,8cm}
\ea
\textit{b-o}\tab [\textsc{imp}-come.\textsc{prs.2sg:S}]\tab `Come!’\tab cf. prs. \textit{e-} `come’ \\
\textit{lu-e}\tab [go\textsc{.prs.imp-2sg:S}]\tab `Go!’\tab  cf. prs. \textit{l-} `go’ \\
\textit{waç-e}\tab  [say.\textsc{prs.imp-2sg:A}]\tab  `Say!'\tab  cf. prs. \textit{aç-} `say' \\
\z 

Of these verbs, the stem for the verb `go' shows variation in appearing as the suppletive stem \textit{lu} or \textit{l-}.

\ea
\textit{bi-lu-e}\tab [\textsc{imp}-go-\textsc{2sg:S}] \tab  `Go!'\\
\textit{lu-e} \tab  [go-\textsc{2sg:S}] \tab  `Go!'\\
\textit{bi-l-e}\tab  [\textsc{imp-}go\textsc{-2sg:S}]\tab `Go!’ \\
\z

\section{Light verb constructions\is{light verb constructions}}
Light verb constructions\is{light verb constructions } consisting of a light verb\is{light verb} and a non-verbal element, equivalent to `pay attention' in English\il{English}, are a productive way of forming new verbs in Tekht H\il{Hewramî!Tekht}. One way to categorise light verb constructions\is{light verb constructions} (LVCs) is according to the light verbs involved in the construction. In the text corpus, the most productive light verbs\is{light verb} are \textit{kerđey} `do’, and \textit{bîyey} `to be, to become’. Less commonly, \textit{kewtey} `fall’, \textit{warđey} `to eat’ and \textit{amay} `to come’ are used in the structure of the LVC. Table \ref{tab:light-V} lays out the common light verbs\is{light verb} with illustrative examples.

\begin{table}[htp!]
{\small\begin{tabular}{lll}
\lsptoprule
\textsc{lv}&\textsc{lvc}&Gloss\\
\midrule
\multirow{4}{*}{\textit{kerđey} `do, make’}&\textit{weş kerđey}&`to make; to build; to cure’ (lit. `well to do’) \\\cline{2-3}
&\textit{zamdar kerđey}& `to injure’ (lit. `wounded to do’) \\\cline{2-3}
&\textit{wiş kerđey} &`to inform’ (lit. `memory to do’) \\\cline{2-3}
&\textit{des kerđey}& `to start' (lit. `hand to do') \\\cline{2-3}
\\
\multirow{3}{*}{\textit{bîyey} `be’}&\textit{řed bîyey} &`to cross’ (lit. `crossing to become’) \\\cline{2-3}
&\textit{peyđa bîyey} &`to appear, to be born’ (lit. `visible to be’) \\\cline{2-3}
&\textit{weş bîyey} &`get healed’ (lit. `well to be’) \\\cline{2-3}
&\textit{nizîk bîyey} &`to approach’ (lit. close to be’) \\\cline{2-3}
\\
\multirow{4}{*}{\textit{kewtey} `fall’} & \textit{neweş kewtey} &`to get ill’ (lit. `ill to fall’) \\\cline{2-3}
&\textit{hetîm kewtey} &`to be left an orphan’ (lit. `orphan to fall’) \\\cline{2-3}
& \textit{pek kewtey }&`to be worried’ (lit. strength to fall’) \\\cline{2-3}
& \textit{paɫ kewtey} &`to lean, to lie down’ (lit. `side to fall’) \\\cline{2-3}
& \textit{nizîk kewtey}& `to get close to’ (lit. `close to fall’) \\\cline{2-3}
\\
\multirow{3}{*}{\textit{warđey} `eat’}&\textit{qesem warđey} &`to swear an oath’ (lit. `oath to eat’) \\\cline{2-3}
&\textit{gîr warđey} &`to get stuck’ (lit. obstacle to eat’) \\\cline{2-3}
&\textit{derđ warđey }&`to be of use’ (lit. `pain to eat’) \\\cline{2-3}
\\
\multirow{2}{*}{\textit{day} `give’} & \textit{tefre day} &`to avoid’ (lit. `avoidance to give’) \\\cline{2-3}
&\textit{biř day} &`to cover a distance’ (lit. `piece to give’) \\\cline{2-3}
\\
\multirow{2}{*}{\textit{amay} `come’}  & \textit{ber amay} &`to rise’ (lit. `out to come’) \\\cline{2-3}
&\textit{tûş amay} &`to run into, to get into trouble’ \\
& & (lit. `accident to come’) \\\cline{2-3}
\\
\multirow{2}{*}{\textit{biřyey} `cut’}  & \textit{mare biřyey} &`to marry’ (lit. `marriage portion to cut’) \\\cline{2-3}
& \textit{sere biřyey} &`to behead’ (lit. `head to cut’) \\\cline{2-3}
\\
\textit{gêrtey} `grab’& \textit{desû dîm gêrtey} &`to perform a Muslim prayer ritual’ \\
&& (lit. `hand and face to take’) \\\cline{2-3}
 \\
\textit{nîyay} `put’&\textit{namê nîyay} &`to name’ (lit. `name to put’) \\ 
\lspbottomrule
\end{tabular}}
    \caption{Common light verbs\is{light verb} in light verb constructions\is{light verb constructions}}
    \label{tab:light-V}
\end{table}

Another way to classify Light verb constructions\is{light verb constructions} is according to the type of non-verbal element that occurs with the light verb\is{light verb}. Nouns and adjectives are the most common preverbal elements in LVCs\is{light verb constructions}. However, note that some nominal elements float between nouns and adjectives, which has relevance for the syntax of complex predicates (see \S\ref{sect:lvc-syntax}). Particles are another category occurring with light verbs\is{light verb}; the number of LVCs with a particle as the non-verbal complement is down to a few cases listed in Table \ref{tab:lVC}.

\begin{table}[t]
\begin{tabular}{lll}    
    \lsptoprule
Category& \textsc{lvc}& Gloss \\
\midrule
\multirow{4}{*}{Noun}&\textit{ʕefwe kerđey} &`to pardon’ (lit. `pardon to do’) \\\cline{2-3}
&\textit{wey kerđey} &`to raise’ (lit. `training to do’) \\\cline{2-3}
&\textit{zînî kerđey} &`to saddle’ (lit. `saddle to do’) \\\cline{2-3}
&\textit{nima kerđey}& `to pray' (lit. `pray to do') \\\cline{2-3}
&& \\
\multirow{5}{*}{Adjective}&\textit{adiz kerđey} &`to upset’ (lit. `upset to do’) \\\cline{2-3}
&\textit{keç kerđey} &`to paralyse’ (lit. `crooked to do’) \\\cline{2-3}
&\textit{dagîr kerđey }&`to occupy’ (lit. `occupied to do’) \\\cline{2-3}
&\textit{řas kerđey} &`to carry out’ (lit. `right to do’) \\\cline{2-3}
&\textit{aşkira kerđey} &`to disclose’ (lit. `visible to do’) \\\cline{2-3}
\\
\multirow{6}{*}{Noun/Adj}&\textit{řed bîyey} &`to cross’ (lit. `crossing to do’) \\\cline{2-3}
&\textit{isɫah kerđey} &`to amend’ (lit. `amendment to do’) \\\cline{2-3}
&\textit{haɫî bîyey}& `to understand’ (lit. `understood to be’) \\\cline{2-3}
&\textit{cemʕ kerđey} &`to gather’ (lit. `addition to do’) \\\cline{2-3}
&\textit{swar kerđey} &`to mount’ (lit. `rider to do’) \\\cline{2-3}
&\textit{bergozar kerđey}& `to hold’ (lit. `accomplished to do’) \\\cline{2-3}
 \\
\multirow{4}{*}{Particle}&\textit{ber amay} &`to rise’ (lit. `out to come’) \\\cline{2-3}
&\textit{ber arđey} &`to take out’ (lit. `out to take’) \\\cline{2-3}
&\textit{ber kerđey} &`to expel (lit. `out to do’) \\\cline{2-3}
&\textit{wer day} &`to release’ (lit. `out to give’) \\
\lspbottomrule
\end{tabular}
\caption{The lexical category of non-verbal elements in LVCs\is{light verb constructions}}
    \label{tab:lVC}
\end{table}

LVCs\is{light verb constructions} may also be categorised according to the transitivity of the light verb\is{light verb} element. The light verbs\is{light verb} \textit{kerđey} and \textit{warđey} are transitive, whereas \textit{bîyey}, \textit{kewtey}, and \textit{amay} are intransitive. In general, the transitivity of the LVC can be determined by the light verb\is{light verb}, hence \textit{hiş kerđey} `to inform’ (< \textit{hiş} `intelligence’ + \textit{kerđey} `do’) is transitive, and \textit{hetîm kewtey} `to be left an orphan’ (< \textit{hetîm} `orphan’ + \textit{kewtey} `to fall’) is intransitive. Therefore, the transitivity in LVCs\is{light verb constructions} is determined based on the lexical transitivity of the light verb, not the semantic transitivity. The LVCs\is{light verb constructions} in (\ref{ex.light-v}) are syntactically transitive, even though some may be semantically considered intransitive.
\TabPositions{2.5cm,5.5cm}
\ea   \label{ex.light-v}
\textit{fewt kerđey}\tab `to pass away’\tab  (lit. `death to make')\\
\textit{zîya kerđey}\tab `to increase’\tab (lit. `addition to make') \\
\textit{kem kerđey}\tab `to decrease’\tab (lit. `little to make')  \\
\textit{derđ warđey}\tab `to be of use’\tab (lit. `pain to eat') \\
\textit{gîr warđey}\tab `to get stuck’\tab  (lit. `hook to eat')\\
\textit{tefre day}\tab `to avoid’\tab  (lit. `evasion to give')\\
\z

Light verb constructions\is{light verb constructions} can also be classified according to the position of the complement relative to the light verb\is{light verb}. While in the majority of the cases, the nominal element precedes the light verb\is{light verb}, as seen above, in the following constructions, the nominal complement consistently follows the verb. The light verb\is{light verb} in these LVCs\is{light verb constructions} is often a mobility verb. The nominal element, thus, can be said to be the goal\is{goal} argument of the LV.
\TabPositions{2.5cm,6cm}
\ea
\textit{ginay řa}\tab `set off on the road’ \\
\textit{luway řa}\tab  `walk' \\
\textit{wistey řa}\tab `carry out’  \\
\textit{amay cuwab}\tab `start to speak’  \\
\textit{nîştey leme}\tab `get pregnant’ \\
\textit{astey cîya}\tab `leave behind’ \\
\textit{menay cîya}\tab `be left behind’ \\
\textit{yaway sinʕe}\tab  `reach adulthood' \tab  [BP.123]
\z

\section{Particle verbs\is{particle verbs}}\label{ptcl-V}
In addition to simple verbs and Light verb constructions\is{light verb constructions}, there are a number of verb forms with figurative meanings composed of a frozen preposition and a simple verb. The particle verbs listed in Table \ref{tab:ptclv} are different from the particle-based LVCs in that they contain frozen prepositions in their structure and the fact that the frozen preposition follows the verb. The lexical transitivity of the simple verb determines the transitivity of such verbs. 
\begin{table}

\begin{tabular}{lllll}
\lsptoprule
Adposition& Particle verb& Gloss \\
\midrule
\multirow{3}{*}{\textit{pene} `to'}&\textit{yaway pene} (\textsc{intr})&`to grow up' (lit. `arrive to') \\\cline{2-3}
&\textit{kewtey pene} (\textsc{intr})& `receive(?)' (lit. `fall to') \\\cline{2-3}
&\textit{zanay pene} (\textsc{tr}) &`learn about' (lit. `know to')\\\cline{2-3}
\\
\multirow{3}{*}{\textit{pore} `at'}&\textit{dîyay pore} (\textsc{tr}) & `to look at' \\\cline{2-3}
& \textit{kêşay pore} (\textsc{tr}) & `to hit' (lit. `hit at')\\\cline{2-3}
&\textit{kewtey pore} (\textsc{intr}) & `stumble' (lit. `fall at') \\\cline{2-3}
\\
\multirow{3}{*}{\textit{wene} `at'}&\textit{qomyay wene} (\textsc{intr})& `to happen' (lit. `accident at') \\\cline{2-3}
&\textit{xuřyey wene} (\textsc{tr})& `to shout' (lit. `shout at')\\\cline{2-3}
&\textit{day wene} (\textsc{tr})&`to set off' (lit. `give at')\\\cline{2-3}
 \\
\multirow{2}{*}{\textit{wer} `out'}& \textit{day wer} (\textsc{tr})& `to herd out' (lit. `give out') \\\cline{2-3}
&\textit{yaray wer} (\textsc{intr})& `to cope with' (lit. `dare at') \\\cline{2-3}
 \\
\multirow{2}{*}{\textit{weɫê} `front'}& \textit{wistey weɫê} (\textsc{tr}) & `to drive forth' (lit. `throw front') \\\cline{2-3}
&\textit{kewtey weɫê} (\textsc{intr})&`move in front of' (lit. `fall front')\\\cline{2-3}
\\
\textit{gel} `with'&\textit{kewtey gel} (\textsc{intr})&`accompany' (lit. `fall with') \\\cline{2-3}
\\

\textit{pêwere} `together'& \textit{kewtey pêwere} (\textsc{intr})&`run into each other' (lit. `fall with')\\
\lspbottomrule
\end{tabular}
\caption{The preposition type in particle verb constructions}
    \label{tab:ptclv}
\end{table}


\section{Valency changing morphology} 
 Causative and passive morphology affect the valency of the verbs. They are neither regular nor productive. They share the commonality of taking the present stem as the basis for morphological derivation. 

\subsection{Causative voice}\label{sect:causative}
As discussed, the present stem is the basis for deriving causative stems. The causative\is{causative} voice is expressed primarily via suffixation, in which case the suffixes \textit{-n} (\textsc{prs}) and \textit{-na} (\textsc{pst}) attach to the present stem of some intransitive verbs and derive transitive counterparts (see below for other alternations). Following \citet{haspelmath_more_1993}, this kind of alternation is called ``causative alternation''\is{causative alternation}, meaning that the inchoative is the basic stem and the causative is derived from it. Examples (\ref{ex.top}-(\ref{ex.topn}) exhibit the causative alternation\is{causative alternation} for the verb \textit{topay} `die'; \textit{topnay} `kill':
\newpage
\TabPositions{1.5cm,2.70cm,4.20cm,6.20cm,7.40cm,8.70cm,9.90cm}
\ea \label{ex:causinch}
\textsc{inf}\tab  Intransitive \tab Gloss \tab Causative\is{causative} \tab Gloss \\
\tab \textsc{prs}\tab \textsc{pst} \tab \tab \textsc{prs}\tab \textsc{pst} \tab\\
\textit{sotey}\tab \textit{soç}-\tab \textit{sot-}\tab `to burn’\tab \textit{soçn-}\tab \textit{soçna-} \tab `to burn’\\
\textit{yaway}\tab \textit{yaw}-\tab \textit{yawa}-\tab `to arrive’\tab \textit{yawn}-\tab \textit{yawna}- \tab  `cause to arrive’ \\
\textit{êşay}\tab \textit{êş}-\tab \textit{êşa}-\tab `to hurt’\tab \textit{êşn-}\tab \textit{êşna}- \tab `to hurt’ \\
\textit{topay}\tab \textit{top}-\tab \textit{topa}-\tab `die’\footnotemark \tab \textit{topn}-\tab \textit{topna}- \tab `kill’ \\  
\textit{cimay}\tab \textit{cim}-\tab \textit{cima}-\tab `move’ \tab \textit{cimn}-\tab \textit{cimna}- \tab `move’ \\ 
\z\footnotetext{Hewramî\il{Hewramî} makes a distinction between the verb used when a human dies and when an animal dies. For the former, \textit{merđey} is used, while for the latter \textit{topay} is used. Likewise, for the causative counterpart, the verbs used are different: \textit{topnay} `kill an animal' vs. \textit{kuştey} `kill a human'. It should be noted that in some contexts, e.g., when dehumanising a person, it is possible to use the verb used for animals to refer to when a person dies, or is killed.}

\ea \label{ex.top}
\textit{dêwe çane topo miro.} \\ 
\gll dêw-e ç=a=ne \textbf{top-o} mir-o \\ 
 ogre\textsc{-def} in=\textsc{dem.dist=post} die\textsc{.prs.ind-3sg:S} die\textsc{.prs.ind-3sg:S} \\ 
\glt `The ogre died there.' \hfill[JP.202]
\z 


\ea \label{ex.topn}
\textit{topneş!} \\
\gll \textbf{topn}-e=ş \\
kill.\textsc{prs-2sg:imp:A=3sg:O} \\
\glt `Kill it (the snake)!' \hfill[MR.42]
\z 

The causative morphology in general only derives a transitive stem from an intransitive one. However, in one case, the causative\is{causative} affix is added to a transitive stem to yield a change in meaning. From (\ref{ex.bexsh}) it can be assumed that the causative suffix would drive a semantically related meaning when added to a transitive verb.
\TabPositions{1.5cm,4.80cm}
\ea   \label{ex.bexsh}
\tab \textit{bexşay} `to forgive’\tab \textit{bexşnay} `to distribute’ \\
\textsc{prs}\tab \textit{bexş}-\tab \textit{bexşn-} \\
\textsc{pst}\tab \textit{bexşa-}\tab \textit{bexşna-} \\
\z

In a few verb pairs, Tekht H.\il{Hewramî!Tekht} has preserved an older pattern of umlaut for the formation of causative stems\is{causative stem}, attested since the Middle Iranian period \citep[220]{skiaervo_middle_2009}{}, e.g., Middle Persian\il{Middle Persian}: \textit{ahram} `go up’ vs. \textit{ahrām} `lead up’ (\textsc{tr}). This pattern is attested in the verb `to break’, where the intransitive/inchoative\is{inchoative} verb appears with the detransitiviser suffix \textit{-ye} (\textsc{prs}); \textit{-ya} (\textsc{pst}). The causative\is{causative stem} counterpart is not only short of the intransitive suffix but is featured by a change in the vowel of the stem.
\TabPositions{1.5cm,6cm}
\ea
\tab \textit{meřyay} `to break' (\textsc{intr})\tab \textit{mařay} `to break' (\textsc{tr}) \\ \textsc{prs}\tab \textit{meřye-}\tab \textit{mař-} \\
\textsc{pst}\tab \textit{meřya-}\tab \textit{mařa-} \\
\z

A similar pattern is attested for the verb `to pour’. Here, the vowel is not changed, but the stem used in the past causative is different.

\ea
\tab \textit{micyay} `to pour' (\textsc{intr})\tab \textit{mitey} `to pour' (\textsc{tr}) \\ \textsc{prs}\tab \textit{micye-}\tab \textit{mic-} \\
\textsc{pst}\tab \textit{micya-}\tab \textit{mit-} \\
\z

In both cases, it is the intransitive/inchoative stem\is{inchoative stem}, rather than the transitive stem, that seves as the base for the formation of the passive stem\is{passive stem}, . This points to the identical stem morphology of passives\is{passive stem} and inchoatives\is{inchoative stem}. Table \ref{tab:stem.passiveʔinchative} highlights the morphological alignment\is{morphological alignment} of passive\is{passive stem} and inchoative stem\is{inchoative stem} morphology for verbs `break' and `pour'. The past stems have been given for ease of comparison. The identical morphology concerns the use of the detransitivising suffix in the formation of both the inchoative and the passive. 

\begin{table}[htp]
    \begin{tabular}{cccc}
\lsptoprule
Verb &  tr. stem & passive\is{passive stem}/inchoative stem\is{inchoative stem} \\
\midrule
`to break'& \textit{mařa-} & \textit{meř-ya} \\
`to pour'& \textit{mit-} & \textit{mic-ya} \\ 
\lspbottomrule
    \end{tabular}
    \caption{Identical morphology of passive\is{passive stem} and inchoative stems\is{inchoative stem}}
    \label{tab:stem.passiveʔinchative}
\end{table}

The causative/transitivising\is{causative stem} suffix is also used with intransitive verbs expressing sound emission without necessarily increasing the verb's valency. Here, the transitivising suffix expresses the agentivity. When the causative/transitivising\is{causative stem} suffix is present on a verb of sound emission, the verb is treated as a transitive verb and is indexed via a clitic pronoun in the past tense. Without a causative\is{causative stem} suffix, the verb is treated as an intransitive verb and is indexed via verbal person suffixes. Consider the difference between (\ref{ex.heshe}) and (\ref{ex.tute}).

\ea \label{ex.heshe}
\textit{heşekê boɫnaş.}\\
\gll heşe-(e)kê \textbf{boɫna=ş}\\
bear-\textsc{def.f.sg} growl.\textsc{pst=3sg:A} \\
\glt `The bear growled.'
\z 

\newpage
\ea \label{ex.tute}
\textit{tûteke gefa.}\\
\gll tûte-(e)ke \textbf{gefa-\O}\\
dog-\textsc{def.m.sg.dir} bark.\textsc{pst-3sg:S} \\
\glt `The dog barked.'
\z 

Note that there is variation in the coding of the verbs of sound emission, such that the same verb may be coded as transitive (in which case it appears with the agentive \textit{-n}) or intransitive; see for example the variation for encoding `it (the donkey) brayed' in Table \ref{tab:tr}.

\begin{table}[htp]
% \small
    \begin{tabular}{lll}
\lsptoprule
`bark’&\textit{gefa} (\textsc{intr})&`It (dog) barked.’ \\
\midrule 
`bleat’&\textit{barya-we} (\textsc{intr})&`It (sheep, goat) bleated.’ \\ \midrule 
`moo’&\textit{qořya-we} (\textsc{intr})&`It (cow) mooed.’ \\ \midrule 
`neigh’&\textit{hîlna꞊ş} (\textsc{tr})&`It (horse, mule) neighed.’ \\ \midrule 
`bray’&\textit{seřa} (\textsc{intr}); \textit{seřna꞊}ş (\textsc{tr})&`It (donkey) brayed.’ \\ \midrule 
`howl’& \multirow{2}{*}{\textit{lûrna꞊ş}\footnotemark (\textsc{tr})}& `It (wolf) howled.’ \\ 
`roar’& &`It (leopard) roared.’ \\ \midrule 
`howl' & \textit{qařna=ş} (\textsc{tr}) & `It (jackal) howled.' \\ \midrule
`gecker’ &\multirow{3}{*}{\textit{wêqna꞊ş} (\textsc{tr})} &`It (fox) geckered.’ \\
&&`It (stone marten) geckered’ \\
`grunt’&&`It (squirrel) grunted.’ \\ \midrule
`growl’&\textit{boɫna꞊ş} (\textsc{tr})&`It (bear) growled.’ \\ \midrule
`cluck’&\textit{qirazna꞊ş} (\textsc{tr})&`It (hen) clucked.’ \\ \midrule
`roar’&\textit{neřna꞊ş} (\textsc{tr})&`It (lion) roared.’ \\ \midrule
`caw’& \multirow{2}{*}{\textit{qiřna꞊ş} (\textsc{tr})}& `It (raven) cawed.’ \\
`quack’&&`It (duck) quacked.’\\ \midrule
`grunt’&\textit{mîzna꞊ş} (\textsc{tr})&`It (turtle) grunted.’ \\ \midrule
`chirp’& \multirow{2}{*}{\textit{cirîkna꞊ş} (\textsc{tr})} &`It (sparrow) chirped.’ \\
`squeak’&& `It (mouse) clicked.’ \\ \midrule
`meow’&\textit{mîyawna꞊ş} (\textsc{tr})&`It (cat) meowed.’ \\ \midrule
`buzz’&\textit{wîzna꞊ş }(\textsc{tr})&`It (bee) buzzed.’ \\ \midrule
`squeak’&\textit{qařya꞊we} (\textsc{intr})&`It (rabbit) squeaked.’ \\ \midrule 
`crow’& \multirow{5}{*}{\textit{wena꞊ş} (\textsc{tr})}& `It (rooster, partridge) crowed.’ \\
`coo’& & `It (dove) cooed.’\\
`croak’& & `It (frog) croaked.’ \\
`hiss’& & `It (snake) hissed.’ \\
`hoot’& &`It (owl) hooted.’ \\
\lspbottomrule
 \end{tabular}
    \caption{Transitivity in verbs of sound emission}
       \label{tab:tr}
\end{table}\footnotetext{This verb may also be expressed as a light verb construction: \textit{lûre꞊ş kerđ} `It howled.'}

Finally, several other alternations occur for derivation of causative\is{causative} from the inchoative\is{inchoative} or vice versa. Most of the verb pairs exhibit what is known as `equipollent alternation\is{equipollent alternation}', meaning that both the causative\is{causative} and the inchoative\is{inchoative} are derived from the same stem, but the affixes used are different. Denominal verbs\is{denominal verbs} in \S\ref{sect:denominal-verbs} feature this alternation.
\TabPositions{2.5cm}
\ea\label{ex.equipollent}
\textit{pijg-ye}\tab `scatter’ (\textsc{intr})\\
\textit{pijg-in}\tab `sow, to scatter’ (\textsc{tr}) \\
\tab\\
\textit{çik-ye}\tab `drip’ (\textsc{intr}) \\
\textit{çik-n}\tab `suckle’ (\textsc{tr}) \\
\tab\\
\textit{xinîk-ye}\tab `suffocate’ (\textsc{intr})\\
\textit{xinîk-n}\tab `strangle’ (\textsc{tr})\\
\tab\\
\textit{temam-ye}\tab `finish’ (\textsc{intr}) \\
\textit{temam-n}\tab `finish’ (\textsc{tr}) \\
\tab\\
\textit{nam-ye}\tab  `bend’ (\textsc{intr})\\
\textit{nam-n} \tab  `bend’ (\textsc{tr}) \\
\tab\\
\textit{dêw-ye}\tab  `become angry’ (\textsc{intr})\\
\textit{dêw-n} \tab  `get angry’ (\textsc{tr}) \\
\tab\\
\textit{gîs-ye=(e)ne}\tab  `shine’ (\textsc{intr}) \\
\textit{gîs-n=ene}\tab  `light, start’ (\textsc{tr}) \\
\z

The following examples illustrate the alternation for \textit{temam-ye} vs. \textit{temam-n} `finish'.

\ea
\textit{awê her kunene her netemamyɛne.} \\ 
\gll awê her kune=ne her ne-\textbf{temamyɛ}=ne \\ 
 water\textsc{.f.sg.obl} \textsc{emph} clay\_pot\textsc{=post} \textsc{emph} \textsc{neg-}finish\textsc{.pst.ptcp.f=cop.3sg.f:S} \\ 
\glt `The water [was] in the clay pots. It would not finish [soon].' \hfill[JE.21]
\z


\ea
\textit{feqîş temamnan.} \\ 
\gll feqî=ş \textbf{temamna}=n \\ 
 theologian\textsc{.m=3sg:A} finish\textsc{.pst.ptcp.m=cop.3sg.m:O} \\ 
\glt `He finished [studying] Islamic jurisprudence.' \hfill[ZP.16]
\z 

Suppletive noncausal/causal alternation is attested as well. Here, different verb roots are used:
\TabPositions{2cm}
\ea
\textit{mir-}\tab `die’ (\textsc{intr})\\
\textit{kuş-}\tab `kill’ (\textsc{tr}) \\
\tab\\
\textit{gin-}\tab `fall’ (\textsc{intr}) \\
\textit{wiz-}\tab `drop’ (\textsc{tr}) \\
% \textit{wir-}&`itch (\textsc{intr})’ \\
% \textit{kiřn-}&`scratch (\textsc{tr})’ \\
\z

Some verb pairs exhibit the `anticausative alternation\is{anticausative alternation}', meaning that the causative verb is basic and the inchoative\is{inchoative} verb is derived from it:

\ea
\textit{fař}\tab `change’ (\textsc{tr})\\
\textit{fař}-\textit{ye}\tab `change’ (\textsc{intr}) \\
\tab \\
\textit{diř}\tab `tear’ (\textsc{tr}) \\
\textit{diř-ye}\tab `tear’ (\textsc{intr}) \\
\tab \\
\textit{yaw=we}\tab `spread’ (\textsc{tr}) \\
\textit{yaw-ye=we}\tab `be spread’ (\textsc{intr}) \\
\tab \\
\textit{pêç}\tab `wrap’ (\textsc{tr}) \\
\textit{pêç-ye}\tab `wrap (oneself)’ (\textsc{intr}) \\
\z

The following examples illustrate the alternation between 
\textit{yawye=we} `be spread' and \textit{yaw=we} `spread'.

\newpage
\ea
\textit{zeře çermekeş ana ça yawyano.} \\
\gll zeř-e çerme-(e)ke=ş ana{-\O} ça \textbf{yawya}=n=o \\
coin-\textsc{ez.cmpd} white-\textsc{def.m.sg.dir=3sg:PSR} \textsc{loc.deic.cop-3sg.m:S} there be\_spread.\textsc{pst.ptcp.m=cop.3sg.m:S=compl} \\
\glt `Her white coins were there, spread out over [the ground].' \hfill[PK.27]
\z

\ea
\textit{qeran qeran yawanşo.} \\
\gll qeran qeran \textbf{yawa}=n=ş=o \\
kurus kurus spread\textsc{.pst.ptcp.m=cop.3sg.m:O=3sg:A=compl} \\
\glt `It (the rat) had spread them (the coins) kurus by kurus.' \hfill[PK.30]
\z 

Some verb pairs exhibit `labile alternation\is{labile alternations}', in which the same verb is used both in the inchoative\is{inchoative} and causative\is{causative}:

\ea
\begin{tabular}{ll}
% \textit{xiz-}&`slip (tr., intr.)’ \\
\textit{toq-} & `scare, be scared' (\textsc{tr}, \textsc{intr})\\
\end{tabular}
\z
The following examples exhibit the verb \textit{toqay} (\textsc{prs:} \textit{toq-}; \textsc{pst:} \textit{toqa-}) `be scared, scare' being used intransitively (\ref{toq.intr}) and transitively (\ref{toq.tr}):

\ea
\textit{pîyakeyç lers kero toqo.} \\
\gll pîya-(e)ke=yç lers ker-o \textbf{toq}-o \\
man-\textsc{def.m.sg.dir=add} shake do.\textsc{prs.ind-3sg:A} be\_scared.\textsc{prs.ind-3sg:S} \\
\glt `The man trembled (in fear); he was scared.' \hfill[JF.29] \label{toq.intr}
\z 


\ea
\textit{sîyamarêwî toqanaş.} \\
\gll sîyamar-êw-î \textbf{toqa}=na=ş \\
black\_snake-\textsc{indf-m.sg.obl} scare.\textsc{pst=cop.1sg:O=3sg:A} \\
\glt `A black snake scared me.' \hfill[JL.68] \label{toq.tr}
\z 

Some verbs do not show any alternation. A subset of these verbs exists only with the causative/agenttive suffix. Most verbs of sound emission seen in Table \ref{tab:tr} belong to this category.
\TabPositions{2.5cm,6cm,8cm}
\ea
\textit{wişknay}\tab  `to scour'\tab  \textit{wişkna-}\tab  \textit{wişkn-} \\
\textit{demnay}\tab  `to start (morning)'\tab  \textit{demna-}\tab  \textit{demn-} \\
\textit{pirûnay} \tab  `rub off (eye)'\tab  \textit{pirûna-} \tab  \textit{pirûn-} \\
\textit{řaznay=we}\tab  `adorn’ (\textsc{tr})\tab  \textit{razna=we}\tab  \textit{razn=we} \\
\z

Another subset occurs only in the inchoative\is{inchoative} form:

\ea
\textit{tişyay} \tab  `go sour’ \tab  \textit{tişya-} \tab \textit{tişye-} \\
\textit{genay} \tab  `rot’ \tab  \textit{gena-} \tab  \textit{gen-}\\
\textit{pûyay} \tab  `rot' \tab  \textit{pûya-} \tab  \textit{pûye-} \\
\z

This classification can capture most derivations between the causative\is{causative stem} and inchoative stems\is{inchoative stem}. Yet, the issue remains complicated for the verb `to break’, as seen below:
\TabPositions{2cm}
\ea
\textit{meř-ye}\tab `break’ (\textsc{intr}) \\
\textit{mař-}\tab `break’ (\textsc{tr}) \\
\z

Here, it seems that the causative stem\is{causative stem} is the basic stem, similar to the anticausative alternation\is{anticausative alternation} seen above and that the inchoative stem\is{inchoative stem} was derived from it by stem modification. The affixation on the intransitive verb seems to be secondary.

\subsection{Passive\is{passive}} \label{passive}
The present stem is the basis for the formation of the the passive\is{passive} stem. The passive\is{passive} suffix has the forms \textit{-ye} (\textsc{prs}) and \textit{-ya} (\textsc{pst}).
\TabPositions{3.25cm}
\ea
\textit{wiretey} `sell’  \\
Present stem\tab \textit{wireş-} \\
Present passive\tab \textit{wireşye-} \\
Past passive\tab \textit{wireşya-} \\
\z

However, the passive stem\is{passive stem} comes with a change in the vowel of the present stem for verbs with the CVC pattern, with the coda being the rhotic consonant \textit{r}, see Table \ref{tab:stem.passive}. This pattern is also available for the verb `to give', which has the present stem in \textit{de-}. For this particular verb, the final \textit{r} in the passive\is{passive stem} is non-etymological. The generalisation seems to be that all the verbs featuring this type of allomorphy have CVC stem, where the vowel is /e/ and the coda is /r/ undergo ablaut in the formation of the passive stem. Additionally, \textit{day} `to give' (present stem \textit{de-}) belongs to this class.\footnote{The stem allomorphy\is{stem allomorphy} associated with these passive verbs has a parallel in unaccusative verbs such as \textit{merđey} `to die': \textsc{pst.} \textit{merđ-}; \textsc{prs.} \textit{mir-}. This could suggest that passive\is{passive stem} and unaccusative stems have identical morphology.} 

\begin{table}[htp]
    \centering
    \begin{tabular}{lllll}
\lsptoprule
Verb & Gloss & Past stem & Prs. stem & Pass. stem \\
\midrule
\textit{kerđey}& `to do' & \textit{kerđ-} & \textit{ker-}& \textit{kir-} \\
\textit{berđey}& `to take'& \textit{berđ-} & \textit{ber-}& \textit{bir-} \\
\textit{warđey}& `to eat'& \textit{warđ-} & \textit{wer-}& \textit{wir-} \\
\textit{day} & `to give' &\textit{da-}& \textit{de-}& \textit{dir-} \\ 
\lspbottomrule
    \end{tabular}
    \caption{Morphological allomorphy in passive stems\is{passive stems}}
    \label{tab:stem.passive}
\end{table}

The passive \is{passive} suffix can be said to be a detransitiviser suffix. A set of inchoative stems\is{inchoative stem} are derived from an unmarked transitive verb by adding the detransitiviser \textit{-ye} (\textsc{prs}); \textit{-ya} (\textsc{pst}). By way of example, consider the derivation of the inchoative\is{inchoative} verb \textit{fařyay} `change (\textsc{intr})’:
\TabPositions{1.5cm,6cm}
\ea
\tab \textit{fařay} `to change (\textsc{tr})’\tab \textit{fařyay} `to change (\textsc{intr})’ \\
\textsc{prs}\tab  \textit{fař-}\tab \textit{fařye-} \\
\textsc{pst}\tab \textit{fařa-}\tab \textit{fařya-} \\
\z

Further examples of intransitive unaccusative verbs with passive\is{passive} morphology are listed below (see \S\ref{sect:causative} for the causative formation of these verbs). There is no extra agent in these verbs, and the verbs express eventualities with internal causation.
\TabPositions{2cm,4.5cm,6.5cm}
\ea
Infinitive\tab \tab Present\tab Past \\
\textit{meřyay}\tab `break’\tab \textit{meřye-}\tab \textit{meřya-} \\
\textit{pêwyay}\tab `be visible’\tab \textit{pêwye-}\tab \textit{pêwya-} \\
\textit{şêwyay}\tab `get confused’\tab \textit{şêwye}-\tab \textit{şêwya-} \\
\textit{qomyay}\tab `happen’\tab \textit{qomye}-\tab \textit{qomya-} \\
\z

This suggests that the same stem morphology is used for inchoative\is{inchoative} and passive\is{passive} verbs. This is shown in Table \ref{tab:pass-inch}, presenting the active and passive stems\is{passive} of the transitive verb \textit{viretey} `to sell’ and the present and past stems of the inchoative\is{inchoative} verb \textit{qomyay} `to happen’:
\begin{table}[htp]
   \resizebox{.9\textwidth}{!}{\begin{tabular}{llll}
\lsptoprule
&Active transitive&Passive\is{passive}&Inchoative\is{inchoative} \\
\midrule
Present stem&\textit{vireş-}&\textit{vireş-ye}&\textit{qom-ye} \\
Past stem&\textit{viret-}&\textit{vireş-ya}&\textit{qom-ya} \\
Participle&\textit{virete-} (\textsc{m}); \textit{viretê-} (\textsc{f, pl})&&\textit{qomya-} (\textsc{m}); \textit{qomyê-} (\textsc{f, pl}) \\
Infinitive&\textit{viretey} `sell'&&\textit{qomyay} `happen'\\
\lspbottomrule
    \end{tabular}}
    \caption{Identical morphology of the passive and inchoative}
    \label{tab:pass-inch}
\end{table}

\section{Derivation of new verb meanings}

Another mechanism for deriving new verb meanings is to add preverbs or postverbs to the verb stems and derive new verbs.

\subsection{Preverbal derivation}
The most common preverb is \textit{hur}- with the approximate meaning of `up’. This prefix is cognate with Kurdish\il{Kurdish} \textit{heɫ-}, \textit{hil-}. The combination of the prefix and the base verb can get lexicalised to the extent that the base verb is no longer identifiable as a regular verb in the language. This is the case, for instance, for \textit{hur-êstey} `wake up' in (\ref{ex.deriv}): the base form is not used in the language.
\TabPositions{2.10cm,5.40cm,7.10cm,8.60cm}
\ea \label{ex.deriv}
Infinitive\tab  \tab \textsc{pst}\tab  \textsc{prs} \\
\textit{hur-êstey}\tab `wake up’\tab \textit{hur-êst-}\tab \textit{hur-êz-} \tab \\
\textit{hur-westey} \tab  `climb'\tab  \textit{hur-west-}\tab  \textit{hur-wez-} \\
\textit{hur-piřay}\tab `dance’\tab \textit{hur-piřa-}\tab \textit{hur-piř-} \tab  cf. \textit{piřay} `jump'\\
\textit{hur-wistey}\tab `hang up’\tab \textit{hur-wist-}\tab \textit{hur-wiz-} \tab  cf. \textit{wistey} `drop'\\
\textit{hur-amay}\tab `emerge, come out’\tab \textit{hur-ama-}\tab \textit{hur-ê-} \tab  cf. \textit{amay} `come'\\
\textit{hur-kewtey}\tab  `appear' \tab  \textit{hur-kewt-}\tab  \textit{hur-gin-}\tab  cf. \textit{kewtey} `fall' \\
\z

\textit{hur-} can also be used ambifixally in Tekht Hewram\^i\il{Hewramî!Tekht}, as suggested by the following examples, where it has been used with the intransitive verb \textit{hur-êstey} `rise'.

\ea
\textit{hurmêzo hur çolekêwe gêro.} \\ 
\gll \textbf{hur}-m-êz-o \textbf{hur} çolek(e)-êwe gêr-o \\ 
\textsc{pvb-ind-}rise\textsc{.prs-3sg:S} \textsc{povb} sparrow\textsc{.f-indf} take\textsc{.prs.ind-3sg:A} \\ 
\glt `He (Jamsher Shah) rose [and] grabbed a sparrow.' \hfill[DP.36]
\z 


\ea
\textit{wêş hurmêzo hur yo minvîso kinaçekê.} \\ 
\gll wê=ş \textbf{hur}-m-êz-o \textbf{hur} yo mi-nvîs-o kinaç(ê)-ekê \\ 
\textsc{refl=3sg:PSR} \textsc{pvb-ind-}rise\textsc{.prs-3sg:S} \textsc{povb} one\textsc{.m} \textsc{ind-}write\textsc{.prs-3sg:A} girl\textsc{.f-def.f.sg} \\ 
\glt `The girl rose [and] wrote one [letter] herself.' \hfill[KŞ.79]
\z 

Other preverbs found in the text corpus include the following: \\
\TabPositions{2cm,4cm,6cm}
\ea
\textit{ber- } `out'\\
\textit{berarđey}\tab `bring out’\tab cf. \textit{arđey}\tab `bring’ \\
\textit{berşîyey}\tab `go out’\tab cf. \textit{şîyey}\tab `go’ \\
\z

\TabPositions{2cm,5cm}
\ea
\textit{wer-} `outward'\\
\textit{wer bîyey}\tab  `release’ (\textsc{intr}) \tab  cf. \textit{bîyey} `be'\\
\textit{wer day}\tab  `set free’ (\textsc{tr}) \tab  cf. \textit{day} `give'\\
\z

\subsection{Postverbal derivation}
This section lists the postverbs in Hewramî. The relevant formatives are \textit{=re/=ra} (\S\ref{sect:povbra} and \textit{=we/=o} (\S\ref{sect:povbwe}). Both these formatives appears at the periphery of the verb after all other suffixes and clitic forms, showing thus typical characteristics of clitics \citep[]{bickel_inflectional_2007}{}; see below for examples. 
\subsubsection{Postverb =re/=ra}\label{sect:povbra}
The postverb \textit{=re, =ra} `down, through, away’ derives new verbs from simple stems.\footnote{Additionally, there is a homophonous postposition \textit{=re}, see \S\ref{sect:postpositions}.} The variant \textit{=ra} occurs in the speech of the narrator from Serû Pîrî. The equivalent formative in Central Kurdish\il{Kurdish!Central} is the prefix \textit{da-}. In the list provided in (\ref{ex.postvr}), the base verb and its translation have been given, provided that the base verb occurs independently of the postverb in the language.

\ea \label{ex.postvr}
\TabPositions{2.5cm,6.5cm}
\textit{weznay=re}\tab `drop down’\footnotemark (\textsc{tr})\tab  cf. \textit{weznay} `drop'\\
\textit{merzîyay=re}\tab `settle down’ (\textsc{intr}) \\
\textit{nîştey=re}\tab `sit down’ (\textsc{intr})\\
\textit{piřnay=re}\tab `throw away’ (\textsc{tr})\tab  cf. \textit{piřnay} `throw'\\
\textit{êjyay=re}\tab `lie down, stretch’ (\textsc{intr})\\
\textit{çiɫakyay=re} \tab  `wake with a start' (\textsc{intr}) \\
\textit{şinyay=re} \tab  `fall down' (\textsc{intr})\\
\textit{westey=re}\tab  `get off' (\textsc{intr})\\
\textit{zinay=re}\tab  `take out' (\textsc{tr}) \\
\textit{zîyay=re}\tab  `go out' (\textsc{intr}) \\
\textit{waɫyay=re} \tab  `notice’ (\textsc{intr}) \\
\z\footnotetext{cf. CK.\il{Kurdish!Central} \textit{da-xistin}}

\textit{=re} occurs at the periphery of the verb after person suffixes and person clitics.

\ea
\textit{kesûkarû xanî kîyanaşre isfehan} \\ 
\gll kesûkar=û xan-î kîyan-a=ş=\textbf{re} {} isfehan \\ 
relative\textsc{.m-ez.gen} chief\textsc{.m-sg.obl} send\textsc{.prs.ind-3pl:A=3sg:O=povb} {} \textsc{pn} \\ 
\glt `The king’s relatives sent it to Isfahan [to the king].' \hfill[KŞ.50]
\z


\ea
\textit{nîyɛneşare} \\ 
\gll nîyɛ=ne=şa=\textbf{re} \\ 
put\textsc{.pst.ptcp.f=cop.3sg.f:O=3pl:A=povb} \\
\glt `They unloaded [the water].' \hfill[JE.17]
\z 
\subsubsection{Completive particle =we/=o}\label{sect:povbwe}
The particle \textit{꞊we} (and its variants \textit{꞊ewe}, \textit{꞊o}) is widely used in Tekht H.\il{Hewramî!Tekht}. It is apparently related to Middle Iranian \textit{baz} `again’. The particle does not take stress, e.g., \textit{bówe} `(it) becomes’ (see \S\ref{stress-position} for stress assignment of verbs). 

The particle has been lexicalised as part of the verb stem, resulting in new verb meanings. Apart from \textit{yawaywe} `spread', in the rest of the verbs listed in (\ref{ex.we}), the simple verb from which the derived verb with \textit{we} is formed does not occur independently as a verb stem in the language with an independent meaning. In other words, the base verb and the particle \textit{we} have been lexicalised.
\TabPositions{2.70cm,6cm}
\ea \label{ex.we}
\textit{yaway-we}\tab  `spread' (\textsc{tr}) \tab  cf. \textit{yaway} `arrive' \\
\textit{yostey=we}\tab  `find’ (\textsc{tr}) \\
\textit{řaznay=we}\tab  `adorn’ (\textsc{tr}) \\
\textit{bařyay=we} \tab  `bleat' (\textsc{intr}) \\
\textit{bijyay=we}\tab  `toss' (\textsc{intr}) \\
\textit{miřoşnay=we}\tab  `collect’ (\textsc{tr}) \\
\textit{qařyay=we} \tab  `squeak' (\textsc{intr}) \\
\textit{qořyay=we}\tab  `moo' (\textsc{intr}) \\
\textit{weyay=we}\tab  `wake up' (\textsc{intr})\\
\textit{yaway=we} \tab  `spread' (\textsc{tr})\\
\textit{zîyay=we} \tab  `get out, grow’ (\textsc{intr})\\
\textit{seyay=we}\tab  `rest' (\textsc{intr})\\
\z

The particle adds a range of meanings to the verbs. It generally adds a sense of completeness to the verb's action without any return, thus segmented as `completive’ throughout the book. More specifically, it can carry a meaning of `returning’, `repetition’, and `opening’. Examples (\ref{ex.completive1})--(\ref{ex.completiver}) represent the meaning of \textit{we} as `returning'.

\ea
\textit{luwaymêwe.} \\
\gll luwa-îmê꞊\textbf{we} \\
go.\textsc{pst-1pl:S꞊compl} \\
\glt `We went back.’ \hfill[ZQ.27] \label{ex.completive1}
\z 


\ea
\textit{amaymêwe.} \\ 
\gll ama-îmê=\textbf{we}\\ 
 come\textsc{.pst-1pl:S=compl}\\ 
\glt `We came back.' \label{ex.completiver}\hfill[ZQ.11] 
\z 

The meaning of the particle in the sense of `repetition' is seen in (\ref{ex.completive2})--(\ref{ex.completiverep}):

\ea
\textit{kûkyowe mûso.} \\ 
\gll kûky(e)-o=\textbf{we} m-ûs-o \\ 
 cough\textsc{.prs.ind}\textsc{-3sg:S}\textsc{=compl} \textsc{ind-}sleep\textsc{.prs}\textsc{-3sg:S} \\ 
\glt `He coughed [and then] slept.' \hfill[KŞ.64] \label{ex.completive2}
\z 


\ea
\textit{jenekaşa lemeşa bîyewe.} \\
\gll \textbf{jen(î)-eka}=şa leme=şa bî-e=\textbf{we} \\
wife\textsc{-def.pl.obl=3pl:PSR} belly.\textsc{f.sg.dir=3pl:NC} be.\textsc{pst-3sg.f:S=compl} \\
\glt `Their wives got pregnant [lit. had bellies] again .' \label{ex.completiverep} \hfill [ME.210]
\z 

Finally, (\ref{ex.completive3}) represents the meaning of the particle as `opening'.

\ea
\textit{bereş kerowe pey.} \\ 
\gll bere=ş ker-o=\textbf{we} pey \\ 
door\textsc{.m=3sg:R} do\textsc{.prs.ind-3sg:A=compl} for \\ 
\glt `She opened the door to him.' \hfill[JH.75] \label{ex.completive3}
\z 

More generally, the particle is involved with the completeness of an action: it expresses that an action is completed without any point of return. The action of the verb is dynamic; it has an endpoint.

\ea
\textit{seʕbê hurmêzowe padşa.} \\ 
\gll seʕbe-ê hur-m-êz-o=we padşa \\ 
 morning\textsc{-f.sg.obl} \textsc{pvb-ind-}rise\textsc{.prs-3sg:S=compl} king\textsc{.m} \\  
\glt `The king woke up in the morning.' \hfill[ZP.37] \\
\z 


\ea
\textit{eçê menmêwe.} \\ 
\gll e=çê m\'en-mê=we \\ 
in=here stay\textsc{.prs.sbjv-1pl:S=compl} \\
\glt `We [cannot] stay here.' \hfill[PM.5] \\
\z 


\ea
\textit{mađam weşeş kerdêbowe ...} \\ 
\gll mađam weş-e=ş kerdê=b-o=we \\ 
 as\_long\_as well\textsc{-f=3sg:A} do\textsc{.pst.ptcp.f}=be\textsc{.prs-3sg:O=compl} \\
\glt `Now that it appears that he has healed her [thoroughly] ...' \hfill[JP.261] \\
\z 


\ea
\textit{dûr ginewe çî şarî!} \\ 
\gll dûr g\'in-e=we çî şar-î \\ 
 far fall\textsc{.prs.imp-2sg:S=compl} from\textsc{=dem.prox} city\textsc{.m}\textsc{-obl}\textsc{.m} \\ 
\glt `Get away from this town!' \hfill[BP.165]
\z 

That the particle is associated with completeness is evident in comparing the following two verbs, one without the particle, i.e., \textit{tawyo}, and one with it, i.e., \textit{tawyowe}. The one without the particle implies that the ice is still in the process of melting without necessarily melting completely. The one with \textit{=we} implies that the melting will surely be completed 

\ea
\textit{yexeke tawyo.} \\
\gll yex-eke tawy(e)-o \\
 ice\textsc{-def.m.sg.dir} melt.\textsc{prs-3sg:S} \\
\glt `The ice is melting.'
\z 


\ea
\textit{yexeke tawyowe.} \\
\gll yex-eke tawy(e)-o=we \\
ice\textsc{-def.m.sg.dir} melt.\textsc{prs-3sg:S=compl} \\
\glt `The ice is melting out.'
\z 

The particle appears at the end of the verbal form, after all other suffixes and clitics:

\ea
\textit{kîyanoşewe.} \\ 
\gll kîyan-o=ş=ewe \\ 
 send\textsc{.prs.ind-3sg:A=3sg:O=compl} \\  
\glt `He sent him back.' \hfill[JP.96] 
\z 


\ea
\textit{milarewe.} \\
\gll mi-l-a=re꞊we \\
\textsc{ind-}go.\textsc{prs-3pl:S=povb=compl} \\
\glt `They went back.’ \hfill[BP.124]
\z 

\section{Infinitive}\label{sect:infinitive}
The infinitive\is{infinitive} is formed by adding the suffix \textit{-ey} to the past stem of the verbs. The vowel \textit{e} gets deleted following vowel-final stems:
\TabPositions{2.5cm,4.5cm}
\ea
Gloss \tab  Infinitive \tab  Past stem \\
`read, study'\tab  \textit{wenay} \tab  \textit{wena-} \\
`do' \tab  \textit{kerđey}\tab  \textit{kerđ-} \\
\z

The infinitive\is{infinitive} has both nominal and verbal functions. In its nominal function, the infinitive\is{infinitive} occurs as a preposition complement (\ref{ex.infinitive1}), as a nominal complement (\ref{ex.infinitive2}), and as a nominal argument in existential or copula clauses (\ref{ex.infinitive3}).

\ea
\textit{şûnû peya bîyeyşre eyakêş merđe.} \\ 
\gll şûn-û peya bîyey=ş=re eđa-(e)kê=ş merđ-e \\ 
after\textsc{-ez.gen} visible be\textsc{.inf=3sg:PSR=post} mother\textsc{.f-def.f.sg=3sg:PSR} die\textsc{.pst-3sg.f:S} \\ 
\glt `Well, the child was born. His mother died after his birth.' \hfill[KŞ.23] \label{ex.infinitive1}
\z 


\ea
\textit{xizone çêro ʕebakê be ʕinwanû witeywew seyaywe.} \\ 
\gll xiz-one çêr=o ʕeba-(e)kê be ʕinwan-û witey=we=û seyay=we \\ 
 creep\textsc{.prs.ind-3sg:S} under\textsc{=post} robe\textsc{.f-def.f.sg} by label\textsc{.m-ez.gen} sleep\textsc{.inf=post}=and rest\textsc{.inf=post} \\ 
\glt `He put it over him. He crept under his robe, supposedly to rest and sleep [lit. in the guise of sleeping and resting].' \hfill[BP.188] \label{ex.infinitive2}
\z 


\ea
\textit{ew aman xeyr amayê weş amayê beynne bîyen.} \\ 
\gll ew ama=n xeyr amay-ê weş amay-ê beyn=ne bîye=n \\ 
 \textsc{dem.dist} come\textsc{.pst.ptcp.m=cop.3sg.m:S} goodness\textsc{.m} come\textsc{.inf-indf} good come\textsc{.inf-indf} between\textsc{=post} be\textsc{.pst.ptcp.m=cop.3sg.m:S} \\ 
\glt `They [lit. He] would come. There were greetings and welcoming.' \\ \hfill[RE.25] \label{ex.infinitive3}
\z 

The infinitive\is{infinitive} also occurs as the complement of the verb \textit{des kerđey} `start' [lit. `hand do'].

\newpage
\ea
\textit{serew des kero gireway.} \\ 
\gll serew des ker-o gireway \\ 
 from\_above hand\textsc{.m} do\textsc{.prs.ind-3sg:A} cry\textsc{.inf} \\  
\glt `He (Little Hama) started to cry on the roof.' \hfill[BP.152]
\z


\ea
\textit{şûnîre amana desim kerđen karêz biřyey.} \\ 
\gll şûnî=re ama=na des=im kerđe=n karêz biřyey \\ 
 afterwards\textsc{=post} come\textsc{.pst.ptcp.m=cop.1sg:S} hand\textsc{.m=1sg:A} do\textsc{.pst.ptcp.m=cop.3sg.m:O} subterranean\_canal\textsc{.m} cut\textsc{.inf} \\ 
\glt `Then, I started [lit. hand do] to dig subterranean canals.' \hfill[JM.6]
\z 

In its verbal function, the infinitive\is{infinitive} is used in purposive clauses following a verb of movement. In most Iranian languages, a subjunctive verb form is expected in this context.

\ea
\textit{kîyanaş pey beẍay wenay.} \\ 
\gll kîyan-a=ş pey beẍa-î wenay \\ 
send\textsc{.prs.ind-3pl:A=3sg:O} to \textsc{pn-m.sg.obl} read\textsc{.inf} \\ 
\glt `They sent him to Baghdad so that he studied.' \hfill[JP.80]
\z 


\ea
\textit{berdênmêşa lo keney bê heq.} \\ 
\gll berdê=nmê=şa lo keney bê heq \\ 
 take\textsc{.pst.ptcp.pl=cop.1pl:O=3pl:A} fodder\_grass mow.\textsc{inf} without salary\textsc{.m} \\ 
\glt `They would take us to mow fodder grass for free [i.e., without wages].' \\\hfill[RE.65]
\z 















\end{sloppypar}
