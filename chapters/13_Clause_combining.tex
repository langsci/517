\chapter{Clause combining}
\begin{sloppypar}

\section{Coordination}
The term coordination refers to syntactic constructions in which two or more units of the same type are combined into a larger unit and still have the same semantic relations with other surrounding elements \citep{Haspelmath_2007}. The coordinate constructions may involve different semantic relations between the coordination units, including conjunction, disjunction, adversative\is{adversative} coordination, and causal coordination. The list of coordination particles is as follows:

\begin{table}
    \begin{tabular}{llll}
\lsptoprule
\multicolumn{2}{l}{Simple coordinator} &   \multicolumn{2}{l}{Bisyndetic coordinator} \\
 \midrule
 \multicolumn{4}{l}{Conjunction} \\ \midrule
\textit{=û} & `and' & \textit{hem ... hem}& `both ... and' \\
\textit{we} &`and' & \textit{ne ... ne} & `neither ... nor' \\
\textit{=îç}& `also, even, etc.' \\

\tablevspace
\multicolumn{4}{l}{Disjunction} \\\midrule 
\textit{yam, ya} & `or' & \textit{ya ... ya}& `either ... or' \\ 
&& \textit{ya meger ... ya meger}& `either ... or' \\
& & \textit{çi ... çi} & `whether ... or' \\
&& \textit{çe ... çe}& `either ... or' \\

\tablevspace
\multicolumn{4}{l}{Adversative} \\\midrule 
\textit{welê} & `but' \\
\textit{emaneketê} & `however' \\
\lspbottomrule
    \end{tabular}
    \caption{Coordinators}
    \label{tab:conjunctions}
\end{table}

\subsection{Simple coordinators}

\subsubsection{Conjunction} \label{sect:conjunction}
The conjunctive\is{conjunctive} coordinator \textit{=û} `and' links the coordination units. It is used both for NP conjunction (including also adjective conjunction, see \S\ref{adjective-infl}) and sentential conjunction. In the former use, consider (\ref{ex.conj}), where the component parts in the nominal compound are joined by the coordinating conjunction `and':
\TabPositions{2.5cm,5.5cm}
\ea \label{ex.conj}
\textit{kesûkar}\tab `relative’ \tab cf. \textit{kes} `person’ + \textit{-û} + \textit{kar} `job’ \\
\textit{kemûkořî} \tab `poverty’ \tab cf. \textit{kem} `little’ + \textit{-û} + \textit{kořî} `enclosure’ \\
\z

In its sentential use, the coordinator is phonologically placed after the first coordination unit, see (\ref{a-co b}), hence A-co B, but syntactically it is part of the second coordination unit, hence A co-B (see below). The conjunctive\is{conjunctive} coordinator can appear on an indefinite number of coordination units (\ref{indefinite-coordinator}).

\ea
\textit{nanşa wardenû çayşa warden.} \\ 
\gll nan=şa warde=n=\textbf{û} çay=şa warde=n \\ 
 bread\textsc{.m.dir=3pl:A} eat\textsc{.pst.ptcp.m=cop.3sg.m:O}=and tea\textsc{.m.dir=3pl:A} eat\textsc{.pst.ptcp.m=cop.3sg.m:O} \\ 
\glt `They would eat food and drink tea.' \hfill[JE.48] \label{a-co b}
\z 


\ea
\textit{êsaɫ miɫkekema nemenenû zeraʕetma nîyaw; bencanîma nîyenew; xîyarma nîyaw; genmîma nîyenew; îne nîyaw; ûne nîyaw; ûne nîya.} \\ 
\gll êsaɫ miɫk-eke=ma ne-mene=n=\textbf{û} zeraʕet=ma nîy(e)=a=\textbf{û} bencanî=ma nîye=ne=\textbf{û} xîyar=ma nîy(e)=a=\textbf{û} genmî=ma nîye=ne=\textbf{û} îne nîy(e)=a=\textbf{û} ûne nîy(e)=a=\textbf{û} ûne nîy(e)=a \\ 
this\_year property\textsc{.m-def.m.sg.dir=1pl:PSR} \textsc{neg-}remain\textsc{.pst.ptcp.m=cop.3sg.m:S}=and agriculture\textsc{.m=1pl:NC} \textsc{neg.exist=cop.3sg.m:S}=and tomato\textsc{.f=1pl:NC} \textsc{neg.exist=cop.3sg.f:S}=and cucumber\textsc{.m=1pl:NC} \textsc{neg.exist=cop.3sg.m:S}=and wheat\textsc{.f=1pl:NC} \textsc{neg.exist=cop.3sg.f:S}=and \textsc{dem.prox.m.3sg.dir} \textsc{neg.exist=cop.3sg.m:S}=and \textsc{dem.dist} \textsc{neg.exist=cop.3sg.m:S}=and \textsc{dem.dist} \textsc{neg.exist=cop.3sg.m:S} \\ 
\glt `[Everyone in Sharazour knows that] we have not cultivated much land this year; we don’t have much agriculture; we don’t have cucumbers; we don’t have tomatoes; we don’t have wheat; This [crop] is not available; that [crop] is not available; that [crop] is not available.' \hfill[PM.37] \label{indefinite-coordinator}
\z 

There is evidence that the particle is syntactically part of the second coordinate. Discussing the coordinating conjunction in Vafsi\il{Vafsi}, Gilaki\il{Gilaki}, and Persian\il{Persian}, \citet[280]{Stilo2004} holds that the conjunction \textit{=û} moves with the second coordination unit (or conjunct) under extraposition but still phonologically attaches to the preceding word. The property holds for Hewramî\il{Hewramî} as well. In the following example, the second conjunct \textit{kinaça} is extraposed and \textit{=û} has moved with it, attaching to the copula index as its phonological host.

\ea
\textit{daraw penc kuřanaw yerê kinaça.} \\ 
\gll dara-û penc kuř-a=na=\textbf{û} yerê kinaç(ê)-a \\ 
 owner\textsc{.m-ez.gen} five son\textsc{.m-pl.obl}\textsc{=cop}\textsc{.1sg:S}=and three daughter\textsc{.f-pl.obl} \\ 
\glt `I have five sons and three daughters.' \hfill[JM.16]
\z 

In a series of coordinated clauses, the particle may express a sense of open-endedness by appearing after the last clause.

\ea
\textit{qaçim meřyanû kewtena yanew.} \\ 
\gll qaç=im meřya=n=û kewte=na yane=\textbf{û} \\ 
 leg\textsc{.m=1sg:PSR} break\textsc{.pst.ptcp.m=cop.3sg.m:S}=and fall\textsc{.pst.ptcp.m=cop.1sg:S} house\textsc{.m}=and \\ 
\glt `My leg is broken. I am confined to home, and so on. [lit. I have fallen at home.]' \hfill[JM.9]
\z 

Alternatively, open-endedness can be expressed by the use of the \textsc{3pl} proximal demonstrative pronoun \textit{înîşa}:

\ea
\textit{kinaçêw luhonî bo teřû çermew înîşa.} \\ 
\gll kinaçê-û luhon-î b-o teř=û çerme=û \textbf{înîşa} \\ 
girl\textsc{.f.sg-ez.gen} \textsc{pn-m.sg.obl} be\textsc{.prs.ind-3sg:S} fresh=and white=and \textsc{dem.prox.3pl.obl} \\ 
\glt `She was from Luhon; she was young, white, and so on.' \hfill[JH.65]
\z 


\ea
\textit{jenîyû yaneş nîyarê înîşa.} \\ 
\gll jenî=û yane=ş nîy(e)=a=rê \textbf{înîşa} \\ 
 woman\textsc{.f}=and house\textsc{.m=3sg:PSR} \textsc{neg.exist=cop.3sg.m:S=povb} \textsc{dem.prox.3pl.obl} \\ 
\glt `He has no wife, nor a house, and so on.' \hfill[ZP.87]
\z 

In a series of more than two clauses expressing sequential events, the conjunctive\is{conjunctive} coordinator \textit{=û} generally connects the final two clauses.

\ea
\textit{ama asawekew şeşkî, kabrake. duwê ferdê hardîş dɛ qewû herekeywe peyşû vatiş } \\ 
\gll ama asaw-eke-û şeşk-î kabra-(e)ke duwê ferd(e)-ê hardî=ş dɛ qew-û her-ekey=we pey=ş=\textbf{û} vat=iş\\ 
come\textsc{.pst.3sg:S} mill\textsc{.m-def.m.sg.dir-ez.gen} \textsc{pn-m.sg.obl} man\textsc{-def.m.sg.dir} two sack\textsc{-pl.dir} flour\textsc{.f.pl.dir=3sg:A} give\textsc{.pst.3pl:O} on\textsc{-ez.gen} donkey\textsc{.m-def.m.sg.obl}\textsc{=post} for\textsc{=3sg}=and say\textsc{.pst=3sg:A} \\ 
\glt `The fellow came to the Shashk mill. He put two sacks of flour on the back of the donkey and said.' \hfill[HB.60]--[HB.62]
\z 
 
In the more formal register, the Arabic\il{Arabic} loan \textit{we} `and' expresses the coordinate particle. This particle occurs rarely in Tekht Hewramî\il{Hewramî!Tekht}.

\ea
\textit{esɫû terîqetû tesewufî anen ke kabra ʕilmiş hen, ʕemeɫ kero be ʕilmekeyş, we be ixlaso ʕemeɫiş pene kero} \\
\gll esɫ-û terîqet-û tesewuf-î ane=n ke kabra ʕilm=iş hen-∅ ʕemeɫ ker-o be ʕilm-ekey=ş \textbf{we} be ixlas=o ʕemeɫ=iş pene ker-o \\
basis-\textsc{ez.gen} doctrine-\textsc{ez.gen} Sufism-\textsc{obl.m} \textsc{dem.dist=cop.3sg.m:S} \textsc{compl} man knowledge=\textsc{3sg:NC} \textsc{exist-3sg.m:S} act do.\textsc{prs.ind-3sg:S} to
knowledge-\textsc{def.obl.m=3sg:PSR} and with virtuosity=\textsc{post} act=\textsc{3sg:R} to do.\textsc{prs.ind-3sg:A} \\
\glt `The basic principle of Sufism is that man has knowledge (about his faith), he fulfils that knowledge, and he fulfils it by virtuosity.’ \\ \hfill\citep[438]{khan_language_2023}{}
\z

The expression of clausal conjunctive\is{conjunctive} coordination is not always handled through the conjunction \textit{=û}. One strategy to link main clauses together is asyndetic coordination through simple juxtaposition. In other words, no coordinator may be used to link the coordination units. This asyndetic strategy may be used for expressing sequential events on the one hand and for events which express temporally overlapping actions or situations on the other. The following examples illustrate sequential events.

\ea
\textit{min bilû yanew kê harđî barû!} \\ 
\gll min bi-l-û yane-û kê harđî b-ar-û \\ 
\textsc{1sg} \textsc{sbjv-}go\textsc{.prs-1sg:S} house-\textsc{ez.gen} who flour\textsc{.f.sg.dir} \textsc{sbjv-}bring\textsc{.prs-1sg:A} \\ 
\glt `Whose house should I go \textbf{[and]} bring flour!' \hfill[HB.20]
\z 


\ea
\textit{eger kinaçekêşa donɛ watɛş, `erê, kerû .’} \\ 
\gll eger kinaç(ê)-ekê=şa don-ɛ wat-ɛ=ş erê ker-û \\ 
 if girl\textsc{.f-def.f.sg=3pl:A} talk\_to\textsc{.pst.cond.aug.3sg:R} say\textsc{.pst.cond.aug=3sg:A} yes do\textsc{.prs.ind-1sg:A} \\ 
\glt `If they had talked to the girl \textbf{[and]} she had said, ‘Yes, I do.’' \hfill[JE.77]
\z 

In the following examples, the asyndetic coordination construction expresses temporally overlapping actions.

\ea
\textit{kûç kera milawe pey germîyanî.} \\ 
\gll kûç ker-a mi-l-a=we pey germîyan-î \\ 
 migration\textsc{.m} do\textsc{.prs.ind-3pl:S} \textsc{ind-}go\textsc{.prs-3pl:S=compl} for \textsc{pn-m.sg.obl} \\ 
\glt `They (i.e., the family) migrated \textbf{[and]} went to Garmiyan.' \hfill[ZB.39]

\z

\ea
\textit{luwanê nizîkû qiřoɫekey bîyawe.} \\ 
\gll luwa-(a)nê nizîk-û qiřoɫ-ekey bî-a=we \\ 
 go\textsc{.pst-1sg:S} close\textsc{-ez.gen} hollow\textsc{.m-def.m.sg.obl} be\textsc{.pst-1sg:S=compl} \\ 
\glt `I went \textbf{[and]} got closer to the tree hollow.' \hfill[ZQ.36]
\z 

\subsubsection{Disjunction}
Disjunction is expressed by the disjunctive\is{disjunctive} coordinator \textit{yam/ya} `or'. The particle appears at the beginning of the second coordinate unit (\ref{or1}). It can appear on more than two conjuncts.

\newpage
\ea
\textit{to qibûɫ kerî ya mekerî?} \\ 
\gll to qibûɫ ker-î ya me-ker-î \\ 
 \textsc{2sg} accepted do\textsc{.prs.ind-2sg:A} or \textsc{neg.ind-}do\textsc{.prs-2sg:A} \\ 
\glt `Will you consent to it or not?' \hfill[RE.50] \label{or1}
\z 


\ea
\textit{pûɫet gereka ya kinaçê.} \\
\gll pûɫe=t gerek=a ya kinaçê \\
money=\textsc{2sg:NC} be\_necessary\textsc{.m=3sg:S} or girl.\textsc{f} \\
\glt `Do you want money or the girl?' \hfill[KT.157]
\z 


\ea
\textit{ya tatema luwan ya mamoma ya kesûkarma.} \\ 
\gll \textbf{ya} tate=ma luwa=n \textbf{ya} mamo=ma \textbf{ya} kesûkar=ma \\ 
either father\textsc{.m=1pl:PSR} go\textsc{.pst.ptcp.m=cop.3sg.m:S} or paternal\_uncle\textsc{=1pl:PSR} or relative\textsc{.m=1pl:PSR} \\ 
\glt `Either our father would go [to the girl’s father], or our uncle, or a relative of ours.' \hfill[RE.4]
\z 

The disjunctive\is{disjunctive} particle can be deleted at the NP level. In (\ref{ex.ya}), \textit{ya} is deleted in the NP \textit{ħewt heşt} `seven (or) eight'.

\ea \label{ex.ya}
\textit{wextê bo be ħewt heşt saɫe xeɫk bero.} \\ 
\gll wext-ê b-o be ħewt heşt saɫe xeɫk ber-o \\ 
 time\textsc{.m-indf} be\textsc{.prs.ind-3sg:S} \textsc{adp} seven eight year\textsc{.f} people\textsc{.m} take\textsc{.prs.ind-3sg:A} \\ 
\glt `When he turned seven [or] eight years old, people would take him [into their houses].' \hfill[KŞ.36]
\z

\subsubsection{Adversative coordination}
Adversative\is{adversative} coordination is expressed by the particle \textit{welê/welî}, which is borrowed from Arabic\il{Arabic}.

\ea
\textit{bereşo memliketû wêta, welê serma deydê.} \\ 
\gll ber-e=ş=o memliket-û wê=ta welê ser=ma d\'e-îdê \\ 
 take\textsc{.prs.imp-2sg:A=3sg:O=compl} country\textsc{.m-ez.gen} \textsc{refl=2pl:PSR} but head\textsc{.m=1pl:R} give\textsc{.prs.imp-2pl:A} \\ 
\glt `Take him to your region, but pay us a visit.' \hfill[DG.68]
\z 

The particle may combine with \textit{emaneketê} `however, but' to give a sense of `however'.

\ea
\textit{welî emaneketê xas xeɫk bawiřiş pene mekero.} \\ 
\gll welî emaneketê xas xeɫk bawiř=iş pene me-ker-o \\ 
 but but well people\textsc{.m} belief\textsc{=3sg:R} in \textsc{neg.ind-}do\textsc{.prs-3sg:A} \\ 
\glt `But people still did not believe in him that much [as a leader].' \hfill[JP.144]
\z 

\subsection{Bisyndetic coordination}
\largerpage
Bisyndetic coordinators\is{bisyndetic coordinators}, alternatively referred to as ``emphatic coordination" \citep{Haspelmath_2007}, are constructions where both coordination units feature a coordinator. The construction \textit{hem ... hem} expresses emphatic coordination of two clauses, which are set in parallel. This construction is also available in NP coordination, which features the use of the coordinate particle \textit{=û} after the first coordinated noun, see (\ref{ex.coordinate}).

\ea
\textit{bê hem mêman bo nîmeřo hem ba qisê kermê.} \\ 
\gll b-ê \textbf{hem} mêman b-o nîmeřo \textbf{hem} ba qis(ê)-ê k\'er-mê \\ 
\textsc{sbjv-}come\textsc{.prs.3sg:S} both guest\textsc{.m} be\textsc{.prs.sbjv-3sg:S} noon both \textsc{hort} talk\textsc{.f-pl.dir} do\textsc{.prs.sbjv-1pl:A} \\ 
\glt `May he come and be our guest for lunch, and we will discuss the matter.' \\\hfill[DP.44]
\z 


\ea
\textit{hem şerêʕetû hem terêqet to amadenî tekmêɫnî.} \\ 
\gll \textbf{hem} şerêʕet=\textbf{û} \textbf{hem} terêqet to amade=nî tekmêɫ=nî \\ 
 both sharia\_law=and and denomination \textsc{2sg} ready\textsc{=cop.2sg:S} complete\textsc{=cop.2sg:S} \\ 
\glt `You are ready both [to practise] Sharia law and [to lead] an order (of dervishes); you are complete.' \hfill[JP.94] \label{ex.coordinate}
\z 

The disjunctive\is{disjunctive} coordinator \textit{ya} has the bisyndetic\is{bisyndetic coordinators} form \textit{ya ... ya}. It can be combined with \textit{meger} to give the alternatives a sense of added emphasis.

\ea
\textit{řûweş mebo waço, `ya metawû ya tawû.'} \\ 
\gll řûe=ş me-b-o w\'aç-o \textbf{ya} me-taw-û \textbf{ya} taw-û \\ 
face\textsc{=3sg:NC} \textsc{neg.ind-}be\textsc{.prs-3sg:S} say\textsc{.prs.sbjv-3sg:A} either \textsc{neg.ind-}can\textsc{.prs-1sg:A} or can\textsc{.prs.ind-1sg:A} \\ 
\glt `He was ashamed to say, ‘I cannot, or I can.’' \hfill[JP.28] \label{or2}
\z 


\ea
\textit{ya meger î kinaçɫê ba ya meger min!} \\
\gll \textbf{ya} \textbf{meger} î kinaçɫê b-a \textbf{ya} \textbf{meger} min \\
or \textsc{emph} \textsc{dem.prox} girl.\textsc{dim.pl} be.\textsc{prs-3pl:S} or \textsc{emph} \textsc{1sg} \\
\glt `Either these girls stay (at the house) or me!' \hfill[ÇK.44]
\z 

The disjunctive\is{disjunctive} particles \textit{çi ... çi} and \textit{çe ... çe} `either ... or' are attested in the text corpus only to function in NP coordination. 

\ea
\textit{řoɫe çi řaseş çi pakeş tatet ehmeđe dize bê.} \\
\gll řoɫe \textbf{çi} řase=ş \textbf{çi} pake=ş tate=t ehmeđ-e diz-e b-ê \\
child\textsc{.voc} either truth=\textsc{3sg:PSR} or clarity\textsc{=3sg:PSR} father=\textsc{2sg:PSR} \textsc{pn-ez.cmpd} thief\textsc{-def} be.\textsc{prs-aug.3sg:S} \\
\glt `Child, to tell you the truth [lit. be it truth or reality], your father was Ahmad, the thief.' \hfill[ED.55]
\z 


\ea
\textit{çe çermeş, çe sîyawiş, çe sûriş her yon.} \\ 
\gll çe çerme=ş \textbf{çe} sîyaw=iş \textbf{çe} sûr=iş her yo=n \\ 
 either white\textsc{=3sg:PSR} or black\textsc{=3sg:PSR} or red\textsc{=3sg:PSR} just one\textsc{.m=cop.3sg.m:S} \\ 
\glt `Be they white, black, or red; they are all the same.' \hfill[JH.102]
\z 

Emphatic negative coordination is expressed by \textit{ne ... ne} `neither ... nor', which, in principle, can be extended over more than two coordination units.

\ea
\textit{ne nan hen, ne çay hen, ne çîw hen, ħîç.} \\ 
\gll \textbf{ne} nan hen-∅ \textbf{ne} çay hen-∅ \textbf{ne} çîw hen-∅ ħîç \\ 
neither bread\textsc{.m} \textsc{exist-3sg.m:S} nor tea\textsc{.m} \textsc{exist-3sg.m:S} nor thing\textsc{.m} \textsc{exist-3sg.m:S} nothing \\ 
\glt `There was neither bread, nor water, nor anything else, [there was] nothing [left].' \hfill[ZB.29]
\z 


\section{Additive\is{additive} particle =îç}\label{sect:additiveclitic}
The additive\is{additive} particle \textit{=îç} is realised as \textit{=yç} and less frequently \textit{=ç} following vowel-final hosts. The particle has a general meaning of an additive focus\is{additive focus}. Its various functions can be divided into those where the focus\is{focus} of the particle extends over a constituent as opposed to those in which the particle has scope over the clause. The following presentation is inspired by \citet[]{khan_language_2023}{}.\footnote{In the presentation of the additive\is{additive} particle functions, I use some of the functions listed in \citegen{Forker2016} cross-linguistic study of additives. The additive\is{additive} particle has a similar set of functions in the neighbouring Kurdish\il{Kurdish} and Neo-Aramaic\il{Neo-Aramaic} dialects \citep[]{Noorlander2022-narrative}{}.}

\subsection{Scope over a constituent}

\subsubsection{Inclusive focus\is{inclusive focus} (`too')}
In this usage, \textit{=îç} has the inclusive focus meaning. The inclusive focus marker adds items with similar properties to the set of items that is inferred from the context.

\ea
\textit{dey bilmê minîç mewt çene.} \\ 
\gll dey bi-l-mê min=îç m-e-û=t çene \\ 
 \textsc{disc.ptcl} \textsc{sbjv-}go\textsc{.prs-1pl:S} \textsc{1sg}\textsc{=add} \textsc{ind-}come\textsc{.prs-1sg:S}\textsc{=2sg:R} with \\ 
\glt `Let’s go, I am coming with you too.' \hfill[JH.11]
\z 


\ea
\textit{milo mêwew maça, `hizmêç bare.’} \\ 
\gll mi-l-o m-ê=we=û m-aç-a hizmê=ç b-ar-e \\ 
\textsc{ind-}go\textsc{.prs-3sg:S} \textsc{ind-}come\textsc{.prs.3sg:S=compl}=and \textsc{ind-}say\textsc{.prs-3pl:A} firewood\textsc{.f.pl.dir=add} \textsc{imp-}bring\textsc{.prs-2sg:A} \\ 
\glt `He went [shepherding and] came back [home]. They said, ‘Bring firewood as well.’' \hfill[JP.24]
\z 

 \ea 
\textit{ađîç melowe.} \\ 
\gll ađ=îç me-l-o=we \\ 
\textsc{3sg.m.dir=add} \textsc{neg.ind-}go\textsc{.prs-3sg:S=compl} \\ 
\glt `He did not go back either.' \hfill[JP.266]
\z 

\subsubsection{Scalar additive focus\is{scalar additive focus} (‘even’)}
\textit{=îç} also allows for scalar additive readings. Scalar additive focus indicates that the focused element is more informative (less expected, more extreme) than (some of) its alternatives.

\ea
\textit{çêweɫ nebîyey bê êtir îseyç xû nîya.} \\ 
\gll çêweɫ nebîyey b-ê êtir îse=îç xû nîy(e)=a \\ 
 previously poverty\textsc{.m} be\textsc{.prs-aug.3sg:S} \textsc{disc.ptcl} now\textsc{=add} \textsc{disc.ptcl} \textsc{neg.exist=cop.3sg.m:S} \\ 
\glt `In the past, there was poverty. But even nowadays, there is poverty.' \\ \hfill[JM.56]
\z 

In addition to \textit{=îç}, there is an inherently scalar-additive marker \textit{heta} `even' in Hewramî. Example (\ref{ex.heta1}) was produced by a young speaker from Hewraman, who seems to copy the Persian pattern of using \textit{heta} with the additive particle in the same clause. Additionally, \textit{heta} means `until, as long as' in Hewramî. It might have a scalar meaning `even' in the speech of the old generation, but it is not compatible with \textit{=îç} (see \ref{ex.heta2}).


\ea \label{ex.heta1}
\textit{heta karê xirabêçiş kerdênê.} \\
\gll heta kar-ê xirab-ê=ç=iş kerdê=nê \\
even job-\textsc{pl.dir} bad-\textsc{pl.dir=add=3sg:A} do.\textsc{pst-ptcp.pl=cop.3pl:O} \\
\glt `She would even do bad things.’ \hfill\citep[439]{khan_language_2023}{}
\z 


\ea \label{ex.heta2}
\textit{heta min jinîyenim.} \\ 
\gll heta min jinîye=n=im\\ 
 even \textsc{1sg} hear\textsc{.pst.ptcp.m=cop.3sg.m:O=1sg:A} \\ 
\glt `I have even heard [that a man came to the service of Sheikh Aladin].' \\\hfill[ZQ.2]
\z

\subsubsection{Establishing a new topic}
Another context for the use of the additive\is{additive} particle with scope over a constituent is when it signals a change in the topic. 


\ea
\textit{hereke mêwe cûwab î qisa kero. ewîç goş miđo pene.} \\ 
\gll her-eke m-ê=we cûwab î qis(ê)-a ker-o \textbf{ew=îç} goş mi-đ(e)-o pene \\ 
 donkey\textsc{.m-def.m.sg.dir} \textsc{ind-}come\textsc{.prs.3sg:S=compl} response\textsc{.m} \textsc{dem.prox} talk\textsc{.f-pl.obl} do\textsc{.prs.ind-3sg:A} \textsc{3sg.m.dir}\textsc{=add} ear\textsc{.m} \textsc{ind-}give\textsc{.prs-3sg:A} to \\ 
\glt `The donkey started to speak; it spoke. \textbf{He} listened to it [i.e., the donkey].' \\\hfill[HB.44]--[HB.45] 
\z 


\ea
\textit{xeberşa da patşay. patşayç xuɫke kerđ.} \\
\gll xeber=şa da-{\O} patşa-î \textbf{patşa-î=ç} xuɫke kerđ--{\O} \\
news\textsc{=3pl:A} give.\textsc{pst-3sg.m:S} king\textsc{-m.sg.obl} king\textsc{-m.sg.obl=add} invitation do.\textsc{pst-3sg.m:O} \\
\glt `They passed the news to the king. The king invited him.' \\ \hfill[DB.310]--[DB.311]
\z 

\subsection{Scope over proposition}

\subsubsection{Thetic clauses\is{thetic clauses}}
Thetic clauses\is{thetic clauses} comment on a situation as a whole rather than stating something about the subject and breaking the clause into subject and predicate \citep[]{sasse1987thetic}{}. In such constructions, the additive\is{additive} particle has scope over the clause as a whole. The thetic clauses\is{thetic clauses} in (\ref{ex.thetic1})--(\ref{ex.thetic2}) give evaluative background to the surrounding discourse.

\ea
\textit{luwaymê kirmaşan. wuɫahî tena dukanê baz nebî. êmeyç pey kelûpelî luwabênmê.} \\
\gll luwa-îmê kirmaşan wuɫahî tena dukan-ê baz ne-bî \textbf{ême=yç} pey kelûpel-î luwa=b-ên-mê\\
go.\textsc{pst-1pl:S} \textsc{pn} by\_God only store-\textsc{indf} open \textsc{neg}-be.\textsc{pst.3sg:S} \textsc{1pl=add} for goods-\textsc{obl.m} go.\textsc{pst.ptcp=}be.\textsc{prs-aug-1pl:S} \\
\glt ‘We went to Kermanshah (K.\il{Kurdish} Kirmaşan): indeed, there was not even one shop open. We had gone there to buy goods.’ \\ \hfill\citep[443]{khan_language_2023}{}\label{ex.thetic1}
\z 


\ea
\textit{êtir ewîç wêş padşa bo.} \\ 
\gll êtir ew=îç wê=ş padşa b-o \\ 
 \textsc{disc.ptcl} \textsc{3sg.m.dir=add} \textsc{refl=3sg:PSR} king\textsc{.m} be\textsc{.prs.ind-3sg:S} \\ 
\glt `Well, he was a king.' \hfill[ZP.35] \label{ex.thetic2}
\z 

\subsubsection{Concessive conditionals\is{concessive conditionals} (`even if')}\label{sect:concessive}
This function of the additive\is{additive} particle is similar to the scalar additive\is{additive} function, with the difference that the additive\is{additive} particle has scope over the proposition in a concessive conditional clause. 

\ea
\textit{îse hetîmbarîç b-o her dewɫet ûsano řa peyş.}\\
\gll îse hetîmbar=îç b-o her dewɫet ûsan-o řa pey=ş \\
now one\_who\_has\_orphans=\textsc{add} \textsc{sbjv}-be.\textsc{prs-3sg:S} \textsc{emph} government grab.\textsc{prs-3sg:A} road for=\textsc{3sg:R} \\
\glt `Now, even if one was a person caring for orphans, the government will
help him.’ \hfill\citep[444]{khan_language_2023}{}
\z 

\subsection{Constituent coordination}
The additive\is{additive} particle can function as a coordinator. In (\ref{ex.iccoord}), featuring NP coordination, \textit{=îç} appears on each conjunct. The first conjunct is followed by the coordinate \textit{=û}. Example (\ref{ex.icclcor}) features ellipsis in clausal coordination.

\ea
\textit{girđû şarezûrî êsaɫ meraʕetû ême kero ce parî zîyater pey toyçû pey êmeyç.} \\ 
\gll girđ-û şarezûr-î êsaɫ meraʕet-û ême ker-o ce par-î zîya-ter pey to=\textbf{îç}=û pey ême=\textbf{îç} \\ 
 all\textsc{-ez.gen} \textsc{pn-m.sg.obl} this\_year care\textsc{.m-ez.gen} \textsc{1pl} do\textsc{.prs.ind-3sg:A} from last\_year\textsc{-m.sg.obl} increased\textsc{-cmpr} for \textsc{2sg=add}=and for \textsc{1pl=add} \\ 
\glt `This year, all [the people in] Sharazour care for us -- both you and I [lit. We] --, more than the previous year.' \hfill[PM.38] \label{ex.iccoord}
\z 


\ea
\textit{îđîç hîç minîç hîçim nîyarê.} \\ 
\gll îđ=\textbf{îç} hîç min=\textbf{îç} hîç=im nîy(e)=a=rê \\ 
 \textsc{3sg.prox.dir.m=add} nothing \textsc{1sg=add} nothing\textsc{=1sg:NC} \textsc{neg.exist=cop.3sg.m:S=povb}\\ 
\glt `She [has] nothing, nor do I.' \hfill[ZP.117] \label{ex.icclcor}
\z

\section{Incremental repetition}
Another strategy for clause linkage in discourse is through the repetition of a clause. In narratives, sequences of clauses may feature a clause being repeated before the next clause starts. The repeated clause has the effect of setting grounds for new information and event cohesion by establishing a bridging linkage between core events. Additionally, repetition can be used to recapitulate, keep track of major themes in the story, and give the speaker time to think \citet{Noorlander2022-narrative}. The repetition may be partial or total. Example (\ref{ex.repetition1}) features a partial replication of the clause. The repeated clause has the overt object NP as an afterthought.

\ea
\textit{a wextîyekey kinaçekê beroş řare. maço, `be xwa îse kuçêş çene werû. kuçêş çene werû î masî.' bero miđoş pene. } \\ 
\gll a wext-î-ekey kinaç(ê)-ekê ber-o=ş řa=re m-aç-o \textbf{be} \textbf{xwa} \textbf{îse} \textbf{kuç-ê=ş} \textbf{çene} \textbf{wer-û} \textbf{kuç-ê=ş} \textbf{çene} \textbf{wer-û} \textbf{î} \textbf{mas-î} ber-o mi-đ(e)-o=ş pene\\ 
 \textsc{dem.dist} time\textsc{.m-sg.obl-def.m.sg.obl} girl\textsc{.f-def.f.sg} take\textsc{.prs.ind-3sg:A=3sg:O} road\textsc{.f=post} \textsc{ind-}say\textsc{.prs-3sg:A} by God\textsc{.m} now little\textsc{-indf}\textsc{=3sg:R} from eat\textsc{.prs.ind-1sg:A} little\textsc{-indf}\textsc{=3sg:R} from eat\textsc{.prs.ind-1sg:A} \textsc{dem.prox} yoghurt\textsc{.m-sg.obl} take\textsc{.prs.ind-3sg:O} \textsc{ind-}give\textsc{.prs-3sg:O=3sg:R} to\\ 
\glt `Then, the girl set off on the way. She said, ‘Well, I shall eat a little of it now. I shall eat a little of it, [of] this yoghurt.’ She took [it and] gave it to him (Hayas).' \hfill[JH.41]--[JH.43] \label{ex.repetition1}
\z 
 
The partial repetition may exclude the nominal subject of the first clause, as illustrated in (\ref{ex.rep}).

\ea \label{ex.rep}

\ea[]{
\textit{pîreke yawa law peɫêw xeley.}\\
\gll pîr-eke yawa la-û peɫ-êw {} xele-î \\
old-\textsc{def.m.dir} arrive.\textsc{pst.3sg.m:S} to\textsc{-ez.gen} \textsc{clf-indf} {} grain\textsc{-m.sg.obl} \\ 
\glt `The old man arrived at a pile of grains.' \hfill[HR.13]
}
\ex[]{
\textit{yawa law peɫêw xeley mîyo ...} \\
\gll yawa la-û peɫ-êw xele-î mi-đy(e)-o \\
arrive.\textsc{pst.3sg.m:S} to\textsc{-ez.gen} \textsc{clf-indf} grain\textsc{-m.sg.obl} \textsc{ind-}look\textsc{.prs-3sg:S}\\ 
\glt `He arrived at a pile of grains [and] realised ...' \hfill[HR.14]
}
\z
\z 

The repetition may involve reversing the order of clausal elements. This is generally used as a stylistic device to draw attention to a particular event in the narrative.

\ea
\textit{xeɫk mê dewreş mido maça, `ane çêşa? çêşa ane?’} \\ 
\gll xeɫk m-ê dewre=ş mi-d(e)-o m-aç-a \textbf{ane} \textbf{çêş=a} \textbf{çêş=a} \textbf{ane} \\ 
 people\textsc{.m} \textsc{ind-}come\textsc{.prs.3sg:S} round\textsc{=3sg:O} \textsc{ind-}give\textsc{.prs-3sg:A} \textsc{ind-}say\textsc{.prs-3pl:A} \textsc{prsv} what\textsc{=cop.3sg.m:S} what\textsc{=cop.3sg.m:S} \textsc{prsv} \\ 
\glt `People came, encircled him, and asked [lit. said.] `What is going on? What is going on?’' \hfill[ZP.113]
\z 


\ea
\textit{înê î qisê çêşene? çêş maçdê? maça çêş?} \\ 
\gll înê î qisê çêşe=ne \textbf{çêş} \textbf{m-aç-dê} \textbf{m-aç-a} \textbf{çêş} \\ 
 \textsc{dem.prox.f.3sg.dir} \textsc{dem.prox} talk\textsc{.f.sg} what\textsc{.f=cop.3sg.f:S} what \textsc{ind-}say\textsc{.prs-2pl:A} \textsc{ind-}say\textsc{.prs-3pl:A} what \\ 
\glt `What is this talk? What are you saying? What are they saying?’' \hfill[JP.223]
\z 


\end{sloppypar}
