\chapter{Phonetics, phonology, and morphophonology} 

\section{Phoneme inventory}

This section presents the inventory and realisation of consonants and vowels in Tekht Hewramî\il{Hewramî!Tekht}. The language has 26 consonants and nine vowels in its phoneme inventory. 

\subsection{Vowel phonemes}

\subsubsection{Description of vowels}\label{DofV}

Tekht Hewramî\il{Hewramî!Tekht} has nine vowel phonemes: four front vowels <î> /i/, <ê> /e/, <ɛ> /ɛ/, <e> /ɛ${\sim}$æ/; three back vowels <û> /u/, <o> /o/, <a> /ɑ/; and two central vowels <i> /ɨ/ and <u> /ʊ/. Of these, /i, e, ɛ, u, ɑ, o/ and /æ/ are realised as long vowels, while /ɨ/ and /ʊ/ are short. The vowel phonemes are distinguished by height and backness, as well as by phonetic realisation (see \sectref{phonetic-realisation}). Vowel length is notphonemically distinctive, e.g., in distinguishing between <ɛ> and <e> (see \sectref{sect:ɛ}).

The circumflex in the grapheme <î> marks the front vowel /i/ in contrast to <i> /ɨ/. The circumflex in <ê> distinguishes the front vowel /e/ from the front vowel [ɛ${\sim}$æ] <e>. For back vowels, the circumflex distinguishes u/ <û> from the short central /ʊ/ <u>. 

 The following examples illustrate the vowels in syllable-medial and syllable-final positions in monosyllabic words. Syllable-initially, vowels are generally preceded by glottal stop [ʔ] because of the restriction against empty onsets.

/i/ <î> is a close front unrounded vowel:
\TabPositions{1.5cm,3cm,4.5cm}
\ea \label{iclsfrt}
\textit{îđ} \tab  [ʔiɹˠ] \tab  `this' \\
\textit{pîr} \tab  [pʰiɾ] \tab  `old' \\
\textit{sî} \tab  [si] \tab  `thirty' \\
\z

/i/ is glided following /o/:
\TabPositions{1.5cm,3cm,4.5cm,6.cm}
\ea
\textit{toyç} \tab  /tʰo.it͡ʃ/ \tab  > \tab  [tojt͡ʃ] \tab  `you too' \\
\z 

/e/ <ê> is a close-mid front unrounded vowel:
\TabPositions{1.5cm,3cm,4.5cm}
\ea
\textit{êş}\tab  [ʔeʃ]\tab  `pain' \\
\textit{şêr}\tab  [ʃeɾ]\tab  `lion' \\
\textit{kê}\tab  [kʰe]\tab  `who' \\
\z

/ɨ/ <i> is a close-central unrounded vowel. This vowel has restricted distribution. Word-initially, it appears to be occurring with disyllabic or trisyllabic words, e.g., \textit{iti} `any more.' Word-finally, it only appears with clitic particles, that is, elements that cannot stand by themselves as a free word, e.g., \textit{gi} `each.'
\ea
\textit{iti}\tab  [ʔɨ.ˈtɨ]\tab  `any more' \\
\textit{çil}\tab  [t͡ʃɨl]\tab  `forty' \\
\textit{gi}\tab  [ɡɨ]\tab  `each' \\
\z

/i/ is used as an epenthetic vowel to break up some consonant clusters syllable-initially, especially in careful speech (see \sectref{anap}).
\TabPositions{1.5cm,3cm,4.5cm,6.cm}
\ea
\textit{bira} \tab  $\sim$ \tab  \textit{bra}\tab  `brother' \\
\z

/i/ may also be used to break up complex consonant clusters in the syllable coda, e.g.,: 
\ea
\textit{gerim} \tab  < \tab \textit{germ} \tab `warm' \\
\z

/ɛ/ <ɛ> is an open-mid front unrounded vowel. /ɛ/ is the result of the contraction of /a/ and /ê/. When realised in open syllables, it is very close in phonetic realisation to a near-open front unrounded vowel /æ/, the difference being that /ɛ/ is more front and has a shorter length than /æ/ (see \sectref{sect:ɛ}).
\ea
\textit{am{ɛ}}\tab  [ˈʔɑm{ɛ}]\tab  `they came' \\
\textit{wat{ɛ}t}\tab  [ˈwɑt{ɛ}t]\tab  `[if] you had said' \\
\z

/æ/ <e> is a near-open front unrounded vowel. Its realisation approaches /ɛ/ in open syllables (see \sectref{sect: e}).
\ea
\textit{esb}\tab  [ʔæsb]\tab  `horse' \\
\textit{ber}\tab  [bæɾ]\tab  `door' \\
\textit{ce}\tab  [d͡ʒæ]\tab  `in, from' \\
\z

/ɑ/ <a> is an open back unrounded vowel: 
\ea
\textit{ađ}\tab  [ʔɑɹˠ]\tab  `he' \\
\textit{par}\tab  [pʰɑɾ]\tab  `last year' \\
\textit{ça}\tab  [t͡ʃɑ]\tab  `there' \\
\z

/o/ <o> is an open-mid back rounded vowel. It does not occur word-initially in monosyllabic words. 
\ea
\textit{oge}\tab  [ʔo.ˈgɛ]\tab  `there'\\
\textit{kor}\tab  [kʰoɾ]\tab  `blind' \\
\textit{to}\tab  [tʰo]\tab  `you (sg.)' \\
\z

/ʊ/ <u> is a near-close near back rounded vowel. It is limited to word-medial position in monosyllabic words in the text corpus.
\ea
\textit{gul}\tab  [ɡʊl]\tab  `bad' \\
\textit{kul}\tab  [kʊl]\tab  `blunt (knife)' \\
\z

/u/ <û> is a close back rounded vowel. Like other back-rounded vowels, /û/ is excluded word-initially, except occasionally in disyllabic words. 
\ea
\textit{ûne} \tab  [ʔu.ˈnæ] \tab  `that' \\
\textit{dûr}\tab  [duɾ]\tab  `far' \\
\textit{ş\^u}\tab  [ʃu]\tab  `husband' \\
\z

\subsubsection{Phonetic realisation of vowels}\label{phonetic-realisation}
\begin{figure}
    \includegraphics[width=.67\textwidth]{figures/vowel_quality.png}
    \caption{Phonetic realisation of vowels}
    \label{fig:vowelquality}
\end{figure}

This section describes the phonetic realisation of individual vowels plotted in vowel charts. The quality of the allophonic realisation of vowels was measured using the acoustic analysis software Praat \citep[][]{boersma2015praat}. The F1 and F2 frequencies for each vowel were plotted on the vowel chart. F1 has an inverse relationship with vowel height: the higher the vowel, the lower the F1. On the other hand, F2 has an inverse relationship with the backness of the vowel, such that back vowels are characterised by lower F2 than front vowels. Diagram \ref{fig:vowelquality} maps the average realisation of vowels onto the cardinal vowel diagram. Square brackets represent cardinal vowels. The realisation of vowels is based on analysing the acoustic properties of vowels averaged out for at least ten words per vowel. A 20-year-old male speaker of Tekht Hewramî\il{Hewramî!Tekht} produced the words. 

\subsubsubsection{/i/}

\begin{figure}[b]
\begin{tabular}{llll}
1. & \textit{met\stackunder[-10pt]{\^{i}}{\'{}}ye} `aunt'& 8. &  \textit{péştî	}`back'\\
2. & \textit{hêz\stackunder[-10pt]{\^{i}}{\'{}}} `yesterday' & 9. & \textit{tiraz\stackunder[-10pt]{\^{u}}{\'{}}yî} `balance'\\
3. & \textit{gic\stackunder[-10pt]{\^{i}}{\'{}}} `shirt' & 10. & \textit{tif\stackunder[-10pt]{\^{i}}{\'{}}} `mulberry'\\
4. & \textit{hévîcî} `carrot' & 11. & \textit{av\stackunder[-10pt]{\^{i}}{\'{}}r} `fire'\\
5. & \textit{tût\stackunder[-10pt]{\^{i}}{\'{}}} `parrot' & 12. & \textit{jîr} `sage'\\
6. & \textit{góşî} `ear' & 13. &  \textit{tar\stackunder[-10pt]{\^{i}}{\'{}}k} `dark'\\
7. & \textit{mékî} `salt'& 14. &  \textit{keɫeş\stackunder[-10pt]{\^{i}}{\'{}}r} `rooster'\\
&  & 15. &  \textit{qey\stackunder[-10pt]{\^{i}}{\'{}}m} `old'\\
\end{tabular}

    \includegraphics[width=.67\textwidth]{figures/ii.png}
    \caption{Phonetic realisation of /i/ <î> }
    \label{fig:ilong}
\end{figure}

As seen in \figref{fig:ilong}, the allophones of the close front unrounded vowel /i/ are generally scattered in the area between cardinal vowels [i] and [e]. The allophones have a tendency to become more back, especially in the environment of an unvoiced obstruent, e.g., \textit{mékî} [ˈmɛ.ki] `salt', \textit{tar\stackunder[-10pt]{\^{i}}{\'{}}k} [tʰɑ.ˈɾik] `dark'.

\subsubsubsection{/e/}

The positional variants of the close-mid front unrounded vowel /e/ show a relatively narrow scatter; see \figref{fig:e mid}. The sample consists of stressed /e/ (1--7) and their non-stressed counterparts (8--10). The allophones of /e/ are primarily positioned close to the cardinal vowel [e] marked by the square. An exception is the unstressed /e/ in \textit{namê} `name', which has a more central realisation, possibly due to the effect of the preceding nasal consonant.

\begin{figure}
\begin{tabular}{llll}
1. & \textit{qij\stackunder[-10pt]{\^{e}}{\'{}}} `hair' & 6. & \textit{h\stackunder[-10pt]{\^{e}}{\'{}}zmê} `firewood (\textsc{pl.dir})' \\
2. & \textit{bir\stackunder[-10pt]{\^{e}}{\'{}}	} `eyebrow'  & 7. & \textit{yag\stackunder[-10pt]{\^{e}}{\'{}}} `place' \\
3. & \textit{kinaç\stackunder[-10pt]{\^{e}}{\'{}}} `daughter, girl' & 8. & \textit{pérê} `the day before yesterday'\\
4. & \textit{sêt\stackunder[-10pt]{\^{e}}{\'{}}	}`husband's sister' & 9. & \textit{námê} `name'\\
5. & \textit{dêy\stackunder[-10pt]{\^{e}}{\'{}}	}`stepmother' & 10. &  \textit{ey\'alê} `mother (vocative)'\\
\end{tabular}

    \includegraphics[width=.67\textwidth]{figures/ee.png}
    \caption{Phonetic realisation of /e/ <ê>}
    \label{fig:e mid}
\end{figure}

\subsubsubsection{/æ/} \label{sect: e}

\figref{fig:a closed syll} shows the allophonic realisation of /æ/ in closed syllables. It can be seen that /æ/ is scattered in an area corresponding to the realisation of the cardinal vowel [ɛ] in most cases, e.g., \textit{tejn\'e} `thirsty'. In addition, some allophonic variants of /æ/ come close to the realisation of [æ], e.g., \textit{p\'el} `leaf'. \figref{fig:a open syll} demonstrates the allophonic variation of /æ/ in open syllables.


\begin{figure}
% \begin{multicols}{2}
\begin{tabular}{llll}
\multicolumn{2}{l}{stressed /æ/ in closed syllables} & \multicolumn{2}{l}{unstressed /æ/ in closed syllables} \\
1. & \textit{kérge} `hen' & 11. & \textit {texté} `wood' \\
2. & \textit{péştî} `back' & 12. & \textit{tersáy} `fear' (\textsc{inf}) \\
3. & \textit{řénge} `colour' & 13. & \textit{heştal\stackunder[-10pt]{\^{u}}{\'{}}î} `plum' \\
4. & \textit{çérme} `white' & 14. & \textit{kerđéy} `do (\textsc{inf})' \\
5. & \textit{kéşkî} `dried whey' & 15. & \textit{berđéy} `take' (\textsc{inf}) \\
6. & \textit{yehér} `liver' & 16. & \textit{merđéy} `die' (\textsc{inf}) \\
7. & \textit{pél} `leaf' & 17. & \textit{pencéw pay} `heel' \\
8. & \textit{çép} `left' & 18. & \textit{jenbirá	} `brother of wife' \\
9. & \textit{hévr} `cloud' & 19. & \textit{tejné	} `thirsty' \\
10. & \textit{héşt} `eight' & 20. & \textit{nemdár} `known' \\
\end{tabular}
    \includegraphics[width=.67\textwidth]{figures/e_close.png}
    \caption{Phonetic realisation of /æ/ <e> in closed syllables}
    \label{fig:a closed syll}
\end{figure}

\begin{figure}
\begin{tabular}{llll}
\multicolumn{2}{l}{stressed /æ/ in open syllables} & \multicolumn{2}{l}{unstressed /æ/ in open syllables} \\
21. & \textit{taté} `father' & 31. & \textit{l\stackunder[-10pt]{\^{u}}{\'{}}te} `nose' \\
22. & \textit{çénî} `needle' & 32. & \textit{vérve} `snow' \\
23. & \textit{mékî} `salt' & 33. & \textit{t\stackunder[-10pt]{\^{i}}{\'{}}re} `arrow' \\
24. & \textit{jénî} `woman' & 34. & \textit{táce} `crown' \\
25. & \textit{îsé }`now' & 35. & \textit{bíze} `goat' \\
26. & \textit{tazé} `new' & 36. & \textit{çóge} `knee' \\
27. & \textit{tûté} `dog' & 37. & \textit{şéve} `night'\\
28. & \textit{kité} `cat' & 38. & \textit{séye} `shade' \\
29. & \textit{pêré} `the day after tomorrow' & 39. & \textit{léme} `belly' \\
30. & \textit{m[é]ye} `sheep' & 40. & \textit{tevérg[e]} `hail' \\
\end{tabular}

    \includegraphics[width=.67\textwidth]{figures/e_open.png}
    \caption{Phonetic realisation of /æ/ <e> in open syllables}
    \label{fig:a open syll}
\end{figure}

As can be seen, allophones of /æ/ in open syllables are predominantly scattered in an area between the cardinal vowels [ɛ] and [e], regardless of whether they are stress-bearing or not. Comparing \figref{fig:a closed syll} to \figref{fig:a open syll}, it can be said that allophones of /æ/ occupy a considerable phonetic space. Notably, the allophonic realisation of /æ/ shows more raising in open syllables than in closed ones.

\newpage
\subsubsubsection{/ɛ/} \label{sect:ɛ}

\figref{fig:ɛ} represents the phonetic realisation of /ɛ/. As seen in \tabref{tab:ɛ vs. e}, the average realisation of /ɛ/ is phonetically very close to the average realisation of /æ/ in open syllables. One difference is that /ɛ/ is more front, i.e., it has a relatively higher F2. The other difference is that /æ/ is longer than /ɛ/. Investigating the average length of these vowels per two words yields the following figures. /ɛ/ has a length of 90 ms, whereas the figure is 170 ms for unstressed /æ/ and 150 ms for stressed /æ/.\footnote{Our phonetic analysis of /ɛ/ and /æ/ in Luhon H.\il{Hewramî!Luhon} shows that the same tendency holds for the difference between /ɛ/ and /æ/, running contrary to \citegen[8]{mackenzie_dialect_1966} description of the vowel system in Luhon, which presents /ɛ/ as a long vowel and /æ/ as a short vowel.}


\begin{tabular}{llll}
1. & \textit{l\'uwɛ} `they went' & 6. &  \textit{zan\'abîyɛm} `(if) I had known'\\
2. & \textit{n\'elɛ} `they didn't go'  & 7. &  \textit{y\'awɛ} `they arrived'\\
3. & \textit{luw\'ɛnî} `you (\textsc{f}) have gone' & 8. & \textit{y\'awɛ} `if they arrived'\\
4. & \textit{luw\'ɛne} `she has gone' & 9. & \textit{w\'atɛt} `[if] you had said'\\
5. & \textit{luw\'ɛnê} `they have gone' & 10. & \textit{w\'atɛş} `if he/she had said' \\
 & & 11. &  \textit{p\stackunder[-10pt]{\^{i}}{\'{}}yɛ} `men'\\
\end{tabular}\\
\begin{figure}
    \includegraphics[width=.67\textwidth]{figures/E.png}
    \caption{Phonetic realisation of /ɛ/ <ɛ>}
    \label{fig:ɛ}
\end{figure} 

\begin{table}
    \begin{tabular}{lll}
    \lsptoprule
vowel& F1& F2 \\
\midrule
/ɛ/ & 490& 2061 \\
stressed /æ/ in open syllables& 519& 1927 \\
unstressed /æ/ in open syllables & 506 & 1824 \\
\lspbottomrule
    \end{tabular}
    \caption{Phonetic realisation difference between /ɛ/ and /æ/}
    \label{tab:ɛ vs. e}
\end{table}

The distinction between /ɛ/ and /e/ <ê> is phonemically distinctive, as shown by the following pairs. 

\ea
\textit{b\stackunder[-10pt]{\^{i}}{\'{}}yê} \tab `they were' \tab  \textit{b\stackunder[-10pt]{\^{i}}{\'{}}yɛ}\tab  `if they had been' \\
\textit{luw\stackunder[-10pt]{\^{e}}{\'{}}nê}\tab  `they were going' \tab  \textit{luw\'ɛnê} \tab  `they have gone' \\
\z



\subsubsubsection{/ɑ/}

\begin{figure}[b]
\begin{tabular}{llll}
1. & \textit{babá} `grandfather' & 8. & \textit{waɫ\stackunder[-10pt]{\^{e}}{\'{}}} `sister' \\
2. & \textit{xuyá} `God'  & 9. & \textit{yag\stackunder[-10pt]{\^{e}}{\'{}} 	}`place' \\
3. & \textit{qurwáqî} `frog' & 10. & \textit{eçá} `there'\\
4. & \textit{gáve} `cow' & 11. & \textit{hamsá} `neighbour'\\
5. & \textit{degá} `village' & 12. & \textit{gîváv} `plant'\\
6. & \textit{laló} `maternal uncle' & 13. &  \textit{xás} `good'\\
7. & \textit{tat\stackunder[-10pt]{\^{e}}{\'{}}} `father' & 14. &  \textit{varán} `rain'\\
&  & 15. & \textit{gemáɫ} `male dog' \\
\end{tabular}
    \includegraphics[width=.67\textwidth]{figures/a.png}
    \caption{Phonetic realisation of /ɑ/ <a>}
    \label{fig:aa vowel}
\end{figure}

As shown in \figref{fig:aa vowel}, the allophonic realisation of /ɑ/ shows a wide scatter between cardinal vowels [a] and [ɑ]. It can be seen that /ɑ/ has a back realisation in the environment of back consonants, e.g., \textit{qurwaqî} `frog', and velarised consonants (caused by the spread of pharyngealisation), e.g., \textit{gemaɫ} `male dog'.
The great diversity in the positioning of /ɑ/ seems to have triggered the raising of [æ] to [ɛ] (see the preceding sections) as an instance of a push chain, meaning that the diversity in the phonetic positioning of /æ/ could have potentially made the distinction between [æ] and [ɑ] difficult and this, in turn, resulted in the raising of [æ] to [ɛ]. 

\subsubsubsection{/o/}


The allophonic realisation of /o/ is mainly scattered between cardinal vowels [o] and [ɔ]; see \figref{fig:o vowel}. /o/ tends to have a more back realisation in the environment of the velarised consonant /ɫ/, e.g., \textit{zoɫfê} `hair (collective)', the voiceless velar plosive /k/, e.g., \textit{çiko} `from where', and in closed syllables, e.g., \textit{sot} `it burnt'; \textit{qoɫ} `deep'.

\begin{figure}
\begin{tabular}{llll}
1. & \textit{góşê} `ear' & 8. &  \textit{sawró} `cow's dung'\\
2. & \textit{laló} `maternal uncle'  & 9. & \textit{séro} `on top'\\
3. & \textit{nîmeřó} `midday' & 10. & \textit{aró} `today'\\
4. & \textit{çikó} `from where' & 11. & \textit{sot} `it burnt'\\
5. & \textit{dóga} `village' & 12. & \textit{pos} `skin'\\
6. & \textit{cûcóɫê} `chicken' & 13. &  \textit{çoɫ} `vacant, empty'\\
7. & \textit{zóɫfê} `hair (collective)' & 14. & \textit{kor} `blind' \\
&  & 15. & \textit{qoɫ} `deep' \\
\end{tabular}
    \includegraphics[width=.67\textwidth]{figures/o.png}
    \caption{Phonetic realisation of /o/ <o>}
    \label{fig:o vowel}
\end{figure}



\newpage
\subsubsubsection{/u/}
Like other vowels, the allophones of the close back rounded vowel /û/ exhibit a wide scatter; see \figref{fig:u long vowel}. It is particularly interesting that the realisation of /û/ is close to the cardinal vowel [o], especially in the environment of velar consonants, e.g., \textit{hang\stackunder[-10pt]{\^{u}}{\'{}}rî} `grape', but also in closed syllables, e.g., \textit{sûr} `red'.

\begin{figure}

\begin{tabular}{llll}
1. & \textit{l\stackunder[-10pt]{\^{u}}{\'{}}te} `nose' & 6. &  \textit{heng\stackunder[-10pt]{\^{u}}{\'{}}rî} `grape'\\
2. & \textit{kîl\stackunder[-10pt]{\^{u}}{\'{}}} `kilo'  &7. & \textit{dûr} `far'\\
3. & \textit{tûté} `dog' & 8. & \textit{sûk} `light'\\
4. & \textit{zû} `early' & 9. & \textit{sûr} `red'\\
5. & \textit{şû} `husband' & 10. & \textit{ber\stackunder[-10pt]{\^{u}}{\'{}}ş} `I will take it'\\
\end{tabular}

    \includegraphics[width=.67\textwidth]{figures/uu.png}
    \caption{Phonetic realisation of /u/ <û>}
    \label{fig:u long vowel}
\end{figure}


\newpage
\subsubsubsection{/ʊ/}


The allophones of /u/ occupy a large horizontal phonetic space. In most environments, they are realised as a central variant, e.g., \textit{xuya} `God'. /u/ is maximally back in the environment of voiceless obstruents, e.g., \textit{puxte} `clean'. It is also noticeable that /u/ gets fronted when followed by liquid consonants, e.g., \textit{kul} `blunt'.

\begin{figure}
\begin{tabular}{llll}
1. & \textit{xuy\'a} `God' & 6. &  \textit{hurgêɫ\'ayve} `return' \\
2. & \textit{gurc} `ready'  & 7. & \textit{cuħm\'e} `Friday'\\
3. & \textit{qurs} `heavy' & 8. & \textit{kul} `blunt'\\
4. & \textit{gul} `bad, dirty' & 9. & \textit{kulk\'e} `finger'\\
5. & \textit{puxt\'e} `clean' & 10. & \textit {wuşh\'aɫ} `happy'\\
\end{tabular}
    \includegraphics[width=.67\textwidth]{figures/u.png}
    \caption{Phonetic realisation of /ʊ/ <u>}
    \label{fig:u short vowel}
\end{figure}

\subsubsubsection{/ɨ/}
As seen in \figref{fig:i short vowel}, the allophonic realisation of /i/ is scattered particularly to the right of the cardinal vowel [e], e.g., \textit{b\'ize} `goat', thus exhibiting characteristics of a central vowel. The allophones of /i/ show fronting in the environment of liquids, e.g., \textit{kirmî} `worm', and backing in the environment of glides, e.g., \textit{w\'inî} `blood'.

\begin{figure}
    

\begin{tabular}{llll}
1. & \textit{bíze} `goat' & 7. & \textit{tíş} `acid' \\
2. & \textit{dívê} `two'  & 8. & \textit{nasík} `gentle' \\
3. & \textit{qíjê} `hair' & 9. & \textit{dir\stackunder[-10pt]{\^{e}}{\'{}}j} `long' \\
4. & \textit{gic\stackunder[-10pt]{\^{i}}{\'{}}} `shirt' & 10. & \textit{viryá} `attentive' \\
5. & \textit{wínî} `blood' & 11. & \textit{pişqélî} `sheep's dung' \\
6. & \textit{kírmî} `worm' & 12. & \textit{birá} `brother' \\
\end{tabular}

    \includegraphics[width=.70\textwidth]{figures/i.png}
    \caption{Phonetic realisation of /ɨ/ <i>}
    \label{fig:i short vowel}
\end{figure}
 
 \newpage

\subsection{Consonant phonemes}

There are overall 29 consonant phonemes (see \tabref{tab:Cchart}). The consonant phonemes put in parentheses are peripheral in that they are restricted to certain positions in the word, or they are limited to loanwords\is{loanwords}, e.g., /γ/.

\begin{table}
\caption{Consonant inventory}
\label{tab:Cchart}\small
\fittable{
	\begin{tabular}{llllllllll}
		\lsptoprule
&{Labial} & {Lab.-dent.} & {Alv.} & {Postalv.} & {Pal.} & {Vel.} & {Uv.} & {Phar.} & {Glott.} \\
\midrule
{Stop} & p b &  & t  d &  &  & k  ɡ& q &  & (ʔ) \\
{Affricate} &  &  &  & t͡ʃ  d͡ʒ &  &  &  &  & \\
{Nasal} &\phantom{0 }m &  &\phantom{0 }n &  &  &\phantom{0 }(ŋ)  &  &  & \\
{Fricative} &  & f  & s  z & ʃ      ʒ &  & x    (γ) &   & ħ  ʕ & h \\
{Tap} &  &  &\phantom{0 }ɾ &  &  &  &  &  & \\
{Trill} &  &  &\phantom{0 }r &  &  &  &  &  & \\
{Lateral} &  &  &\phantom{0 }l (ɫ) & &  & &  &  & \\
{Approximant} &\phantom{0 }w &  &         &  &\phantom{0 }j &  & & & \\
\lspbottomrule
	\end{tabular}
	}
\end{table}

 Minimal pairs or near-minimal pairs distinguish between voiced-voiceless pairs in each set. The sets in (14)--(17) and those in (24)--(28) have the same place of articulation, while the rest differ in their manner of articulation.
\ea{Bilabials /p-b-m-w/ \\
\begin{tabular}{lll}
\textit{pîr}& `old'& [pʰiɾ] \\
\textit{bîre}& `well'& [ˈbi.ɾɛ] \\
\textit{mîr}& `prince'& [miɾ] \\
\textit{wîr}& `memory'& [wiɾ] \\ 
\end{tabular}}
\z

\ea{Alveolars /t-d-n-s-z-r-l/ \\
\begin{tabular}{lll}
\textit{tame}& `taste'& [ˈtʰɑ.mɛ] \\
\textit{dame}& `a type of game'& [dɑ.ˈmɛ]  \\
\textit{namê}& `name'& [ˈnɑ.me]  \\
\textit{zame}& `wound'& [ˈzɑ.mɛ]  \\
\textit{řame}& `egg put under hen'& [ˈrɑ.mɛ]  \\
\textit{leme}& `belly'& [ˈlɛ.mɛ]  \\
\end{tabular}}
\z

\ea{Velars and glottal: /k-g-w-h/. /ŋ/ does not occur word-initially; ʔ occurs word-initially before vowels. \\
\begin{tabular}{lll}
\textit{kewe}& `blue'& [kʰɛ.ˈwɛ] \\
\textit{gawe}& `cow'& [ˈɡɑ.wɛ] \\
\textit{wawe}& `again'& [ˈwɑ.wɛ] \\
\textit{hewa}& `weather'& [hɛ.ˈwɑ] \\
\textit{ew}& `he'& [ʔæw] \\
\end{tabular}}
\z


\ea{Affricates / t͡ʃ-d͡ʒ/\\
\begin{tabular}{lll}
\textit{ça}& `there'& [t͡ʃɑ]\\
\textit{ca}& `then'& [d͡ʒɑ] \\
\end{tabular}}
\z

\ea{Voiceless stops /p-t-k-q/ \\
\begin{tabular}{lll}
\textit{peř}& `full'& [pʰær] \\
\textit{teř}& `wet'& [tʰær] \\
\textit{keř}& `deaf'& [kʰær] \\
\textit{qen}& `sugar cube'& [qæn] \\
\end{tabular}}
\z

\ea{Voiced stops /b-d-g/ \\
\begin{tabular}{lll}
\textit{bar}& `load'& [bɑɾ] \\
\textit{dar}& `tree'& [dɑɾ] \\
\textit{gaw}& `bull'& [ɡɑw] \\
\end{tabular}}
\z

\ea{Voiceless fricatives /f-s-ʃ-ħ-h/ \\
\begin{tabular}{lll}
\textit{fire}& `much, very'& [fɨ.ˈɾɛ] \\
\textit{sere}& `head'& [sɛ.ˈɾɛ] \\
\textit{şewe}& `night'& [ˈʃɛ.wɛ] \\
\textit{ħeře}& `mud'& [ˈħɛ.rˤɛ] \\
\textit{here}& `the donkey'& [hɛ.ˈɾɛ] \\
\end{tabular}}
\z

\ea{Voiced fricatives /z-ʒ-ʕ-γ/ \\
\begin{tabular}{lll}
\textit{zaɫe}& `gall bladder'& [zɑ.ˈɫɛ] \\
\textit{jaɫe}& `apiary'& [ʒɑ.ˈɫɛ] \\
\textit{ʕale}& `good' (\textsc{f})& [ˈʕɑ.lɛ] \\
\textit{teẍare}& `a unit of weight equal to 120 kilos'& [tɛ.ˈγɑ.ɾɛ] \\
\end{tabular}}
\z

\ea{Nasals /m-n-ŋ/ \\
\begin{tabular}{lll}
\textit{dem}& `mouth'& [dæm] \\
\textit{deŋ}& `voice'& [dæŋ] \\
\textit{ben}& `woollen string'& [bæn] \\
\end{tabular}}
\z


\ea{Approximants /w-y/ \\
\begin{tabular}{lll}
\textit{yeher}& `liver'& [yɛ.ˈhæɾ] \\
\textit{wehar}& `spring'& [wɛ.ˈhɑɾ] \\
\end{tabular}}
\z

\ea{Bilabial stops /p-b/ \\
\begin{tabular}{lll}
\textit{par}& `last year'& [pʰɑɾ] \\
\textit{bar}& `load'& [bɑɾ] \\
\end{tabular}}
\z

\ea{Alveolar stops /t-d/ \\
\begin{tabular}{lll}
\textit{taɫ}& `bitter'& [tʰɑɫ] \\
\textit{daɫ}& `falcon'& [dɑɫ] \\
\end{tabular}}
\z 

\ea{Velar stops /k-g/ \\
\begin{tabular}{lll}
\textit{kul}& `blunt'& [kʊl] \\
\textit{gul}& `bad'& [ɡʊl] \\
\end{tabular}}
\z

\ea{Rhotics /ɾ-r-rˤ/ \\
\begin{tabular}{lll}
\textit{ħerˤe}& `mud'& [ˈħɛ.rˤɛ] \\
\textit{here}& `the donkey'& [hɛ.ˈɾɛ] \\
\textit{mere}& `meadow'& [ˈmɛ.ɾɛ] \\
\textit{merˤe}& `cave'& [ˈmɛ.rˤɛ] \\
\end{tabular}}
\z

\newpage
\ea{Laterals /l-ɫ/ \\
\begin{tabular}{lll}
\textit{kel}& `mountain pass'& [kʰæl] \\
\textit{keɫ}& `mountain goat'& [kʰæɫ] \\
& \\
\textit{çil}& `forty'& [t͡ʃɨl] \\
\textit{çiɫ}& `branch'& [t͡ʃɨɫ] \\
\end{tabular}}
\z

\subsubsection{Description of consonants}\label{Dofcons}
% \subsubsubsection{Stops}\label{stops}

\phonemesubsubsubsection{/p/}{
    $\left\{
    \begin{tabular}{l}
        {}[pʰ]/\#{\longrule}\\
        {}[p̚]/{\longrule}\#  %you need the {} for the \\ not to catch the []
    \end{tabular}
    \right.$
    }

/p/ is a voiceless bilabial stop. It occurs both syllable-initially and in the syllable-final position. In the former position, it is generally aspirated, except when followed by central vowels, where aspiration is weakened. Similarly, aspiration is weakened when /p/ is preceded by a voiceless fricative. /p/ is unreleased word-finally. 
\TabPositions{2cm,5cm,7cm}
\ea
\textit{pîr}\tab  [pʰiɾ]\tab  `old' \\
\textit{pirđî}\tab  [ˈpɨɾ.ɹˠi]\tab  `bridge' \\
\textit{espeřêz}\tab  [ʔæs.pɛ.ˈrez]  \tab  `a village in Hewraman' \\
\textit{çep}\tab  [t͡ʃæp̚]\tab  `left' \\
\z

% \begin{tabular}{l}

\phonemesubsubsubsection{/b/}{[b]}
% \end{tabular}

/b/ is a voiced bilabial stop that occurs in syllable-initial and syllable-final position. 
\ea
\textit{bar}\tab  [bɑɾ]\tab  `load' \\
\textit{qesab} \tab  [qɛ.ˈsɑb]\tab  `butcher' \\
\z

In peripheral Gorani dialects, e.g., Zerde\il{Gorani!Zerde}, and Gewrecû\il{Gorani!Gewrecû}, the post-vocalic /b/ often lenites to /w/. This lenition process is not active in core Hewramî varieties such as Tekht Hewramî\il{Hewramî!Tekht} \citep[][]{mohammadirad_lenition_nodate}.
\ea
\textit{cuwa\textbf{b}} \tab  cf. G. Zerde \textit{cuwa\textbf{w}}\tab  `answer' \\
\textit{xira\textbf{b}}\tab  cf. G. Gewrecû \textit{xira\textbf{w}}\tab  `bad' \\
\z

\phonemesubsubsubsection{/t/}{
        $\left\{
        \begin{tabular}{l}
            {}[tʰ]/\#{\longrule}\\
            {}[t̚ ]/{\longrule}\#  %you need the {} for the \\ not to catch the []
        \end{tabular}
        \right.$
}

/t/ is a voiceless alveolar stop. It has the same phonetic realisations as /p/ syllable-initially and syllable-finally. Similarly, aspiration is weakened when /t/ is followed by central vowels. 
\ea
\textit{to}\tab  [tʰo]\tab  `you (\textsc{sg})' \\
\textit{tiş}\tab  [tɨʃ]\tab  `acid' \\
\textit{dastan}\tab  [dɑs.ˈtɑn]\tab  `story' \\
\textit{xizmet}\tab  [xɨz.ˈmæt̚]\tab `service' \\
\z

\phonemesubsubsubsection{/d/}{
        $\left\{
        \begin{tabular}{l}
            {}[d]/\#{\longrule}\\
            {}[ɹˠ] ∼ [j] ∼ 0/ V {\longrule} V  \\ %you need the {} \\for the \\ not to catch the []
            {}[ɹˠ]/{\longrule}\#
        \end{tabular}
        \right.$
}

/d/ is a voiced alveolar stop. It is realised as /d/ word-initially. Its realisation is not stable in intervocalic position, where it is sometimes lenited as a velarised alveolar approximant [ɹˠ] and sometimes as a palatal approximant [j]. When /d/ is realised as a [ɹˠ], the grapheme <đ> is used for its representation; when it is realised as a glide, the grapheme <y> represents it. These two allophones are in free variation with each other. The same speaker may use both allophones in their speech, sometimes for the same word. 

Note that /j/ <y> is a separate phoneme in Hewramî\il{Hewramî}. Thus, the distinction between /d/ and /j/ is phonetically neutralised in the intervocalic position. 

The lenition of /d/ is a widespread phenomenon across Iranian and non-Iranian languages spoken along the Zagros mountains, and the phenomenon is called ``Zagros /d/'' \citep[see the chapters in][]{haig_languages_2018}.
\TabPositions{1.5cm,4cm,6.5m}
\ea
\textit{daresan}\tab  [dɑ.ɾɛ.ˈsɑn]\tab  `forest' \\
\textit{miđo}\tab  [mɨ.ˈɹˠo]\tab  `He/she gives.' \\
\textit{xuya}\tab  [xʊ.ˈjɑ]\tab  `God' \\
\z

In some intervocalic contexts, /d/ tends to be deleted entirely. Similarly, deletion is attested after the flap consonant across the syllable boundary.
\TabPositions{2cm,5cm,8cm}
\ea
\textit{qeyîm}\tab  [qɛ.ˈim] \tab  `old, past'\tab  < cf. Ar.\il{Arabic} \textit{qa\textbf{d}im} \\
\textit{bîye}\tab  [ˈbi.ɛ] \tab  `Look!' \tab < \textit{bi + \textbf{d}y(e) + e} \\
\textit{aye, ađe}\tab  [ˈʔɑ.jɛ], [ˈʔɑ.ɹˠɛ]\tab  `she'\tab  \\
\z

/d/-deletion also occurs in post-consonantal slots, where the relevant consonant is a rhotic, though not deleted in \textit{merđ}.
\ea
\textit{kursan}\tab  [kʊɾ.ˈsɑn]\tab `Kurdistan' \tab <*\textit{Kur\textbf{d}isan} \\
\textit{zer} \tab  [zæɾ]\tab  `yellow' \tab  < cf. CK.\il{Kurdish!Central} \textit{zerd} \\
\textit{merđ}\tab  [mæɾɹˠ]\tab  `he died' \\
\z

Similarly, /d/ often undergoes lenition to a velarised alveolar approximant [ɹˠ] in post-vocalic position.\footnote{See \citet{mohammadirad_lenition_nodate} for an overview of the lenition of voiced stops within Kurdic, including in Gorani\il{Gorani} dialects.} 
\TabPositions{1.5cm,3cm,4.5cm}
\ea
\textit{ađ}\tab  [ʔɑɹˠ]\tab  `he' \\
\textit{dađ}\tab  [dɑɹˠ]\tab  `shout' \\
\z 


\phonemesubsubsubsection{/k/}{
        $\left\{
        \begin{tabular}{l}
            {}[kʰ]/\#{\longrule}\\
            {}[k̚̚ ]/{\longrule}\#  %you need the {} for the \\ not to catch the []
        \end{tabular}
        \right.$
}

/k/ is a voiceless velar stop. It is usually strongly aspirated syllable-initially, except when followed by central vowels or voiceless fricative consonants. In word-final position, it is unreleased. Therefore, /k/ forms a natural class with the voiceless stops /p/ and /t/.
\TabPositions{2cm,4cm,7cm}
\ea
\textit{keř}\tab  [kʰær]\tab  `deaf' \\
\textit{kinaçê}\tab  [kɨ.nɑ.ˈt͡ʃe]\tab  `girl' \\
\textit{kul}\tab  [kʊl]\tab  `blunt' \\
\textit{bêşkê}\tab  [beʃ.ˈkɛ] \tab  `cot' \\
\textit{sûk}\tab  [suk̚ ]\tab  `light' \\
\z

% \begin{tabular}{l}

\phonemesubsubsubsection{/g/}{[ɡ]}
% \end{tabular}

/ɡ/ is a voiced velar stop which occurs both syllable-initially and syllable-finally. 
\ea
\textit{gave}\tab  [ˈɡɑ.vɛ]\tab  `cow' \\
\textit{beg}\tab  [bæɡ]\tab  `chieftain' \\
\z

% \begin{tabular}{l}

\phonemesubsubsubsection{/q/}{[q]}
% \end{tabular}

/q/ is an unvoiced uvular stop. It occurs both syllable-initially and syllable-finally. Due to its back articulation, /q/ is not generally aspirated in the syllable-initial position.
\ea
\textit{qesab}\tab  [qɛ.ˈsɑb]\tab  `butcher' \\
\textit{qaqez}\tab  [qɑ.ˈqæz]\tab  `paper' \\
\textit{teq}\tab  [tʰæq]\tab  `knocking' \\
\z

% \begin{tabular}{l}

\phonemesubsubsubsection{/ʔ/}{[ʔ]/\#{\longrule}V}
% \end{tabular}

/ʔ/ is a voiceless glottal stop. It does not have contrastive phonemic status in Hewramî\il{Hewramî} and only occurs in syllable-initial position. Its occurrence is conditioned by the avoidance of empty onsets in word-initial position.   
\ea
\textit{asaw}\tab  [ʔɑ.ˈsɑw]\tab  `mill' \\
\textit{avî}\tab  [ˈʔɑ.vi]\tab  `water' \\
\z

% \subsubsubsection{Affricates}



% \begin{tabular}{l} \\

\phonemesubsubsubsection{/t͡ʃ/}{[t͡ʃʰ]/\#{\longrule}}
% \end{tabular}

/t͡ʃ/ is a voiceless post-alveolar affricate, represented as <ç>. It occurs both in syllable-initial and syllable-final position. In the former position, it is generally aspirated. 
\ea
\textit{çêş}\tab  [t͡ʃʰeʃ]\tab  `what' \\
\z

/t͡ʃ/ undergoes assimilation\is{assimilation} in voicing when preceded by a vowel and followed by alveolar /d/. Here, the stop part of the affricate is deleted and becomes voiced.
\ea
\textit{wajdê}\tab  [wɑʒ.ˈde]\tab  `You (pl.) say!'\tab  cf. \textit{waçdê} 
\z

% \begin{tabular}{l}

\phonemesubsubsubsection{/d͡ʒ/}{[d͡ʒ]}
% \end{tabular}

/d͡ʒ/ is a voiced post-alveolar affricate, represented as <c>. It occurs in syllable-initial and syllable-final position.
\ea
\textit{cîya}\tab  [d͡ʒi.ˈjɑ]\tab  `separate' \\
\textit{bêcge}\tab  [ˈbed͡ʒ.ɡɛ]\tab  `apart from' \\
\z 

Note that /d͡ʒ/ has a low phonemic load in Hewramî\il{Hewramî}. One of the known isoglosses within Iranian is that the Old Iranian palatal approximant /j/ is preserved syllable initially in Gorani\il{Gorani} while it has shifted to /d͡ʒ/ in Kurdish\il{Kurdish}, e.g.:
\TabPositions{2cm,4cm,6cm,8.cm}
\ea
Gorani\il{Gorani}\tab  Kurdish\il{Kurdish} \\
\textit{yewe}\tab  \textit{co, ceh}\tab  `barely'\tab  cf. Av.\il{Avestan} \textit{yauua} \\
\textit{yeher}\tab  \textit{cerg, ciger}\tab  `liver' \\
\z 

The low functional load of <c> in Hewramî\il{Hewramî} may explain the realisation of [d͡ʒ] in the Arabic\il{Arabic} loan \textit{aciz} `upset' as /d/, hence \textit{adiz} [ʔɑ.dˈɨz].

% \subsubsubsection{Nasals}

% \begin{tabular}{l} \\

\phonemesubsubsubsection{/m/}{[m]}
% \end{tabular}

[m] is a voiced bilabial nasal that occurs syllable-initially and syllable-finally. 
\ea
\textit{mûso}\tab  [mu.ˈso]\tab  `he/she is sleeping' \\
\textit{kam}\tab  [kʰɑm]\tab  `which' \\
\z

\phonemesubsubsubsection{/n/}{
        $\left\{
        \begin{tabular}{l}
            {}[n] \\
            {}[ŋ]/{\longrule} C [ +sibilant] 
        \end{tabular}
        \right.$
}

/n/ is a voiced alveolar nasal that occurs syllable-initially and syllable-finally. 
\ea
\textit{namê}\tab  [ˈnɑ.me]\tab  `name' \\
\textit{nan}\tab  [nɑn]\tab  `food' \\
\z

/n/ is weakened before the voiced sibilant consonants /z/ and /s/, and sounds like a nasal glide:
\ea
\textit{paŋze}\tab  [pɑŋ.ˈzɛ]\tab  `fifteen' \\
\textit{sêŋze}\tab  [seŋ.ˈzɛ]\tab  `thirteen' \\
\textit{paŋsew}\tab  [pɑŋ.ˈsæw]\tab  `five-hundred' \\
\z

% \begin{tabular}{l}

\phonemesubsubsubsection{/ŋ/}{[ŋ]}
% \end{tabular}

The phoneme /ŋ/ is a voiced velar nasal, represented by the grapheme <ŋ> in words where it always appears, as in \textit{paŋze} `fifteen' (above). In some cases, however, the phoneme is not stable in the current state of the language, and some speakers pronounce it as <ng>. Notably, /ŋ/ only occurs syllable-finally.
\ea
\textit{deng}\tab  [dæŋ]\tab  `voice' \\
\textit{ceng}\tab  [d͡ʒæŋ]\tab  `war' \\
\z

% \subsubsubsection{Fricatives}

% \begin{tabular}{l} \\

\phonemesubsubsubsection{/f/}{[f]}
% \end{tabular}

/f/ is a voiceless labio-dental fricative. It occurs syllable-initially and syllable-finally. 
\ea
\textit{fire}\tab  [fɨ.ˈɾɛ]\tab  `much, very' \\
\textit{sêf}\tab  [sef]\tab  `potato' \\
\z

\phonemesubsubsubsection{/s/}{
        $\left\{
        \begin{array}{l}
            {}[\text{s}] \\
            {}[\text{ṣ}]/\#{\longrule} V\left[\begin{array}{c}+\text{low}\\
                                                        + \text{back}\\
                                                        -\text{rounded}
                                                    \end{array}\right] ∼ V
                                                    \left[\begin{array}{c}+\text{low}\\
                                                                                                    + \text{back}\\
                                                                                                    -\text{rounded}
                                                                                                \end{array}\right]
        \end{array}
        \right.$
}

/s/ is a voiceless alveolar fricative. It occurs both word-initially and word-finally.

\ea
\textit{saɫe}\tab  [sɑɫɛ]\tab  `year' \\
\textit{mas}\tab  [mɑs]\tab  `yoghurt' \\
\z

/s/ becomes pharyngealised in the environment of back vowels. This has been shown with acoustic analysis for the following pairs (see \citealt[24]{khan_language_2023}). The greater F1 frequency in the pharyngealised words suggests the lowering of the vowel in the environment of the pharyngeal phoneme. Similarly, the lower F2 for the pharyngealised vowels means that these vowels are realised further back compared to the vowels in non-pharyngealised words.
\TabPositions{1.5cm,3cm,5cm,7cm,9cm}
\ea
\textit{şeṣ}\tab  [ʃɑsˤ]\tab  `sixty'\tab  /ɑṣ/\tab  F1=623\tab  F2=982 \\
\textit{mes}\tab  [mæs]\tab  `drunk'\tab  /æs/\tab  F1=589\tab  F2=1464 \\
\z

\ea
\textit{ṣe}\tab  [sˤɑ]\tab  `hundred'\tab  /ṣɑ/\tab  F1=604\tab  F2=942 \\
\textit{îse}\tab  [ʔi:ˈsɛ]\tab  `now'\tab  /sɛ/\tab  F1=438\tab  F2=1708 \\
\z


\phonemesubsubsubsection{/z/}{
        $\left\{
        \begin{tabular}{l}
            {}[z]\\
            {}[j] / V {\longrule} V  (optional)  %you need the {} \\for the \\ not to catch the []
        \end{tabular}
        \right.$
}

/z/ is a voiced alveolar fricative that occurs syllable-initially and syllable-finally.

\ea
\textit{zînan}\tab  [zi.ˈnɑn]\tab  `prison' \\
\textit{payîz}\tab  [pɑ.ˈiz]\tab  `autumn' \\
\z

/z/ is sometimes lenited to a palatal approximant [j] in intervocalic position. 
\TabPositions{2cm,4cm,8cm}
\ea
\textit{meyan\^u}\tab  [mɛ.jɑ.ˈnu]\tab  `I don't know' \tab  cf. \textit{mezanû}\\
\z

% \begin{tabular}{l}

\phonemesubsubsubsection{/ʃ/}{[ʃ]}
% \end{tabular}

/ʃ/ is a voiceless post-alveolar fricative, represented as <ş>. It occurs both syllable-initially and syllable-finally.
\TabPositions{1.5cm,3cm,4.5cm}
\ea
\textit{şot}\tab  [ʃot]\tab  `milk' \\
\textit{tiş}\tab  [tɨʃ]\tab  `acid' \\
\z

/ʃ/ undergoes assimilation\is{assimilation} in voicing when preceded by a vowel and followed by alveolar /d/:
\TabPositions{2cm,4cm,8cm}
\ea
\textit{kujdê}\tab  [kʊʒ.ˈde]\tab  `You (\textsc{pl}) kill!'\tab  cf. \textit{kuşdê} \\
\textit{gojd}\tab  [ɡoʒd]\tab  `meat' \tab  cf. \textit{goşt, goşd} \\
\z

% \begin{tabular}{l}

\phonemesubsubsubsection{/ʒ/}{[ʒ]}
% \end{tabular}

/ʒ/ is a voiced post-alveolar fricative, represented as <j>. It occurs both syllable-initially and syllable-finally. 
\TabPositions{1.5cm,3cm,4.5cm}
\ea
\textit{jenî}\tab  [ˈʒɛ.ni]\tab  `woman' \\
\textit{řoj}\tab  [roʒ]\tab  `daylight' \\
\z

% \begin{tabular}{l}

\phonemesubsubsubsection{/x/}{[x]}
% \end{tabular}

/x/ is a voiceless velar fricative that occurs syllable-initially and syllable-finally. 
\ea
\textit{xele}\tab  [xɛ.ˈlɛ]\tab  `grain' \\
\textit{xas}\tab  [xɑs]\tab  `good' \\
\textit{puxte}\tab  [pʊx.ˈtɛ]\tab  `clean' \\
\textit{bax}\tab  [bɑx]\tab  `garden' \\
\z

% \begin{tabular}{l}

\phonemesubsubsubsection{/ɣ/}{[ɣ]}
% \end{tabular}

/ɣ/ is a voiced velar fricative, represented as <ẍ>. It only occurs syllable-initially.
\ea
\textit{aẍe}\tab  [ʔɑ.ˈɣɛ]\tab  `lord' \\
\textit{ẍeyb}\tab  [ɣ{\ae}jb]\tab  `disappeared' \\
\z

% \begin{tabular}{l}

\phonemesubsubsubsection{/ħ/}{[ħ] /\#{\longrule}}
% \end{tabular}

/ħ/ is a voiceless pharyngeal fricative. It occurs both in words of Iranian stock and loanwords\is{loanwords} from Semitic languages\il{Semitic}. In the native lexicon, the non-etymological /ħ/ has developed through either pharyngealisation of the glottal fricative /h/ or in the environment of back vowels; see \textit{ʕine}.
\TabPositions{2cm,4cm,7cm}
\ea
\textit{ħot}\tab  [ħot]\tab  `seven'\tab  cf. P.\il{Persian} \textit{haft} \\
\textit{ʕine}\tab  [ˈʕɨnɛ]\tab  `buttock'\tab  cf. CK. \il{Kurdish!Central}/NK.\il{Kurdish!Northern} \textit{qûn} \\
\textit{ħîç}\tab  [ħit͡ʃ]\tab  `nothing'\tab  cf. P.\il{Persian} \textit{hîç} \\
\z

Note that the pharyngealisation of /ħ/ in \textit{ħîç} `nothing' seems to be conditioned by information structure, such that when the word is in focus\is{focus}, /h/ tends to be pharyngealised. This is likely due to the ``perceptual magnet effect'' associated with pharyngeals \citep[][]{blevins2017areal}.

/ħ/ is preserved in loanwords\is{loanwords} from Arabic\il{Arabic}, where it occurs mainly syllable-initially. The syllable-final position of /ħ/ results either from the retention of /ħ/ from the source word, e.g., \textit{nîkaħ} (see below) or from shifting the originally syllable-initial voiced pharyngeal [ʕ] to the syllable coda through metathesis\is{metathesis} and then devoicing it to [ħ], see \textit{cuħme}. Note that variation exists between speakers for devoicing [ʕ] syllable-finally. Therefore, no categorical rule can be posited here. 
\ea
\textit{ħeywan}\tab  [ħæj.ˈwɑn] \tab  `animal' \\
\textit{ħukm}\tab  [ħʊkm] \tab  `rule' \\
\textit{nîkaħ}\tab  [ni.ˈkɑħ] \tab  `marriage' \\
\textit{cuħme}\tab  [d͡ʒʊħ.ˈmɛ] \tab  `Friday'\tab  cf. Ar.\il{Arabic} \textit{jomʕa} \\
\textit{ħegaɫ}\tab  [ħɛˈɡɑɫ] \tab  `scarf'\tab  cf. Ar.\il{Arabic} \textit{ʕiqal} \\ 
\z


\phonemesubsubsubsection{/ʕ/}{
        $\left\{
        \begin{tabular}{l}
            {}[ʕ]\\
            {}[ħ]{\longrule}\# (optional)
        \end{tabular}
        \right.$
}


/ʕ/ is a voiced pharyngeal fricative. It is loaned from Arabic and can occur only syllable-initially. [ʕ] sometimes undergoes devoicing to [ħ]. 
\ea
\textit{ʕal}\tab  [ʕɑl]\tab  `good'\tab  cf. Ar.\il{Arabic} \textit{ʕaliya} \\
\textit{temaʕe} \tab  [tʰɛ.ˈmɑ.ʕɛ]\tab  `greed'\tab  cf. Ar.\il{Arabic} \textit{ṭamaʕiyya} \\
\textit{cuħme}\tab  [d͡ʒʊħ.ˈmɛ]\tab  `Friday'\tab  cf. Ar.\il{Arabic} \textit{jomʕa} \\
\textit{deħfe}\tab  [ˈdæħfɛ]\tab  `exclusion'\tab  cf. Ar.\il{Arabic} \textit{dafʕa} \\
\z

% \begin{tabular}{l}
\phonemesubsubsubsection{/h/}{[h]}
% \end{tabular}
 
/h/ is a voiceless glottal fricative that occurs only syllable-initially.
\ea
\textit{her}\tab  [hæɾ] \tab  `each, every' \\
\textit{hamin}\tab  [hɑ.ˈmɨn] \tab  `summer' \\
\z

/h/ is regularly deleted from the syllable-final position in cognate words.
\ea
\textit{mêman}\tab [me.ˈmɑn]\tab  `guest'\tab  cf. P.\il{Persian} \textit{mehmān} \\
\textit{şa}\tab [ʃɑ]\tab  `king'\tab  cf. P.\il{Persian} \textit{şāh} \\
\z

% \newpage
% \subsubsubsection{Rhotics}

\phonemesubsubsubsection{/ɾ/}{
        $\left\{
        \begin{tabular}{l}
            {}[ɾ] /σ{\longrule},\; {\longrule}σ \;\; \text{(not word-initial)}\\
            {}[r]/\#{\longrule}  \\ %you need the {} \\for the \\ not to catch the []
            {}[rˤ]/σ{\longrule}
        \end{tabular}
        \right.$
}

There is a three-way distinction of rhotics in Tekht Hewramî\il{Hewramî!Tekht}. The rhotics can be realised as a tap /ɾ/, alveolar trill /r/, or an emphatic /rˤ/.

 /ɾ/ is a voiced alveolar tap, represented as <r>. It occurs syllable-initially and syllable-finally.
\ea
\textit{hesere}\tab  [hɛ.ˈsɛ.ɾɛ]\tab  `mule' \\
\textit{here}\tab  [hɛ.ˈɾɛ]\tab  `the donkey' \\
\textit{pîr}\tab  [pʰiɾ]\tab  `old' \\
\textit{şar}\tab  [ʃɑɾ]\tab  `city' \\
\z

/ɾ/ is excluded word-initially, where voiced alveolar trill /r/, represented as <ř>, is used instead. In other words, the contrast between /r/ and /ɾ/ is neutralised word-initially. /r/ occurs syllable-initially and syllable-finally. In the latter position, it usually appears with low and back vowels. 
\ea
\textit{řas}\tab  [rɑs]\tab  `correct, right' \\
\textit{kuř}\tab  [kʊr]\tab  `boy' \\
\textit{keř}\tab  [kʰær]\tab  `deaf' \\
\z

Hewramî\il{Hewramî} also has the emphatic /rˤ/, which occurs word-internally but is limited to the syllable initial position.  
\ea
\textit{merˤe}\tab  [ˈmɛ.rˤɛ]\tab  `cave' \\
\textit{herˤe}\tab  [ˈhɛ.rˤɛ]\tab  `mud' \\
\z

The distinction between these rhotic consonants has been measured by instrumental acoustic analysis in \citet[30]{khan_language_2023}. The pharyngealised /rˤ/ has lower F2 than /ɾ/ and /ř/, suggesting its back articulation in the vocal cavity. The adjacent vowels also show lower F2 in the environment of pharyngealised /rˤ/, suggesting flat resonance. Similarly, F1 is higher in the environment of pharyngealised /rˤ/, meaning that the tongue is closer to the bottom of the oral cavity. It can also be seen that F2 is lower for the alveolar trill /ř/ in \textit{beř} `product' in comparison to the flap /ɾ/ in \textit{here} `donkey', reflecting flatter resonance.\footnote{As the pharyngealised /rˤ/ has a very low functional load in Hewramî\il{Hewramî}, and indeed there is considerable cross-speaker variation in its production, I use the trilled <ř> to represent it.}
\TabPositions{2cm,4cm,6cm,7cm,9cm}
\ea
\textit{mere}\tab  [ˈmɛ.ɾɛ]\tab  `grassland'\tab  /e/\tab  F1=523 \tab  F2=1526 \\
\tab  \tab  \tab  /ɾ/\tab  F1=427 \tab  F2=1542 \\
\tab  \tab  \tab  /e/\tab  F1=490 \tab  F2=1562 \\
\textit{merˤe}\tab  [ˈmɑ.rˤɛ]\tab  `cave'\tab  /e/\tab  F1=725 \tab  F2=1119 \\
\tab  \tab  \tab  /rˤ/\tab  F1=637 \tab  F2=1036 \\
\tab  \tab  \tab  /e/\tab  F1=643 \tab  F2=1131 \\
\z

\ea
\textit{here}\tab  [hɛ.ˈɾɛ]\tab  `donkey'\tab  /e/\tab  F1=559 \tab  F2=1526 \\
\tab  \tab  \tab /ɾ/\tab  F1=486 \tab  F2=1525 \\
\tab  \tab  \tab /e/\tab  F1=515 \tab  F2=1561 \\
\textit{herˤe}\tab  [ˈhɑ.rˤɛ]\tab  `mud'\tab  /e/\tab  F1=668 \tab  F2=1145 \\
\tab  \tab  \tab /rˤ/\tab  F1=634 \tab  F2= 1049 \\
\tab  \tab  \tab  /e/\tab  F1=637 \tab  F2=1164 \\
\z

\ea
\textit{beř}\tab  [bær]\tab `product'\tab  /æ/\tab  F1=528 \tab  F2=1403 \\
\tab  \tab  \tab  /r/\tab  F1=447 \tab  F2=1469 \\
\textit{ber}ˤ\tab  [bærˤ]\tab  `dried'\tab  /æ/\tab  F1=653 \tab  F2=1112 \\
\tab  \tab  \tab  /rˤ/\tab  F1=705 \tab  F2=1143 \\
\z

% \subsubsubsection{subsection{Laterals}



% \begin{tabular}{l} \\
\phonemesubsubsubsection{/l/}{[l]}
% \end{tabular}

/l/ is a voiced alveolar lateral. It occurs both syllable-initially and syllable-finally.
\ea
\textit{lalo}\tab  [lɑ.ˈlo]\tab  `maternal uncle' \\
\textit{çil}\tab  [t͡ʃɨl]\tab  `forty' \\
\z

\phonemesubsubsubsection{/ɫ/}{
        $\left\{
        \begin{tabular}{l}
            {}[l]/\#{\longrule}\\
            {}[ɫ]/{\longrule}\#    
            \end{tabular}
        \right.$
}

/ɫ/ is a velarised alveolar lateral. It occurs mostly syllable-finally. It can sometimes occur syllable-initially; for instance, when the feminine\is{feminine} suffix \textit{-e} is added to masculine\is{masculine} nouns and adjectives with /ɫ/ in their coda, resyllabification occurs and /ɫ/ ends up as the onset of the final syllable. Unlike its plain counterpart, /ɫ/ cannot occur word-initially. 
\ea
\textit{laɫe}\tab  [ˈlɑ.ɫɛ]\tab  `deaf' (\textsc{f}) \\
\textit{saɫe}\tab  [ˈsɑ.ɫɛ]\tab  `year' \\
\z

% \subsubsubsection{Approximants}



% \begin{tabular}{l} \\

\phonemesubsubsubsection{/j/}{[j]}
% \end{tabular}

/j/ is a palatal approximant represented as <y>. It occurs syllable-initially and syllable-finally.
\TabPositions{2cm,5cm,9cm}
\ea
\textit{yerê}\tab  [ˈjɛ.re]\tab  `three' \\
\textit{berzepey}\tab  [bær.zɛ.ˈpæj]\tab  `standing' \\
\z

\phonemesubsubsubsection{/w/}{
        $\left\{
        \begin{tabular}{l}
            {}[v] ∼ [w]/\#{\longrule}\\
            {}[w]/{\longrule}\#  \\ 
            {}[ʋ]/ V {\longrule} V (optional)
            \end{tabular}
        \right.$
}

/w/ is a bilabial approximant. It is usually realised as a labio-dental fricative [v] syllable-initially. 
\TabPositions{2cm,4cm,7cm}
\ea
\textit{watiş}\tab  [ˈvɑt.ɨʃ]\tab  `he/she said' \\
\textit{wînû}\tab  [vi.ˈnu]\tab   `I see' \\
\z

In syllable-final position, realisation as /w/ is more common.
\ea
\textit{asaw}\tab  [ʔɑ.ˈsɑw]\tab  `mill' \\
\textit{masaw}\tab  [mɑs.ˈɑw]\tab  `fish' \\
\z

The sound is sometimes lenited in intervocalic position and realised as a labio-dental approximant [ʋ].
\TabPositions{2.5cm,6cm,9cm}
\ea
\textit{aweyanî}\tab  [ɑ.ʋɛ.jɑ.ˈni]\tab  `prosperity' \\
\textit{miro ken\stackunder[-10pt]{\^{e}}{\'{}}we}\tab  [mɨ.ˈɾo kɛ.ˈne.ʋɛ]\tab  `pear-picking' \\
\z

\subsection{Phoneme-grapheme associations}\label{orthography}
The transcription system used in this book follows the `Hawar' standard Kurdish\il{Kurdish} script \citep[][]{loc_kurdish_roman}. \tabref{tab:phon-graph} exhibits how the graphemes in the Hawar script correspond to IPA symbols.

\begin{table}
\begin{tabular}{cc@{\qquad\qquad}cc}
\lsptoprule
phoneme & grapheme & phoneme & grapheme \\
\midrule
p & p & ɣ & ẍ \\
b & b & ħ & ħ \\
t & t & ʕ & ʕ \\
d & d & h & h \\
k & k & ɾ & r \\
g & g & r & ř \\
q & q & l & l \\
t͡ʃ & ç & ɫ & ɫ \\
d͡ʒ & c & w & w \\
m & m & j & y \\
n & n & i & î \\
ŋ & ŋ & e & ê \\
f & f & ɛ∼æ & e \\
s & s & ɛ & ɛ \\
z & z & ɨ & i \\
ʃ & ş & u & û \\
ʒ & j & ʊ & u \\
x & x & o & o \\
 & & ɑ & a \\
\lspbottomrule
\end{tabular}
\caption{Phoneme-grapheme associations}
\label{tab:phon-graph}
\end{table}

The Standard Kurdish\il{Kurdish} transcription system used in this book differs from \citegen{mackenzie_dialect_1966} transcription system used for Luhon Hewram\^i\il{Hewramî!Luhon} and, more broadly, the transcription systems in Iranian philology in the following aspects; see \tabref{tab:transcription_systems}.
\begin{table}
\begin{tabular}{lcc}
 \lsptoprule
IPA & Standard Kurdish\il{Kurdish} & MacKenzie (1966) \\
\midrule
t͡ʃ& ç & č \\
\raisebox{-0.2ex}{d͡ʒ}& c & \v{\j}\\
ʃ& ş & š\\ 
ʒ& j & ž\\
i & î & i \\
e& ê & e\\
ɛ∼æ& e & a\\
u& û &  ū \\
ɨ∼ɪ & i & ɪ \\
ɑ& a & ā \\
\lspbottomrule 
\end{tabular}
    \caption{Correspondences between Standard Kurdish\il{Kurdish} orthography and the transcription system used in Iranian philology}
    \label{tab:transcription_systems}
\end{table}

\section{Phonotactics}\label{phonotactics}

The phonotactics of vowels and consonants have been described separately for each phoneme in \sectref{DofV} and \sectref{Dofcons}. Here, I present the distribution of vowels and consonants in separate tables. The most frequent syllable structures are CV, CVC, and CVCC.

\subsection{Phoneme distribution}

\subsubsection{Phonotactics of Consonants}\label{phonotactic-conson}

There are some restrictions on the co-occurrence of consonants. All consonants can occur syllable-initially except /ŋ/. The liquid consonants /ɾ/ and /ɫ/ can occur syllable-initially but are excluded in word-initial position. In coda position, <ẍ> and /ʔ/ are excluded word-finally, see \tabref{tab:cons-phonotactics}.

\begin{table}[htp!]
\begin{tabular}{lcc}
 \lsptoprule
&\#-initial & \#-final  \\
\midrule
p & + & +  \\
b & + & +  \\
t & + & + \\
d & + & +  \\
k & + & +  \\
g & + & +  \\
q & + & +  \\
ʔ & + & –  \\
t͡ʃ & + & – \\
d͡ʒ & + & –  \\
m & + & +  \\
n & + & + \\
ŋ & – & +  \\
f & + & +  \\
s & + & + \\
 \lspbottomrule
\end{tabular}\hspace{1cm}
\begin{tabular}{lcc}
 \lsptoprule
&\#-initial & \#-final \\
\midrule
z & + & + \\
ʃ & + & + \\
ʒ & + & + \\
x & + & + \\
γ & + & – \\
ħ & + & + \\
ʕ & + & + \\
h & + & + \\
ɾ & – & + \\
r & + & + \\
l & + & + \\
ɫ & – & + \\
w & + & + \\
j & + & + \\
&&\\
 \lspbottomrule
\end{tabular}
    \caption{Phonotactics of consonants}
    \label{tab:cons-phonotactics}
\end{table}

Sequences of two consonants may appear across syllable boundaries, e.g., \textit{herben} [hɛɾ.ˈbɛn] `donkey keeper'. \tabref{tab:cons-comb}, inspired by \citet[]{Visser2022}, summarises the most frequent combinations resulting from such sequences. Compounding and derivational suffixes are also considered in these combinations. The most frequent consonants in the coda of the first syllable are <r>, <n>, <m>, <ş>, <s>,<w>,<y>, and <k>, respectively (also taking into account the combinations not appearing in \tabref{tab:cons-comb}). The most common consonants in the onset of the second syllable are <m>,<l>,<y>,<r>,<t>,<k>,<d>,<s>, <n>, and <b>.
\begin{table}
\begin{tabular}{clcccccccccc}
  \lsptoprule
 \diagbox{coda}{onset}& \textbf{b}&\textbf{t}&\textbf{d}&\textbf{k}&\textbf{m}&\textbf{n}&\textbf{s}&\textbf{r}&\textbf{l}&\textbf{w}&\textbf{y}\\
 \midrule
\textbf{t}&-&-&+&+&+&-&-&+&-&-&- \\
\textbf{k}&+&+&-&-&+&-&+&-&+&-&+ \\
\textbf{m}&-&+&+&+&-&+&+&+&+&-&- \\
\textbf{n}&+&+&+&+&-&+&+&-&-&+&+ \\
\textbf{s}&+&+&-&-&+&+&-&+&+&-&+ \\
\textbf{ş}&-&+&+&+&+&+&-&+&-&+&- \\
\textbf{r}&+&+&+&+&+&+&+&-&+&+&+ \\
\textbf{w}&-&+&-&+&+&-&+&+&-&-&+ \\
\textbf{y}&-&-&-&+&-&-&-&-&+&+&+ \\
 \lspbottomrule
\end{tabular}
    \caption{The most frequent combinations of consonants across syllable boundaries}
    \label{tab:cons-comb}
\end{table}

It is known that syllable boundaries are sensitive to the sonority hierarchy, so sonority must not rise across syllable boundaries \citep{gouskova_falling_2001}. This could explain the lack of sequences such as <kn>, <kr>, <nr>, <nl>, and so on in the text corpus across syllable boundaries.

In some cases, the onset after the syllable boundary undergoes deletion. The consonant in the coda is generally a coronal. The effect of this deletion is that disyllabic CVC.CVC breaks up into CV.CVC.
\TabPositions{2cm,4cm,6cm,8cm}
\ea
\textit{zînan}\tab [zi.ˈnɑn]\tab `prison'\tab < *\textit{zîndan} \\
\textit{mezeb}\tab [mɛ.ˈzæb]\tab `religion'\tab < *\textit{mezheb} \\
\textit{desûr}\tab [dɛ.ˈsur]\tab `order'\tab < *\textit{destûr} \\
\z

\subsubsection{Phonotactics of vowels}

Vowels cannot start a syllable. When appearing word-initially, they are preceded by the glottal stop /ʔ/. Excepting /ɑ/, other back vowels do not generally occur word-initially. The same is true for the central vowels /i/ and /u/, which do not appear initially in (especially monosyllabic) words. 
\begin{table}
\begin{tabular}{lcc}
\lspbottomrule
&\#-initial & \#-final \\ 
\midrule
i& +& + \\
e& +& + \\
ɛ& +& + \\
ɑ& +& + \\
u& –& + \\
o& –& + \\
ʊ& –& + \\
ɨ& –& – \\
\lspbottomrule
\end{tabular}\hspace{1cm}
\caption{Vowel distribution in monosyllabic words}
\label{V distribution}
\end{table}


\subsection{Syllable structure} \label{Syll structure}
A syllable in Tekht Hewramî\il{Hewramî!Tekht} consists minimally of a vowel and maximally of a vowel flanked by two consonants, yielding (C)(C)V(C)(C). Note that according to the automatic phonetic rule of glottal stop insertion, empty onsets are avoided, which would contradict the statement just made. The glottal stop insertion in this sense is a rule that shapes the actual surface form, not the underlying assumption of a syllable structure. Given that one vowel is allowed per syllable maximally, the number of vowels is equal to the number of syllables in a word. 

\largerpage
The syllable pattern containing only V is often avoided due to the condition that avoids empty onsets. Empty onsets require a glottal stop /ʔ/ to occupy the onset position of vowel-initial syllables. However, /ʔ/ is often deleted in casual speech. \\

\begin{tabular}{llllll}
V& \textit{êge}& [e.ˈgɛ]& ∼& [ʔe.ˈgɛ]& `here'
\end{tabular}\\

The most frequent syllabic patterns are listed below: \\

\begin{tabular}{lllllll}
VC& \textit{ađ}& [ɑɹˠ]& `he'& \textit{êş}& [eʃ]& `pain' \\
CV& \textit{wa}& [wɑ]& `wind'& \textit{ça}& [t͡ʃɑ]& `there' \\
CVC& \textit{pos}& [pos]& `skin'& \textit{herben}& [hæɾ.ˈbæn]& `donkey keeper' \\
\end{tabular}\\

The CCV pattern is the second most frequent. This syllabic pattern can be broken up into a disyllabic CVCV, through the insertion of an epenthetic <i>. The epenthetic <i> occurs often when the word is stressed and in careful speech. Thus, \textit{fre} `much, very' can be either monosyllabic [fɾɛ] or disyllabic [fɨ.ˈɾɛ] depending on context. \\

\begin{tabular}{llll}
CCV& \textit{fre}& [fɾɛ]& `much, very' \\
& \textit{bra}& [bɾɑ]& `brother' \\
& \textit{knaçê}& [knɑ.ˈt͡ʃe]& `girl' \\
\end{tabular}\\

The first segment in the CCV cluster is generally an obstruent, while the second is often a sonorant, i.e., a liquid, an approximant, or a nasal. There is also the CCVC structure, where the second consonant in the onset is an approximant.\\

\begin{tabular}{llll}
CCVC& \textit{çwar} & [t͡ʃwɑɾ]& `four' \\
& \textit{křêɫ}& [kreɫ]& `key' \\
& \textit{qřoɫ}& [qroɫ]& `tree hollow' \\
\end{tabular}
\\

CVCC and CCVCC are other permitted syllable patterns. The consonant cluster in the coda usually consists of a sibilant or an approximant as the first segment and an obstruent as the second segment. \\

\begin{tabular}{llll}
CVCC& \textit{merđ}& [mɛɾɹˠ]& `died' \\
& \textit{heşt}& [hæʃt̚]& `eight' \\
& \textit{pilt}& [pɨlt̚]&`short' \\
& \textit{quɫf}& [qʊɫf]&`locked' \\
& \textit{fewtno}& [fæwt.ˈno]&`he kills' \\
CCVCC& \textit{drext}& [dɾ\ae{}xt̚]& `tree' \\
\end{tabular}

\section{Stress position}\label{stress-position}
Stresss at the level of individual words and phrases is marked by an acute accent (v́). Lexical stress is only marked when the stress position is discussed. On the other hand, in presenting linguistic examples, the most prominent word within an intonation group is sometimes marked by a grave accent (v̀). In such cases, the word is said to take nuclear stress. This is only marked when the role of nuclear stress is discussed.  

Hewramî\il{Hewramî} is a language with phonemic stress\is{phonemic stress} placement: for the majority of verbs, stress is the only cue to distinguish between subjunctive and indicative verbs derived from the present tense (see \sectref{verb-stress}). In Hewramî\il{Hewramî}, there is variability in the positioning of stress in the words. In most cases, this is determined by the lexical category of the words and the interaction between syntax and discourse. The following sections review stress placement for the major word classes.

Lexical stress is associated with high intensity. \figref{fig:hesepraat} shows the stress pattern for the feminine noun \textit{h\'eşe} `bear'. The stressed syllable has a higher amplitude than the unstressed syllable.
\begin{figure}
    
    \includegraphics[width=.7\textwidth]{figures/hese.png}
    \caption{Spectogram, intensity and waveform for \textit{h\'eşe} `bear' (\textsc{f})}
    \label{fig:hesepraat}
\end{figure}

\subsection{Nouns}\label{nouns-stress}

There is some variation in stress placement in nouns and pronouns. Masculine\is{masculine} nouns follow what may be considered the general rule of word-final stress placement, which applies when nouns are pronounced in the citation form.
\TabPositions{2cm,4cm}
\ea
\textit{pîyá}\tab  `man' \\
\textit{sawró}\tab  `cow's dung' \\
\textit{gic\stackunder[-10pt]{\^{i}}{\'{}}}\tab  `shirt' \\
\textit{texte}́\tab  `wood' \\
\textit{av\stackunder[-10pt]{\^{i}}{\'{}}r}\tab  `fire' \\
\z

Similarly, the word-final stress pattern applies to a subset of feminine\is{feminine} nouns ending in \textit{-ê} and nouns ending in \textit{-a}:
\ea
\textit{sêt\stackunder[-10pt]{\^{e}}{\'{}}}\tab  `husband's sister' \\
\textit{dêy\stackunder[-10pt]{\^{e}}{\'{}}}\tab  `stepmother' \\
\textit{yag\stackunder[-10pt]{\^{e}}{\'{}}}\tab `place' \\
\textit{qeɫ\'a}\tab  `castle' \\
\textit{ʕeb\'a}\tab  `robe' \\
\z

However, the stress falls on the penultimate syllable in most feminine\is{feminine} nouns, including those ending in the unstressed vowels \textit{î}, \textit{e}, \textit{ê}. 
\ea
\textit{wínî}\tab  `blood' \\
\textit{séye}\tab  `shadow' \\
\textit{y\'erê}\tab  `three' \\
\z

Nominal formatives such as definite suffixes\is{definite suffix}, the singular masculine\is{masculine} oblique case\is{oblique case} suffix, and the infinitive\is{infinitive} receive word stress. 
\ea
\textit{degak\stackunder[-10pt]{\^{e}}{\'{}}}\tab  `the village' \\
\textit{kuř\stackunder[-10pt]{\^{i}}{\'{}}}\tab  `boy (\textsc{obl.m})' \\
\textit{berđéy}\tab `to take' \\
\z

The derivational suffixes \textit{-î} and \textit{-gerî} are also stress-bearing. 
\ea
\textit{wişkesaɫ\stackunder[-10pt]{\^{i}}{\'{}}}\tab `drought' \\
\textit{gewreger\stackunder[-10pt]{\^{i}}{\'{}}}\tab `grandeur' \\
\z

However, the direct plural\is{direct case} affix \textit{-ê} and the indefinite suffixes\is{indefinite suffix} \textit{-êw / -êwe / -ê} are not stress-bearing. The stress thus remains penultimate.
\ea
\textit{h\stackunder[-10pt]{\^{e}}{\'{}}zmê}\tab `firewood' \\
\textit{řóê}\tab `days' \\
\textit{k\'uřêw}\tab `a boy' \\
\textit{kinaç\stackunder[-10pt]{\^{e}}{\'{}}we}\tab `a girl' \\
\z

Nouns in the vocative\is{vocative} have the stress on the penultimate syllable:
\ea
\textit{tàte}\,\suppipe{}\tab `Father!' \\
\textit{èđa}\,\suppipe{}\tab `Mother!' \\
\textit{kinàçê}\,\suppipe{}\tab `Girl!' \\
\z

Pronominal clitics are not stress-bearing and thus do not cause a change in the stress pattern of nouns they attach to.
\ea
\textit{yané꞊ta}\tab `your house' \\
\textit{w\stackunder[-10pt]{\^{e}}{\'{}}꞊ma}\tab `ourselves' \\
\textit{yó꞊şa}\tab `one of them' \\
\z

Likewise, the additive\is{additive} clitic \textit{꞊îç} is not stress-bearing.
\ea
\textit{mín꞊îç}\tab `I too' 
\z

\subsection{Adjectives}

The stress placement pattern in adjectives is similar to that in nouns. Thus, their stress pattern is not completely predictable. Masculine\is{masculine} adjectives follow the basic syllable-final stress pattern. On the other hand, the stress is on the penultimate syllable in feminine\is{feminine} and plural\is{plural} adjectives. 
\ea
\textit{ʕaqíɫ}\tab  `wise (\textsc{m})' \\
\textit{ʕaqíɫe}\tab  `wise (\textsc{f})' \\
\textit{ʕaqíɫê}\tab  `wise (\textsc{pl})' \\
\z

The comparative\is{comparative} and superlative\is{superlative} suffixes \textit{-ter} and \textit{-terîn} receive word stress. 
\ea
\textit{zil-tér}\tab  `bigger' \\
\textit{zil-ter\stackunder[-10pt]{\^{i}}{\'{}}n}\tab  `biggest' \\
\z

\subsection{Adverbs}

Adverbials, like masculine\is{masculine} nouns, follow the syllable-final stress pattern.
\ea
\textit{êgé}\tab  `here' \\
\textit{îsé}\tab  `now' \\
\textit{it\'i} \tab  `any more, any way'\\
\z

In some adverbials, the stress falls on the penultimate syllable:
\ea
\textit{wélê}\tab  `but' \\
\textit{pérê}\tab  `the day before yesterday' \\
\textit{b\stackunder[-10pt]{\^{e}}{\'{}}cge}\tab  `except for' \\
\textit{péwkî}\tab `for the reason' \\
\z

\subsection{Verbs}\label{verb-stress} 

Most verbs derived from the present tense stem have no morphological distinction between the subjunctive and imperfective verb forms (see §\ref{section-vinfmorph}). Stress positioning is the only way to distinguish between identical subjunctive and imperfective moods for such verbs. In the imperfective, the stress is placed on the syllabic inflectional person suffix (\ref{ind.prs.stress1}). In the subjunctive, the stress is placed on the first syllable of the verb (\ref{sbjv.prs.stress1}). 
\ea \label{ind.prs.stress1}
\textit{yaw-\stackunder[-10pt]{\^{u}}{\'{}}}\tab  `I arrive' \\
\textit{yaw-\stackunder[-10pt]{\^{i}}{\'{}}}\tab  `you arrive' \\
\textit{yaw-ó}\tab  `he/she arrives' \\
\textit{yaw-m\stackunder[-10pt]{\^{e}}{\'{}}}\tab  `we arrive' \\
\textit{yaw-d\stackunder[-10pt]{\^{e}}{\'{}}}\tab  `you arrive' \\
\textit{yaw-á}\tab  `they arrive' \\
\z

\ea \label{sbjv.prs.stress1}
\textit{yáw-û}\tab  `that I arrive' \\
\textit{yáw-î}\tab  `that you arrive' \\
\textit{yáw-o}\tab  `that he/she arrives' \\
\textit{yáw-mê}\tab  `that we arrive' \\
\textit{yáw-dê}\tab  `that you arrive' \\
\textit{yáw-a}\tab  `that they arrive' \\
\z

\figref{fig:bero_ind} and \ref{fig:bero_sbjv} exhibit different stress patterns associated with the verbs \textit{ber\'o} `he takes' [JP.29] and \textit{b\'ero} `that he takes' [HB.11]. For each verb, the higher intensity peak highlights the stressed vowel.

\begin{figure}[!htp]
  
  \begin{subfigure}[b]{0.5\textwidth}
    \includegraphics[width=\textwidth]{figures/text_bero_he_takes.png}
    \caption{indicative `he takes'}
    \label{fig:bero_ind}
  \end{subfigure}\begin{subfigure}[b]{0.5\textwidth}
    \includegraphics[width=\textwidth]{figures/text_bero_that_he_takes.png}
    \caption{subjunctive `that he takes'}
    \label{fig:bero_sbjv}
  \end{subfigure}
  \caption{The stress position for the verb `he takes'}
\end{figure}

With disyllabic present stems in the subjunctive, the stress is on the first syllable. 
\ea
 \textit{g\'irîno} \tab  `that he/she boils' 
\z 

If the vowel of the inflectional person affix becomes a glide following a vowel-final verb stem, the stress is placed on the last syllable of the stem (see \ref{ex.give}). Note additionally that `give' is one of the few verbs that take the indicative prefix (see \sectref{indicative m-}).
\ea \label{ex.give}
\textit{mi-đé-y}\tab  `you (\textsc{sg}) give'
\z

In verb forms derived from the past stem, the stress is placed on the last syllable of the verb stem. The bound person markers used to inflect past stem verbs do not bear stress. These include inflectional person affixes, used to inflect past intransitive verbs, and clitic pronouns, used to inflect past transitive verbs (see \sectref{bound PMs}). Inflectional person affixes in past intransitive do not take stress because historically they originate from the enclitic copula, which underwent univerbation with the past participle verb.
\ea Past intransitive \\
\textit{yawá-nê}\tab  `I arrived' \\
\textit{yawá-y}\tab  `you arrived' \\
\textit{yawá-∅}\tab  `he/she arrived' \\
\textit{yawá-ymê}\tab  `we arrived' \\
\textit{yawá-ydê}\tab  `you arrived' \\
\textit{yaw\'ɛ}\tab  `they arrived' \\
\z

\ea Past transitive \\
\textit{wát꞊im}\tab  `I said' \\
\textit{wát꞊it}\tab  `you said' \\
\textit{wát꞊iş}\tab  `he/she said' \\
\textit{wát꞊ma}\tab  `we said' \\
\textit{wát꞊ta}\tab  `you said' \\
\textit{wát꞊şa}\tab  `they said' \\
\z

Similarly, in past imperfective forms, the stress falls on the penultimate syllable.
\ea
\textit{yaw-\stackunder[-10pt]{\^{e}}{\'{}}n-ê}\tab  `I was arriving' \\
\textit{yaw-\stackunder[-10pt]{\^{e}}{\'{}}n-mê}\tab  `we were arriving' \\
\z

The stress pattern of the imperative/subjective prefixes depends on the syllable structure of the verb. The formative is stressed with stems starting with a consonant cluster or if the stem consists solely of a consonant. However, the stress shifts to the stem in verb stems with CVC structure (see \sectref{sect:sbjv} for details).
\ea Imperative/subjunctive \\
\textit{bí-nvîs-e}\tab  `Write!' \\
\textit{bí-l-a}\tab  `that they go' \\
\textit{b-z\'an-mê}\tab  `that we know' \\
\textit{p-s\'an-û}\tab  `that I know' \\
\z

The negative prefixes are stress-bearing. 
\ea Negative/prohibitive \\
\textit{mé-don-e꞊m}\tab  `do not talk to me!' \\
\textit{né-ker-mê}\tab  `let's not do (it)' \\
\textit{né-yaw-ên-mê}\tab  `we were not arriving' \\
\z

As shown, stress retraction applies to the subjunctive form of the verb, even when the subjunctive prefix is absent. Historically, the loss of the subjunctive prefix led to shift of stress to the stem (see \citealt[]{karim_demorphologization_nodate}, \citealt[]{karim_imperfective_inreview}).
\ea
\textit{yáw-û}\tab  `(if) I arrive' \\
\textit{yáw-î}\tab  `(if) you arrive' \\
\textit{yáw-o}\tab  `(if) he/she arrives' \\
\textit{yáw-mê}\tab  `(if) we arrive' \\
\textit{yáw-dê}\tab  `(if) you arrive' \\
\textit{yáw-a}\tab  `(if) they arrive' \\
\z

The completive particle \textit{꞊we/꞊o} is not stress-bearing.
\ea
\textit{amá=we}\tab  `he came back' \\
\textit{ker-\stackunder[-10pt]{\^{u}}{\'{}}꞊we}\tab  `I will open' \\
\z

\subsection{Copula}

Present copulas are not stress-bearing. The stress is thus realised on the predicate. Note that feminine\is{feminine} and plural\is{plural} adjectives do not take syllable-final stress, as is the case with nominals (see \sectref{nouns-stress}).
\ea
\textit{x\'as꞊na}\tab  `I (\textsc{m}) am well' \\
\textit{x\'ae꞊na}\tab  `I (\textsc{f}) am well' \\
\textit{x\'as꞊a}\tab  `he is well' \\
\textit{x\'asê꞊nmê}\tab  `we are well' \\
\z

Similarly, past copula suffixes do not receive word stress.
\TabPositions{2.5cm,4cm}
\ea
\textit{xas b-\stackunder[-10pt]{\^{e}}{\'{}}n-ê}\tab  `I (\textsc{m}) was well' \\
\textit{xase b-\stackunder[-10pt]{\^{e}}{\'{}}n-ê}\tab  `I (\textsc{f}) was well' \\
\textit{xasê b-ên-mê}\tab  `we were well' \\
\z

\section{Major morphophonemic processes}

\subsection{h-initial insertion} \label{h-initial}
As discussed in \sectref{Syll structure}, vowel-initial words are preceded by a glottal stop [ʔ] due to the restriction against onsetless syllables. Sometimes, word-initial [ʔ] is realised as a glottal fricative /h/. A survey of the text corpus reveals that /h/ is initially inserted primarily before the front vowels <e> and <ê>, both in the native lexicon and in loanwords\is{loanwords}.
\TabPositions{2cm,5cm,7cm}
\ea Before /e/ \\
\textit{hetîm}\tab  `orphan'\tab  cf. Tr.\il{Turkish} \textit{etim} \\
\textit{heɫbet}\tab  `of course'\tab  cf. Ar. \textit{albatte} \\
\textit{hesere}\tab  `mule'\tab  cf. NK.\il{Kurdish!Northern} \textit{êstir} \\
\textit{hesare}\tab  `star'\tab  cf. CK. \il{Kurdish!Central}\textit{estêre} \\
\textit{henar}\tab  `pomegranate'\tab  cf. P.\il{Persian} \textit{anār} \\
\textit{heke}\tab  `if, well'\tab  cf. P.\il{Persian} \textit{agar} `if' \\
\z

In some cases, /h/ alternates freely with /ʔ/:
\TabPositions{2cm,3cm,5cm,7cm}
\ea
\textit{hesp}\tab  $\sim$\tab  \textit{esb}\tab  `horse'\tab  cf. P.\il{Persian} \textit{asb} \\
\textit{hême}\tab  $\sim$\tab  \textit{ême}\tab  `we'\tab cf. CK.\il{Kurdish!Central} \textit{ême} \\
\z

Less commonly, /h/ is added before words starting with vowels other than <e>:
\ea
\textit{hêɫeke}\tab  `fine sieve'\tab  cf. Tr.\il{Turkish} \textit{elek} \\
\textit{harđî}\tab  `flour'\tab  cf. P.\il{Persian} \textit{ārd} \\
\textit{hiɫoşe}\tab  `sour \textit{}plum'\tab  cf. P.\il{Persian} \textit{ālûçe} \\
\z

\subsection{Inversion in the voicing of pharyngeals} \label{inversion-phryng}

A pharyngeal phoneme is sometimes borrowed, but its voicing is inverted in Hewramî\il{Hewramî}.
\TabPositions{2cm,4cm,6cm,8cm}
\ea
\textit{seʕbe}\tab  [ˈsæʕ.bɛ]\tab  `morning'\tab  cf. Ar.\il{Arabic} \textit{sˤabāħ} \\
\textit{deħfe}\tab  [ˈdæħ.fɛ]\tab  `expulsion'\tab  cf. Ar.\il{Arabic} \textit{dafʕ} \\
\textit{cuħme}\tab  [d͡ʒʊħ.ˈmɛ]\tab  `Friday'\tab  cf. Ar.\il{Arabic} \textit{jomʕa} \\
\textit{ħegaɫ}\tab  [ħɛ.ˈɡɑɫ]\tab  `scarf'\tab  cf. Ar.\il{Arabic} \textit{ʕiqāl} \\
\z

\subsection{Glide insertion}\label{glide-ins}

An epenthetic glide is inserted between syllable-initial vowel sequences to avoid hiatus. /w/ is used following [u]; elsewhere /j/ is used.
\ea
/piɑ/\tab  > [pi.ˈjɑ]\tab  `man' \\
/guinɛ/\tab  > [ɡu.wi.ˈnɛ]\tab  `sickly' \\
/dɛ.gɑ.ɑ/\tab  > [dɛ.ɡɑ.ˈjɑ]\tab  `villages (\textsc{pl.obl})' \\
\z

\subsection{Anaptyctic\is{anaptyctic} vowel insertion} \label{anap}
 In Hewramî\il{Hewramî} and other regional languages, the vowel generally employed to break up illegal consonant clusters is <i> (IPA [ɨ]) or <u> (IPA [ʊ]) depending on the quality of adjacent consonants. These vowels may break up consonant clusters in the onset of the syllable.
 \ea
\textit{bira}  \tab  < \textit{bra} \tab  `brother' \\
\textit{şuwane}\tab  < \textit{şwane}\tab  `shepherd' \\
\z

 The anaptyctic\is{anaptyctic} \textit{i} and \textit{u} may also break up consonant clusters in the coda. The clusters that allow breaking up by <i> include rm, ɫm, sk, rs, řk, wr. As can be seen, the first segment in the cluster is a sibilant or a liquid, whereas the second segment is usually an occlusive (nasals and obstruents) and less frequently a rhotic. 
\ea
\textit{zoɫim}\tab  < \textit{zoɫm}\tab  `tyranny' \\
\textit{gerim}\tab  < \textit{garm}\tab  `warm' \\
\textit{nasik}\tab  < \textit{nask}\tab  `thin (stick)' \\
\textit{biřik}\tab  < \textit{biřk}\tab  `thick (stick)' \\
\textit{quris}\tab  < \textit{qurs}\tab  `heavy' \\
\textit{hewir}\tab  < \textit{hewr}\tab  `cloud' \\
\z

The clusters that allow breaking up by \textit{u} appear to be less common. The first segment in such clusters is often a back consonant, e.g., /k/, /q/.
\ea
\textit{şukur}\tab  < \textit{şukr} \tab  `praise' \\
\textit{nuquɫ}\tab  < \textit{nuqɫ}\tab  `candy' \\
\z

\subsection{Metathesis\is{metathesis}} \label{sect: metathesis}

Many disyllabic loans from Arabic \il{Arabic} containing a pharyngeal consonant in the onset of the second syllable or in the code of the second syllable are subject to metathesis\is{metathesis} in Hewramî\il{Hewramî}. These loans are typically treated as feminine\is{feminine}, with an unstressed \textit{-e} added if the loan does not already end in a \textit{-e}. Therefore, in the loan word the pharyngeal phoneme would occur in the onset (or the coda of) the second syllable, but when metathesis\is{metathesis} is applied, it ends up as the coda of the first syllable. 
\TabPositions{2cm,4cm,7cm,9cm}
\ea
\textit{seʕbe}\tab  [ˈsæʕ.bɛ]\tab  cf. Ar.\il{Arabic} \textit{sˤabāħ}\tab  `morning' \\
\textit{weʕze}\tab  [ˈwæʕ.zɛ]\tab  cf. Ar.\il{Arabic} \textit{wazʕ}\tab  `situation' \\
\textit{cuʕme}\tab  [d͡ʒʊʕ.ˈmɛ]\tab  cf. Ar.\il{Arabic} \textit{jomʕa}\tab  `Friday' \\
\textit{deħfe}\tab  [ˈdæħ.fɛ]\tab  cf. Ar.\il{Arabic} \textit{dafʕ}\tab  `expulsion' \\
\textit{meʕsûd}\tab  [mæʕ.ˈsud]\tab  cf. Ar.\il{Arabic} \textit{masʕûd}\tab  `Masoud' \\
\z 

The metathesis\is{metathesis} occurs to avoid non-permitted consonant clusters across syllable boundaries (see \sectref{phonotactic-conson} for consonants phonotactics). The motivation behind the metathesis\is{metathesis} here is that following the sonority hierarchy, the coda of the first syllable should be more sonorant than the onset of the second syllable. In other words, sonority must not rise across syllable boundaries \citep{gouskova_falling_2001}. The non-permitted patterns in \tabref{tab:metath} all result in a rise of sonority in the syllable boundary. Therefore, the metathesis\is{metathesis} occurs to resolve the issue.

\begin{table}
    \centering
    \begin{tabular}{ccl}
    \lsptoprule
permitted cluster & non-permitted cluster & \multirow{2}{*}{Gloss} \\
across σ-boundaries & across σ-boundaries \\ 
\midrule
\textit{seʕbe} [ˈsæʕ.bɛ]& *[ˈsæb.ʕɛ]& `morning' \\
\textit{weʕze} [ˈwæʕ.zɛ]& *[wæz.ʕɛ]& `situation' \\
\textit{cuʕme} [d͡ʒʊʕ.ˈmɛ]& *[d͡ʒʊm.ʕɛ]& `Friday' \\
\textit{deħfe} [ˈdæħ.fɛ]& *[dæf.ħɛ]& `expulsion' \\
\textit{meʕsûd} [mæʕ.ˈsud]& *[mæs.ˈʕud]& `Masoud' \\
\lspbottomrule
    \end{tabular}
    \caption{Non-permitted consonant sequences across syllable boundaries leading to metathesis}
    \label{tab:metath}
\end{table}

In the following pairs, onset metathesis\is{onset metathesis} is attested across multiple CV syllable structures; the onset of the second syllable is metathesised with the onset of the third syllable.
\TabPositions{2cm,4cm,6cm,7cm,9cm}
\ea
\textit{heteqî}\tab  [hɛ.tɛ.ˈqi]\tab  `truly'\tab  vs.\tab  \textit{heqetî}\tab  [hɛ.qɛ.ˈti] \\
\textit{kiřêɫe}\tab  [kɨ.re.ˈɫæ]\tab  `key'\tab  vs.\tab  \textit{kɫêře}\tab  [kɨ.ɫe.ˈrɛ] \\
\z

Coda metathesis\is{coda metathesis} is only attested with CVCV structures:	
\ea
\textit{nûrî}\tab  [nu.ˈri]\tab  `force'\tab  vs.\tab  \textit{nîrû}\tab  [ni.ˈru] \\
\z

Less commonly, metathesis occurs within the consonant clusters in the coda.
\ea
\textit{tiɫf}\tab  [tɨɫf]\tab  `child'\tab  vs.\tab  \textit{tifɫ}\tab  [tɨfɫ] \\
\z

\subsection{Vowel hiatus}\label{vowel hiatus}

Vowel hiatus is generally avoided across or within word boundaries. There are several strategies to resolve hiatus. The first is that the second vowel is glided, yielding a rising diphthong, e.g., /oy/, /ay/:
\TabPositions{2cm,5cm,7cm}
\ea
\textit{toyç}\tab  `you too'\tab  < [ˈtʰo.it͡ʃ] \\
\textit{pîyay}\tab  `man (\textsc{m.sg.obl})'\tab  < [pʰi.ˈɑi] \\
\z

The second strategy is to add an epenthetic glide /j/ or /w/ between vowels (see \sectref{glide-ins}):
\TabPositions{2cm,4cm,7cm}
\ea
\textit{gûwînê}\tab  /ɡu.i.nɛ/\tab  > [ɡu.wiˈnɛ]\tab  `sickly' \\
\textit{degaya}\tab  /dɛ.ɡɑ.ɑ/\tab  > [dɛ.ɡɑ.ˈjɑ]\tab  `villages (\textsc{pl.obl})' \\
\z

Another strategy is to omit the first vowel, which is generally unstressed:
\TabPositions{2cm,6cm,8cm}
\ea
\textit{kîseɫê}\tab  `tortoise (\textsc{f.sg.obl})'\tab  < \textit{kîséɫî + -ê} (\textsc{f.sg.obl}) \\
\textit{miđo}\tab  `he/she gives'\tab  < \textit{mi-đe + -o} \\
\textit{jena}\tab  `women (\textsc{pl.obl})'\tab  < \textit{jénî + -a} \\
\textit{jenû}\tab  `wife of'\tab  < \textit{jenî + -û} \\
\z

Yet another strategy is to coalesce the two vowels into one:
\ea
\textit{pîy{ɛ}}\tab  `men (\textsc{pl.dir})\tab  < \textit{pîya + -ê} \\
\textit{mê} \tab  `he/she comes' \tab  < \textit{m-e-o}  \\
\z


\subsection{Assimilation}\label{assimilation}

Assimilation\is{assimilation} is a process in which a pair of adjacent segments become similar. There are two kinds of assimilation\is{assimilation}: progressive\is{progressive assimilation} and regressive\is{regressive assimilation}. These assimilation\is{assimilation} processes have the effect of creating geminate consonants which are otherwise not attested in the language. Total progressive assimilation\is{progressive assimilation} is seen in the following words, where /d/ fully assimilates to the preceding nasal sound:
\ea
\textit{çinne}\tab  `how much'\tab  < *\textit{çinde}\tab  cf. CK.\il{Kurdish!Central} \textit{çende} \\
\textit{inne}\tab  `this much'\tab  < *\textit{inde}\tab  cf. CK.\il{Kurdish!Central} \textit{ewende} \\
\z

Regressive assimilation\is{regressive assimilation} takes place in the following example, where the rhotic /ř/ assimilates to the following liquid.
\ea
\textit{kulle}\tab  `small boy'\tab  < *\textit{kuřle}
\z
























