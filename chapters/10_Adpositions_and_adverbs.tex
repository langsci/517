\chapter{Adpositions and adverbs}
\begin{sloppypar} 
\section{Adpositions}
 The category of adpositions includes prepositions, postpositions, and circumpositions. This means that Hewramî\il{Hewramî} has a mixed adpositional typology. This trait is a common feature of Northwestern Iranian languages such as Vafsi\il{Vafsi} and most varieties of Kurdish. It is assumed to be a reflection of their geographical distribution between OV languages, e.g., Armenian\il{Armenian}, Turkish\il{Turkish}, Indic\il{Indic}, and VO languages, e.g., Arabic\il{Arabic}, Aramaic\il{Aramaic} \citep[cf.][18]{stilo_circumpositions_2009}{}.
 
 Additionally, Tekht Hewramî\il{Hewramî!Tekht} features formatives of (mainly) prepositional origin, commonly called ``absolute adpositions'' within Iranian linguistics. As discussed in \S\ref{sect:absolute_prep}, one way of analysing these formatives is in terms of applicative morphology.
 
\subsection{Prepositions}
The majority of adpositions belong to the category of prepositions. Prepositions have both grammatical and spatial-temporal functions. They usually trigger oblique marking\is{oblique case} on the nominal that follows them, though oblique marking\is{oblique case} is contingent on certain factors, e.g., animacy and pre- and post-verbal positioning of the prepositional phrase. By way of example, in (\ref{ex.prep1})--(\ref{ex.prep2}), the preposition \textit{be} marks the instrument. However, only the animate complement is marked in the oblique case\is{oblique case}. Investigating factors influencing differential oblique marking\is{differential oblique marking} remains a topic for feature research.

\ea
\textit{miqdarêş ađî be sey misefay ardeniş,} \\ 
\gll miqdarê=ş ađî \textbf{be} \textbf{sey} \textbf{misefa-î} arde=n=iş \\ 
 some\textsc{=3sg:PSR} \textsc{3sg.obl.m} by \textsc{pn} \textsc{pn-m.sg.obl} bring\textsc{.pst.ptcp.m=cop.3sg.m:O=3sg:A} \\  
\glt `Some of the money which he had made Say Mustafa take. [Lit. He brought some of it \textbf{by Say Mustafa}.]' \hfill[JP.113] \label{ex.prep1}
\z 


\ea
\textit{be kune awîşa ardêne.} \\ 
\gll \textbf{be} \textbf{kune} awî=şa ardê=ne \\ 
 by clay\_pot\textsc{.m} water\textsc{.f.sg.dir=3pl:A} bring\textsc{.pst.ptcp.f=cop.3sg.f:O} \\ 
\glt `They used to fetch water \textbf{using clay pots}.' \hfill[JE.16] \label{ex.prep2}
\z 

Prepositions fall into two sets: basic prepositions, and locational nouns\is{locational nouns}. The locational nouns\is{locational nouns} can be used independently or in combination with prepositions. These two sets are morphologically distinguished. Simple prepositions occur with nominals through simple juxtaposition. Locational nouns\is{locational nouns} are linked to the nominal via a linker dubbed ``genitive ezafe\is{genitive ezafe}'' (\S\ref{sect:gen-ez}).

\subsubsection{Basic prepositions}\label{sect:basicprep}
The following is the list of basic prepositions, most of which are monomorphemic.

\TabPositions{2.5cm}
\ea
\textit{be}\tab `to, by, with, in' \\
\textit{ce}\tab `at, from' \\
\textit{pey}\tab `for, to' \\
\textit{çenî/çenû} \tab `with' \\
\textit{bê}\tab `without' \\ 
\textit{ta}\tab `until' \\
\textit{e=}\tab `in, from' \\
\textit{ne}\tab `into' (occurring rarely, see \S\ref{sect:postpositions}) \\
\z

Among these, \textit{bê} and \textit{ta} occur only with non-bound complements, i.e., with nouns and independent pronouns, which are not prosodically deficient. On the other hand, \textit{be} and \textit{ce} are unique in that they exhibit allomorphy when attaching to demonstratives and clitic pronouns. When combined with demonstratives, the initial consonants are devoiced, yielding the forms \textit{ç=} and \textit{p=}, e.g., \textit{ç=î mentêqe-î} [in=\textsc{dem.prox} region\textsc{.m-sg.obl}] `in this region' [PM.5]; \textit{p=a mêman mizgî} [to\textsc{=dem.dist} guest\textsc{.m} mosque\textsc{.m}] `to the guest in the mosque' [JH.40]. 
The resulting forms sometimes grammaticalise into spatial adverbs\is{spatial adverbs}: \textit{ça} `there'; \textit{çê} `here'. 
When combined with clitic pronouns \textit{be} and \textit{ce} are realised as \textit{pene} and \textit{çene}, respectively, e.g., \textit{pene=m} `to me'; \textit{çene=ş} `in it'. In traditional Iranian philology, these allomorphs are referred to as ``absolute prepositions'' (see \citealt[55]{mackenzie_dialect_1966} for Hewramî Luhon\il{Hewramî!Luhon}, \citealt[601]{mccarus_kurdish_2009} for Central Kurdish\il{Kurdish!Central}, and \citealt[]{lazard_dialecte_1992}{} for Laki\il{Laki}). \textit{çenî/çenû} `with' exhibits a slightly different pattern: its ``absolute'' form is \textit{çene}. \tabref{tab:be_ce_allomorp} summarises the allomorphs of \textit{be}, \textit{ce}, and \textit{çenî/çenû}.

\begin{table}
    \begin{tabular}{llll}
\lsptoprule
&with nouns& with demonstratives& with clitic pronouns \\
\midrule
be & \textit{be}& \textit{p=}& \textit{pene} \\
ce & \textit{ce} & \textit{ç=}& \textit{çene} \\
\textit{çenî/çenû} & \textit{çenî/çenû}& \textit{çenî/çenû} & \textit{çene} \\
\lspbottomrule
    \end{tabular}
    \caption{The allomorphs of \textit{be} and \textit{ce} and \textit{çenî}}
    \label{tab:be_ce_allomorp}
\end{table}

Basic prepositions occur before nominals through simple juxtaposition. An exception is \textit{çenû} `with', which is a combination of \textit{çenî} and the genitive ezafe\is{genitive ezafe} form \textit{-û}. The form \textit{çenû} is now grammaticalised as the citation form for most speakers. The original form \textit{çenî} seems to be less frequent. 

\subsubsubsection{\textit{be} `to, by, with, in'}
This preposition has a range of meanings. It typically marks the following roles: recipient\is{recipient} (\ref{ex.be.prp.1}), addressee\is{addressee} (\ref{ex.be.prp.2}), (metaphorical) goal\is{goal} (\ref{ex.be.prp.3}), comitative (\ref{ex.be.prp.4}), beneficiary (\ref{ex.be.prp.5}), passive agent\is{passive agent} (\ref{ex.be.prp.6}), and manner (\ref{ex.be.prp.7}).\footnote{See \citet{mohammadirad_instrumental_2017,mohammadirad_functions_2018} for the study of the multifunctionality of Hewramî instrumental and dative markers within the context of Iranian languages. The list of functionalities of adpositions provided here is more comprehensive.} 


\ea
\textit{qurban to desûrêt da be sey misefay.} \\ 
\gll qurban to desûr-ê=t da-{\O} \textbf{be} \textbf{sey} \textbf{misefa-î} \\ 
 sir.\textsc{voc} \textsc{2sg} order\textsc{.m-indf=2sg:A} give\textsc{.pst-3sg.m:O} to \textsc{pn} \textsc{pn-m.sg.obl} \\ 
\glt `Sir, you ordered Say Mustafa [lit. gave an order to].' \hfill[JP.126] \label{ex.be.prp.1}
\z 


\ea
\textit{maço be xanî.} \\ 
\gll m-aç-o \textbf{be} \textbf{xan-î} \\ 
 \textsc{ind-}say\textsc{.prs}\textsc{-3sg:A} to chief\textsc{.m-sg.obl} \\ 
\glt `He said \textbf{to the chief}.' \hfill[KŞ.97] \label{ex.be.prp.2}
\z 


\ea
\textit{xeber yawo be baba xwađaw ħemey ẍeybî.} \\ 
\gll xeber yaw-o \textbf{be} \textbf{baba} \textbf{xwađa=û} \textbf{ħeme-î} \textbf{ẍeybî} \\ 
 news\textsc{.m.sg.dir} arrive\textsc{.prs.ind-3sg:S} to \textsc{pn} \textsc{pn}{=and} \textsc{pn-ez.attr} invisible \\ 
\glt `The news reached Baba Khwada and Hama the Invisible.' \hfill[BP.46] \label{ex.be.prp.3}
\z 


\ea
\textit{êtir çayşa kerden be nanekey.} \\ 
\gll êtir çay=şa kerde=n \textbf{be} \textbf{nan-ekey} \\ 
\textsc{disc.ptcl} tea\textsc{.m.sg.dir=3pl:A} do\textsc{.pst.ptcp.m=cop.3sg.m:O} with bread\textsc{.m-def.m.sg.obl} \\ 
\glt `Then, they made tea [to be served] with the food.' \hfill[JE.38] \label{ex.be.prp.4}
\z 


\ea
\textit{xizmet be merđimî kero.} \\ 
\gll xizmet \textbf{be} \textbf{merđim-î} ker-o \\ 
 service\textsc{.m} to people\textsc{.m-sg.obl} \textsc{sbjv-}do\textsc{.prs-3sg:A} \\ 
\glt `[And he] be of service \textbf{to people}.' \hfill[JP.87] \label{ex.be.prp.5}
\z 


\ea
\textit{sewzê xel kiryan be sahêbîçiş.} \\
\gll sewzê xel kir-ya=n \textbf{be} \textbf{sahêb-î=ç=iş} \\
crop grain do.\textsc{prs-pass=cop.3sg.m:S} by owner-\textsc{obl.m=add=3sg:PSR} \\
\glt `The crop has been piled up [lit. turned into corn] by its owner too.' \\ \hfill[HR.16]\label{ex.be.prp.6}
\z 


\ea
\textit{be adizî luwan.} \\ 
\gll \textbf{be} \textbf{adizî} luwa=n \\ 
in desperation go\textsc{.pst.ptcp.m=cop.3sg.m:S} \\ 
\glt `He had left [town] in desperation.' \hfill[JH.62] \label{ex.be.prp.7}
\z 
 
\textit{be} may also mark the nominal argument of inchoative\is{inchoative} verbs `become, turn into', indicating a change of state.

\ea
\textit{bîyen be feqî.} \\ 
\gll bîye=n \textbf{be} \textbf{feqî} \\ 
 be\textsc{.pst.ptcp.m=cop.3sg.m:S} \textsc{adp} theologian\textsc{.m} \\ 
\glt `[He] became a theologian.' \hfill[ZP.15]
\z 


\ea
\textit{miniş kerdena be wekêɫ.} \\ 
\gll min=iş kerde=na \textbf{be} \textbf{wekêɫ} \\ 
\textsc{1sg=3sg:A} do\textsc{.pst.ptcp.m=cop.1sg:O} \textsc{adp} advocate\textsc{.m} \\ 
\glt `He has put me in charge. [Lit. made me advocate.]' \hfill[ZP.89]
\z 

 It should be mentioned, though, that the use of the preposition is optional here, as the nominal argument of inchoative\is{inchoative} verbs may be realised without any flagging: 


\ea
\textit{min, taze padşay kerdena wekêɫ.}\\ 
\gll min taze padşa-î kerde=na \textbf{wekêɫ}\\ 
\textsc{1sg} anyway king\textsc{.m-sg.obl} do\textsc{.pst.ptcp.m=cop.1sg:O} advocate\textsc{.m}\\ 
\glt `As for me, well, the king has put me in charge.' [Lit. The king has made me advocate.] \hfill [JP.206]
\z 


\ea
\textit{bawe yo. keraşawe yo}\\ 
\gll b-a=we \textbf{yo} ker-a=şa=we \textbf{yo} \\ 
 become\textsc{.prs.ind-3pl:S=compl} one\textsc{.m} do\textsc{.prs.ind-3pl:A=3pl:O=compl} one\textsc{.m} \\ 
\glt `They became one. They were put in the same [cell] [lit. they made them one].' \hfill[BP.141]
\z 

\subsubsubsection{\textit{pey} `for, to'}
This preposition generally triggers oblique case\is{oblique case} marking on its complement. It typically expresses a beneficiary argument and a goal\is{goal} argument of a verb of movement, whether human or inanimate.

\ea
\textit{hêɫeřûwenîşa kerde pey sahêbû hereyû pey herbenew herey.} \\ 
\gll hêɫeřûwenî=şa kerd-e \textbf{pey} \textbf{sahêb-û} \textbf{her-e-î}=û \textbf{pey} \textbf{herbene-û} \textbf{her-e-î} \\ 
 fried\_egg\textsc{f.sg.dir=3pl:A} do\textsc{.pst-3sg.f:O} for owner\textsc{.m-ez.gen} donkey\textsc{.m-def-m.sg.obl}{=and} for donkey\_keeper\textsc{.m-ez.gen} donkey\textsc{.m-def-m.sg.obl} \\ 
\glt `They (the host’s family) cooked fried eggs for the donkey's owner [i.e.] for the donkey keeper.' \hfill[HB.56]
\z 


\ea
\textit{luwewe pey yaney!} \\ 
\gll lu-e=we \textbf{pey} \textbf{yane-î} \\ 
go.\textsc{prs.imp-2sg:S=post} to house\textsc{.m-sg.obl} \\ 
\glt `[Now] go back home!' \hfill[JH.118]
\z 


\ea
\textit{î kinaçê narîwe êtir pey min.} \\ 
\gll î kinaçê n(e)-ar-î=we êtir \textbf{pey} \textbf{min} \\ 
 \textsc{dem.prox} girl\textsc{.f} \textsc{neg.sbjv-}bring\textsc{.prs-2sg:A=compl} any\_more to \textsc{1sg} \\ 
\glt `[Please] do not bring this daughter [of mine] back to me.' \hfill[ZP.48]
\z 

The nominal complement of \textit{pey} occurs regularly in the oblique case\is{oblique case}. In rare cases, the complement appears in the direct case\is{direct case}.

\ea
\textit{kîyasenşa pey beẍa.} \\ 
\gll kîyase=n=şa \textbf{pey} \textbf{beẍa} \\ 
 send\textsc{.pst.ptcp.m=cop.3sg.m:O=3pl:A} to \textsc{pn.m} \\
\glt `[Then] they sent him to Baghdad.' \hfill[ZP.14]
\z 

\subsubsubsection{\textit{çenî} `with'}
This preposition occurs in two forms in the corpus: \textit{çenî} and \textit{çenû}. The latter results from the unimorphation of \textit{çenî} with the genitive ezafe\is{genitive ezafe} \textit{-û}. \textit{çenî} expresses a comitative argument. The complement is generally marked in the oblique case\is{oblique case}.

\ea
\textit{î masî bere deş pa mêman mizgî çenîyû nanî.} \\ 
\gll î mas-î b\'er-e d\'e-(e)=ş p=a mêman mizgî \textbf{çenî-û} \textbf{nan-î} \\ 
 \textsc{dem.prox} yoghurt\textsc{.m-sg.obl} take\textsc{.prs.imp-2sg:A} give\textsc{.prs.imp-2sg:A=3sg:O} to\textsc{=dem.dist} guest\textsc{.m} mosque\textsc{.m} with\textsc{-ez.gen} bread\textsc{.m-sg.obl} \\ 
\glt `Take this yoghurt [and] together with [some] bread [and] give it to the guest in the mosque.’' \hfill[JH.40]
\z 


\ea
\textit{nîşare çenû lalowekeyş.} \\ 
\gll nîş-a=re \textbf{çenû} \textbf{lalo-ekey=ş} \\ 
 sit\textsc{.prs.ind-3pl:S=povb} with\textsc{.ez.gen} maternal\_uncle\textsc{.m-def.m.sg.obl=3sg:PSR} \\ 
\glt `She sat [on the stone] with her uncle.' \hfill[ZP.56]
\z 

\subsubsubsection{\textit{ce} `at, from'}
This preposition takes an oblique argument\is{oblique argument} in its spatial sense, `in, from' or in the sense of `of'. \textit{ce} is also used to express the standard of comparison; see (\ref{ex.ce.prp1}).

\newpage
\ea
\textit{maça neferê înarê ce menteqew hewramanatî.} \\ 
\gll m-aç-a nefer-ê îna-∅=rê \textbf{ce} \textbf{menteqe-û} \textbf{hewraman-at-î}\\ 
 \textsc{ind-}say\textsc{.prs-3pl:A} person\textsc{.m-indf} \textsc{loc.deic.cop-3sg.m:S=povb} in region\textsc{-ez.gen} \textsc{pn-pl-m.sg.obl} \\ 
\glt `It was said [to him in the dream] that there is a person in the Hewraman region.' \hfill[ZP.31]
\z 


\ea
\textit{ane ce zemînî pêse waçî xway ketê pey kîyasen.} \\ 
\gll ane \textbf{ce} \textbf{zemîn-î} pêse w\'aç-î xwa-î ket-ê pey kîyase=n \\ 
 \textsc{dem.dist.m.3sg.dir} from earth\textsc{.m-sg.obl} as\_if say\textsc{.prs.sbjv-2sg:A} God\textsc{.m-sg.obl} bed\textsc{-indf} to send\textsc{.pst.ptcp.m=cop.3sg.m:R} \\ 
\glt `[He had levitated] this much from the ground, as if God had sent him a bed.' \hfill[JP.69]
\z  


\ea
\textit{ce taranî kuweyt dewɫetmenter bê.} \\ 
\gll \textbf{ce} \textbf{taran-î} kuweyt dewɫetmen-ter b-ê \\ 
 from \textsc{pn-m.sg.obl} \textsc{pn} rich\textsc{-cmpr} be\textsc{.prs-aug.3sg:S} \\  
\glt `Kuwait was more affluent than Tehran.' \hfill[JM.44] \label{ex.ce.prp1}
\z 

\subsubsubsection{\textit{bê} `without'} 
This preposition marks peripheral arguments which are not in the verb's argument structure. The preposition has a privative sense. It usually triggers oblique marking\is{oblique case} on the nominal complement.

\ea
\textit{bê a bega ême netawanma ħîç kermê.} \\ 
\gll \textbf{bê} \textbf{a} \textbf{beg-a} ême ne-tawa=n=ma ħîç k\'er-mê \\ 
without \textsc{dem.dist} chief\textsc{.m-pl.obl} \textsc{1pl} \textsc{neg-}can\textsc{.pst.ptcp.m=cop.3sg.m:O=1pl:A} nothing do\textsc{.prs.sbjv-1pl:A} \\ 
\glt `Without those chiefs, we were not able to do anything.' \hfill[RE.64]
\z 


\ea
\textit{mîyo bê heyasîş mekiryo.} \\ 
\gll mi-đy(e)-o \textbf{bê} \textbf{heyas-î}=ş {} me-kiry(e)-o \\ 
 \textsc{ind-}look.\textsc{prs-3sg:S} without \textsc{pn-m.sg.obl=3sg:NC} {} \textsc{neg.ind}-do.\textsc{do.prs-pass-3sg:S}\\ 
\glt `He realised that he cannot cope without Hayas.' \hfill[HS.19]
\z 

\subsubsubsection{\textit{ta} `until'} 
This preposition marks oblique arguments\is{oblique arguments} in a spatial-temporal sense, `till, until' and `duration of time'. Unlike other prepositions, the complement of \textit{ta} overwhelmingly occurs in the direct case\is{direct case}.

\ea
\textit{menawe ta zimsanê.} \\ 
\gll men-a=we \textbf{ta} \textbf{zimsan-ê} \\ 
remain\textsc{.prs.ind-3pl:S=compl} until winter\textsc{.m-indf} \\ 
\glt `They stayed until winter.' \hfill[BP.33]
\z


\ea
\textit{meselen eçê luwan ta milemarfa wêşû xanewadeş.} \\ 
\gll meselen e=çê luwa=n \textbf{ta} \textbf{milemarfa} wê=ş=û xanewade=ş \\ 
 for\_example from=here go\textsc{.pst.ptcp.m=cop.3sg.m:S} until \textsc{pn} \textsc{refl=3sg:PSR}=and family\textsc{.m=3sg:PSR} \\ 
\glt `Let's say they might have gone together with their family as far as Milamarfa.' \hfill[BP.167]
\z 


\ea
\textit{ta yerê saɫê eça keşo ađ guzeran kero la birakeyşo.} \\ 
\gll \textbf{ta} \textbf{yerê} \textbf{saɫ(e)-ê} e=ça keş=o ađ guzeran ker-o la bira-(e)key=ş=o \\ 
 until three year\textsc{.f-pl.dir} in=there mountain\textsc{.m=post} \textsc{3sg.m.dir} subsistence\textsc{.m} do\textsc{.prs.ind-3sg:A} with brother\textsc{.m-def.m.sg.obl=3sg:PSR=post} \\ 
\glt `For three years, he stayed with his brothers in those mountains' \hfill[DG.21]
\z 

Like \textit{ce}, \textit{ta} may be used to express standard of comparison.

\ea
\textit{êtir îse weʕzû hewramanî ta menteqê terê kuçê xastera.} \\ 
\gll êtir îse weʕz(e)-û hewraman-î \textbf{ta} \textbf{menteq(e)-ê} \textbf{ter-ê} kuç-ê xas-ter=a \\ 
 \textsc{disc.ptcl} now situation\textsc{.f-ez.gen} \textsc{pn-m.sg.obl} than region\textsc{-pl.dir} other\textsc{-pl.dir} little\textsc{-indf} good\textsc{-cmpr}\textsc{=cop.3sg.m:S} \\ 
\glt `The situation in Hewraman is a bit better than in other regions.' \hfill[JM.55]
\z 

\subsubsubsection{e= `in'} 
This preposition is limited to spatial and manner adverbials. Unlike most adpositions, \textit{e=} does not have morphological allomorphs triggered by the morphophonological type of argument it takes. Yet, it is possible that \textit{e=} can be regarded as an allomorph of \textit{ce}.

\ea
\textit{nemendêwe eçê!} \\ 
\gll ne-men-dê=we \textbf{e=çê} \\ 
 \textsc{proh-}remain\textsc{.prs-2pl:S=compl} in=here \\ 
\glt `Don’t stay here!' \hfill[PM.22]
\z 

The preposition \textit{e=} is homophonous with the emphatic particle \textit{e}.

\ea
\textit{e înen ardenim xizmetû wêt.} \\ 
\gll \textbf{e} îne=n arde=n=im xizmet-û wê=t \\ 
 \textsc{emph} \textsc{\textsc{dem.prox}.dir.m.3sg=cop.3sg.m:S} bring\textsc{.pst.ptcp.m=cop.3sg.m:O=1sg:A} service\textsc{.m-ez.gen} \textsc{refl=2sg:PSR} \\ 
\glt `He [the boy] is this [man here] that I have brought to your service.' \\\hfill[ZQ.54]
\z 

\subsubsection{Locational nouns} \label{sect:locationalnouns}
Locational nouns are emerging prepositions that are the result of grammaticalisation from nouns. The source of grammaticalisation for some of the locational nouns\is{locational nouns} is body-part terms. Some are originally borrowings, such as \textit{beyn}, \textit{payîn}, and \textit{teref}.

\TabPositions{2.5cm,6cm}
\ea
\textit{ser}\tab `on'\tab cf. \textit{sere} `head' \\
\textit{mil} \tab `over' \tab cf. \textit{mil} `neck' \\
\textit{dil} \tab `inside' \tab cf. \textit{dil} `heart' \\
\textit{peşt} \tab `back' \tab cf. \textit{peştî} `back' \\
\textit{ber}\tab `front' \tab cf. \textit{ber} `front part of the body' \\
\textit{qew}\tab `on' \tab cf. \textit{qew} `stature' \\
\textit{şûn}\tab `after' \\
\textit{beyn}\tab `between' \\
\textit{çêr} \tab `under' \\
\textit{dewr}\tab `around' \\
\textit{gel} \tab `with' \\
\textit{gerûber} \tab `around' \\
\textit{la} \tab `at the place of, with' \\
\textit{nam}\tab `inside' \\
\textit{pêse}\tab `like' \\
\textit{payîn} \tab `down' \\ 
\textit{teref} \tab `from' \\
\textit{werawer}\tab `opposite' \\ 
\z

Complex prepositions may be formed by combining locational nouns\is{locational nouns} with basic adpositions, though this strategy is uncommon in the text corpus. The resulting complex forms tend to reinforce the original meaning of the basic preposition, on one hand, and may be informed by humanness. The two complex prepositions listed in (\ref{ex.complexprep}) are used with human referents.

\TabPositions{2.5cm,6cm}
\ea \label{ex.complexprep}
 \textit{ce teref} \tab `from' \tab [BP.36]\\  
\textit{pey la} \tab `to' \tab [ŞC.32] \\
\z 
 
Locational nouns\is{locational nouns} are morphologically distinguished from basic prepositions by two criteria. First, unlike the basic prepositions \textit{be} and \textit{ce}, they do not feature allomorphy when heading bound and non-bound arguments. In (\ref{la-1})--(\ref{la-2}), the locational noun \textit{la} has an invariant form.

\ea \label{la-1}
\textit{şewêne milûwe laş.} \\ 
\gll şew(e)-ê=ne mi-l-û=we \textbf{la=ş} \\ 
 night\textsc{-f.sg.obl=post} \textsc{ind-}go\textsc{.prs-1sg:S=compl} to\textsc{=3sg:R} \\ 
\glt `I go back to him at night.' \hfill[DG.41]
\z 


\ea \label{la-2}
\textit{zaroɫe bo hetîm bo milo la ađî.} \\ 
\gll zaroɫe b-o hetîm b-o mi-l-o \textbf{la} \textbf{ađî} \\ 
child be\textsc{.prs.ind-3sg:S} orphan be\textsc{.prs.ind-3sg:S} \textsc{ind-}go\textsc{.prs-3sg:S} to \textsc{3sg.obl.m} \\  
\glt `He was a child. He was an orphan. He went to him (his uncle).' \hfill[JP.14]
\z 

The other difference is that locational nouns\is{locational nouns} are generally linked to their non-bound complements via the genitive ezafe\is{genitive ezafe}.

\ea
\textit{serû řûçinekeywe des kero gireway.} \\ 
\gll \textbf{ser-û} řûçine-(e)key=we des ker-o {} gireway \\ 
 on\textsc{-ez.gen} chimney\textsc{.m-def.m.sg.obl=post} hand\textsc{.m} do\textsc{.prs.ind-3sg:A} {} cry\textsc{.inf} \\  
\glt `He started to cry on the chimney.' \hfill[BP.153]
\z 


\ea
\textit{cuwanê bê pêsew î girdû cuwana.} \\ 
\gll cuwan-ê b-ê \textbf{pêse-û} î gird-û cuwan-a \\ 
 youth\textsc{.m-indf} be\textsc{.prs-aug.3sg:S} like\textsc{.ez.gen} \textsc{dem.prox} all\textsc{-ez.gen} youth\textsc{.m-pl.obl} \\ 
\glt `He was a young man like all the young men.' \hfill[ZQ.54]
\z 


\ea
\textit{berem piştû berew wiɫaxeka.} \\ 
\gll b\'er-e=m \textbf{pişt-û} bere-û wiɫax-eka \\ 
 take\textsc{.prs.imp-2sg:A=1sg:O} back\textsc{-ez.gen} door\textsc{.m-ez.gen} horse/donkey\textsc{.m-def.pl.obl} \\  
\glt `Take me behind the door of the horses’ stable!' \hfill[ŞC.62]
\z 

The locational noun \textit{nam} `inside' is exceptional in not taking any linker. This may indicate that \textit{nam} is treated like a basic adposition.

\ea
\textit{nîy{ɛ}nêm nam bêşkekê.} \\ 
\gll nîy{ɛ}=nê=m \textbf{nam} \textbf{bêşke-(e)kê} \\ 
put\textsc{.pst.ptcp.pl=cop.3pl:O=1sg:A} inside cot\textsc{.f-def.f.sg} \\  
\glt `I gave them food [and] put them back in the cot.' \hfill[JE.67]
\z 

On the other hand, \textit{la} and \textit{pêse} `like' can occur both with and without the genitive ezafe\is{genitive ezafe}. In both cases, the ezafe-less variant seems to be more frequent. This may indicate that \textit{la} and \textit{pêse} are on their way to becoming grammaticalised (at least in part) as basic prepositions.

\ea
\textit{milo law mamoyş.} \\ 
\gll mi-l-o \textbf{la-û} \textbf{mamo-î=ş}\\ 
 \textsc{ind-}go\textsc{.prs-3sg:S} to\textsc{-ez.gen} paternal\_uncle\textsc{-m.sg.obl=3sg:PSR}\\ 
\glt `He went to his uncle.' \hfill[JP.11]
\z 


\ea
\textit{mila la řuwasê.} \\ 
\gll mi-l-a \textbf{la} \textbf{řuwas(e)-ê} \\ 
\textsc{ind-}go\textsc{.prs-3pl:S} to fox\textsc{-f.sg.obl} \\  
\glt `They went to the fox.' \hfill[MR.31]
\z 


\ea
\textit{kerđiş pêsew paɫaya.} \\ 
\gll kerđ=iş \textbf{pêse-w} \textbf{paɫa-ya} \\ 
 do.\textsc{pst=3sg:S} like-\textsc{ez.gen} shoe\textsc{.m-pl.obl} \\ 
\glt `He made it like shoes.' \hfill[DB.268]
\z 


\ea
\textit{beraşa pey menteqê pêse esfehanî.} \\ 
\gll ber-a=şa pey menteq(e)-ê \textbf{pêse} \textbf{esfehan-î} \\ 
take\textsc{.prs.ind-3pl:A=3pl:O} to region\textsc{-indf} like \textsc{pn-m.sg.obl} \\  
\glt `They were taken to a region like Isfahan.' \hfill[KŞ.47]
\z 

\subsection{Postpositions} \label{sect:postpositions}
Postpositions are a smaller class compared to prepositions. The following is a list of postpositions attested in the text corpus. The postpositions are treated as clitics, as they do not bear stress.\footnote{Note that some of the prepositions and locational nouns are not stressed either. The reason they are not marked by the equal sign is due to orthographic convention. }

\ea
\textit{=ne}\tab `in, at, from' \\
\textit{=we}\tab `at, out, etc.' \\
\textit{=hur}\tab `up' \\
\textit{=we, o}\tab `from, on, in' \\
\textit{=re/=ere} \tab `down'\\
\z 
 
Postpositions do not generally trigger oblique-case\is{oblique case} marking on their nominal complements, as exemplified by (\ref{ex.postp1})--(\ref{ex.postp2})--(\ref{ex.postp3}).

\ea
\textit{şamne a tome pîr şelîyarî wêş arden.} \\ 
\gll \textbf{şam=ne} a tom=e pîr şelîyar-î wê=ş arde=n \\ 
 \textsc{pn.m.dir=post} \textsc{dem.dist} seed\textsc{.m=dem} \textsc{pn} \textsc{pn-m.sg.obl} \textsc{refl=3sg:A} bring\textsc{.pst.ptcp.m=cop.3sg.m:O} \\ 
\glt `From Damascus, Pir Shaliyar had brought its seeds.' \hfill[ZP.95] \label{ex.postp1}
\z


\ea
\textit{jenekêm qomyaş venî kelekewe.} \\ 
\gll jen(î)-ekê=m qomya=ş venî \textbf{kel-eke=we} \\ 
 woman\textsc{.f-def.f.sg=1sg:PSR} happen\textsc{.pst.3sg:S=3sg:R} at mountain\textsc{-def.m.sg.dir=post} \\  
\glt `My wife was about to deliver a baby in the mountains.' \hfill[ZQ.14] \label{ex.postp2}
\z 


\ea
\textit{mila řahur.} \\ 
\gll mi-l-a \textbf{řa=hur} \\ 
 \textsc{ind-}go\textsc{.prs-3pl:S} road\textsc{.f.dir=post} \\  
\glt `They continued [along] the road.' \hfill[JH.14] \label{ex.postp3}
\z 
The postpositions seemingly trigger oblique\is{oblique case} marking on the nominal complement in the following examples. However, the case marking\is{case marking} happens because of the genitive construction that is embedded in the postpositional phrase.

\ea
\textit{çowe saraɫû êranîwe gêɫaymêwe.} \\ 
\gll ç=o=we \textbf{saraɫ-û} \textbf{êran-î=we} gêɫa-îmê=we \\ 
 in=\textsc{dem.dist=post} \textsc{pn-ez.gen} \textsc{pn-m.sg.obl=post} turn\textsc{.pst-1pl:S=compl} \\  
\glt `On the way back, we returned from the Saral region of Iran.' \hfill[ZQ.10]
\z 


\ea
\textit{cezîrew kîşîne karma kerđen şiş mangê.} \\ 
\gll \textbf{cezîre-û} \textbf{kîş-î=ne} kar=ma kerđe=n şiş mang(e)-ê \\ 
 island\textsc{-ez.gen} \textsc{pn-m.sg.obl=post} work\textsc{.m=1pl:A} do\textsc{.pst.ptcp.m=cop.3sg.m:O} six month\textsc{.f-pl.dir} \\ 
\glt `[We would go to] and work at the Island of Kish for six months.' \hfill[JM.25]
\z 

The following examples illustrate the rare cases of compatibility of postpositions with oblique case.

\ea
\textit{desiş nîya barîwe}\\
\gll des=iş nîya \textbf{bar-î=we}\\
hand=\textsc{3sg:A} put.\textsc{pst} load-\textsc{m.sg.obl=postp}\\
\glt `He stretched out his hands to the load.' \hfill[ÇH.73] \label{ex.postpobl}
\z 


\ea
\textit{tomekeş şana zemînekeynew.}\\ 
\gll tom-eke=ş şana zemîn-ekey=ne=û\\ 
seed\textsc{.m-def.m.sg.dir=3sg:A} scatter\textsc{.pst.3sg:O} land\textsc{.m-def.m.sg.obl=post}=and\\ 
\glt `He scattered the seeds in the field.' \hfill[JP.51]
\z

The postposition \textit{=ne} seems to have a preposition counterpart which occurs rarely in the speech of one story-teller. This makes \textit{ne} an ambiposition. In (\ref{ex.ne1})--(\ref{ex.ne2}), \textit{ne} is a preposition triggering oblique case marking on the following noun. However, note that \textit{=ne} also functions as a post-verb in Hewramî, rendering the adposition analysis in these examples tentative. Yet, the oblique marking on \textit{şar} and \textit{kor} makes the preposition analysis possible, as non-core arguments that are bare generally lack case marking (see \S\ref{sect:diffobl}).

\ea \label{ex.ne1}
\textit{kewt ne şarî.}\\
\gll kewt ne şar-î\\
fall.\textsc{pst.3sg.m:S} into town-\textsc{m.sg.obl} \\
\glt `He stormed into town.' \hfill [ÇH.167]
\z


\ea \label{ex.ne2}
\textit{kewt ne kora.}\\
\gll kewt ne kor-a \\
fall.\textsc{pst.3sg.m:S} into blind-\textsc{pl.obl}\\
\glt `He rushed into blind people.' \hfill [DB.309]
\z 

\subsection{Circumpositions} \label{sect:circum}
Circumpositions occur frequently in Hewramî\il{Hewramî}. They are constructed mainly in two ways: by combining a basic preposition and a postposition, e.g., \textit{ce ... ne} `in, (less so) from' or by combining a locational noun and a postposition, e.g., \textit{mil ... hur} `on top'. There are also more complex forms such as \textit{pêwere} `together' containing the circumposition \textit{be .... ere} combined with the indefinite suffix\is{indefinite suffix} \textit{-êwe}. 
Within basic prepositions, \textit{be} and \textit{ce} can form circumpositions with postpositional elements. The circumpositions may serve to disambiguate the multiple meanings of basic prepositions. For instance, \textit{ce} has both locational and source meanings. The addition of postpositions \textit{ne} and \textit{o/ew} helps distinguish between these spatial senses.

\ea\label{ex.cricum1}
\textit{ce ... ne}\tab `in, (less so) from' \\
\textit{ce ... o/we/ew} \tab `from' \\
\textit{be ... we}\tab `about'\\
\textit{be ... ne}\tab `on'\\
\textit{be ... re}\tab `in'\\
\textit{dilê ... re}\tab `in'\\
\textit{dilê ... ne}\tab `in' 
\z

The circumpositions listed in (\ref{ex.cricum1}) differ in the rate to which they trigger oblique marking on their complement \citep[]{Mohammadiradnon-core}. For instance, the complement of \textit{be ... we} (\ref{ex.be_we}) is more likely to be oblique\is{oblique case}-marked than the complement of \textit{ce ... ne} (\ref{ex.ce_ne}) and \textit{ce ... o/we} (\ref{ex.ce_o}).

\newpage
\ea
\textit{min înem dî be sey misefaywe.} \\ 
\gll min îne=m dî-∅ be sey misefa-î=we \\ 
 \textsc{1sg} \textsc{dem.prox.m.3sg.dir=1sg:A} see\textsc{.pst-3sg.m:O} about \textsc{pn} \textsc{pn-m.sg.obl=post} \\ 
\glt `I witnessed this about Say Mustafa.' \hfill[JP.74] \label{ex.be_we}
\z 


\ea
\textit{ce hewramanne karî naşerʕî kerênê.} \\ 
\gll ce hewraman=ne kar-î naşerʕî ker-ên-ê \\ 
 at \textsc{pn=post} task\textsc{.m-ez.attr} unlawful do\textsc{.prs-aug-3pl:A} \\ 
\glt `They were committing unlawful acts in Hewraman.’' \hfill[BP.62] \label{ex.ce_ne}
\z 


\ea
\textit{ce qeredaxo maroş ta heɫebce.} \\ 
\gll ce qeredax=o m-ar-o=ş ta heɫebce \\ 
 from \textsc{pn}\textsc{=post} \textsc{ind-}bring\textsc{.prs-3sg:A=3sg:O} until \textsc{pn} \\ 
\glt `He took his brother from Qaradax to Halabja.' \hfill[DG.29] \label{ex.ce_o}
\z 


 Circumpositions consisting of locational nouns\is{locational nouns} and a postposition comprise a far bigger class than circumpositions consisting of a basic preposition and a postposition, see (\ref{ex.compound.circ}). This could be due to the bigger size of the class of locational nouns\is{locational nouns}.
 
\TabPositions{2.5cm,6cm}
\ea \label{ex.compound.circ}
\textit{ber ... ne}\tab `in front of' \tab [HB.73] \\
\textit{beyn ... ne}\tab `in between' \tab [BP.178] \\
\textit{çêr ... ne}\tab `under' \tab [BP.194] \\
\textit{dil ... ne}\tab `inside' \tab [KŞ.57] \\
\textit{dewr ... ne}\tab `at around' \tab [ZP.67] \\
\textit{dewr ... hur}\tab `around' \tab \\
\textit{la ... we}\tab `at the place of'\tab [ZB.59] \\
\textit{la ... o}\tab `with'\tab [JH.1] \\
\textit{la ... ne}\tab `at the place of' \tab [HB.73] \\
\textit{la ... re} \tab `next to' \\
\textit{ce la ... we}\tab `to, at the place of'\tab [BP.211] \\
\textit{mil ... we}\tab `because of, over' \tab [JM.27] \\
\textit{mil ... re}\tab `on' \tab [JP.107] \\
\textit{mil ... hur}\tab `on top' \tab \\
\textit{payîn ... re}\tab `down at' \tab [ZQ.5] \\
\textit{qew ... we}\tab `on, around' \tab [HB.61] \\
\textit{ser ... we}\tab `on top of' \tab [BP.153] \\
\textit{şûn ... re}\tab `after' \tab [JM.4] \\
\textit{şûn ... o/we}\tab `behind' \tab [DP.8] \\
\textit{teref ... we}\tab `from' \tab [PM.34] \\
\textit{werawer ... we}\tab `in front of' \tab [JP.201] \\
\z

The nominal complement of the circumpositions listed in (\ref{ex.compound.circ}) may be marked in the oblique case\is{oblique case}.

\ea
\textit{înê dilê namekeyne minvîso.} \\ 
\gll înê dilê name-(e)key=ne mi-nvîs-o \\ 
 \textsc{dem.prox.3pl} inside letter\textsc{.m-def.m.sg.obl=post} \textsc{ind-}write\textsc{.prs-3sg:A} \\ 
\glt `He wrote these [words] inside the letter [and gave it to the shepherd’s son].' \hfill[KŞ.57]
\z 


\ea
\textit{min la şêxîwe menîşû.} \\ 
\gll min la şêx-î=we me-nîş-û\\
\textsc{1sg} with sheikh\textsc{.m-sg.obl=post} \textsc{neg.ind-}sit\textsc{.prs-1sg:S} \\ 
\glt `I won’t sit together with the sheikh [any more!].' \hfill[HB.67]
\z 


\ea
\textit{da luwe şûnû sey misefayre} \\ 
\gll da lu-e şûn-û sey misefa-î=re \\ 
 \textsc{hort} go.\textsc{prs.imp-2sg:S} after\textsc{-ez.gen} \textsc{pn} \textsc{pn-m.sg.obl}\textsc{=post} \\ 
\glt `Go after Say Mustafa!' \hfill[JP.43]
\z 

The components of some circumpositions may be grammaticalised into compound adpositions. The resultant forms often convey spatial and temporal adverbial meanings (see \S\ref{sect:loc-adv}).
\TabPositions{2.5cm,8cm}
\ea
\textit{beynne} \tab `in-between' \tab [RE.25]\\
\textit{serew}\tab `from top' \tab [BP.152]\\
\textit{dilne}\tab `inside' \tab [JE.26]\\
\textit{çêrwe} \tab `from underneath' \tab [KŞ.31]\\
\textit{çêrhur} \tab `from below facing upwards' \tab [ZB.45]\\
\textit{şûnîre}\tab `afterwards' \tab [JM.6] \\
\z 


\subsection{Absolute adpositions} \label{sect:absolute_prep}

As hinted in \S\ref{sect:basicprep}, Tekht Hewramî\il{Hewramî!Tekht} features a set of formatives that are morphological allomorphs of basic adpositions. These formatives have been labelled ``absolute prepositions'' in works on Iranian languages. Contrary to all other adpositions seen so far, which can be used with prosodically independent complements (i.e., independent pronouns, nouns), the complement of some of the `absolute adpositions' can only be a bound pronominal element, as seen in the difference between \textit{be} (\ref{ex.be_}) and \textit{pene} (\ref{ex.pene}). Additionally, absolute adpositions are associated with a special syntax (see below).


\ea
\textit{newatim be to?} \\ 
\gll ne-wat=im \textbf{be} \textbf{to} \\ 
 \textsc{neg-}say.\textsc{pst=1sg:A} to \textsc{2sg}  \\ 
\glt `Didn't I say [that] to you?' \hfill[PW.87]  \label{ex.be_}
\z 


\ea
\textit{watim peneş} \\ 
\gll wat=im \textbf{pene=ş} \\ 
 say\textsc{.pst=1sg:A} to\textsc{=3sg:R} \\ 
\glt `I said to him.' \hfill[JH.31] \label{ex.pene}
\z 

\begin{table}[b]
    \begin{tabular}{llll}
\lsptoprule
Adposition/locational noun\is{locational nouns} & Gloss& Absolute form \\
\midrule
\textit{be} & `to, by' & \textit{pene} \\
— & `with' & \textit{pene} \\
\textit{be} ... \textit{re}& `on' & \textit{pore}  \\
\textit{be} ... \textit{we}& `on' & \textit{powe}  \\
\textit{pey} & `for' & \textit{pey}  \\
\textit{ce}, \textit{ce} ... \textit{ne} & `from' & \textit{çene}  \\
\textit{çenî} & `with' & \textit{çene} \\
\textit{la} ... \textit{we} & `to'& \textit{la ... we} / \textit{lawe}  \\
\textit{mil} ... \textit{re} & `to'& \textit{mil ... re} / \textit{milre} \\
\textit{ser} ... \textit{o} & `on top' & \textit{ser ... o} / \textit{sero}  \\
\textit{bê} & `without' & —  \\
\textit{ta} & `until' & —  \\
\textit{ser} & `on, to' & \textit{ser}  \\
\lspbottomrule
    \end{tabular}
    \caption{Adpositions and their corresponding absolute forms}
    \label{tab:applicatives}
\end{table}

\tabref{tab:applicatives} demonstrates the absolute form of adpositions found in the main text corpus. As can be seen, adpositions can be classified into four general sets concerning their `absolute forms': (1) formatives which are derivable by transparent phonological process from their basic forms, e.g., \textit{be} vs. \textit{pene}. Those adpositions that have an absolute form will go into this form when the adpositional complement is a bound pronominal argument, rather than an NP (as seen in the difference between (\ref{ex.be_}) and (\ref{ex.pene}); (2) formatives for which the basic form and the absolute form are the same, see \textit{pey}, \textit{ser}; (3) formatives with no basic forms, see \textit{pene} `with'; (4) formatives with no absolute forms, e.g., \textit{bê}, \textit{ta}.



Absolute adpositions are associated with a specific syntax. They can only take bound pronominal arguments, realised through different markers depending on the tense and transitivity of the clause. There is an additional complication, however. Due to independent processes of clitic movement, the clitic pronoun may move from its governing adposition and attach to another element of the clause. Crucially, the adposition remains in the absolute form even if its complement has moved off it to attach to a distinct host. In clauses with verbs deriving from the present tense, clitic movement is generally leftward and is realised on the first element of the VP. The VP-initial element is an object NP in (\ref{ex.absoluteprep1}), a light verb\is{light verb} complement in (\ref{ex.absoluteprep2}), and a verb in (\ref{ex.absoluteprep3}), on which the preposition complement lands.\footnote{Discussing a similar phenomenon in neighbouring Central Kurdish\il{Kurdish!Central} dialects, \citet{karim_applicative_2022} propose an applicative analysis of absolute prepositions. While insightful, their analysis runs into problems in cases like (\ref{ex.absoluteprep1}), where the absolute form is not a derivational morpheme on the verb, which is one of the defining features of applicative morphemes \citep{pacchiarotti_introduction_2022}. Notice that nor in the corresponding Central Kurdish\il{Kurdish!Central} rendering of (\ref{ex.absoluteprep1}) is the formative a derivational preverb on the root: \textit{nan=yan bo peya e-ka}.}

\ea
\textit{nanşa pey peya kero.} \\ 
\gll nan=\textbf{şa} \textbf{pey} peya ker-o \\ 
bread\textsc{.m=3pl:R} for visible do\textsc{.prs.ind-3sg:A} \\ 
\glt `He finds food for them.' \hfill[BP.148] \label{ex.absoluteprep1}
\z  


\ea
\textit{bawiřiş pene kero.} \\ 
\gll bawiř=\textbf{iş} \textbf{pene} ker-o \\ 
 belief\textsc{=3sg:R} in do\textsc{.prs.ind-3sg:A} \\ 
\glt `He (Hama Yoso) believed him.' \hfill[JP.59] \label{ex.absoluteprep2}
\z 


\ea
\textit{bero miđoş pene.} \\ 
\gll ber-o mi-đ(e)-o=\textbf{ş} \textbf{pene} \\ 
 take\textsc{.prs.ind-3sg:A} \textsc{ind-}give\textsc{.prs-3sg:A=3sg:R} to \\ 
\glt `She took [it and] gave it to him (Hayas).' \hfill[JH.43] \label{ex.absoluteprep3}
\z 

In the examples above, the VP-initial element is immediately realised to the left of the absolute prepositions. The clitic complement of the adposition can still land on the VP-initial element provided the host is at a reasonable distance from the preposition; see the difference between (\ref{ex.absoluteprep4})--(\ref{ex.absoluteprep5}), on the one hand, and (\ref{ex.absoluteprep6}) on the other. In (\ref{ex.absoluteprep6}) the clitic pronoun can, in principle, move leftward and be realised on \textit{esb-î}. The lack of clitic movement may suggest that the mobility is not obligatory.

\ea
\textit{min şûyş kerû pene.} \\ 
\gll min şû-î=\textbf{ş} ker-û \textbf{pene} \\ 
 \textsc{1sg} husband\textsc{.m-sg.obl=3sg:R} do\textsc{.prs.ind-1sg:A} to \\ 
\glt `I will marry him.' \hfill[JH.59] \label{ex.absoluteprep4}
\z 


\ea
\textit{yewêş kera sero.} \\ 
\gll yew(e)-ê=\textbf{ş} ker-a \textbf{ser=o} \\ 
 lucerne\textsc{-f.sg.obl=3sg:R} do\textsc{.prs.ind-3pl:A} on\textsc{=post} \\ 
\glt `They will put [some] lucerne on top of the straw.' \hfill[HB.37] \label{ex.absoluteprep5}
\z 


\ea
\textit{a esbî zînî kere peym.} \\
\gll a esb-î zînî k\'er-e \textbf{pey=m} \\ 
 \textsc{dem.dist} horse\textsc{-m.sg.obl} saddle do\textsc{.prs.imp-2sg:A} for\textsc{=1sg:R} \\ 
\glt `Saddle up the horse for me.' \hfill[ŞC.53] \label{ex.absoluteprep6}
\z 

Likewise, the clitic complement of the adposition is realised in situ when the adposition is the first element within the VP. Note that the element before \textit{pey} is the verb of the preceding clause, hence not counted as the immediate VP-internal element to the left as a host for the clitic pronoun.

\ea
\textit{çowe ew mê peym maro.} \\ 
\gll ç=o=we ew m-ê \textbf{pey=m} m-ar-o \\ 
 in=\textsc{dem.dist=post} \textsc{dem.dist} \textsc{ind-}come\textsc{.prs.3sg:S} for\textsc{=1sg:R} \textsc{ind-}bring\textsc{.prs-3sg:A} \\ 
\glt `From that direction, he comes [to me] and brings [food] for me.' \hfill[PM.46]
\z 

Similarly, with intransitive verbs derived from the present tense and frequently also in the past tense, the complement of the adposition is realised as a clitic pronoun. Here, the clitic floats leftward. The leftward movement takes even the S argument (\ref{ex.adp-clc-1}) or the copula subject (\ref{ex.adp-clc-3}) as host. This could be interpreted as a remnant of the original clausal second-position rule for clitic placement in Hewram\^i\il{Hewramî}. The leftward movement of the clitic pronoun in intransitive clauses may also target the verb as the host, as shown in (\ref{ex.adp-clc-2}).

\ea
\textit{haminiş aman milre.} \\ 
\gll hamin=\textbf{iş} ama=n \textbf{milre} \\ 
 summer\textsc{.m=3sg:R} come\textsc{.pst.ptcp.m=cop.3sg.m:S} on \\ 
\glt `Then it became summer.' [Lit. Summer came upon him.] \hfill[DP.10] \label{ex.adp-clc-1}
\z 


\ea
\textit{eger minit çene bîy{ɛ}nê ...} \\
\gll eger min=\textbf{it} \textbf{çene} bî-{ɛ}n-ê \\
if \textsc{1sg=2sg:R} with be.\textsc{pst.cond.aug-1sg:S} \\
\glt `If I had been with you ...' \hfill[PW.88] \label{ex.adp-clc-3}
\z 


\ea
\textit{kuřekêş germîyanne nimêniş çene.} \\ 
\gll kuř-ekê=ş germîyan=ne nim(e)-ê-nê=\textbf{iş} \textbf{çene} \\ 
 son\textsc{.m-def.pl.dir=3sg:PSR} \textsc{pn=post} \textsc{neg.ind-}come\textsc{.prs-3pl:S=3sg:R} with \\ 
\glt `His sons did not accompany him [they stayed in] Garmiyan.' \hfill[ZB.9] \label{ex.adp-clc-2}
\z

The adposition complement is realised as a person index in the following example featuring a past intransitive construction.

\ea
\textit{diɫim şîyenî pene.} \\ 
\gll diɫ=im şîye=n-\textbf{î} \textbf{pene} \\ 
heart=\textsc{1sg:PSR} go.\textsc{pst.ptcp.m=cop.3sg:S=2sg:R} to \\ 
\glt `I am fond of you. [Lit. My heart has gone to you.]' \hfill[SK.06]
\z

In clauses based on a past transitive verb, the adposition complement does not move leftward. Instead, it can be realised through a verbal person affix or a copula PM on the verb (depending on the TAM category of the verb), with the same feature values. If (\ref{ex.kete}) were in the present tense, the complement of \textit{pey} would move leftward to land on \textit{ketê}.

\ea
\textit{xway ketê pey kîyasen.} \\ 
\gll xwa-î ket-ê \textbf{pey} kîyase=\textbf{n} \\ 
God\textsc{.m-sg.obl} bed\textsc{-indf} to send\textsc{.pst.ptcp.m=cop.3sg.m:R} \\ 
\glt `[As if] God had sent him a bed.' \hfill[JP.69] \label{ex.kete}
\z


\ea
\textit{îna qaqezêçiş dana pey.} \\ 
\gll în(e)=a qaqez-ê=ç=iş da=\textbf{na} \textbf{pey} \\ 
 \textsc{dem.prox.m.3sg.dir=ptcl} letter\textsc{.m-indf=add=3sg:A} give\textsc{.pst.ptcp.m=cop.1sg:R} to \\ 
\glt `Look, he has given me a letter.' \hfill[KŞ.84]
\z 

\section{Adverbs}
The class of adverbs is composed of non-derived and derived adverbs. Adverbs are typically formed from a combination of an adpositional or a demonstrative element and a nominal. Among adverbial expressions, manner adverbs\is{manner adverbs} can also be used adjectivally. 

\subsection{Spatial adverbs\is{spatial adverbs}} \label{sect:loc-adv}
Spatial adverbs\is{spatial adverbs} are classified based on having a demonstrative in their structure or not. Following \citet{diessel_demonstratives_1999}, the forms based on demonstratives are called `local adverbial demonstratives\is{local adverbial demonstratives}' here. Spatial adverbs\is{spatial adverbs} based on demonstratives are listed in \tabref{tab:spatial-adv} (see \S\ref{sect:loc-adv-dem} for discussion).
\begin{table}[htp]
    \centering
    \fittable{\begin{tabular}{llll}
\lsptoprule
`here' &\textit{çêge, êge} &< \textit{ç} `in' + \textit{ê} `this' + \textit{-ge} \\
`there (immediate context)&\textit{çage}&< \textit{ç} `in' + \textit{a} `that' + \textit{-ge} \\
`there (distant but visible)' &\textit{çoge, oge} &< \textit{o} `that' + \textit{-ge} \\
`there (invisible)' &\textit{pagene, epagewe} &\\
`down here'&\textit{epêre} &\\
`down there (invisible)' &\textit{epare} &\\
`up there (visible)'&\textit{ewagehur} &\\
`up there (invisible)'&\textit{epagehur} & \\
\lspbottomrule
    \end{tabular}}
    \caption{Spatial adverbs based on demonstratives}
    \label{tab:spatial-adv}
\end{table}

Spatial adverbs\is{spatial adverbs} not derived from demonstratives are formed from a combination of spatial nouns and often a postposition. In addition to adverbial use, they may have adpositional uses (see \S\ref{sect:circum}). The most common spatial adverbs\is{spatial adverbs} are listed in (\ref{ex.spatialadv}).

\TabPositions{2.5cm,8cm}
\ea \label{ex.spatialadv}
\textit{beynne} \tab `in-between' \tab [RE.25] \\
\textit{serew}\tab `from the top' \tab [BP.152]\\
\textit{dilne}\tab `inside' \tab [JE.26]\\
\textit{çêrwe} \tab `from underneath' \tab [KŞ.31]\\
\textit{çêrhur} \tab `from below facing upwards' \tab [ZB.45]\\
\textit{berewe}\tab `outside' \tab \\
\textit{ce dûrew }\tab `from far away' \tab [DP.4] \\
\z

The following examples illustrate the use of spatial adverbs\is{spatial adverbs}.

\ea
\textit{çêrwe kuřekey memew bizekêş ward.} \\ 
\gll \textbf{çêrwe} kuř-ekey meme-û bize-(e)kê=ş ward \\ 
from\_underneath boy\textsc{.m-def.m.sg.obl} breast\textsc{.m-ez.gen} goat\textsc{.f-def.f.sg=3sg:A} eat\textsc{.pst} \\ 
\glt `From underneath, the boy fed from the goat’s udder.' \hfill[KŞ.31]
\z 


\ea
\textit{serew des kero gireway.} \\ 
\gll \textbf{serew} des ker-o {} gireway \\ 
 from\_above hand\textsc{.m} do\textsc{.prs.ind-3sg:A} {} cry\textsc{.inf} \\ 
\glt `He (Little Hama) started to cry from the top [of the roof].' \hfill[BP.152]
\z 


\ea
\textit{kabrayç ama berewe.} \\ 
\gll kabra=yç ama \textbf{berewe} \\ 
 fellow=\textsc{add} come.\textsc{pst.3sg:S} outside\\ 
\glt `The fellow came outside.' \hfill[KK.34]
\z 

\subsection{Temporal adverbs\is{temporal adverbs}}
Temporal adverbs\is{temporal adverbs} are classifiable into five categories: general deictic adverbs\is{general deictic adverbs} (\S\ref{sect:general-deic}); time-of-day adverbs\is{time-of-day adverbs} (\S\ref{sect:time-day-adv}), calendrical adverbs\is{calendrical adverbs} (\S\ref{sect:calend-adv}), calendrical cyclic adverbs\is{calendrical cyclic adverbs} (\S\ref{sect:calend-cyc-adv}), and other temporal adverbs\is{temporal adverbs} (\S\ref{sect:time-gen-adv}). These categories present different morphological properties. 

\subsubsection{General deictic adverbs\is{general deictic adverbs}} \label{sect:general-deic}
General deictic adverbs\is{general deictic adverbs} may, in principle, inflect for case\is{case}, as seen in \tabref{tab:gen.deictic}, though some occur only in one case in the text corpus.

\begin{table}[htp]
    \centering
    \fittable{\begin{tabular}{llll}
\lsptoprule
& \textsc{dir} & \textsc{obl} \\ \midrule
`now' & \textit{\^ise} & \textit{\^isey} \\
`back then, in the past' & \textit{çaweɫ} & \textit{çaweɫî} & < * \textit{ç= a eweɫ} `from that beginning'\\
`at that time, then' & \textit{ewsa} & - \\
`after some time'&-&\textit{ça dimay} \\
`afterwards'& \textit{şûnîre}& - \\
`before now' & - & \textit{çêweɫî} \\
\lspbottomrule
    \end{tabular}}
    \caption{General deictic adverbs}
    \label{tab:gen.deictic}
\end{table}

The following examples exhibit the use of \textit{çaweɫ} and \textit{çaweɫî}.

\ea
\textit{çaweɫ ta cawe nelabê pane pêwyê tewenekê.} \\ 
\gll \textbf{çaweɫ} ta cawe ne-la=b-ê p=ane pêwy(e)-ê tewen(î)-ekê \\ 
 in\_the\_past until road \textsc{neg-}go\textsc{.pst.ptcp.m}=be\textsc{.prs-aug.3sg:S} at=\textsc{dem.dist.m.3sg.dir} be\_visible\textsc{.prs-aug.3sg:S} stone\textsc{.f-def.f.sg} \\ 
\glt `In the past, when no road was constructed there [lit. The road had not gone there], the stone was visible.' \hfill[ZP.54]
\z 


\ea
\textit{çaweɫî sextî nebîyen.} \\ 
\gll \textbf{çaweɫ-î} sextî ne-bîye=n \\ 
 in\_the\_past\textsc{-m.sg.obl} difficulty\textsc{.m.dir} \textsc{neg-}be\textsc{.pst.ptcp.m=cop.3sg.m:S} \\ 
\glt `In the past, there was no hardship.' \hfill[JE.56]
\z 

\subsubsection{Time-of-day adverbs} \label{sect:time-day-adv}
Time-of-day adverbs\is{time-of-day adverbs} have the base forms as follows:

\ea
\textit{seʕbe} \tab `morning' \\
\textit{nîmeřó}\tab `midday' \\
\textit{wêrega}\tab `evening' \\
\textit{şewe}\tab `night' \\
\z

When used in the sense of temporal location, they may be followed by a postposition or case marking. Note that the oblique case\is{oblique case} is blocked before the postposition \textit{=ne} (see \S\ref{sect:postpositions}).

\ea
\textit{seʕbê, seʕbne} \tab `in the morning' \\
\textit{nîmeřone}\tab `in the midday' \\
\textit{wêregane}\tab `in the evening' \\
\textit{şewê}\tab `at night' \\
\z 

Examples:

\ea
\textit{î kabr{ɛ}çe ce ʕêraqo am{ɛ}nê, wêreganew nîmeřonew seʕbne bexşnayşa bexşn{ɛ}nêwe.} \\ 
\gll î kabr{ɛ}=ç=e ce ʕêraq=o am{ɛ}=nê wêrega=ne=û nîmeřo=ne=û seʕb=ne bexşnay=şa bexşn{ɛ}=nê=we \\ 
 \textsc{dem.prox} man\textsc{.pl.dir=add=dem} from \textsc{pn=post} come\textsc{.pst.ptcp.pl=cop.3pl:S} evening\textsc{=post}=and noon\textsc{=post}=and morning\textsc{=post} distribute\textsc{.nmlz=3pl:A} distribute\textsc{.pst.ptcp.pl=cop.3pl:R=compl} \\ 
\glt `They (people) would donate [food] to the fellows (the tax collectors) who had come from Iraq, in the evenings, mornings, and at noon.' \hfill[BP.38]
\z 


\ea
\textit{hurmêzowe seʕbê.} \\ 
\gll hur-m-êz-o=we seʕb(e)-ê \\ 
 \textsc{pvb-}\textsc{ind-}rise\textsc{.prs-3sg:S=compl} morning\textsc{-f.sg.obl} \\ 
\glt `He woke up in the morning.' \hfill[ŞC.47]
\z 

\subsubsection{Calendrical adverbs\is{calendrical adverbs}} \label{sect:calend-adv}
Some calendrical adverbs\is{calendrical adverbs} contain a deictic element, e.g.\textit{ êşew} < \textit{ê} `this' + \textit{şew} `night'. The terms \textit{pérê} and \textit{pêré} are synchronically opaque in this regard.

\ea
\textit{aró} \tab `today' \\
\textit{êşew}\tab `tonight' \\
\textit{seba}\tab `tomorrow' \\
 \textit{pêré} \tab `the day after tomorrow' \\
\textit{hêz\stackunder[-10pt]{\^{i}}{\'{}}} \tab`yesterday' \\
\textit{pérê}\tab `the day before yesterday’ \\
\textit{êsaɫ} \tab `this year' \\
\z

Unlike their cyclic counterparts (\S\ref{sect:calend-cyc-adv}), calendrical adverbs are not inflected for case\is{case}. Rather, the base form is used.

\ea
\textit{seba kewçê teriş pene miđa.} \\ 
\gll \textbf{seba} kewç(e)-ê ter=iş pene mi-đ(e)-a \\ 
 tomorrow\textsc{.m} six\_kilos\textsc{-indf} another\textsc{=3sg:R} to \textsc{ind-}give\textsc{.prs-3pl:A} \\ 
\glt `The next day, he (the uncle) would give him another six kilos [of barley seed].' \hfill[JP.41]
\z 


\ea
\textit{êsaɫ miɫkekema nemenenû zeraʕetma nîya.} \\ 
\gll \textbf{êsaɫ} miɫk-eke=ma ne-mene=n=û zeraʕet=ma nîy(e)=a \\ 
this\_year property\textsc{.m-def.m.sg.dir=1pl:PSR} \textsc{neg-}remain\textsc{.pst.ptcp.m=cop.3sg.m:S}=and agriculture\textsc{.m=1pl:NC} \textsc{neg.exist=cop.3sg.m:S} \\ 
\glt `We have not cultivated much land this year; we don’t have much agriculture.' \hfill[PM.37]
\z 

\subsubsection{Calendrical cyclic adverbs} \label{sect:calend-cyc-adv}
Calendrical cyclic adverbs\is{calendrical cyclic adverbs} refer to words for cyclic events, such as `day', `year', `hour', `spring'. They are inflected for case\is{case} in their non-base forms, though the case marking\is{case marking} is blocked before the postposition \textit{=ne} (see \S\ref{sect:postpositions}).

\TabPositions{1.5cm,6cm,8.5cm}
\ea
 Base form  \tab Tempral locational `in' \\
\textit{hamin} \tab `summer' \tab \textit{hamin=ne} \tab `in the summer' \\
\textit{řo} \tab `day' \tab \textit{řo=ne } \tab `during the day' \\
\z
 
Calendrical cyclic adverbs\is{calendrical cyclic adverbs} can be used with the quantifier \textit{her} `every' and \textit{gi} `all, every'.\footnote{\textit{gi} and \textit{gir} are truncated variants of quantifier \textit{girđ}.}

\TabPositions{2.5cm}
\ea
\textit{her wêregane} \tab `every evening' \\ 
\textit{her şewê } \tab `each night' \\
\textit{her seʕatê} \tab `each hour' \\
\textit{gi řowê} \tab `every day' \\
\textit{gir carê} \tab `each time' \\
\textit{gir şewêwe } \tab `every night' \\
\z

Example:

\ea
\textit{yanew yo terî, her şewê her wêregane î mêmanê mêmanû yoy bîyênê.} \\ 
\gll yane-û yo ter-î \textbf{her} \textbf{şew(e)-ê} \textbf{her} \textbf{wêrega=ne} î mêman-ê mêman-û yo-î bîyê=nê \\ 
 house\textsc{.m-ez.gen} one\textsc{.m} other\textsc{-m.sg.obl} each night\textsc{-f.sg.obl} each evening\textsc{=post} \textsc{dem.prox} guest\textsc{.m-pl.dir} guest\textsc{.m-ez.gen} one\textsc{.m-sg.obl} be\textsc{.pst.ptcp.pl=cop.3pl:S} \\ 

\glt `[likewise, one tax collector] would be the guest in the house of one [fellow from Hewraman] each evening, each night.' \hfill[BP.43]
\z 


\ea
\textit{awekê biřo. milonû wer werwe maro be nîrûwekeyş de koɫê penj koɫê gi řowê.} \\ 
\gll aw(î)-ekê biř-o mi-l-on=û wer werwe m-ar-o be nîrû-ekey=ş de koɫ-ê penj koɫ-ê \textbf{gi} \textbf{řo-ê} \\ 
 water\textsc{.f-def.f.sg} cut\textsc{.prs.ind-3sg:A} \textsc{ind-}go\textsc{.prs-3sg:S}=and out snow\textsc{.f} \textsc{ind-}bring\textsc{.prs-3sg:A} by force\textsc{.m-def.m.sg.obl=3sg:PSR} ten load/shoulder\textsc{.m-pl.dir} five load/shoulder\textsc{.m-pl.dir} each day\textsc{.m-pl.dir} \\ 
\glt `He cut off the water supply. He (i.e., Jamsher Shah) fetched snow, five or ten loads daily, using his men.' \hfill[DP.34]
\z 

\subsubsection{Other temporal adverbs\is{temporal adverbs}} \label{sect:time-gen-adv}
General temporal adverbs\is{general temporal adverbs} comprise the following. The first two can be used as adjectives as well.

\TabPositions{2.5cm}
\ea
\textit{zû} \tab `soon' \\ 
\textit{dêr} \tab `late' \\
\textit{keřetê} \tab `once' \\
\textit{eweɫêne} \tab `at first' \\
\textit{deredaymê} \tab `always' \\
\textit{hemîşe} \tab `always' \\
\z 


\ea
\textit{seʕbê dêr mê.} \\ 
\gll seʕb(e)-ê dêr m-ê \\ 
 morning\textsc{-f.sg.obl} late \textsc{ind-}come\textsc{.prs.3sg:S} \\ 
\glt `In the morning, he went late [to the meeting].' \hfill[BP.74]
\z 


\ea
\textit{zû biřewbare bisana bênêwe.} \\ 
\gll zû biřewbare bi-san-a b-ê-nê=we \\ 
 quickly tax\textsc{.m} \textsc{sbjv-}buy\textsc{.prs-3pl:A} \textsc{sbjv-}come\textsc{.prs-3pl:S=compl} n \\ 
\glt `They were supposed to collect taxes soon and return [to Iraq].' \hfill[BP.58]
\z 


\ea
\textit{eweɫêne duwê jenî kîyana.} \\
\gll eweɫêne duwê jenî kîyan-a \\
at\_first two woman.\textsc{f} \textsc{ind}-send.\textsc{prs-3pl:A} \\
\glt `At first, they (i.e., the family of the boy) send two women (to the family of the girl).’ \hfill\citep[558]{khan_language_2023}
\z

\subsection{Adverbs of change and continuation}
These adverbs express the following meanings: `already', `still', `not yet', and `no longer'. `Still' is expressed by \textit{heɫa} and \textit{her}:

\ea
\textit{heɫa nefama.} \\ 
\gll heɫa nefam=a \\ 
 still inexperienced\textsc{=cop.3sg.m:S} \\ 
\glt `He was still inexperienced.' \hfill[BP.135]
\z 

 \ea 
\textit{her hewarne bîyênê.} \\ 
\gll her hewar=ne bîyê=nê \\ 
 still summer\_habitat\textsc{.m=post} be\textsc{.ptcp.pl=cop.3pl:S} \\ 
\glt `They were still at the summer habitat [until 40 or 50 days after autumn began].' \hfill[JE.10]
\z

`Not yet' is expressed by \textit{heɫay}, which is a variant of \textit{heɫa}.

\ea
\textit{heɫay neyawan sinʕe.} \\ 
\gll heɫay ne-yawa=n sinʕe \\ 
yet \textsc{neg-}arrive\textsc{.pst.ptcp.m=cop.3sg.m:S} age\textsc{.f} \\ 
\glt `He (Little Hama) has not reached adulthood yet.' \hfill[BP.123]
\z 

`No longer' and `any more' are expressed by \textit{êtir}.


\ea
\textit{êtir mebo ême karêş pene kermê.} \\ 
\gll êtir me-b-o ême kar-ê=ş pene k\'er-mê \\ 
no\_longer \textsc{neg.ind-}be\textsc{.prs-3sg:S} \textsc{1pl} task\textsc{.m-indf=3sg:R} to do\textsc{.prs.sbjv-1pl:A} \\ 
\glt `We can no longer do anything for him.' \hfill[JP.75]
\z 


\ea
\textit{pekema megino êtir.} \\ 
\gll peke=ma me-gin-o êtir \\ 
 strength\textsc{.f=1pl:PSR} \textsc{neg.ind-}fall\textsc{.prs-3sg:S} no\_longer \\ 
\glt `we should not be worried any more.' \hfill[HB.42]
\z 

\subsection{Degree adverbs}
Degree adverbs\is{degree adverbs} modify adjectives and verbs. They are listed in (\ref{ex.degreeadv}). Of the adverbs listed, \textit{fire}, \textit{kem}, and \textit{kuçê} can also be used as adjectives.
\TabPositions{2.5cm}
\ea \label{ex.degreeadv}
\textit{fire} \tab `very, much' \\ 
\textit{kem}\tab `little' \\
\textit{kuçê}\tab `little' \\
\textit{mêqdarê}\tab `a bit' \\
\textit{xeylê}\tab `a lot' \\
\z

Examples:

\ea
\textit{xeylê raziɫû diɫîş kerdêş.} \\ 
\gll xeylê raziɫ-û diɫ-î=ş kerd-ê=ş \\ 
 so\_much request\textsc{-ez.gen} heart\textsc{.m-sg.obl=3sg:PSR} do\textsc{.pst-3pl:O=3sg:A} \\ 
\glt `He struggled a lot.' \hfill[HB.28]
\z


\ea
\textit{miqdarê yawo pene.} \\ 
\gll miqdarê yaw-o pene \\ 
a\_little arrive\textsc{.prs.ind-3sg:S} to \\
\glt `He grew up a bit.' \hfill[JP.15]
\z 


\ea
\textit{fire mûsaw kem mûsa.} \\ 
\gll fire m-ûs-a=û kem m-ûs-a \\ 
 a\_lot \textsc{ind-}sleep\textsc{.prs-3pl:S}=and a\_little \textsc{ind-}sleep\textsc{.prs-3pl:S} \\ 
\glt `They slept a lot; they slept a little.' \hfill[JP.66]
\z 
 
\subsection{Manner adverbs\is{manner adverbs}} \label{sect:mann-adv}
Manner adverbs\is{manner adverbs} may be classified according to whether they have a demonstrative in their structure or not. Following \citet{diessel_demonstratives_1999}, the forms based on demonstratives are referred to as ``manner adverbial demonstratives\is{manner adverbial demonstratives}''. Manner adverbs\is{manner adverbs} based on demonstratives are listed here (see \S\ref{sect:man-adv-dem} for discussion).

\ea
\textit{eçîne, epêse} \tab `like this, in this way, as' \\
\textit{eçane, epase} \tab `like that, in that way, as' \\
\z 

Manner adverbs\is{manner adverbs} may be derived from adjectives:

\ea
\textit{eçîne, epêse} \tab `like this, in this way' \\
\textit{eçane, epase} \tab `like that, in that way' \\
\z 

Manner adverbs\is{manner adverbs} may be expressed by the preposition \textit{be} combined with an abstract noun:

\ea
\textit{îne eçê çenû aẍeyş cengşa bîyen. be adizî luwan.} \\ 
\gll îne e=çê çenû aẍe-î=ş ceng=şa bîye=n \textbf{be} \textbf{adizî} luwa=n \\ 
 \textsc{dem.prox.m.3sg.dir} in=here with.ez\textsc{.gen} agha\textsc{.m-sg.obl=3sg:PSR} quarrel\textsc{=3pl:NC} be\textsc{.pst.ptcp.m=cop.3sg.m:S} with desperation go\textsc{.pst.ptcp.m=cop.3sg.m:S} \\ 
\glt `He (Hayas) got into a quarrel with his master. He had left [town] in desperate mood.' \hfill[JH.62]
\z 

Another strategy for expressing manner adverbs\is{manner adverbs} is through a reduplicated construction consisting of the adjective and the noun deriving from it, see \textit{weşɫe weşɫeyî} in (\ref{ex.man.adv}).

\ea
\textit{pane weşɫe weşɫeyî yanekoɫêşa kerden.} \\ 
\gll p=ane weşɫe weşɫeyî yanekoɫê=şa kerde=n \\ 
at=\textsc{dem.dist.m.3sg.dir} well happy\_manner moving\_house\textsc{=3pl:A} do\textsc{.pst.ptcp.m=cop.3sg.m:O} \\ 
\glt `They left the house in a happy manner.' \hfill[JE.71] \label{ex.man.adv}
\z 

Other manner adverbs\is{manner adverbs} include \textit{kirj} `quickly', \textit{yewaşê} `slowly', etc. The latter is used predominantly as a discourse marker roughly equivalent in meaning to English\il{English} `then'.

\ea
\textit{kirj mila.} \\ 
\gll kirj mi-l-a \\ 
 quickly \textsc{ind-}go\textsc{.prs-3pl:S} \\ 
\glt `They went quickly.' \hfill[ŞC.68]
\z 

\end{sloppypar}
