\chapter{Nouns and nominal morphology} 


Nouns are marked for gender\is{gender}, number\is{number}, and case\is{case}, though gender\is{gender} is not expressed in the plural\is{plural}. These categories are represented on nouns via fusional suffixes. The inflection of a noun is predictable from the phonological shape of the base and from its gender. 

\section{Nominal inflection}\label{sect:nom-inflection}

The categories case\is{case}, number\is{number}, and gender\is{gender} can be marked morphologically on nouns. Each category involves two distinctions, thus singular\is{singular} and plural\is{plural} for number\is{number}, masculine\is{masculine} and feminine\is{feminine} for gender\is{gender} (only in singular\is{singular} number), and \textsc{direct}\is{direct case} and \textsc{oblique}\is{oblique case} for case\is{case}. 

The gender assignment is phonologically based (see \S\ref{gendersection}). For now, it suffices to say that the nouns ending in a consonant, stressed \textit{-é}, \textit{-í}, \textit{-\stackunder[-10pt]{\^{u}}{\'{}}}, and \textit{-ó} are masculine\is{masculine}. Additionally, some nouns ending in stressed \textit{-\'a} are masculine. On the other hand, all the nouns ending in \textit{-ê}, whether stressed or not, and nouns ending in unstressed \textit{-e} and \textit{-î} are feminine\is{feminine}. The class of feminine\is{feminine} nouns also includes a subset in stressed \textit{-á}.

As remarked, the inflection of a noun is predictable from the phonological shape of the base and from its gender. Masculine and feminine nouns are in their base form when singular\is{singular} and in the direct case\is{direct case}. Masculine nouns end in \textit{-î} and feminine nouns in \textit{-ê}, when singular and in the oblique case\is{oblique case}. In the plural, gender distinction is lost. The relevant inflectional suffixes are \textit{-ê} in the direct case\is{direct case}, and \textit{-a} in the oblique case\is{oblique case}, see Table \ref{tab:nom-infl-under}. 
\begin{table}

    \begin{tabular}{lllllll} 
    \lsptoprule
& \textsc{sg.dir}& \textsc{sg.obl}& \textsc{pl.dir} & \textsc{pl.obl} \\
\midrule
\textsc{m}& {-\O}& -î & \multirow{2}{*}{-ê}& \multirow{2}{*}{-a} \\
\textsc{f}& {-\O}& -ê & & \\
\lspbottomrule
\end{tabular}
    \caption{Nominal inflectional suffixes--underlying forms}
    \label{tab:nom-infl-under}
\end{table}
 

As seen in Table \ref{tab:nom-infl-under}, Feminine nouns feature syncretism\is{syncretism} in \textsc{sg.obl} and \textsc{pl.dir}, though some extend it to \textsc{sg.dir}, too (see below). Table \ref{nom-infl} illustrates the surface form of the inflectional suffixes when combined with the base-final segment.

\begin{table}

    \begin{tabular}{|c|c|c|c|c|c|c|c|c|c|c|c|}
     \hline
        & \multicolumn{6}{c|}{{masculine}} & \multicolumn{4}{c|}{ {feminine}} \\
        \hline
        Base-final  & \multirow{2}{*}{-C} & \multirow{2}{*}{-\stackunder[-10pt]{\^{i}}{\'{}}} & \multirow{2}{*}{-\stackunder[-10pt]{\^{u}}{\'{}}} & \multirow{2}{*}{-o} & \multirow{2}{*}{-\'e} & \multirow{2}{*}{-a} & \multirow{2}{*}{-a} & \multirow{2}{*}{-ê} & \multirow{2}{*}{-e} & \multirow{2}{*}{-î} \\
        segment & & & & & & & & & &\\
        \hline
        \textsc{sg.dir} & \multicolumn{10}{c|}{-\emptyset} \\  % Fixed the uppercase "C"
        \hline
        \textsc{sg.obl} & -î &-(î) &\multicolumn{4}{c|}{-y}  & -ɛ & -(ê) & -ê& -î/ -ê \\  % Added missing column
        \hline
        \textsc{pl.dir} & \multicolumn{5}{c|}{-ê} & \multicolumn{2}{c|}{-ɛ} & -(ê) & -ê& -î/ -ê \\
        \hline
        \textsc{pl.obl} & \multicolumn{5}{c|}{-a}&\multicolumn{2}{c|}{-ya} & \multicolumn{3}{c|}{-a}\\
        \hline
    \end{tabular}
      \caption{Nominal inflection--surface forms}
    \label{nom-infl}
\end{table}

The analysis of the inflectional morphology proposed in this book is different from that of \citet[14--16]{mackenzie_dialect_1966}{} proposed for the neighbouring Luhon Hewram\^i\il{Hewramî!Luhon}. MacKenzie suggests that there are three classes of nouns in Hewramî. His classification is based on the surface forms of the suffixes when combined with the base-final segment. However, as remarked, the form of the suffix is predictable from the base-final segment, which would suggest that the suffixes are phonologically conditioned allomorphs and that there is one underlying inflectional class. 

In what follows, we look at different citation forms of nouns and describe major morphophonemic processes that each citation form may undergo when combined with the nominal inflectional suffixes. For masculine nouns ending in a consonant, the underlying suffixes are used without any change in the citation form of the nouns:

\TabPositions{2cm,4cm,6cm}
\ea \textit{kuř} `boy, son'\\
\textsc{sg.dir}\tab  \textit{kuř-\O} \tab [ZQ.45] \\
\textsc{sg.obl}\tab  \textit{kuř-î} \tab [ZB.35] \\
\textsc{pl.dir}\tab  \textit{kuř-ê} \tab [ZB.9] \\
\textsc{pl.obl}\tab  \textit{kuř-a} \tab [JM.16] \\
\z

Masculine nouns ending in \textit{-\stackunder[-10pt]{\^{i}}{\'{}}} are inflected the same as consonant-final nouns, except the identical vowel in the \textsc{sg.obl} is deleted following the base-final segment \textit{î}:

\ea \textit{mizgî} `mosque' \\  
\textsc{sg.dir}\tab  \textit{mizgî-\O} \\
\textsc{sg.obl}\tab  \textit{mizgî} (< \textit{mizgî-î}) \\
\textsc{pl.dir}\tab  \textit{mizgî-ê} \\
\textsc{pl.obl}\tab  \textit{mizgî-a} \\
\z 

Masculine nouns ending in \textit{\stackunder[-10pt]{\^{u}}{\'{}}}, and \textit{-\'o} have identical inflection. Their inflection is the same as that of consonant-final nouns, except that the underlying \textsc{sg.obl} \textit{-î} is glided following the base-final segment.
\TabPositions{2cm,5cm,7cm}
\ea 
\tab \textit{şû} `husband'\tab  \textit{řo} `day'\\  
\textsc{sg.dir}\tab  \textit{şû-\O} \tab  \textit{řo-\O}\\
\textsc{sg.obl}\tab  \textit{şû-y} \tab  \textit{řo-y}\\
\textsc{pl.dir}\tab  \textit{şû-ê} \tab  \textit{řo-ê}\\
\textsc{pl.obl}\tab  \textit{şû-a} \tab  \textit{řo-a} \\
\z

For masculine nouns ending in \textit{\'e}, the underlying \textsc{sg.obl} \textit{-î} is glided following the base-final segment. Additionally, the final vowel of the base is dropped before the inflectional suffixes in the plural.
\ea \textit{yané} `house'\\  
\textsc{sg.dir}\tab  \textit{yane-\O}\tab  [ZP.85] \\
\textsc{sg.obl}\tab  \textit{yane-y} \tab [ZP.85] \\
\textsc{pl.dir}\tab  \textit{yanê} (< \textit{yane} + \textit{-ê})\\
\textsc{pl.obl}\tab  \textit{yana} (< \textit{yane} + \textit{-a}) \\
\z

The inflection of masculine nouns ending in their base form in \textit{a} feature gliding of the \textsc{sg.obl} \textit{-î}, merging of the base-final segment and the \textsc{pl.dir} \textit{-ê} into -\textit{ɛ}, and glide insertion in the \textsc{pl.obl}. 
\TabPositions{2cm,6cm,9cm,10cm}
\ea
\tab \textit{zema} `groom'\tab  \textit{pîya} `man' \\  
\textsc{sg.dir}\tab  \textit{zema-\O}\tab  \textit{pîya-\O} \\
\textsc{sg.obl}\tab  \textit{zema-y}\tab  \textit{pîya-y} \tab  [JP.262] \\
\textsc{pl.dir}\tab  \textit{zem{ɛ}} (< \textit{zema} + \textit{-ê})\tab  \textit{pîy{ɛ}} (< \textit{pîya} + \textit{-ê})\tab [JE.78]  \\
\textsc{pl.obl}\tab  \textit{zema-ya}\tab  \textit{pîya-ya} \tab [HM.39]\\
\z

Feminine nouns ending in \textit{a} are inflected the same as their masculine counterparts, except the underlying \textsc{sg.obl} merges with final-base segment \textit{a}, resulting in \textit{ɛ} or \textit{e}.
\TabPositions{1.5cm,6.5cm,9cm}
\ea
\tab \textit{eya/eđa} `mother' \tab \textit{dega} `village' \\  
\textsc{sg.dir} \tab  \textit{eya} [JH.2]  \tab \textit{dega} [ZP.46]\\
\textsc{sg.obl} \tab  \textit{eye, ey{ɛ}} (< \textit{eya} + \textit{-ê}) [KŞ.1]  \tab \textit{deg{ɛ}} (< \textit{dega} + \textit{-ê}), \textit{dege} [KŞ.59] \\
\textsc{pl.dir} \tab  \textit{ey{ɛ}} (< \textit{eya} + \textit{-ê})  \tab \textit{deg{ɛ}} (< \textit{dega} + \textit{-ê})\\
\textsc{pl.obl} \tab  \textit{eya-ya} \tab \textit{dega-ya} \\
\z


In the inflection of feminine nouns ending in -ê, the underlying \textsc{sg.obl} and \textsc{pl.dir} \textit{-ê} are merged into the identical base-final vowel. In the \textsc{pl.obl}, the base-final vowel is dropped before \textit{-a}.
\TabPositions{1.5cm,6.75cm,9cm}
\ea
\tab \textit{kinaç\stackunder[-10pt]{\^{e}}{\'{}}} `girl' \tab \textit{qis\stackunder[-10pt]{\^{e}}{\'{}}} `word'\tab  \\  
\textsc{sg.dir}\tab  \textit{kinaçê} [KŞ.92]  \tab \textit{qisê} [ZP. 46] \\
\textsc{sg.obl}\tab  \textit{kinaçê} (< \textit{kinaçê} + \textit{-ê}) [RE.49]  \tab \textit{qisê} (< \textit{qisê} + \textit{-ê}) [JP.150]  \\
\textsc{pl.dir}\tab  \textit{kinaçê} (< \textit{kinaçê} + \textit{-ê}) [JH.21]  \tab \textit{qisê} (< \textit{qisê} + \textit{-ê}) [HB.46]  \\
\textsc{pl.obl}\tab  \textit{kinaça} (< \textit{kinaçê} + \textit{-a}) [JM.16]  \tab \textit{qisa} (< \textit{qisê} + \textit{-a}) [HB.44]  \\
\z

In the inflection of feminine nouns ending in \textit{-e}, the base-final segment is dropped before the inflectional suffixes for \textsc{sg.obl}, \textsc{pl.dir}, and \textsc{pl.obl}.
\TabPositions{1.5cm,6cm,9cm}
\ea
\tab \textit{bíze} `goat'\tab  \textit{y\'ewe} `barley' \\  
\textsc{sg.dir}\tab  \textit{bize-\O} [KŞ.28]  \tab \textit{yewe-\O}\\
\textsc{sg.obl}\tab  \textit{bizê} (< \textit{bize} + \textit{-ê}) \tab \textit{yewê} (< \textit{yewe} + \textit{-ê}) \\
\textsc{pl.dir}\tab  \textit{bizê} (< \textit{bize} + \textit{-ê}) [ZB.27]  \tab \textit{yewê} (< \textit{yewe} + \textit{-ê}) [JP.45] \\
\textsc{pl.obl}\tab  \textit{biza} (< \textit{bize} + \textit{-a}) [JP.22]  \tab \textit{yewa} (< \textit{yewe} + \textit{-a}) [JP.29]  \\
\z

Finally, the inflection of the feminine nouns ending in \textit{-î} seems to depend on the lexical semantics of the nouns. For inherently mass nouns, e.g., \textit{mekî} `salt', \textit{hardî} `flour', the singular forms and the plural direct form may be identical (see \S\ref{inh-pl-section}), whereas in the \textsc{pl.obl}, the base-final vowel is dropped before the inflectional suffix. For the rest of the nouns, the base-final -î is dropped before the inflectional suffixes for \textsc{sg.obl}, \textsc{pl.dir}, and \textsc{pl.obl}, same as the inflection of feminine nouns ending in \textit{-e}. 
\TabPositions{1.5cm,6.5cm,9cm}
\ea
\tab \textit{h\'ardî} `flour'\tab  \textit{j\'enî} `woman'\tab  \\  
\textsc{sg.dir}\tab  \textit{hardî}  \tab \textit{jenî} [ZP.115]  \\
\textsc{sg.obl}\tab  \textit{hardî} [HB.13] \tab \textit{jenê} (< \textit{jenî} + \textit{-ê}) [HS.77]\\
\textsc{pl.dir}\tab  \textit{hardî} \tab \textit{jenê} (< \textit{jenî} + \textit{-ê}) \\
\textsc{pl.obl}\tab  \textit{harda} (< \textit{hardî} + \textit{-a}) [HB.72] \tab \textit{jena} (< \textit{jenî} + \textit{-a}) [RE.59]  \\
\z

The above-mentioned paradigms show the shape of cases in each paradigm. They do not indicate the conditions under which a particular case is realised. For example, a feminine direct object may be in the direct case or in the oblique case. This is a matter of syntax and information structure and will be treated under differential argument flagging in \S\ref{sect:arg_marking}. 

\subsection{Gender}\label{gendersection}
\subsubsection{Gender assignment}
Nouns are overtly marked for the category of gender\is{gender}. Tekht Hewramî has two genders, masculine\is{masculine} and feminine\is{feminine}. Gender\is{gender} is assigned to nouns primarily based on the ending the nouns take. In other words, the language has a phonological assignment system \citep{corbett_gender_1991}. The phonological gender assignment rule also applies to nouns with biological gender (see below). It is, however, notable that in some cases, semantic and morphological clues play a role in gender assignment\is{gender assignment}. For instance, mass nouns overwhelmingly have feminine gender (see \S\ref{inh-pl-section}). The gender assignment of loanwords seems to be primarily determined by semantics, which in turn results in phonological adaptation (see \S\ref{sect:loanword}). Similarly, derived place names are feminine, e.g., \textit{awîrga} `fireplace' (< \textit{awîr} `fire' (\textsc{m.} + \textit{-ga} `place'). The phonological assignment system can lead to near-minimal pairs which have different genders.
\TabPositions{1.5cm,4cm,6cm}
\ea
    \textit{her\'e}\tab  `donkey' \tab  \textit{h\'eře}\tab  `mud'\\
    \textit{mer\'e}\tab  `meadow' \tab  \textit{m\'eře}\tab  `cave' \\
\z

The gender assignment\is{gender assignment} rule was explained above. Table \ref{gender-ass} summarises gender assignment\is{gender assignment} based on the nominal endings.

\begin{table}[htp!]

    \begin{tabular}{ccc}
 \lsptoprule
& Masculine\is{masculine}& Feminine\is{feminine} \\ 
 \midrule
&-C & \\
& -é& -e \\
& -\stackunder[-10pt]{\^{i}}{\'{}}& -î \\
& -ó& -ê, -\stackunder[-10pt]{\^{e}}{\'{}}\\
& -á & -á \\
& -\stackunder[-10pt]{\^{u}}{\'{}}& \\
 \lspbottomrule
    \end{tabular}
    \caption{Gender assignment\is{gender assignment} based on the nominal endings}
    \label{gender-ass}
\end{table}

Examples of masculine\is{masculine} nouns ending in:
\begin{itemize}
    \item[-C]  \textit{hamin} `summer'; \textit{hewr} `cloud'; \textit{asman} `sky'; \textit{qiřoɫ} `tree hollow'; \textit{daresan} `woodland'; \textit{nan} `bread'; \textit{şar} `town'; \textit{asaw} `mill'; \textit{pos} `skin'; \textit{gîyan} `soul'; \textit{deɫek} `stone marten'.
    \item[-é] \textit{nere} `oak tree'; \textit{sayqe} `lightening'; \textit{yane} `house'; \textit{meme} `breast'; \textit{hane} `water spring'; \textit{şuwane} `shepherd'; \textit{hêɫe} `egg';  \textit{gorewe} `sock'; \textit{peme} `cotton'; \textit{laşe} `body'; \textit{çîçe} `breast'; \textit{qaɫîçe} `carpet'; \textit{derwaze} `gate'; \textit{gele} `herd'; \textit{duçerxe} `bicycle'.
    \item[-î́] \textit{mizgî} `mosque'; \textit{aweyanî} `inhabited place'; \textit{sînî} `tray'; \textit{gonî} `udder'; \textit{gicî} `shirt'; \textit{dewrî} `plate'.
    \item[-û́] \textit{şû} `husbnad'; \textit{nîrû} `force'; \textit{herû} `gum'; \textit{heɫû} `eagle'; \textit{petû} `blanket'
    \item[-ó]  \textit{řo} `day'; \textit{ko} `mountain'; \textit{yo} `one (person)'; \textit{lalo} `maternal uncle'.
    \item[-á ] \textit{pîya} `man'; \textit{zema} `groom'; \textit{xaneqa} `monastry'; \textit{geɫa} `leaf'; \textit{ijdeha} `serpent'; \textit{poɫa} `steel'; \textit{tiɫa} `gold'; \textit{lempa} `lamp'.
\end{itemize}

Examples of feminine\is{feminine} nouns ending in:
\begin{itemize}
    \item[-ế] \textit{yagê} `place'; \textit{qisê} `talk'; \textit{pîfê} `burr'; \textit{namê} `name'; \textit{xeplê} `bread (of cereal other than wheat) baked in oven'; \textit{manê} `bleached skin bag'
    \item[-ê] \textit{yerê} `three'; \textit{qijê} `hair'; \textit{ađê} `they'
    \item[-e] \textit{çawre }`tent'; \textit{bize} `goat'; \textit{meře} `cave'; \textit{eşkewte} `cave'; \textit{yewe} `barley'; \textit{werwe} `snow'; \textit{leme} `stomach'; \textit{tace} `crown'; \textit{hesere} `mule'; \textit{beye} `quince'; \textit{ʕinne} `buttock'; \textit{daɫane} `gate-house'; \textit{şewe} `night'; \textit{piştîye} `back support'; \textit{neware} `cassette'; \textit{lase} `hoard of grass'; \textit{çaɫe} `ditch, pit'; \textit{tenûre} `oven'; \textit{deşte} `field'; \textit{penîre} `cheese'; \textit{tewerge} `hail'; \textit{werêse} `rope'; \textit{tîre} `arrow';  \textit{lelûwe} `cradle'; \textit{řûwe} `face'; \textit{esɫehe} `gun'; \textit{şike} `doubt'; \textit{nuweye} `chickpea'
    \item[-î] \textit{hêzmî} `wood'; \textit{awî} `water'; \textit{harđî} `flour'; \textit{genmî} `wheat'; \textit{bencanî} `tomato'; \textit{winî }`blood'; \textit{çaştî} `food'; \textit{pirđî} `bridge'; \textit{kawetirî} `pigeon'; \textit{karđî} `knife'; \textit{keştî} `boat'; \textit{meşî} `fly'; \textit{pişqelî} `sheep dung'
    \item[-á] \textit{dega} `village'; \textit{řa} `road'; \textit{seringa} `pillow'; \textit{ʕeba} `robe'; \textit{dinya} `world'; \textit{qeɫa} `castle'; \textit{çira} `light bulb'; \textit{ʕasa} `crook'; \textit{îştîba} `mistake'; \textit{siđa} `voice'
\end{itemize}

Some nouns exhibit variation in gender assignment\is{gender assignment} depending on the dialects. In the vernacular of Hewraman Tekht, \textit{hesar\'e} `fence' is masculine\is{masculine}. In the Nwên vernacular, the cognate form ends in unstressed -î: \textit{hesárî}, which is feminine\is{feminine}. 

Nouns ending in \textit{-y} are typically masculine\is{masculine}, including as well verbal nouns, e.g., \textit{çay} `tea'; \textit{aw\'ay} `village, habitat'. An interesting case is \textit{nuweye} `chickpea' (\textsc{f}), for which the cognate form in Luhon Hewramî is \textit{nuwey} (\textsc{f}) according to \citet[]{mackenzie_dialect_1966}. \textit{nuwey} ending in a consonant is an exception to the gender assignment\is{gender assignment} rule in Luhon Hewramî \il{Hewramî!Luhon}. This apparent anomaly with the gender assignment\is{gender assignment} rule might be solved if the final \textit{-y} on \textit{nuwey} be taken as the unstressed \textit{-î}, shifting to \textit{-y} to avoid hiatus with \textit{-e}. Note that the plural\is{plural} form is also \textit{nuwey}, behaving thus like a subset of feminine\is{feminine} nouns ending in \textit{-î}. In Tekht Hewram\^i, this apparent anomaly has been dealt with differently. The form \textit{nuweye} has been attested in the speech of the young generation, who seem to have regularised the class membership of \textit{nuwey} based on feminine\is{feminine} nouns ending in \textit{-e}.  

Another apparent exception to gender assignment\is{gender assignment} is \textit{kité} `cat', which is feminine\is{feminine} despite ending in stressed \textit{-é}. As noted by \citet[9]{mackenzie_dialect_1966}{}, the underlying stress is on the central vowel \textit{i}. Following the reduction of \textit{i}, the stress shifts to the right, hence the stressed \textit{-é} in \textit{kité}.

For kinship terms, the phonological assignment system overlaps with biological sex. Therefore, nouns denoting females are feminine\is{feminine} and nouns denoting males are masculine\is{masculine}. In other words, nouns denoting to females have the endings and stress patterns associated with feminine nouns, and nouns denoting males have endings and stress pattern associated with masculine nouns.
\TabPositions{1.5cm,4.5cm,6.75cm}
\ea
Feminine\is{feminine}\tab  \tab  Masculine\is{masculine}\tab  \\
\textit{mamá }\tab  `grandmother'\tab \textit{babá }\tab `grandfather' \\  
\textit{eđ\'a}\tab  `mother'\tab  \textit{taté}\tab `father' \\
\textit{waɫ\stackunder[-10pt]{\^{e}}{\'{}}}\tab  `sister'\tab  \textit{birá }\tab `brother' \\
\textit{hes\'irwe}\tab `mother-in-law'\tab  \textit{hesûr\'e}\tab `father-in-law' \\
\textit{dêđ\stackunder[-10pt]{\^{e}}{\'{}}}\tab `step-mother'\tab \textit{baba pîyaré}\tab `step-father' \\
\textit{met\stackunder[-10pt]{\^{i}}{\'{}}ye}\tab `father's sister'\tab  \textit{mam\'o}́ \tab `father's brother' \\
\textit{met\stackunder[-10pt]{\^{i}}{\'{}}ye}\tab `mother's sister'\tab  \textit{lal\'o}\tab `mother's brother' \\
\textit{sêt\stackunder[-10pt]{\^{e}}{\'{}}}\tab `husband's sister'\tab \textit{hêw\'er}\tab  `husband's brother' \\
\textit{biraj\'enî}\tab `brother's wife'\tab  \textit{jenbir\'a }\tab `wife's brother' 
\z

The last pair clearly illustrate that the gender assignment is semantically motivated: \textit{birajenî} `brother's wife' is a female person, hence gender assignment is feminine. 

In compound nouns whose elements have different genders\is{gender}, gender\is{gender} is assigned based on the second element. For example, \textit{hêɫeřûwenî} `fried eggs', is a compound noun with feminine\is{feminine} gender. It is composed of < \textit{hêɫe} `egg' (\textsc{m}) + \textit{řûwenî} `oil' (\textsc{f}). This rule also applies where the coordinator conjunction =\textit{û} `and' (see \S\ref{sect:conjunction}) joins the elements in an echo compound. For example, \textit{awû çîwî} `water and things like it' (cf. [JE.7]) is a masculine\is{masculine} compound comprised of \textit{awî} (\textsc{f}) the coordinator particle \textit{=û}, and the masculine\is{masculine} noun (in oblique case\is{oblique case}) \textit{çîwî} `thing'. Another example is coordinate noun phrases like \textit{eđa=w tate} `mother and father', composed of different genders. The gender agreement\is{gender agreement} here can be based on the second coordinate noun. 
\ea
\textit{eđaw tatew jenê merđ} \\
\gll eđa=w \textbf{tate}-û jenê merđ-\textbf{\O} \\
mother=\textsc{and} father-\textsc{ez.gen} wife\textsc{.f.sg.obl} die.\textsc{pst-3sg.m} \\
\glt `The wife's parents died.' \hfill [XX.110]
\z 

Likewise, gender distinction is phonologically coded for some animals. 
\ea \label{ex.animalphon}
Feminine\is{feminine}\tab  \tab  Masculine\is{masculine}\tab \\
\textit{gawe}\tab  `cow'\tab \textit{gaw}\tab `bull' \\  
\textit{bizɫê}\tab  `female kid-goat'\tab  \textit{bizɫe}\tab  `male kid-goat' \\
\z

For a subset of animals, gender\is{gender} distinction is semantically encoded. This would include using different lexical items for expressing gender\is{gender} distinctions between females and males:\footnote{See also \citet[14, ff.2]{mackenzie_dialect_1966}}{}
\TabPositions{2cm,6cm,8cm}
\ea \label{split}
Feminine\is{feminine}\tab  \tab  Masculine\is{masculine}\tab \\
\textit{jerejî}\tab `female partridge'\tab  \textit{beq}\tab `male partridge' \\
\textit{meye}\tab `sheep'\tab  \textit{beran}\tab  `ram' \\
\textit{kerge}\tab  `hen'\tab  \textit{keɫeşîr}\tab  `rooster' \\
\textit{mayînî}\tab  `mare'\tab  \textit{esb}\tab  `horse' \\
\textit{qişqêre}\tab `female raven'\tab  \textit{qaɫawe}\tab  `male raven' \\
\textit{bize}\tab `goat'\tab \textit{sabrîn}\tab `male goat' \\
 \textit{bize neçîre}\tab  `female mountain goat'\tab  \textit{keɫ}\tab  `male mountain goat'\\
\z 

A subset of semantically-encoded gender\is{gender} nouns are morpho-phonologically encoded. Thus, the difference between \textit{her} `donkey' and \textit{ma-her-e} `female donkey' is expressed by the prefix \textit{ma-} `female' combined with the feminine suffix \textit{-e}.

For some animals, a fixed gender\is{gender} is used to encompass both male and female members:
\TabPositions{2.5cm}
\ea \label{nosplit}
\textit{verg} (\textsc{m})\tab  `wolf' \\
\textit{şêr} (\textsc{m})\tab  `lion' \\
\textit{meřekur} (\textsc{m})\tab  `grasshopper' \\ 
\textit{hewrêşe} (\textsc{m})\tab  `rabbit' \\
\textit{řûwáse} (\textsc{f})\tab  `fox' \\
\textit{héşe} (\textsc{f})\tab  `bear' \\
\textit{kité} (\textsc{f})\tab  `cat' 
\z 

As seen, the animals in (\ref{ex.animalphon})--(\ref{split}) have a gender split, whereas those in (\ref{nosplit}) do not. This split cannot be easily accounted for in terms of the animacy hierarchy, which is generally understood as large animals with which humans interact regularly are higher in animacy, which would consequently be reflected by the gender split in languages. The difference between the setS in (\ref{ex.animalphon})--(\ref{split}) and the one in (\ref{nosplit}) seems to come from the degree of closeness of animals to the immediate human environment, or being traditionally part of the food chain, e.g., in the case of `partridge' or `mountain goat'. 

As said above, the morphological shape of words also has a role in gender assignment\is{gender assignment}. This concerns morphologically derived nouns, whose gender is determined by the derivational suffix (see \S\ref{nominal-affixation}) added to the noun. For instance, \textit{awîrga} `fireplace', a feminine\is{feminine} noun, is composed of \textit{awîr} `fire' (\textsc{m}) and the feminine\is{feminine} suffix \textit{-ga} `place'. Some tendencies suggest themselves:
\begin{itemize}
    \item Abstract nouns formed from other nouns and adjectives by the derivational suffix \textit{-î} are masculine\is{masculine}: 
\ea
\textit{sextî} \tab  `difficulty' \\
\textit{weşî}\tab  `happiness' \\
\textit{hemahengî} \tab  `collaboration' \\
\textit{neweşî}\tab  `illness' \\
\textit{dewayî}\tab  `medication' \\
\z

\item Abstract nouns formed with the suffix \textit{-gerî} are masculine\is{masculine}: 
\ea
\textit{hewramîgerî} \tab  `Hewram\^ihood' \\
\textit{gewregerî} \tab  `nobleness' \\
\textit{pîyagerî}\tab  `manhood' \\
\textit{aẍegerî}\tab  `lordship' \\
\textit{paşagerî}\tab  `kingship' \\
\z

\item Place names formed with \textit{-ga/ -ge} are feminine\is{feminine}:
\TabPositions{2cm,4cm}
\ea
\textit{dega}\tab  `village'\tab  < \textit{de} `village' + \textit{-ga} `place' \\
\textit{awîrga}\tab  `fireplace'\tab  < \textit{awîr} (\textsc{m}) `fire' + \textit{-ga} `place' \\
\textit{seringa}\tab  `pillow'\tab  < \textit{serin} (\textsc{m})+ \textit{-ga} `place' \\
\textit{ewêge}\tab  `there'\tab  < \textit{awê} `that' + \textit{-ge} `over there' \\
\textit{coge}\tab  `stream'\tab  < *\textit{cû} `stream' + \textit{-ge} `place' \\
\z

\newpage
\item Place names ending in \textit{-xane} are masculine\is{masculine}. 
\ea
\textit{kitêbxane} \tab  `library' \tab  < \textit{kitêb} `book' + \textit{-xane} `place' \\
\textit{darûxane} \tab  `pharmacy' \tab  < \textit{darû} `medicine' + \textit{-xane} `place' \\
\textit{eđebxane} \tab  `toilet' \\
\textit{řoxane} \tab  `river' \\ 
\textit{aşpezxane} \tab  `kitchen' \\
\z

\item Nouns ending in \textit{-waɫe} are masculine\is{masculine}. Examples: 
\ea
\textit{zerdewaɫe}\tab `bee'\tab  < \textit{zerd} `yellow' + \textit{-e} + \textit{waɫe} `wing' \\
\textit{tûwaɫe}\tab  `eggshell'\tab  \tab  \\
\z

\item Nouns of meals ending in \textit{-îne} are feminine\is{feminine}:
\TabPositions{2.5cm}
\ea
\textit{şelemîne}\tab  `dish made of cracked wheat and turnip' \\
\textit{dowîne}\tab  `dish made of cracked wheat and diluted yoghurt' \\
\textit{xeple zeřatîne}\tab  `a type of bread made of corn flour' \\
\z
\end{itemize}

Semantic factors also play a role in gender assignment\is{gender assignment}. Small entities, e.g., small fruits and grains, tend to be feminine\is{feminine}. The nouns in this class are further phonologically marked to be feminine\is{feminine}. An exception is \textit{tif\stackunder[-10pt]{\^{i}}{\'{}}} `mulberry', which should be masculine\is{masculine} according to the phonological assignment rule. This apparent anomaly seems to be motivated by the loss of the central vowel in \textit{tif\stackunder[-10pt]{\^{i}}{\'{}}} and the stress shift to \textit{î}. Alternatively, one might reflect that semantic criteria take precedence over phonological criteria in the gender assignment\is{gender assignment} of tiny fruit. 
\ea
\textit{mîjûyî}\tab  `lentil' \\
\textit{nuweye}\tab  `chickpea' \\
\textit{genmî}\tab  `wheat' \\
\textit{yewe, yewê}\tab  `barley' \\
\textit{tifî}\tab  `mulberry' \\
\textit{wamî}\tab  `almond' \\
\textit{wezî}\tab  `walnut' \\
\textit{çeqalê} \tab  `bitter almond' \\
\textit{hengûrî}\tab  `grapes' \\
\textit{şêɫanê}\tab  `apricot' \\
\textit{heştaɫûyî}\tab  `plum' \\
\textit{tifɫe şêxanê}\tab  `strawberry' \\
\z

It is notable that \textit{lûbya} `beans' and \textit{maş} `black lentils' are masculine\is{masculine} and, thus, exceptions to the generalisation made above. These nouns appear to be recent borrowings into Hewramî\il{Hewramî}, probably from Persian\il{Persian}. The gender assignment\is{gender assignment}, at least for \textit{maş} follows from the default phonological assignment rule.

\subsubsection{Gender\is{gender} of loanwords\is{loanwords}} \label{sect:loanword}

Words borrowed from other languages acquire gender\is{gender} in Tekht Hewram\^i\il{Hewramî!Tekht}. Arabic loanwords\is{loanwords} generally constitute an earlier layer of loans in the language, especially the ones related to the realm of religion. Arabic\il{Arabic} loans denoting abstract concepts are borrowed predominantly as feminine\is{feminine} nouns, regardless of the original gender\is{gender} (see below for some exceptions). As can be seen, these words are often borrowed with an extra unstressed vowel, which is not present in the source word. The category of nouns exhibiting this trait also includes the temporal nouns such as \textit{seʕbe} `morning'. This reflects that there is a semantic basis for gender assignment with loanwords\is{gender assignment}. However, it is notable that the great majority of abstract nouns are formed via the native derivational suffixes \textit{-î} and \textit{-gerî} which assign masculine gender, e.g., \textit{sextî} `difficulty', \textit{gewregerî} `nobleness' (see \S\ref{gendersection}); therefore no claim can be made about an underlying semantic basis, already operating in the native lexicon, for assigning feminine gender to abstract nouns borrowed from Arabic.
\TabPositions{2cm,5cm,7cm}
\ea
\textit{temaʕe} (\textsc{f}) \tab  `greed'\tab  [JE.58] \tab  cf. Ar.\il{Arabic} \textit{ṭamaʕiyya} (\textsc{f}) \\
\textit{ʕemre} (\textsc{f})\tab  `order'\tab  \tab cf. Ar.\il{Arabic} \textit{ʕamr} (\textsc{m}) \\
\textit{řeza} (\textsc{f})\tab  `satisfaction'\tab  [JE.83] \tab  cf. Ar.\il{Arabic} \textit{riḍaʔ} (\textsc{m}) \\
\textit{qisê} (\textsc{f})\tab  `remark'\tab  [ZB.59] \tab  cf. Ar.\il{Arabic} \textit{qiṣṣa }(\textsc{f}) \\
\textit{dawa} (\textsc{f})\tab  `plea'\tab  [DP.13] \tab  cf. Ar.\il{Arabic} \textit{daʕwa} (\textsc{m}) \\
\textit{dinya} (\textsc{f})\tab  `world'\tab  [ZP.128] \tab  cf. Ar.\il{Arabic} \textit{dunya} (\textsc{f}) \\
\textit{duʕa} (\textsc{f})\tab  `praying'\tab  [DG.6] \tab  cf. Ar. \textit{duʕa} (\textsc{m}) \\
\textit{şerte} (\textsc{f})\tab  `condition'\tab  [ŞC.98] \tab  cf. Ar.\il{Arabic} \textit{şarṭ} (\textsc{m}) \\
\textit{weʕze} (\textsc{f})\tab  `situation'\tab  [DG.34] \tab  cf. Ar.\il{Arabic} \textit{waḍʕ} (\textsc{m}) \\
\textit{ʕefwe} (\textsc{f})\tab  `mercy'\tab  [DG.4] \tab  cf. Ar.\il{Arabic} \textit{ʕafw} (\textsc{m}) \\
\textit{sinʕe} (\textsc{f})\tab  `age'\tab  [BP.123] \tab  cf. Ar.\il{Arabic} \textit{sinn} (\textsc{m}) \\
\textit{seʕbe} (\textsc{f})\tab  `morning'\tab   \tab  cf. Ar.\il{Arabic} \textit{s\textsuperscript{ʕ}abāħ} (\textsc{m}) 
\z

Exceptions to the above generalisation can be found. In such cases, Hewramî\il{Hewramî} assigns gender\is{gender assignment} to the Arabic loan based on its phonological system of gender assignment\is{gender assignment}. Thus, if an abstract feminine\is{feminine} noun in the source language ends in a consonant, Hewramî\il{Hewramî} seems to treat it as masculine\is{masculine}. 

\ea
\textit{xizmet (\textsc{m})} \tab  `service'\tab  [RE.12] \tab  cf. Ar.\il{Arabic} \textit{xidmat} (\textsc{f}) \\ 
\textit{şerîʕet} (\textsc{m})\tab  `Sharia law’\tab  [JP.84] \tab  cf. Ar.\il{Arabic} \textit{şariʕat} (\textsc{f}) \\
\textit{terîqet} (\textsc{m})\tab  `denomination’\tab  [JP.84] \tab  cf. Ar.\il{Arabic} \textit{tariqat} (\textsc{f}) \\
\textit{wiɫat} (\textsc{m})\tab  `region, country’\tab  \tab  cf. Ar.\il{Arabic} \textit{wilayat} (\textsc{f}) \\
\textit{wext} (\textsc{m})\tab  `time’\tab  \tab  cf. Ar.\il{Arabic} \textit{waqt} (\textsc{m}) \\
\textit{tekbîr} (\textsc{m})\tab  `planning’\tab  \tab   cf. Ar.\il{Arabic} \textit{tadbîr} (\textsc{m}) \\
\textit{fikr} (\textsc{m}) \tab  `thought' \tab \tab  cf. Ar.\il{Arabic} \textit{fikr} (\textsc{m}) \\
\z

Elsewhere, the gender of Arabic loans is retained in Hewramî as long as the noun endings correspond in gender\is{gender}. 
\ea
\textit{ħegaɫ} (\textsc{m})\tab  `scarf’\tab  \tab cf. Ar.\il{Arabic} \textit{ʕîqal} (\textsc{m}) \\
\textit{ʕeba} (\textsc{f})\tab  `robe’\tab  [BP.187] \tab  cf. Ar.\il{Arabic} \textit{ʕaba} (\textsc{f}) \\
\textit{lêfe} (\textsc{m})\tab  `quilt’\tab  \tab  cf. Ar.\il{Arabic} \textit{lîhaf} (\textsc{m}) \\
\textit{xet} (\textsc{m}) \tab  `line' \tab \tab  cf. Ar.\il{Arabic} \textit{xat} (\textsc{m}) \\
\textit{qifɫ} (\textsc{m}) \tab  `lock' \tab \tab  cf. Ar.\il{Arabic} \textit{qufl} (\textsc{m}) \\
\textit{cins} (\textsc{m}) \tab  `material, stuff' \tab \tab  cf. Ar.\il{Arabic} \textit{jins} (\textsc{m}) \\
\z

\textit{wezî} (\textsc{f}) `walnut’ is an exception, apparently borrowed from the Arabic \textit{jawz} (\textsc{m}). The lack of gender\is{gender} correspondence here seems to have a semantic reason since, as seen in \S\ref{gendersection}, small fruits are feminine\is{feminine} in Hewramî\il{Hewramî}. Alternatively, it is possible that Arabic \textit{jawz} was borrowed from Middle Iranian, which would explain the feminine marking of the native word \textit{wezî} as being part of small fruits. 

Persian\il{Persian} provides the primary source for modern words in Tekht Hewram\^i\il{Hewramî!Tekht}. Names denoting new objects are usually assigned a gender based on their ending. Thus, consonant-final nouns are masculine\is{masculine}. 
\TabPositions{3.5cm}
\ea
\textit{lêwan} (\textsc{m})\tab  `cup'\\
\textit{maşîn} (\textsc{m})\tab  `car’ \\
\textit{televizyon} (\textsc{m})\tab  `television’ \\
\textit{telefon} (\textsc{m})\tab  `telephone’ \\
\textit{mebal, mubayl} (\textsc{m})\tab  `mobile phone’ \\
\textit{cîp} (\textsc{m})\tab  `jeep’ \\
\textit{sîmkart} (\textsc{m})\tab  `SIM card’ \\
\textit{aɫbum} (\textsc{m})\tab  `album’ \\
\textit{vaksen} (\textsc{m})\tab  `vaccine’ \\
\textit{mahware} (\textsc{m})\tab  `satellite’ \\
\textit{zilûbeya} (\textsc{m})\tab  `Zalabiyeh’ \\
\textit{desga} (\textsc{m})\tab  `machine’ \\
\textit{duçerxe} (\textsc{m})\tab  `bicycle' \\
\z

An exception is modern place names such as `restaurant’ and `hotel’, which are borrowed through regional languages. These items end in a consonant in Kurdish\il{Kurdish} and Persian\il{Persian} but appear with an added unstressed \textit{-e} in Hewramî\il{Hewramî}, thus feminine\is{feminine}. This exception can be understood because the names of places are generally feminine\is{feminine}, including all the ones derived from the place suffix \textit{-ga} (see \S\ref{gendersection}).
\TabPositions{2.5cm}
\ea
\textit{ristûřane} (\textsc{f})\tab  `restaurant’ \\
\textit{hutêle} (\textsc{f})\tab  `hotel’ \\
\textit{sendelîye} (\textsc{f})\tab  `chair’     
\z

\subsubsection{Functions of the base-final feminine\is{feminine} morphemes}
The base-final feminine\is{feminine} morphemes additionally express the female counterpart of masculine\is{masculine} nouns denoting occupations, and derive fruit names from the corresponding fruit tree. These functions are discussed below.

\subsubsubsection{Expressing the female counterpart of masculine\is{masculine} nouns}

A feminine\is{feminine} noun can express the female counterpart of masculine\is{masculine} general nouns and occupational titles (see also \citealt{sadjadi_grammatical_2019}): 
\TabPositions{2.5cm,5cm}
\ea
\tab  Masculine\is{masculine}\tab  Feminine\is{feminine}\\
`baker'\tab  \textit{nanpeç}\tab  \textit{nanpeçe} \\
`cook'\tab  \textit{çaçker}\tab  \textit{çaçkere} \\
`patient'\tab  \textit{neweş}\tab  \textit{neweşe} \\
`physician'\tab  \textit{duktir}\tab  \textit{duktire} \\
`worker'\tab  \textit{kareker}\tab  \textit{karekere} \\
`miller'\tab  \textit{asawan}\tab  \textit{asawane} \\
\z

This list can be extended to some items in the kin domain:
\ea
`betrothed'\tab  \textit{desgîran} (\textsc{m})\tab  \textit{desgîrane} (\textsc{f}) 
\z
\subsubsubsection{Deriving the fruit names}

The feminine\is{feminine} suffixes \textit{-î} and \textit{-\stackunder[-10pt]{\^{e}}{\'{}}} derive some fruit names from the corresponding fruit tree. This can be seen in the derivation of the `fig' (\textsc{f}) from `fig tree' (see \ref{ex.fig}). Note, however, that in most cases, both the tree and fruit names share the same gender\is{gender}, rendering it hard to determine the direction of derivation (see \ref{ex.tree}). 
\TabPositions{2cm,5.75cm,8cm}
\ea \label{ex.fig}
\textit{hencîr} (\textsc{m})\tab  `fig tree'\tab  \textit{hencîrî} (\textsc{f})\tab  `fig' 
\z


\ea \label{ex.tree}
\textit{sawî} (\textsc{f})\tab  `apple tree'\tab  \textit{sawî} (\textsc{f})\tab  `apple' \\
\textit{çeqalê} (\textsc{f})\tab  `green apricot tree'\tab  \textit{çeqalê} (\textsc{f})\tab  `green apricot' \\
\textit{wezî} (\textsc{f})\tab  `walnut tree'\tab  \textit{wezî} (\textsc{f})\tab  `walnut' \\
\textit{wamî} (\textsc{f})\tab  `almond tree'\tab  \textit{wamî} (\textsc{f})\tab  `almond' \\
\z 

\subsubsection{Gender agreement\is{gender agreement}} \label{sect:gend-agreement}

So far, we have seen that phonological, semantic, and morphological criteria can predict the gender\is{gender} of a noun. In addition, the gender\is{gender} of a noun is expressed by formal variation on agreeing elements, or agreement targets. At the level of noun phrases, agreement in gender\is{gender agreement} can be found on two levels: (i) agreement between syntactically associated words (e.g., a noun and an adjective) (ii) the gender-sensitive allomorphy of certain suffixes, e.g., the definiteness suffix.

Adjectives agree in gender\is{gender agreement} with the nouns they modify both attributively and predicatively. In the following examples, the attributive adjective agrees with the head nouns in gender (see below for gender agreement with predicative adjectives).  
\ea
\textit{kuřê cuwanxas} \\
\gll kuř-ê cuwanxas-∅ \\
boy.\textsc{m-indf} good\_looking\textsc{-m} \\
\glt `a good-looking boy' \hfill [KŞ.68] 
\z

\ea
\textit{jenê xase} \\
\gll jenî-ê xas-e \\
wife\textsc{.f-indf} charming\textsc{-f} \\
\glt `a charming wife' \hfill [JH.64] 
\z
 
The definite suffix\is{definite suffix} \textit{-eke} takes on the gender\is{gender} of the singular\is{singular} noun it attaches to. When it appears on a masculine\is{masculine} noun, it takes the forms \textit{-eke} in the direct case\is{direct case} and \textit{-ekey} in the oblique case\is{oblique case}. With feminine\is{feminine} nouns, the addition of suffix \textit{-ekê} can results in the deletion of the base-final vowel (see \S\ref{Definiteness} for other outcomes). 
\ea
\tab  \textit{jenî} `woman' (\textsc{f}) \tab  \textit{kuř} `boy' (\textsc{m})\\
\textsc{sg.dir}\tab  \textit{jen(î)-ekê} [ZQ.14] \tab  \textit{kuř-eke} [ZQ.38] \\
\textsc{sg.obl}\tab  \textit{jen(î)-ekê} [BP.185] \tab  \textit{kuř-ekey} [KŞ.31] \\
\z

With derived nouns, the use of the definite suffixes is triggered by the gender of the derived noun. 
\TabPositions{2.5cm,6cm}
\ea
\textit{neweşî-eke}\tab  `the illness' \tab  < \textit{neweşî} (\textsc{m})\\
\textit{awîrga-(e)kê} \tab  `the fireplace' \tab  < \textit{awîrga} (\textsc{f})\\
\z

At the clause level, agreement targets for the category of gender\is{gender agreement} are predicative adjectives, \textsc{3sg} copula markers (\textsc{3sg.m} \textit{꞊n / ꞊a}; \textsc{3sg.f} \textit{꞊ne}), and \textsc{3sg} inflectional person suffixes in verbs derived from the past stem (\textsc{3sg.m} \textit{-∅}; \textsc{3sg.f} \textit{-e}). In (\ref{adj}), the predicative adjective, the participle, and the \textsc{3sg} copula agree with the gender of \textit{dega} `village', which has been relativised. (\ref{adjb}) illustrates another example of gender agreement\is{gender agreement} on predicative adjectives.
\ea
\textit{î dega toş vînî çoɫe bîyêne.} \\ 
\gll î dega to=ş vîn-î çoɫ-e bîyê=ne \\ 
 \textsc{dem.prox} village{\textsc{.f}} \textsc{2sg=3sg:O} see\textsc{.prs.ind-2sg:A} deserted-\textsc{f} be\textsc{.pst.ptcp.f=cop.3sg.f:S} \\ 
\glt `This village, which you see, was deserted.' \hfill [JE.4] \label{adj}
\z

\ea 
\textit{řama dûrene.} \\ 
\gll řa=ma dûr-e=ne \\ 
 road\textsc{.f=1pl:PSR} far\textsc{-f=cop.3sg.f:S} \\ 
\glt `We have a long way [to go]. [Lit. Our way is far.]' \hfill [BP.191] \label{adjb}
\z 

In verbs derived from the past stem, the 3SG inflectional affix, \textsc{m}: \textit{-∅}, \textsc{f}: \textit{-e}, agrees in gender\is{gender agreement} with the subject of intransitive clauses and object of transitive clauses.
\ea
\textit{kinaçê wite.} \\
\gll \textbf{kinaçe}̂ wit-\textbf{e} \\
girl.\textsc{f} sleep.\textsc{pst-\textbf{3sg.f:S}} \\
\glt `The girl slept.' 
\z 

\ea
\textit{kuř wit.} \\
\gll \textbf{kuř} wit-\textbf{\O} \\
boy.\textsc{m} sleep.\textsc{pst-\textbf{3sg.m:S}} \\
\glt `The boy slept.' 
\z

\ea 
\textit{hêɫeřûwenîşa kerde.} \\
\gll \textbf{hêɫeřûwenî}꞊şa kerd-\textbf{e} \\
fried\_egg\textsc{.f}\textsc{\textsc{.sg.dir}}\textsc{=3pl:A} do\textsc{.pst-}\textsc{\textbf{3sg.f:O}} \\
\glt `They cooked fried eggs.' \hfill [HB.56]  
\z



\subsection{Number\is{number}}\label{Number-section}

Nouns mark number\is{number}. Tekht Hewram\^i\il{Hewramî!Tekht} has two number\is{number} values: singular\is{singular} and plural\is{plural}. As seen in \S\ref{sect:nom-inflection}, the two number\is{number} values are expressed by different allomorphs depending on the base-final vowels (see \S\ref{sect:nom-inflection}). All nouns, regardless of gender and count vs. mass distinction have an oblique plural in \textit{-(y)a}. There are complications with mass nouns in direct plural as explained in \S\ref{inh-pl-section}.
\begin{table}

    \begin{tabular}{lll}
    \lsptoprule
&\textsc{sg}& \textsc{pl} \\ 
\midrule
\textsc{dir.m}&-∅ & \multirow{2}{*}{-ê} \\ 
\textsc{dir.f}&-∅ &  \\ \midrule
\textsc{obl.m}&-î & \multirow{2}{*}{-a} \\ 
\textsc{obl.f}&-ê & \\
\lspbottomrule
    \end{tabular}
    \caption{Number\is{number} inflection}
    \label{tab:num-inf}
\end{table}

Number\is{number} is a morphosyntactic feature in Tekht Hewram\^i\il{Hewramî!Tekht}. The reason is that it is involved in agreement \citep{kibort_grammatical_2008}. Number agreement occurs only when the nouns are in the direct case. There is no number agreement with a noun in the plural oblique. At the level of the noun phrase, the targets for number agreement\is{number agreement} are adjectives (\ref{ex.numb-agr}), classifiers\is{classifiers} (\ref{ex.nmb-agr-2}), and quantifiers\is{quantifiers} (\ref{ex.numb-agr-3}). 
\ea
\textit{karê xerabê} \\
\gll kar-ê xerab-ê\\
thing-\textsc{pl} bad-\textsc{pl} \\
\glt `bad things' \hfill \citep[439]{khan_language_2023} \label{ex.numb-agr}
\z

\ea
\textit{yerê danê hêɫê} \\ 
\gll yerê dan(e)-ê hêɫ(e)-ê \\ 
 three \textsc{clf}\textsc{-dir}\textsc{.pl} egg\textsc{.m}\textsc{-pl.dir} \\ 
\glt `three eggs' \hfill [JH.81] \label{ex.nmb-agr-2}
\z

\ea
\textit{çinnê saɫê} \\
\gll çinn(e)-ê saɫê\\
some\textsc{-pl} year\textsc{\textsc{.f}}\textsc{-dir}\textsc{.pl} \\
\glt `several years’ \hfill [HB.68] \label{ex.numb-agr-3}
\z

At the clause level, the agreeing elements are predicative adjectives (\ref{ex.numb-agr-4}) and adjective complement of the light verb\is{light verb} (\ref{ex.numb-agr-5})--(\ref{ex.numb-agr-6}); see \S\ref{sect:lvc-syntax} for the syntax of light verb constructions. 
\ea 
\textit{ême firê bênmê.} \\ 
\gll ême fir(e)-\textbf{ê} b-ên-mê \\ 
 \textsc{1pl} a\_lot\textsc{-pl} be.\textsc{prs}\textsc{-aug}\textsc{-\textsc{1pl:S}} \\ 
\glt `We were a large number.' \hfill [BP.110]  \label{ex.numb-agr-4}\\
\z 
\ea
 \textit{ême keç-ê ne-b-îmê.} \\
\gll ême keç-\textbf{ê} ne-b-îmê \\
\textsc{1pl} crooked-\textsc{pl} \textsc{neg.sbjv}-be\textsc{.prs-1pl:S} \\
\glt `May we not be crooked.’ \hfill [DG.66] \label{ex.numb-agr-5}
\z 

\ea
\textit{ême zamdarê nekero.} \\
\gll ême zamdar-\textbf{ê} ne-ker-o \\
\textsc{1pl} wounded-\textsc{pl} \textsc{neg.sbjv}-do.\textsc{prs-3sg:A} \\
\glt `He should not injure us.’ \hfill [DG.64] \label{ex.numb-agr-6}
\z 

In addition to nouns and adjectives, pronouns also carry the morphological expression of number\is{number} (see \ref{sect:perso-pro}). 

\subsubsection{Inherently plural\is{plural} nouns}\label{inh-pl-section}
A number of words appear only in the plural\is{plural}. These are nouns denoting collective entities\is{collective noun} and entities that come in pairs. The singular\is{singular} of these nouns appears in the feminine\is{feminine} gender. This then suggests that there is a semantic basis for the gender assignment of mass nouns. The nouns in this category end in unstressed \textit{-î} and \textit{-ê}. 
\TabPositions{1.5cm,3.75cm,5.15cm,7.10cm,9.15cm}
\ea
\textit{mekî}\tab  `salt’ \tab  \textit{hardî} \tab  `flour' \tab  \textit{xurmawî}\tab  `date'\\
\textit{simêɫê} \tab  `moustache’ \tab  \textit{awî} \tab  `water' \tab  \textit{çeqalê} \tab  `bitter almond'\\
\textit{pantoɫê} \tab  `trousers’ \tab  \textit{winî}\tab  `blood' \tab  \textit{hengûrî}\tab  `grape'\\
\textit{şiɫwarê}\tab  `trousers’ \tab  \textit{mîjûyî}\tab  `lentil' \tab  \textit{şêɫanê}\tab  `apricot'\\
\textit{birê}\tab  `eyebrows’ \tab  \textit{qijê}\tab  `hair' \tab  \textit{heştaɫûyî}\tab  `plum'\\
\textit{qijê}\tab  `hair’ \tab  \textit{genmî}\tab  `wheat' \tab  \textit{tifɫe şêxanê}\tab  `strawberry'\\
\textit{wezî}\tab  `walnut’ \tab  \textit{tifî} \tab  `mulberry' \tab  \textit{hencîrî} \tab  `fig'\\
\textit{yewê}\tab  `barley' \tab  \textit{wamî}\tab  `almond'\\
\textit{hêzmî}\tab  `wood' \tab  \textit{wezî}\tab  `walnut'\\
\z

Proof for the plural\is{plural} inflection of these nouns comes in the agreement patterns found in the noun phrase and within the clause. 

\newpage
\ea
\textit{simêɫê řeşênê.} \\
\gll simêɫê řeş-ê꞊nê \\
moustache.\textsc{pl} black-\textsc{pl꞊3pl:S} \\
\glt `The moustache is black.’ \\
\z 

\ea
\textit{pilekanî dirêjênê.} \\
\gll pilekanî dirêj-ê꞊nê \\
stair.\textsc{pl} long-\textsc{pl꞊3pl:S} \\
\glt `The stairs are long.’ \\
\z 

\ea
\textit{pantoɫê teskênê.} \\
\gll pantoɫê tesk-ê꞊nê \\
trousers.\textsc{pl} narrow-\textsc{pl꞊3pl:S} \\
\glt `The trousers are narrow.’ \\
\z 

\ea
\textit{qijêş ecaybênê.}\\
\gll qijê=ş ecayb-ê=nê \\
hair.\textsc{pl.dir=3sg:PSR} extraordinary-\textsc{pl-cop.3pl} \\
\glt `His hair was extraordinary.' \hfill [BM. 79]
\z

It is, however, notable that when individuated, the singular form of these nouns can be used. In \ref{ex:inh_pl}, the singular oblique form of `water' has been used. 
\ea
\textit{heywane awê berde} \\ 
\gll heywane \textbf{aw(î)-ê} berd-e \\ 
 animal\textsc{.f.sg.dir} water\textsc{.f.sg.obl} take\textsc{.pst-3sg.f:O} \\  
\glt `The flood [lit. water] took away the animals.' \hfill [ZB.21] \label{ex:inh_pl}
\z

\subsubsection{Associative plural\is{associative plural}}
The associative plural\is{associative plural} expresses a meaning such as ``noun X and other people associated with X'' and ``noun X and other things associated with X''. Largely speaking, two different words are used for expressing the associative plural\is{associative plural}, the choice between which is triggered by animacy and humanness of the noun X. \textsc{3pl} direct pronoun \textit{ađê} `they' is used with humans, while \textit{çîw} `thing' is used with inanimate nouns. 
\ea 
\textit{berqû çîw nebîyen.} \\ 
\gll berq=û çîw ne-bîye=n \\ 
electricity\textsc{.m.dir}=and thing\textsc{.m.dir} \textsc{neg-}be\textsc{.pst.ptcp.m=cop.3sg.m:S} \\ 
\glt `There were only embers [to light the house]. There was no electricity or anything.' \hfill [JE.40]
\z

\ea 
\textit{tirêştew kardû çîwiş mebo.} \\ 
\gll tirêşte=û kard(î)=û çîw=iş me-b-o \\ 
 axe\textsc{.m}=and knife\textsc{.f}=and thing\textsc{.m=3sg:NC} \textsc{neg.ind-}be\textsc{.prs-3sg:S} \\ 
\glt `He had neither axe, nor knife, nor anything else.' \hfill [BP.71]
\z 

\ea
\textit{Emînû ađê} \\
\gll Emîn=û ađê \\
\textsc{pn}=and \textsc{3pl.dir} \\
\glt `Amin and people associated with him’ \\
\z 

\subsection{Case\is{case}}\label{case-section}

Nouns inflect for case\is{case} based on their syntactic function. The feature of the case\is{case} has two values: `direct\is{direct case}’ and `oblique\is{oblique case}’. By way of example, Table \ref{tab: case-inflection} illustrates the underlying case inflection.

\begin{table}

    \begin{tabular}{ccc}
 \lsptoprule
& \textsc{dir}& \textsc{obl} \\ 
\midrule
\textsc{m}&-∅ &-î \\
\textsc{f}& -∅& -ê \\
\textsc{pl}& -ê& -a \\
 \lspbottomrule
    \end{tabular}
    \caption{case\is{Case} inflection}
    \label{tab: case-inflection}
\end{table}

Case distinction in the singular is not visible for masculine nouns ending in \textit{-î} (e.g., \textit{mizgî}), and feminine nouns ending in \textit{-ê}, e.g., \textit{kinaçê} -- including also feminine nouns ending in the definite suffix \textit{-ekê} (see -- \S\ref{Definiteness}), as the direct and oblique forms are identical. Likewise, case distinction is not visible for a subset of nouns followed by the additive particle \textit{=îç}. These nouns are glossed in the examples without any specification for case. 

The two-term case system, also attested in some modern West Iranian languages (see \citealt{haig_alignment_2008}: Chapter 4, and \citealt{stilo_case_2008} for an overview), is a reduction from an (up to) eight-term case\is{case} system in Old Iranian languages. The direct case\is{direct case} is a continuation of the old nominative case, and the oblique case\is{oblique case} is a continuation of the functions expressed by old non-nominative cases. 

The direct case\is{direct case}, by default, expresses the following syntactic functions. Following \citet{dixon_ergativity_1994}, I use the abbreviations S for the single argument of an intransitive verb, A for an agent of a transitive verb, and O for a patient of a transitive verb.  
\begin{enumerate}[label=\Roman*]
    \item \textbf{Intransitive subject (S)} \\
An intransitive subject in all tenses is expressed by the direct case. 
\ea
\textit{seʕbê wiɫaxdarê mila.} \\
\gll seʕbê \textbf{wiɫaxdar-ê} mi-l-a \\
morning.\textsc{f.sg.obl} stableman-\textsc{pl.dir} \textsc{ind}-go.\textsc{prs-3pl:S} \\
\glt `In the morning, the horse grooms went.’ \hfill [ŞC.66] 
\z 

\ea
\textit{ħewt gawê leřê amênê.} \\
\gll ħewt \textbf{gaw(e)-ê} leř-ê amê=nê \\ 
seven cow.\textsc{f-pl.dir} thin-\textsc{pl} come.\textsc{pst.ptcp.pl=cop.3pl:S} \\
\glt `Seven thin cows came.' \hfill[PP.26]
\z 

\item \textbf{Agent of a present tense verb (A-\textsc{prs})}\\
The transitive subject of a clause built on the present stem verb is expressed by the direct case. The relevant tenses are the present tense, future tense, and past imperfect. 
\ea 
\textit{dêwê řas maça.} \\ 
\gll \textbf{dêw-ê} řas m-aç-a \\ 
 ogres\textsc{-pl.dir} truth \textsc{ind-}tell\textsc{.prs}\textsc{-3pl:A} \\ 
\glt `The ogres are telling the truth.' \hfill [SK.64] 
\z

\ea
\textit{zaroɫê bazî kerênê.} \\
\gll \textbf{zaroɫ(e)-ê} bazî ker-ên-ê \\
child\textsc{-pl.dir} game do.\textsc{prs-aug-3pl:A} \\
\glt `The children were playing a game.' \hfill[ŞE.09]
\z 

\item \textbf{O of a past tense verb (O-past)} 
\ea
\textit{hewarêşa wişkin{ɛ}nê.} \\ 
\gll \textbf{hewar-ê}=şa wişkin{ɛ}=nê \\ 
 summer\_habitat\textsc{.m-pl.dir=3pl:A} scour\textsc{.pst.ptcp.pl=cop.3pl:O} \\  
\glt `They scoured the summer habitats [searching for food, etc.].' \hfill [JE.3] 
\z

\item \textbf{Possessed NP in predicative possessive constructions\is{predicative possessive constructions}} \\
The possessed NP in predicative possessive constructions of all tenses is marked by the direct case. 
\ea 
\textit{zemanê ême î begêma bîyênê.} \\ 
\gll zeman-ê ême î \textbf{beg-ê}=ma bîyê=nê \\ 
 time\textsc{.m-indf} \textsc{1pl} \textsc{dem.prox} chief\textsc{.m-pl.dir=1pl:NC} be\textsc{.ptcp.pl=cop.3pl:S} \\ 
\glt `Once we had these noblemen [in our region].' \hfill [RE.56] 
\z

\ea
\textit{marêş henê.} \\
\gll \textbf{mar-ê}=ş hen-ê \\
snake-\textsc{pl.dir=3sg:A} \textsc{exist-3pl:S} \\
\glt `He (Pharaoh) has snakes.' \hfill[MF.295]
\z 

\item \textbf{Grammatical subject in non-canonical subject constructions\is{non-canonical subject constructions}} \\
The grammatical subject in a non-canonical subject construction\is{non-canonical subject constructions} (see \ref{sect:cliticpers} for definition) is marked in a direct case\is{direct case}. On the other hand, the subject-like argument is marked in the oblique case\is{oblique case} (see below). 
\ea
\textit{min tateta gerekma.} \\
\gll min \textbf{tate-\O}=ta gerek=m=a \\
\textsc{1sg} father\textsc{-m.sg.dir=2pl:PSR} necessary.\textsc{m=1sg:NC=cop.3sg.m:S} \\
\glt `I want your father.' \hfill [ÇK.17]
\z 

\ea
\textit{a zařoɫa jenîşa gerekene.} \\
\gll a zařoɫ(e)-a \textbf{jenî}=şa gerek-e=ne \\
\textsc{dem.dist} child-\textsc{pl.obl} wife.\textsc{f.sg.dir=3pl:NC} necessary-\textsc{f-cop.3sg.f:S} \\
\glt `The children want to get married. [Lit. wife is necessary to the children].' \hfill [ME.13]
\z 
\end{enumerate}

Nouns are marked for the oblique case\is{oblique case} in the following syntactic functions: 
\begin{enumerate}[label=\Roman*]
\item \textbf{Direct object of a present tense verb (O-prs)}
\ea
\textit{yewayç bero.} \\ 
\gll \textbf{yew(e)-a}=yç {} ber-o \\ 
 barley\textsc{.f-pl.obl=add} {} take\textsc{.prs.ind-3sg:A} \\ 
\glt `[Out of obligation], he took the barley seeds, too.' \hfill [JP.29] 
\z

\ea
\textit{ba î pîyay bermê.} \\ 
\gll ba î pîya-î b\'er-mê \\ 
 \textsc{hort} \textsc{dem.prox} man\textsc{-m.sg.obl} take.\textsc{prs.sbjv-1sg:A}\\ 
\glt `Let's take this man.' \hfill [DB.119] 
\z 
\item \textbf{A of a past tense verb (A-past)} \\
The case\is{case}-marked A-past argument usually has a prominent role in discourse, e.g., it takes the nuclear stress (\ref{ex.A-obl1})--(\ref{ex.A-obl2}) or is associated with referents who are typically agents (\ref{ex.A-obl3}). See \S\ref{sect:DAM} for discussion.
\ea
\textit{pase herey wat...} \\ 
\gll pase \textbf{her-e-î} wat \\ 
 like donkey\textsc{.m}-\textsc{2sg}\textsc{-obl}\textsc{.m} say\textsc{.pst} \\ 
\glt `As the donkey said...' \hfill [HB.54] \label{ex.A-obl1}
\z 

\ea
\textit{min, taze padşay kerdena wekêɫ.} \\ 
\gll min taze \textbf{padşa-î} kerde=na wekêɫ \\ 
\textsc{1sg} anyway king\textsc{.m-sg.obl} do\textsc{.pst.ptcp.m=cop.1sg:O} advocate\textsc{.m} \\ 

\glt `As for me, well, the king has put me in charge.' \hfill [JP.206] \label{ex.A-obl2}
\z

\ea
\textit{xway ketê pey kîyasen.} \\ 
\gll \textbf{xwa-î} ket-ê pey kîyase=n \\ 
 God\textsc{.m-sg.obl} bed\textsc{-indf} to send\textsc{.pst.ptcp.m=cop.3sg.m:R} \\ 
\glt `God had sent him a bed.' \hfill [JP.69]  \label{ex.A-obl3}
\z

In addition to animate referents, as seen in examples (\ref{ex.A-obl1})--(\ref{ex.A-obl2}), inanimate agents are also marked in the oblique case.\footnote{According to \citet[51]{mackenzie_dialect_1966}, oblique marking of past transitive subject is available only for inanimate agents in the Hewramî variety of Luhon. In Tekht, by contrast, it is available for animate referents as well, including human and non-human referents (see \ref{ex.A-obl1}--\ref{ex.cawecase}).} 
\ea
\textit{pejarey berđwe.}\\
\gll \textbf{pejare-î} berđ-\O=we\\
grief-\textsc{m.sg.obl} take.\textsc{pst-3sg.m:O=compl}\\
\glt `Grief overtook him.'\hfill[DB.24]
\z 

\ea 
\textit{îse cawê kerdêne çêro.} \\ 
\gll îse caw(e)-\textbf{ê} kerdê=ne çêr=o \\ 
 now road\textsc{-f.sg.obl} do\textsc{.pst.ptcp.f=cop.3sg.f:O} under\textsc{=post} \\ 
\glt `Now, the stone is laid under the road.' \hfill[ZP.53] \label{ex.cawecase}
\z 

\item \textbf{Complement of an adposition}
\ea 
\textit{min bere la patşay.} \\ 
\gll min b\'er-e \textbf{la} \textbf{patşa-î} \\ 
 \textsc{1sg} take.\textsc{imp-2sg:A} to king.\textsc{m-sg.obl} \\ 
\glt `Take me to the king.' \hfill[PP.27]
\z  

\ea
\textit{milawe pey germîyanî.} \\ 
\gll mi-l-a=we \textbf{pey} \textbf{germîyan-î} \\ 
 \textsc{ind-}go\textsc{.prs-3pl:S=compl} for \textsc{pn-m.sg.obl} \\ 
\glt `They went to Garmiyan.'  \hfill[ZB.39]
\z 


\item \textbf{Possessor noun in a genitive construction} 

\ea 
\textit{ane kuřû şuwaneyn.} \\ 
\gll ane \textbf{kuř-û} \textbf{şuwane-î}=n \\ 
 \textsc{dem.dist.m.3sg.dir} son\textsc{.m-ez.gen} shepherd\textsc{.m-sg.obl=cop.3sg.m:S}  \\ 
\glt `That is the shepherd’s son [on your horse].' \hfill [KŞ.100]
\z  

\ea 
\textit{tatew pîr şelîyarîn.} \\ 
\gll \textbf{tate-û} \textbf{pîr} \textbf{şelîyar-î}=n \\ 
 father\textsc{.m-ez.gen} \textsc{pn} \textsc{pn-m.sg.obl=cop.3sg.m:S} \\ 
\glt `He is Pir Shaliyar’s father.' \hfill[BP.7]
\z

\item \textbf{Adverbial time expressions}
\ea 
\textit{şewê firê bijyawe.} \\ 
\gll şew(e)-ê fire-ê bijye-a=we \\ 
 night\textsc{.f-f.sg.obl} a\_lot\textsc{-indf} toss\textsc{.prs-3pl:S=compl} \\  
\glt `During the night, [while lying down] they tossed and turned [trying to get comfortable].' \hfill [ZB.18]
\z 

\ea 
\textit{hurmêzowe seʕbê.} \\ 
\gll hur-m-êz-o=we seʕbe-ê \\ 
 \textsc{pvb-ind-}rise\textsc{.prs-3sg:S=compl} morning\textsc{.f-f.sg.obl} \\ 
\glt `He woke up in the morning.' \hfill [ŞC.47]
\z 

\newpage
\item \textbf{possessor in a predicative possessive construction} 
\ea
\textit{xway deseɫatê çaneş bîyen.} \\ 
\gll \textbf{xwa-î} deseɫat-ê çane=ş bîye=n \\ 
 God\textsc{.m-sg.obl} power\textsc{.m-indf} like\_that\textsc{=3sg:NC} be\textsc{.pst.ptcp.m=cop.3sg.m:S} \\ 
\glt `God had such a power.' \hfill [ZQ.42]
\z

\ea
\textit{sîyaweħşî kuřêwiş bîyen.} \\
\gll \textbf{sîyaweħş-î} kuř-êw=iş bîye=n \\
\textsc{pn-m.sg.obl} son\textsc{-indf=3sg:NC} be.\textsc{pst.ptcp.m=cop.3sg.m:S}\\
\glt `Siyawehsh has had a son.' \hfill [SK.87]
\z 

\ea
\textit{jenekaşa lemeşa bîyewe.} \\
\gll \textbf{jen(î)-eka}=şa leme=şa bî-e=we \\
wife\textsc{-def.pl.obl=3pl:PSR} belly.\textsc{f.sg.dir=3pl:NC} be.\textsc{pst-3sg.f:S=compl} \\
\glt `Their wives got pregnant [lit. had bellies] again .'\hfill [ME.210]
\z 

\ea
\textit{a zařoɫa jenîşa gerekene.} \\
\gll a \textbf{zařoɫ(e)-a} jenî=şa gerek-e=ne \\
\textsc{dem.dist} child-\textsc{pl.obl} wife.\textsc{sg.f.dir=3pl:NC} necessary-\textsc{f-cop.3sg.f:S} \\
\glt `Those children (of yours) ask for wives [lit. wife is necessary to the children].' \hfill [ME.13]
\z 

\item \textbf{subject-like argument in non-canonical subject constructions\is{non-canonical subject constructions}} \\
A subject-like argument in non-canonical subject constructions is different from the grammatical subject in these constructions in two respects: (i) the subject-like argument is generally marked in the oblique case (see the examples below), as opposed to the direct marking for the grammatical subject; (ii) the subject-like argument is indexed by clitic pronouns, whereas the grammatical subject is indexed by the copula or person suffixes (see \S\ref{sect:cliticpers} for discussion on the second point). 
\ea 
\textit{zařoɫam awraşan.} \\
\gll \textbf{zařoɫ(e)-a}=m awra=şa=n \\
child.\textsc{m-pl.obl=1sg:PSR} hungry=\textsc{3pl:NC=cop.3sg.m:S} \\
\glt `My children are hungry.' \hfill [SH.230]
\z 

\ea 
\textit{xway îrayeşa.} \\ 
\gll \textbf{xwa-î} îraye=ş=a \\ 
  God\textsc{.m-sg.obl} volition\textsc{=3sg:NC=cop.3sg.m:S} \\ 
\glt `[This] is God’s will. [Lit. God, his volition is.]' \hfill [ZB.34]
\z 

\ea
\textit{dêđêş qînîş kewte.} \\
\gll \textbf{dêđê}=ş qînî=ş kewt-e \\
stepmother\textsc{f.sg.obl=3sg:PSR} rage\textsc{.f.sg.dir=3sg:NC} fall.\textsc{pst-3sg:S} \\
\glt `His step-mother got into a rage. [lit. her rage fell.]'  \hfill [SK.15]
\z 

\ea 
\textit{tateyt miqdarêş hoşa.} \\ 
\gll \textbf{tate-î}=t miqdarê=ş hoş=a \\ 
father\textsc{.m-sg.obl=2sg:PSR} a\_little\textsc{=3sg:NC} memory\textsc{.m=cop.3sg.m:S} \\ 
\glt `Your father remembers it a bit. [Lit. Your father, his memory is.]'  \\ \hfill [BP.13]
\z 

\item \textbf{The second coordinate noun in a co-ordinated NP (e.g., when standing alone)}
\ea
\textit{minû hesenî} \\
\gll min=û hesen-î \\
\textsc{1sg=}and \textsc{pn-m.sg.obl} \\
\glt `Hasan and I'
\z 
\item \textbf{the copula complement in a prepositional non-verbal predicate} \\
The prepositional complement of a copula clause may be marked by the oblique case\is{oblique case}. This is especially the case when the preposition is \textit{pêse} `like, similar to'. In this kind of clauses, the clitic copula exceptionally moves on the preposition (see \S\ref{sect:copula_clasue} for discussion).
\ea
\textit{to pêsenî emîrî.} \\
\gll to pêse=nî emîr-î \\
\textsc{2sg} like\textsc{=cop.2sg:S} \textsc{pn-sg.obl.m} \\
\glt `You are like Emir (to me).'
\z

\ea 
\textit{řisq pêsen miɫey.} \\
\gll řisq pêse=n miɫe-î \\
rat like=\textsc{cop.3sg.m:S} mouse-\textsc{sg.obl.m} \\
\glt `A rat is like a mouse.'  \hfill [PK.36]
\z 
\end{enumerate}

It should be noted that there are certain complications for deployment of case marking\is{case marking}. For example, case marking\is{case marking} of O-\textsc{prs} is sensitive to the status of nouns as definite and specific (see \S\ref{sect:DOM} for details). Generic Os are generally marked in the direct case\is{direct case} (see \ref{ex.hilose}).
\ea 
\textit{min hiɫoşet miđew.} \\ 
\gll min \textbf{hiɫoşe}=t mi-đe-û \\ 
 \textsc{1sg} cracked\_wheat\textsc{.f.sg.dir=2sg:R} \textsc{ind-}give\textsc{.prs-1sg:A} \\ 
\glt `I will give you cracked wheat.'  \hfill [JP.236] \label{ex.hilose}
\z 

Similarly, there is some variation in the oblique marking\is{oblique case} of non-core arguments such as possessors in predicative possession construction, the subject-like argument of a non-canonical construction\is{non-canonical subject constructions}, etc. (see \S\ref{sect:diffobl} for an overview); see (\ref{ex.nonc2})--(\ref{ex.nonc3}) for direct\is{direct case} marking of these arguments.  

\ea 
\textit{yo sêfêş bîyêne.} \\ 
\gll \textbf{yo} sêfê=ş bîyê=ne \\ 
 one\textsc{m.sg.dir} potato\textsc{.f.sg=3sg:NC} be\textsc{.pst.ptcp.f=cop.3sg.f:S} \\ 
\glt `One had potatoes.' \hfill [JE.32] \label{ex.nonc2}
\z 

\ea
\textit{padşa gerekşa} \\
\gll \textbf{padşa} gerek=ş=a \\
king\textsc{.m.sg.dir} necessary=\textsc{3sg:NC=cop,3sg.m} \\
\glt `The king wanted...].' \hfill [BM. 135] \label{ex.nonc3}
\z 

Another feature of Hewramî\il{Hewramî} that affects the case marking\is{case marking} is the notion of transitivity and how it is applied to light verb constructions. As mentioned in Chapter \S\ref{ch.typology}, tense-sensitive alignment\is{tense-sensitive alignment} depends not on the transitivity of the clause in a semantic sense but on the lexical transitivity of individual verbs. Likewise, in light verb constructions, transitivity is determined not in a semantic sense but by the transitivity of the light verb\is{light verb}. In terms of case marking\is{case marking}, the subject of a transitive light verb\is{light verb} should be assigned an oblique case\is{oblique case} in past transitive construction. In the following examples, the oblique marking\is{oblique case} on the subject is triggered by the transitivity of the light verb\is{light verb} \textit{kerđ} `do'. 
\ea 
\textit{xuđay eta kerden.} \\
\gll \textbf{xuđa-î} eta kerde=n \\
God\textsc{.m-obl} granting do.\textsc{pst.ptcp.m=cop.3sg.m:O} \\
\glt `God granted him.’ \hfill [DB.125]
\z 

\ea 
\textit{jenê desiş kerd qisekerdey.} \\
\gll \textbf{jen(î)-ê} des=iş kerd{-\O} qisekerdey \\
woman.\textsc{f-obl} hand=\textsc{3sg:A} do.\textsc{pst-3sg.m:O} talking \\
\glt `The woman started [lit. hand do] to talk.' \hfill [SH.184]
\z 

There is no designated case to express the vocative\is{vocative}. Therefore, a vocative\is{vocative} function can be expressed in several ways. Singular\is{singular} nouns may be express the vocative\is{vocative} in a direct case\is{direct case}. 
\TabPositions{1.5cm,3.5cm}
\ea
\textit{táte}\,\suppipe{}\tab `Father!'\tab [JH.22]  \\
\textit{éđa}\,\suppipe{}\tab `Mother!’ \\
\z

The vocative\is{vocative} is expressed via the oblique\is{oblique case} form for plural\is{plural} nouns.
\ea 
\textit{hewramîya xwahafêz!} \\
\gll hewramî-a xwahafêz \\
Hewram\^i-\textsc{pl.obl} farewell \\
\glt `O Hewram\^i [people]! Goodbye!’
\z

The vocative\is{vocative} particle \textit{ya}, borrowed from Arabic, may also express the vocative\is{vocative}. 
\ea 
\textit{ya qaɫîçew silêmanî!} \\
\gll ya qaɫîçe-û silêman-î \\
\textsc{voc} carpet-\textsc{ez.gen} pn-\textsc{obl.m} \\
\glt `O Solomon's carpet!’ \hfill [DB.164]
\z

A similar strategy, sometimes employed in tales, is to use the particle \textit{ey} to express the vocative\is{vocative}. 
\ea
\textit{maço, `ey kitêk ey çira ey bira!'} \\
\gll m-aç-o ey kitik ey çira ey bira \\
\textsc{ind-}say.\textsc{prs-3sg:A} \textsc{voc} cat\textsc{.dim} \textsc{voc} lamp \textsc{voc} brother \\
\glt `He said, `Oh Cat; oh Lamp; oh Brother!’' \hfill [SH.111]
\z 

Another strategy for expressing the vocative\is{vocative} is to use the bare noun\is{bare noun} in the direct case\is{direct case} followed by the modifier \textit{gîyan} `dear’:

\TabPositions{2cm,9.5cm}
\ea 
\textit{bira gîyan}\tab  `Dear brother! (addressing the neighbour)\tab [RE.5]  \\
\z

The vocative\is{vocative} can also be expressed by adding the diminutive\is{diminutive} suffix \textit{-le} (\textsc{m}), \textit{-lê} (\textsc{f}) to the kin terms. Here, more emotional engagement is involved with the interlocutor. Notably, the diminutive\is{diminutive} suffix is frozen in words referring to `sister’ and `child’. In neighbouring Southern Kurdish\il{Kurdish!Southern} varieties, the term for `child' is the cognate form \textit{zarû}. Similarly, the general term for `sister' in Kurdish, i.e., \textit{xoşk}, has the frozen diminutive suffix \textit{-ik}.
\TabPositions{2cm,5cm}
\ea
\textit{tate-le}\tab  `Father!’ \\
\textit{eđa-lê} \tab  `Mother!’ \\
\textit{mama-lê} \tab  `Grandmother!’ \\
\textit{waɫê}\tab  `Sister!’ \\
\textit{zawɫe}\tab  `Child!\tab  < *\textit{zaro} + \textit{ɫe}’ \\
\z

Alternatively, the definite suffix\is{definite suffix} may express vocative\is{vocative} case, mainly when used with kin terms `husband’ and `wife’. The definite suffix\is{definite suffix} agrees with the referent in gender\is{gender agreement}.
\ea
\textit{jenekê}\tab  `Wife!’\tab  [ZB.13]  \\
\textit{pîyake}\tab  `Husband!’\tab  [DB.221]\\
\z

\subsection{Classifiers\is{classifiers} and measure nouns\is{measure nouns}}\label{sect:classifiers}
Hewramî\il{Hewramî} has several linguistic items that exhibit characteristics of noun classifiers\is{classifiers}; that is, their choice depends on the semantics of the head nouns, and they may be used anaphorically \citep[]{aikhenvald_classifiers_2000}{}. Classifiers\is{classifiers} make up a small word class in Hewramî\il{Hewramî}, and a limited number of nouns have them. Table \ref{tab:classifier} lists a preliminary list of classifiers. They can be classified into sortal classifiers \is{classifiers!sortal}, which are used with count nouns, and mensural classifiers\is{classifiers!mensural}, used with mass nouns. 
\begin{table}
    \begin{tabular}{lll}
\lsptoprule
        Type& item & Used with / for \\
        \midrule
         \multirow{6}{*}{Sortal} & \textit{dane} & count nouns \\
         & \textit{ser} & domestic animals, e.g., sheep, goat\\
         & \textit{des} & clothes \\
         & \textit{boɫ} & counting grape \\
         & \textit{hoşe} & counting a cluster of grape \\
         & \textit{bine} & trees\\
         & \textit{telane} & farmland \\\midrule
         Mensural & \textit{peɫ} & mass nouns, e.g., ``grain'' \\
         \lspbottomrule
    \end{tabular}
    \caption{Common classifiers}
    \label{tab:classifier}
\end{table}

The most common classifier\is{classifiers} is \textit{dane} (literally `seed'), which can be used with various head nouns. In terms of their syntax, classifiers\is{classifiers} agree in number\is{number agreement} with the head noun. In (\ref{ex.clf.1}), \textit{dane} agrees with the head noun; in (\ref{ex.clf.2}), it is used anaphorically to refer back to `eggs' in the previous discourse. 
\ea 
\textit{duwê danê dêwê} \\ 
\gll duwê dan(e)-ê dêw-ê\\ 
 two \textsc{clf}\textsc{-pl.dir} ogre\textsc{.m-pl.dir}  \\ 
\glt `two ogres' \hfill [JP.176] \label{ex.clf.1}
\z 

\ea 
\textit{a wextî danêşa sîyaw kero.} \\ 
\gll a wext-î dan(e)-ê=şa sîyaw ker-o \\ 
 \textsc{dem.dist} time\textsc{.m-sg.obl} \textsc{clf-indf=3pl:PSR} black do\textsc{.prs.ind-3sg:A} \\ 
\glt `The girl coloured one of them black.' \hfill [JH.83] \label{ex.clf.2}
\z 

Additionally, classifiers are optional and may or may not appear with the head noun. Compare the following examples.
\ea 
\textit{yerê danê hêɫê maro.} \\ 
\gll yerê dan(e)-ê hêɫ(e)-ê m-ar-o \\ 
 three \textsc{clf-pl.dir} egg\textsc{.m-pl.dir} \textsc{ind-}bring\textsc{.prs-3sg:A} \\ 
\glt `She [went and] brought three eggs.' \hfill[JH.81]
\z 

\ea
\textit{yerê hêɫêş ardê.}\\
\gll yerê hêɫ(e)-ê=ş ard-ê \\ 
three egg\textsc{.m-pl.dir=3sg:A} take.\textsc{pst-3pl:O} \\
\glt `She took three eggs.' \hfill[HS.82]
\z 
The following classifiers\is{classifiers} have been identified in the fieldwork. The list in Table \ref{tab:classifier} is not exhaustive, and no claims of thoroughness can be made here. \textit{peɫ} refers to `bugs and insects' and grains en masse. 
\ea 
\textit{peɫê merekuř mê şarezûr.} \\ 
\gll peɫ-ê merekuř m-ê şarezûr \\ 
 \textsc{clf}\textsc{-indf} grasshopper \textsc{ind-}come\textsc{.prs.3sg:S} \textsc{pn} \\ 
\glt `A swarm of grasshoppers came to Sharazour.' \hfill [PM.1]
\z 

\ea 
\textit{pîreke yawa law peɫêw xeley.} \\ 
\gll pîr-eke yawa la-û peɫ-êw xele-î \\ 
 old-\textsc{def.m.dir} arrive.\textsc{pst.3sg.m:S} to\textsc{-ez.gen} \textsc{clf-indf} grain\textsc{-m.sg.obl} \\ 
\glt `The old man arrived at a pile of grains.' \hfill [HR.13]
\z 

\textit{Ser} is used for domestic animals like sheep and goats, especially when used as a sacrifice. 
\ea 
\textit{serê heywan sere biřo.} \\ 
\gll ser-ê heywan sere biř-o \\ 
 \textsc{clf-indf} animal\textsc{.m} head\textsc{.m} cut\textsc{.prs.ind-3sg:A} \\ 
\glt `He butchered one animal.' \hfill [JP.253]
\z 

\ea
\textit{ya mûsa serê heywanma de.} \\
\gll ya mûsa serê heywan=ma de \\
\textsc{voc} \textsc{pn} \textsc{clf} animal\textsc{m.sg.dir=1pl:R} give.\textsc{imp.2sg:A} \\
\glt `Oh Moses! Give us an animal.' \hfill [MF.209]
\z 

\textit{Boɫ} and \textit{hoşe} are measure nouns\is{measure nouns} for one grape, and a cluster of grape, respectively. 
\ea 
\textit{boɫê hengûrî} \\
\gll  boɫ-ê hengûrî \\ 
 \textsc{clf}\textsc{-indf} grape \\ 
\glt `one grape' 
\z 

\ea 
\textit{duwê hoşê hengûrî} \\
\gll  duwê hoş(e)-ê hengûrî\\ 
 two \textsc{clf}\textsc{-pl.dir} grape\textsc{.pl.dir} \\ 
\glt `two clusters of grape'
\z 

Other classifiers\is{classifiers} include \textit{bine}, and \textit{teɫane}. \textit{bine} is used as a measure of unit for trees. \textit{teɫane} is used as a measure of unit for farmland.
\ea 
\textit{duwê binê wezî} \\ 
\gll  duwê bin(e)-ê wezî \\ 
 two \textsc{clf}\textsc{-pl.dir} walnut\textsc{.pl.dir} \\ 
\glt `two walnut trees' 
\z 

\ea 
\textit{duwê teɫanê bencacnî} \\ 
\gll  duwê teɫan(e)-ê bencacnî \\ 
 two \textsc{clf}\textsc{-pl.dir} tomato\textsc{.pl.dir} \\ 
\glt `two pieces of tomato farm' 
\z 

There are different measure words for walnuts, reflecting their importance in the local economy. These are summarised below:
\TabPositions{2.5cm}
\ea
  \textit{laye}  \tab  `a collection of five walnuts' \\
   \textit{dese}  \tab  `a collection of ten walnuts' \\
   \textit{nîme hezar} \tab  `a collection of five-hundred walnuts' \\
   \textit{hezar} \tab  `a collection of one-thousand walnuts' \\
\z

Examples:
\ea 
\textit{yerê desê wezî} \\ 
\gll yere des(e)-ê wezî \\ 
 three \textsc{clf}\textsc{-pl.dir} walnut\textsc{.pl.dir} \\ 
\glt `thirty walnuts' 
\z

\ea 
\textit{desew layê wezî} \\ 
\gll dese=û lay(e)-ê wezî \\ 
 \textsc{clf=}and \textsc{clf-indf} walnut\textsc{.pl.dir} \\ 
\glt `fifteen walnuts' 
\z

Among these, \textit{des} may also be used to refer to `a unit of ten domestic animals'. Other measure nouns\is{measure nouns} include \textit{hît} `pair', \textit{gez} `a unit of measure equivalent to 72 cm of textile', \textit{teẍar} `a unit of weight equivalent to 120 kilograms', etc. 
\ea 
\textit{zemanû a anê hîtê paɫ{ɛ} bîyênêw çwar gezê parçe bîyen.} \\ 
\gll zeman-û a anê hît-ê paɫ{ɛ} bîyê=nê=û çwar gez-ê parçe bîye=n \\ 
 time\textsc{.m-ez.gen} \textsc{dem.dist} \textsc{dem.dist.3sg.f.sg.obl} pair\textsc{-pl} shoe\textsc{.pl.dir} be\textsc{.ptcp.pl=cop.3pl:S}=and four ell\textsc{.m-pl.dir} textile\textsc{.m} be\textsc{.pst.ptcp.m=cop.3sg.m:S} \\  
\glt `Then, let us say, at the time, [the wedding gift] consisted of (some) pairs of shoes [and] four gaz of textile.' \hfill [RE.11]
\z 

\subsection{Bare nouns\is{bare noun}}\label{sect: Bare_nouns}
A bare noun\is{bare noun} can fulfil different functions in Tekht Hewramî\il{Hewramî!Tekht}. It may express indefinite referents and definite referents alike (see below), and have singular\is{singular} or plural\is{plural} reference. Thus, its use radically differs from well-known languages like English\il{English}. In (\ref{ex.bare1})--(\ref{ex.bare2}), the bare nouns\is{bare noun} \textit{meɫa} and \textit{gaw} express singular\is{singular} nouns with non-specific referents.

\newpage
\ea 
\textit{meɫa mara mareş biřa peyş.} \\ 
\gll \textbf{meɫa} m-ar-a mare=ş biř-a pey=ş \\ 
mullah\textsc{.m} \textsc{ind-}bring\textsc{.prs-3pl:A} marriage\textsc{=3sg:O} cut\textsc{.prs.ind-3pl:A} for\textsc{=3sg:R} \\ 
\glt `They fetched a Mullah [and] married her (the girl) to him (the shepherd’s son).' \hfill [KŞ.88] \label{ex.bare1}
\z 

\ea 
\textit{pêse gaw til bowe ...} \\ 
\gll pêse \textbf{gaw} til b-o=we \\ 
 as\_if bull\textsc{.m} rolling be\textsc{.prs.sbjv-3sg:S=compl} \\  
\glt `As if a bull had rolled down [into the valley] ...' \hfill [ZP.79] \label{ex.bare2}
\z 

Similarly, a bare noun\is{bare noun} can express a noun with an indefinite specific reference which is known to the speaker. In the following example, the narrator talks about a specific woodland he has seen while searching for a mountain shelter. 
\ea 
\textit{kelêwe dîmandîmê bê fire tûl bê; \textbf{daresan} bê.} \\ 
\gll kel-êwe dîmandîm-ê b-ê fire tûl b-ê daresan b-ê\\ 
 mountain\textsc{-indf} wide\textsc{-indf} be.\textsc{prs-aug.3sg:S} a\_lot long be\textsc{.prs-aug.3sg:S} woodland\textsc{.m} be.\textsc{prs-aug.3sg:S}\\ 
\glt `It was a big mountain; it was very high; it was a woodland.' \hfill [ZQ.12]
\z

A bare noun\is{bare noun} can also be used in a generic sense to express indefinite non-specific plural items. The noun would have plural\is{plural} inflection in equivalent situations in English\il{English}.
\ea 
\textit{be kune awîşa ardêne.} \\ 
\gll be \textbf{kune} awî=şa ardê=ne \\ 
by clay\_pot\textsc{.m} water\textsc{.f.sg.dir=3pl:A} bring\textsc{.pst.ptcp.f=cop.3sg.f:O} \\ 
\glt `They used to fetch water using \textbf{clay pots}.' \hfill [JE.16]
\z 

\ea 
\textit{be hesere hêzmîşa ardênê pey zimsanî.} \\ 
\gll be \textbf{hesere} hêzmî=şa ardê=nê pey zimsan-î \\ 
 by mule\textsc{.f} firewood\textsc{.pl.dir=3pl:A} bring\textsc{.pst.ptcp.pl=cop.3pl:O} for winter\textsc{.m-sg.obl} \\ 
\glt `They fetched firewood for the winter on \textbf{mules}.' \hfill [JE.35]
\z 

Finally, a bare noun\is{bare noun} may have a definite reading. This happens when the reference to the noun has been established both for the speaker and the listener (see \S\ref{Definiteness}). In the following excerpt, \textit{çawre} `tent' is first unmarked as a bare noun\is{bare noun} and has a generic plural reference. In the second mention, it takes the definite suffix\is{definite suffix} \textit{-ekê}. In the third mention, it has a definite reference but appears as a bare noun\is{bare noun}. 
\ea 
\textit{ca a wextî çawre bo. çawrekêne fire ginawe kem. wextê weyowe, lawaw aman çawre peře bîye awî.} \\ 
\gll ca a wext-î \textbf{çawre} b-o \textbf{çawre-(e)kê}=ne fire gin-a=we kem wext-ê weye-o=we lawaw ama=n \textbf{çawre} peř-e bî-e awî\\ 
afterwards \textsc{dem.dist} time\textsc{.m-sg.obl} tent\textsc{.f} be\textsc{.prs.ind-3sg:S} tent\textsc{.f-def.f.sg=post} a\_lot fall\textsc{.prs.ind-3pl:S=compl} little time\textsc{.m}\textsc{-indf} wake\textsc{.prs.ind}\textsc{-3sg:S}\textsc{=compl} flood\textsc{.m} come\textsc{.pst}\textsc{.ptcp}\textsc{.m}\textsc{=cop}\textsc{.3sg}\textsc{.m:S} tent\textsc{.f} full\textsc{-f} be\textsc{.pst}\textsc{.3sg.f:S} water\textsc{.f} \\ 
\glt `Back then, there were tents. They tossed around [in bed] in the tent for a while. When they finally woke up, the flood had come [and] the tent was filled with water.' \hfill [ZB.19-ZB.20]
\z 

\subsection{Definiteness}\label{Definiteness}
Nouns are marked for definiteness\is{definiteness} by the following suffixes: \textit{-eke}; \textit{-e}. The definiteness marker \textit{-eke} has distinct allomorphs depending on the gender, case, and number of the base to which it attaches:

\begin{table}

    \begin{tabular}{lll}
 \lsptoprule
 & \textsc{dir}& \textsc{obl} \\ 
 \midrule
\textsc{sg.m}& \textit{-eke}& \textit{-ekey} \\
\textsc{sg.f}& \cellcolor{gray!25}\textit{-ekê}& \cellcolor{gray!25}\textit{-ekê} \\
\textsc{pl}& \cellcolor{gray!25}\textit{-ekê}& \textit{-eka} \\
 \lspbottomrule
    \end{tabular}
    \caption{Definiteness paradigm}
    \label{tab:definite}
\end{table}

The following paradigms illustrate the inflection of the definite suffix \textit{-eke} on consonant-final nouns.  
\TabPositions{1.5cm}
\ea
\tab  \textit{kuř} `boy, son’ (\textsc{m})\\
\textsc{sg.dir}\tab \textit{kuř-eke} [ZQ.38]  \\
\textsc{sg.obl}\tab \textit{kuř-ekey}  [KŞ.31]  \\
\textsc{pl.dir} \tab \textit{kuř-ekê} [ZB.9]   \\
\textsc{pl.obl}\tab \textit{kuř-eka}  \\
\z

The addition of the definite suffix to the base-final vowel results in vowel hiatus, which is resolved in several ways: (i) the initial vowel of the definite suffix is dropped; (ii) vowel reduction; (iii) the final vowel of the noun is dropped; (iv) an epenthetic consonant is added between the two vowels. These are discussed below.  

With nouns ending in \textit{-a}, the initial vowel of the definite suffix is dropped.
\TabPositions{1.5cm,7cm}
\ea
 \tab  \textit{pîya} `man (\textsc{m})' \tab  \textit{dega} `village' (\textsc{f})\\
\textsc{sg.dir}\tab  \textit{pîyake} (< \textit{pîya-eke}) [JH.110] \tab  \textit{degakê} (< \textit{dega-ekê})\\
\textsc{sg.obl}\tab  \textit{pîyakey} (< \textit{pîya-ekey}) [DG.53] \tab  \textit{degakê} (< \textit{dega-ekê})\\
\textsc{pl.dir}\tab  \textit{pîyakê} (< \textit{pîya-ekê})\tab  \textit{degakê} (< \textit{dega-ekê})\\
\textsc{pl.obl}\tab  \textit{pîyaka} (< \textit{pîya-eka})\tab  \textit{degaka} (< \textit{dega-eka})\\
\z

With nouns ending in \textit{-e}, the sequence of two identical vowels is reduced into one. 
\ea
\tab  \textit{bize} `goat' (\textsc{f}) \tab  \textit{yane} `house’ (\textsc{m}) \\
\textsc{sg.dir}\tab  \textit{bizekê} (< \textit{bize-ekê}) [KŞ.31] \tab \textit{yaneke} (< \textit{yane-eke}) \\
\textsc{sg.obl}\tab  \textit{bizekê} (< \textit{bize-ekê}) [KŞ.31] \tab \textit{yanekey} (< \textit{yane-ekey}) \\
\textsc{pl.dir}\tab  \textit{bizekê} (< \textit{bize-ekê}) [JP.37] \tab \textit{yanekê} (< \textit{yane-ekê}) \\
\textsc{pl.obl}\tab  \textit{bizeka} (< \textit{bize-eka}) \tab \textit{yaneka} (< \textit{yane-eka}) \\
\z


With feminine nouns ending in \textit{-ê} and \textit{-î}, these final vowels are dropped before the the initial vowel of the definite suffix.
\ea
\tab \textit{kinaçê} `girl' (\textsc{f}) \tab  \textit{jenî} `wife' (\textsc{f}) \\
\textsc{sg.dir}\tab  \textit{kinaçekê} (< \textit{kinaçê-ekê}) 
 \tab \textit{jenekê} (< \textit{jenî-ekê}) [ZQ.14]  \\
\textsc{sg.obl}\tab \textit{kinaçekê} (< \textit{kinaçê-ekê})\tab  \textit{jenekê} (< \textit{jenî-ekê}) [BP.185]  \\
\textsc{pl.dir}\tab \textit{kinaçekê} (< \textit{kinaçê-ekê})\tab  \textit{jenekê} (< \textit{jenî-ekê}) [RE.21]  \\
\textsc{pl.obl}\tab \textit{kinaçeka} (< \textit{kinaçê-eka})\tab  \textit{jeneka} (< \textit{jenî-eka}) \\
\z

The definite suffix\is{definite suffix} \textit{-e} appears to have a more restricted usage than \textit{-eke}. Data from the text corpus shows that it is limited to occur with masculine\is{masculine} singular\is{singular} nouns which are prominent in discourse. 
\ea
\tab  \textit{her} `donkey' \tab  \textit{kuř} `boy' \\
\textsc{sg.dir}\tab  \textit{her-e} [HB.48]  \tab \textit{kuř-e} [KŞ.49]  \\
\textsc{sg.obl}\tab  \textit{her-e-y} [HB.15]  \tab \textit{kuř-e-y}  [BP.206]  
\z

The definite suffix\is{definite suffix} \textit{-eke} originates from the diminutive\is{diminutive} suffix \textit{-ak} of Iranian (\citealt{haig_definiteness_2019}; \citealt{nourzaei_definiteness_2021}; \citealt{karim_synchrony_2021}). This sense has been preserved down to the present day, for instance, to address a loved one:
\TabPositions{2cm,4.5cm}
\ea
\textit{jenekê}\tab  `Wife!’\tab  [ZB.13]  \\
\textit{pîyake}\tab  `Husband!’\tab  [JF.21]\\
\z

\subsection{Indefiniteness}\label{Indefiniteness}

Indefiniteness\is{indefiniteness} is marked on singular\is{singular} nouns via the suffix \textit{-êwe}, which alternates freely with \textit{-êw}. In the plural, the indefiniteness is marked by \textit{-ê} in the direct case, and \textit{-a} in the oblique case. The singular forms are often truncated into \textit{-ê}. Vowel hiatus is generally tolerated when \textit{-êwe}/-\textit{ê} attaches to stressed vowel-final nouns:
\ea
\textit{mizgî-ê} \tab  `mosque' \tab <  \textit{mizgî + -ê} \\
\textit{řo-ê} \tab  `a day' \tab  <  \textit{řo + -ê} \\
\textit{pîya-ê}\tab  `a man’\tab  < \textit{pîya + -ê} \\ 
\z

The addition of the indefinite suffix to nouns ending in \textit{-a} can have two outcomes. In most cases, the result is the merger of the two vowels into /ɛ/. Occasionally, an epenthetic <y> resolves vowel hiatus.
\ea
\textit{pîyɛ} \tab  `a man' \tab < \textit{pîya + -ê} \\ 
\textit{pîyayê}\tab  `a man’\tab < \textit{pîya + -ê} \\ 
\z

However, the affixation of the indefinite suffix\is{indefinite suffix} results in the deletion of the unstressed final vowel of nouns ending in unstressed \textit{î} and \textit{ê}. 
\TabPositions{2cm,5cm}
\ea
\textit{jenêwe}\tab  `a woman’\tab  < \textit{jenî + -êwe} \\
\textit{kîseɫê}\tab  `a tortoise’\tab  < \textit{kîseɫî + -ê} \\
\textit{şewê}\tab  `night (\textsc{f.sg.obl})’\tab  < \textit{şewe + -ê} \\
\z

In the vernacular of Hewraman Tekht, \textit{-êwe }is used with masculine\is{masculine} and feminine\is{feminine} nouns alike. \citet[15]{mackenzie_dialect_1966}{} reports that in Luhon H.\il{Hewramî!Luhon} the indefinite suffix\is{indefinite suffix} is \textit{-êw} for masculine\is{masculine} nouns and \textit{-êwe} for feminine\is{feminine} nouns. This distinction is not held in Tekht H.\il{Hewramî!Tekht}, at least in places where the main text corpus was collected. Rather, \textit{-êwe} has taken over as the predominant form (\ref{ex/êwe}). Nevertheless, it is noticeable that \textit{-êw} can be attested occasionally (\ref{ex.êw}). 

\ea \label{ex/êwe}
\textit{jenêwe}\tab `a woman' (\textsc{f})\tab [ZB.8]  \\
\textit{darêwe}\tab `a tree’ (\textsc{m})\tab [ZB.41]  \\
\textit{kuřêwe}\tab `a boy’ (\textsc{m})\tab [KŞ.30]  \\
\textit{kîseɫêwe}\tab `a tortoise’ (\textsc{f})\tab [DG.61]  \\
\z
\TabPositions{3cm}
\ea \label{ex.êw}
\textit{bizêw}\tab `a goat' (\textsc{f}) \\
\textit{duwe qeranî-êw}\tab `a two-kurus coin’ \\
\z

The singular indefinite\is{indefinite suffix} forms are often reduced to \textit{-ê}, showing no gender\is{gender} distinction.
\TabPositions{2cm,5cm,7cm}
\ea
\textit{nefer-ê}\tab `a person’\tab [JH.24]  \\
\textit{tewenê} \tab `a stone’\tab [ZP.50] \tab cf. \textit{tewenî} `stone’ \\
\textit{yagê}\tab `a place’\tab [JP.20] \tab cf. \textit{yagê} `place’ \\
\textit{pîya-ê}\tab `a man’\tab [JP.154] \tab  \\
\textit{mebal-ê}\tab  `a cell phone’\tab [RE.30] \\
\z

The data from the text corpus suggests that the indefinite suffix\is{indefinite suffix} is not compatible with case marking\is{case marking}, illustrated by \textit{pilekanî} `ladder’ and \textit{kuř} `son’ in Table \ref{tab:indef_infl}. However, case distinction is maintained in the plural.
\begin{table}

    \begin{tabular}{llll}
    \lsptoprule
& \textsc{indf.m}&\textsc{indf.f}&\textsc{indf.pl} \\ 
\midrule
\textsc{dir}&\textit{kuř-ê(we)}&\textit{pilekanê(we)}&\textit{kuř-ê / pilekan-ê} \\
\textsc{obl}&\textit{kuř-ê(we)}&\textit{pilakanê(we)}&\textit{kuř-a / pilekan-a} \\
\lspbottomrule
    \end{tabular}
    \caption{The indefinite nominal inflection}
    \label{tab:indef_infl}
\end{table}

Therefore, the distinction between direct\is{direct case} and oblique cases\is{oblique case} is neutralised in the singular. Examples are listed below. In (\ref{ex.indf-neutr1})--(\ref{ex.indf-neutr2}), the indefinite suffix\is{indefinite suffix} has blocked the case marking\is{case marking} on the direct object arguments.
\ea 
\textit{mebalê gêro desşo.} \\ 
\gll mebal-ê gêr-o des=ş=o \\ 
 cell\_phone\textsc{.m-indf} take\textsc{.prs.ind-3sg:A} hand\textsc{.m=3sg:PSR=post} \\ 
\glt `He holds a mobile phone in his hand.' \hfill [RE.30] \label{ex.indf-neutr1}\\
\z

\ea 
\textit{pilekanêwe binye!} \\ 
\gll pilekanî-êwe bi-ny(e)-e\\ 
 stairway\textsc{.f}\textsc{-indf} \textsc{sbjv-}put\textsc{.prs}-\textsc{2sg:A} \\ 
\glt `Set up a stairway!' \hfill [JH.12] \label{ex.indf-neutr2} \\
\z 
In (\ref{ex.indf-neutr3})--(\ref{ex.indf-neutr4}), the indefinite suffix\is{indefinite suffix} \textit{-ê} has blocked the case marking\is{case marking} on the preposition complement and the genitive, respectively.
\ea
\textit{çenû pîy{ɛ}we mênê.} \\ 
\gll çenû pîy{ɛ}we m-ê-nê \\ 
 with\textsc{.ez.gen} man\textsc{.m}\textsc{.indf} \textsc{ind-}come\textsc{.prs}\textsc{-3pl:S} \\ 
\glt `She was coming [towards him] with a man.' \hfill [ZP.66] \label{ex.indf-neutr3} \\
\z
 
\ea 
\textit{minîç şimşû yerû heywanêt midew pene.'} \\ 
\gll min=îç şimş(î)=û yeher-û heywan-ê=t mi-de-û pene \\ 
 \textsc{1sg=add} spleen=and liver\textsc{.m-ez.gen} animal\textsc{.m-indf=2sg:R} \textsc{ind-}give\textsc{.prs-1sg:A} to \\ 
\glt `I will give you the spleen and liver of an animal.' \hfill [JP.230] \label{ex.indf-neutr4}
\z 

By contrast, in the Tekht varieties of Silên and Nwên, there is evidence of the compatibility of case marking\is{case marking} with indefinite suffixes\is{indefinite suffix}, as suggested by the following examples:
\ea
\textit{îse mildê tûşo derbenêwî meydê.} \\
\gll îse mi-l-dê tûş-û derben-êw-î m-e-îdê \\
now \textsc{ind}-go.\textsc{prs-2pl:S} encountering-\textsc{ez.gen} canyon-\textsc{indf.m-sg.obl} \textsc{ind-}come.\textsc{prs-2pl:S} \\
\glt `On your way, you will run into a canyon.' \hfill [ME.33] 
\z 

\ea 
\textit{minîç dewayêwîş şanî miđew.} \\
\gll min=îç deway-êw-î=ş şanî mi-đe-û \\
\textsc{1sg=add} medication.\textsc{m-indf.m-sg.obl=3sg:R} showing \textsc{ind-}give.\textsc{prs-1sg:A} \\
`I will show him a medication.' \hfill [DB.260]
\z 

\section{Nominal word formation}

New nominal lexemes are created either through derivation, or compounding. Affixation is more productive than compounding for the derivation of nouns. 

\subsection{Derivation} \label{nominal-affixation}
Derivational affixes are mainly of two types: those deriving nouns from other nouns, yielding denominal nouns\is{denominal nouns}, and those deriving nouns from adjectives, resulting in deadjectival nouns. Some derivational affixes, e.g., \textit{-î}, \textit{-gerî} derive nouns from both adjectives and nouns. 

\subsubsection{-î}
The suffix \textit{-î} is one of the most productive derivational affixes. It derives abstract nouns with roughly the meaning `in the state of being’ from adjectives.
\TabPositions{1.75cm,5.5cm,8cm}
\ea
\textit{sextî}\tab `difficulty’\tab  cf. \textit{sext}\tab `difficult’ \\
\textit{mecbûrî}\tab `obligation’\tab  cf. \textit{mecbûr}\tab `obliged’ \\
\textit{azađî}\tab `freedom’\tab  cf. \textit{aza(đ)}\tab `free’ \\
\textit{genekarî}\tab `adultery, debauchery’\tab  cf. \textit{genekar}\tab `adulterous’ \\
\textit{adizî}\tab `anger’\tab  cf. \textit{adiz}\tab `angry, anxious’ \\
\z

\textit{-î} is also used to derive nouns from both simple nouns, e.g. \textit{zarole}, and derived nouns, e.g. \textit{heywandar}.

\TabPositions{2.15cm,5.5cm,8cm}
\ea
\textit{zaroɫeyî}\tab  `childhood’\tab  cf. \textit{zaroɫe} `child’ \\
\textit{ħeywandarî}\tab `animal husbandry'\tab  cf. \textit{ħeywandar}   `one who owns\\
\tab \tab  domesticated animals' 
\z

\subsubsection{-dar}
This suffix derives concrete masculine nouns with the sense of ``owning'' from the base noun.
\ea
\textit{wiɫaxdar}\tab  `stable-keeper’\tab cf. \textit{wiɫax}\tab  `horse’ \\
\textit{goşdar}\tab `listener’\tab cf. \textit{goş}\tab  `ear’ \\
\textit{baɫdar}\tab `bird’\tab cf. \textit{baɫ}\tab  `wing’ \\
\textit{gîyandar}\tab `living being’\tab cf. \textit{gîyan}\tab  `soul’ \\
\textit{dûkandar}\tab `shopkeeper’\tab cf.\textit{dûkan}\tab  `shop’ \\
\textit{ħeywandar}\tab `one who owns\tab  cf. \textit{ħeywan}\tab  `animal’ \\
\tab  domesticated animals'\tab  \\
\z

The feminine form \textit{-dare} expresses the female counterpart of the above items. 
\ea
\textit{dûkan-dare}\tab `female shopkeeper’ \\
\z 

\subsubsection{-eke/ -ekê} 
In some animal nouns, what seems to be the definite suffix\is{definite suffix} \textit{-eke} (\textsc{m}), \textit{-ekê} (\textsc{f}) now forms part of the base noun and is thus lexicalised. This reflects the origin of the definite suffix\is{definite suffix} as a diminutive\is{diminutive} suffix in West Iranian. 
\ea
\textit{çolekê}\tab  `sparrow’ \\
\textit{goreke}\tab `calf’ \\
\textit{çaleke}\tab  `badger' \\
\z

\subsubsection{-îne} 
The derivational feminine suffix \textit{-îne} derives nouns denoting `meal’ from nouns denoting what the meal is principally made of.
\TabPositions{2.25cm,8cm,9cm}
\ea
\textit{hiɫoşîne}\tab `dish made from sour plum’\tab  cf. \textit{hiɫoşe} `sour plum’ \\
\textit{xeple zeřatîne}\tab  `a type of bread made of corn flour’\tab  cf. \textit{zeřat} `corn’ \\
\textit{şelemîne}\tab  `dish made of cracked wheat \\
\tab and turnip’\tab  cf. \textit{şelem} `turnip’ \\
\textit{dowîne}\tab  `dish made of cracked wheat \\
\tab  diluted and yoghurt’\tab  cf. \textit{do} `diluted \\
\tab \tab yoghurt’ \\
\z 

\subsubsection{-le /-ɫe}
The deminutive suffix \textit{-le} marks gender\is{gender} and number\is{number}: \textit{-le / -ɫe} (\textsc{m.sg}); \textit{-lê / - ɫê} (\textsc{f.sg}); \textit{-lê / -ɫê} (\textsc{pl}). It is used to derive nouns (see below) and adjectives (see \S\ref{deriv-adj}). In its denominal use, the suffix has primarily the sense of `small’.
\TabPositions{3cm}
\ea
\textit{zaroɫe}\tab `child’ \\ 
\textit{bizɫe} (\textsc{m})\tab `small goat’ \\
\textit{bizɫê} (\textsc{f})\tab `small goat’ \\ 
\textit{mişarle}\tab  `small hoe’ \\
\textit{gicîle}\tab `small shirt’ \\
\textit{duwê seringaɫê}\tab `two cushions’ \\
\z 

The suffix has become lexicalised with kin terms such as \textit{waɫê} `sister’, \textit{zaroɫe} `child', as well as with nouns like \textit{paɫê} `shoes’ (< \textit{pa} `foot’ + \textit{ɫê}). 

By extension from the semantic meaning `small’, the diminutive\is{diminutive} suffix is used more pragmatically to denote affection \citep[for a typology of diminutives see][]{jurafsky_universal_1996}{}{}. This is particularly true with kinship terms:
\TabPositions{2cm,6cm,8cm}
\ea
\textit{tate-le}\tab `(dear) father!’\tab \textit{tate}\tab `father’ \\
\textit{eya-lê} \tab `(dear) mother’\tab \textit{eya}\tab `mother’ \\
\textit{mama-lê}\tab `(dear) grandmother’\tab \textit{mama}\tab `grandmother’ \\
\z 

As noted by \citet[553]{jurafsky_universal_1996}{}, the diminutive\is{diminutive} can have an ``imitation" sense, viz., it can mark nouns which are viewed as imitations or copies of natural objects, often body parts. Tekht H.\il{Hewramî!Tekht} exhibits this function of diminutive\is{diminutive}. As seen below, the flora terms \textit{tisle} and \textit{şotîle} exhibit similarities with \textit{tis} and \textit{şot}, respectively. The similarity in case of \textit{şotîle} `euphorbia plant' is linked to the plant producing milky sap. 
\ea \label{ex.dim-imitation}
\textit{tisle}\tab  `a prickly fetid plant'\tab  cf. \textit{tis}\tab `silent fart' \\
\textit{şotîle}\tab  `euphorbia plant' \tab  cf. \textit{şot} \tab `milk' \\
\z 

Related diminutive\is{diminutive} suffixes are \textit{-oɫe}, and \textit{-île}: 
\TabPositions{2cm,4cm,6cm}
\ea
\textit{camoɫe}\tab `small pot’\tab ( <\textit{cam} `pot’ + \textit{-oɫe}) \\
\z

\subsubsection{-ga/ -ge, -gê}
The derivation suffix \textit{-ga/ -ge, -gê} derives place names from nouns denoting objects.
\ea
\textit{awîrga}\tab `fire place’\tab < \textit{awîr} (\textsc{m}) `fire’+ \textit{-ga} `place’ \\
\textit{seringa}\tab `pillow’\tab < \textit{serin} (\textsc{m}) + \textit{-ga} `place’ \\
\z

The suffix has become lexicalised in nouns that already denote a place name. In addition, it has become lexicalised as part of the adverbial noun \textit{wêrega} `evening' and \textit{yagê} `place'.
\ea
\textit{ewêge}\tab `there’ \tab < \textit{awê} `that’ + -\textit{ge} `over there’ \\
\textit{yagê}\tab `place’ \tab < \textit{ya} `place’ + -\textit{gê} `over there’ \\
\textit{cûwe}\tab `stream’\tab < *\textit{cû} `stream' + \textit{-ge} `place’\\
\textit{dega}\tab `village’\tab < \textit{de} `village’ + -\textit{ga} `place’ \\
\textit{milega}\tab `ridge’\tab < \textit{mile} `neck?’ + -\textit{ga} `place’ \\
\z

\subsubsection{-gerî} 
The derivational masculine suffix \textit{-gerî }derives quality nouns with the sense of ``association of a person or a group of people with certain features'' from nouns and adjectives.
\TabPositions{2.5cm,5.5cm,8cm}
\ea
\textit{gewregerî} \tab `nobleness’\tab \textit{gewre}\tab `big’ \\
\textit{hewramîgerî}\tab `Hewram\^ihood’\tab \textit{hewramî} \tab `Hewram\^i’ \\
\textit{pîyagerî} \tab `manhood’\tab \textit{pîya}\tab `man’ \\
\textit{padşagerî}\tab `king-hood’\tab \textit{padşa}\tab `king’ \\
\textit{aẍegerî}\tab `lordship’\tab \textit{aẍe}\tab `lord’ \\
\textit{qeyẍagerî} \tab  `chieftain-hood'\tab  \textit{qeyẍa} \tab  `chieftain' \\
\z
 
\subsubsection{-wan}
The derivational masculine suffix \textit{-wan} has the approximate meaning of ``taking care of''. The feminine\is{feminine} form is \textit{-wane}.

\ea
\textit{asawan}\tab `miller' (\textsc{m})\tab  cf. \textit{asaw}\tab  `mill' \\
\textit{asawane}\tab `miller' (\textsc{f})\tab  cf. \textit{asaw}\tab  `mill' \\
\textit{neçîrewan}\tab `hunter’\tab  cf. \textit{neçîre}\tab `hunt' \\
\textit{řoçewan}\tab `one who fasts’\tab  cf. \textit{řoçe}\tab  `fasting' \\
\z
 
\subsubsection{-yane}
The derivational masculine suffix \textit{-yane} occurs in the sense of `vernacular associated with a region'. 
\ea
\textit{hewramî-yane}\tab  `Hewram\^i vernacular' \\
\textit{paweyane}\tab  `Pawe vernacular' \\
\z

\subsubsection{-gel}
The derivational suffix \textit{-gel} is derived from the \textit{gele} `herd, flock' (\textsc{m}). The suffix adds a collective meaning to the base noun. This suffix is assumed to be the origin of the plural suffix \textit{-gel} in the neighbouring CK\il{Kurdish!Central} varieties of the Sanandaj region and in Southern Kurdish\il{Kurdish!Southern}. 
\ea
\textit{gaw-gel}\tab  `flock of cows' \\
\textit{pez-gel}\tab  `flock of sheep' \\
\z

\ea
\textit{şuwanew pezgelî!}\\
\gll şuwane-û pezgel-î\\
shepherd-\textsc{ez.gen} sheep\_herd-\textsc{m.sg.obl}\\
\glt `Shepherd!'  \hfill[ÇH.118]
\z

\subsection{Nominal compounding} 

Compound nouns are primarily formed by juxtaposing the component parts. The component parts form a single word in terms of stress marking, gender assignment, and agreement. As discussed in \S\ref{gendersection}, the compound word \textit{hêɫeruwenî} `fried eggs' ( < \textit{hêɫe} `egg' (\textsc{m}) + \textit{řuwenî} `oil' (\textsc{f}) has feminine gender, which is determined by the non-final stress in \textit{řuwenî} `oil'. 

The most frequent compounds are N-N and N-V compounds, where the V equals the verb's present stem. \\

\subsubsection{N-N compounds}
\TabPositions{2cm,5cm,8.35cm}
\ea
\textit{hêɫeřûwenî}\tab `fried eggs’\tab \textit{hêɫe} `egg’ + \textit{řûwenî} `oil’ \\
\textit{yanekoɫê}\tab `(leaving) houses’\tab \textit{yane} `house’ + \textit{koɫê} `shoulder’ \\
\textit{herbene}\tab `donkey keeper’\tab \textit{her} `donkey’ + \textit{bene} `servant’ \\
\z

Less commonly, compound nouns are formed by adding a spatial particle to the semantic, phonological and morphosyntactic head noun. 
\TabPositions{2cm,5.5cm,8.35cm}
\ea
\textit{werdes}\tab `servant’\tab cf. \textit{wer-} `front’ + \textit{des} `hand’ \\
\textit{çêrxan}\tab  `basement' \tab  cf. \textit{çêr-} `under' + \textit{xan} `house' \\
\textit{serxan}\tab  `top floor in a house' \tab  cf. \textit{ser-} `up' + \textit{xan} `house' \\
\z

\subsubsection{N-V compounds} \label{sect:n-v-compounds}
N-V compounds feature the present stem of the verb in the compound. The most common verb stem used is \textit{-ker} `do’, which derives nouns denoting occupations from nouns which either denote the object manufacture or a process.
\TabPositions{2.5cm,6cm,8.35cm}
\ea
\textit{wêjen-ker}\tab `gum producer’ (\textsc{m})\tab cf. \textit{wêjen} \tab `Tragacanth gum’ \\
\textit{çaşt-ker}\tab `cook’ (\textsc{m}) \tab cf. \textit{çaştî} \tab `food’ \\
\textit{selem-ker}\tab `pre-seller' (\textsc{m})\tab cf. \textit{selem} \tab `pre-selling’ \\
\textit{kar-ker, kareker}\tab `worker' (\textsc{m})\tab cf. \textit{kar} \tab `work’ \\
\textit{coɫa-ker}\tab `weaver’\tab cf. \textit{coɫa}\tab `weaving’ \\
\z

The feminine\is{feminine} counterpart of the above names is derived by adding the unstressed \textit{-e} to the compound noun: 
\TabPositions{3.5cm}
\ea
\textit{çaştker-e}\tab `cook' (\textsc{f}) \\
\textit{karker-e, kareker-e}\tab `worker' (\textsc{f}) \\
\z

Other verb stems are found marginally in N-V compounds: 
\TabPositions{2cm,5.5cm,8cm}
\ea
\textit{çareniwîs}\tab `fortune-teller’\tab cf. \textit{çare} `destiny’ + \textit{niwîs-} `write’ \\
\textit{hermanber}\tab `domestic worker’\tab cf. \textit{herman} `work’ + \textit{ber-} `take’ \\
\textit{nanpeç}\tab  `baker'\tab  cf. \textit{nan} `bread' + \textit{peç-} `bake' \\
\textit{xuɫekêş}\tab  `soil carrier'\tab  cf. \textit{xuɫe} `soil' + \textit{kêş-} `pull' \\
\z

There are also a few fauna terms which seem to be formed by reduced clauses similar to the English\il{English} flower term \textit{forget-me-not}. The term for cricket insect is \textit{cêř cêř kere}, containing \textit{cêř} `chirping’, \textit{kere} `do’, yielding `does chirping’. The \textit{-e} element on the present stem of `do’ seems to be a nominaliser. Alternatively, these terms can be analysed as N-V compounds with the final \textit{-e} marking the feminine gender. Similarly, the term for `mantis' is \textit{coɫakere}, and \textit{heɫîzewere} refers to a type of grasshopper: 
\TabPositions{2cm,7cm,9cm}
\ea
\textit{coɫakere} \tab < \textit{coɫa} `weaving’ + \textit{kere} `do' \tab `mantis’ [lit. `weaver']  \\
\textit{heɫîzewere}\tab  < \textit{heɫîze} `churn' + \textit{were} `eating' \tab  `a type of grasshopper' \\\tab \tab  [lit. `churn eater'] \\
\z

\subsubsection{N-ADJ compounds}
The compound nouns in this category consists of lexicalised items for expressing concepts such as `old man', `old woman', etc. The positioning of the adjective may vary depending on the compound.
\TabPositions{2cm,6.5cm}
\ea
\textit{řîşçerme}\tab `old man [lit. white beard]’\tab cf. \textit{řîş} `beard’ + \textit{çerme} `white’ \\
\textit{pîrejenî}\tab `old woman’\tab cf. \textit{pîr} `old’ + -e + \textit{jenî} `woman’ \\
\z


\subsubsection{Echo compounds} \label{sect:echo-compounds}
Echo compounds\is{echo compounds} are a feature of regional languages such as Kurdish\il{Kurdish}, Persian\il{Persian} and Turkish\il{Turkish}. They are formed by a partial reduplication strategy in which the initial consonant in the base word is replaced by \textit{m}. The reduplicant can be juxtaposed to the base or preceded by the coordinate particle \textit{꞊û} `and’.
\TabPositions{2cm,5.5cm,9cm}
\ea
\textit{jen꞊û menî}\tab ( < \textit{jenî û menî}) \tab  `women and so on’\tab [BP.49]  \\
\textit{qise=û mise}\tab  \tab  `gossip and such' \tab  [RE.23] \\
\textit{çîw mîw}\tab  \tab `things and stuff'\tab  \\
\z




