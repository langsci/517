\addchap{\lsAbbreviationsTitle}
% \addchap{Abbreviations and symbols}
\section*{General abbreviations}

\begin{tabularx}{.45\textwidth}{lQ}
Ar.& Arabic \\
Av. & Avestan \\
CK.& Central Kurdish \\
H. & Hewramî \\
\end{tabularx}
\begin{tabularx}{.45\textwidth}{lQ}
NK.& Northern Kurdish \\
P.& Persian \\
SK.& Southern Kurdish \\
Tr.& Turkish \\
\end{tabularx}

\section*{Grammatical abbreviations}
\begin{tabularx}{.45\textwidth}{lQ}

\textsc{1} & first person\\
\textsc{2} & second person \\
\textsc{3} & third person \\
\textsc{A} & agent-like argument of a transitive clause \\
\textsc{add} & additive \\
\textsc{adj}& adjective \\
\textsc{adjzr}& adjectivizer \\
\textsc{adp}& adposition \\
\textsc{adv}& adverbial \\
\textsc{aug} & augment \\
\textsc{aux} & auxiliary \\
\textsc{clf} & classifier \\
\textsc{cmpr} & comparative \\
\textsc{compl} & completive \\
\textsc{conj} & conjunctive \\
\textsc{cop} & copula \\
\textsc{def} & definite \\
\textsc{deic} & deictic \\
\textsc{dem} & demonstrative \\
\textsc{dir} & direct \\
\textsc{disc.ptcl} & discourse particle \\
\textsc{dist} & distal \\
\textsc{emph} & emphatic \\
\end{tabularx}
\begin{tabularx}{.45\textwidth}{lQ}
\textsc{ep} & epenthesis \\
\textsc{exist} & existential particle \\
\textsc{ez} & ezafe linker \\
\textsc{ez.attr} & attributive ezafe \\
\textsc{ez.cmpd} & ezafe compound \\
\textsc{ez.gen} & genitive ezafe \\
\textsc{f} & feminine \\
\textsc{hort} & hortative \\
\textsc{imp} & imperative \\
\textsc{ind} & indicative \\
\textsc{indf} & indefinite \\
\textsc{inf} & infinitive \\
\textsc{intj}& interjection \\
\textsc{intr}& intransitive \\
\textsc{loc}& locational \\
\textsc{m} & masculine \\
\textsc{NC} & non-canonical subject \\
\textsc{n.f} & feminine noun \\
\textsc{n.m} & masculine noun \\
\textsc{neg} & negative \\
\textsc{nmlz}& nominalizer \\
\textsc{O} & direct object of a transitive verb \\
\textsc{obl} & oblique \\
\end{tabularx}

\begin{tabularx}{.45\textwidth}{lQ}
\textsc{ord}& ordinal \\
\textsc{P}& patient-like argument of a transitive clause\\
\textsc{pl} & plural \\
\textsc{pn} & proper noun \\
\textsc{post} & postposition \\
\textsc{povb} & post-verb \\
\textsc{pro} & pronoun \\
\textsc{proh} & prohibitive \\
\textsc{prox} & proximal \\
\textsc{prs} & present \\
\textsc{prsv} & presentative \\
\textsc{PSR} & possessor \\
\textsc{pst} & past \\
\textsc{ptcl} & particle \\
\textsc{ptcp} & participle \\
\textsc{pvb} & preverb \\
\textsc{q.ptcl} & question particle \\
\end{tabularx}
\begin{tabularx}{.45\textwidth}{lQ}
\textsc{R} & recipient-like argument in ditransitive clauses \\
\textsc{R}& non-core arguments of intransitive and transitive clauses \\
\textsc{recp}& reciprocal base\\
\textsc{refl} & reflexive \\
\textsc{S} & single argument of an intransitive verb \\
\textsc{sbjv} & subjunctive \\
\textsc{sbrd} & subordinate particle\\
\textsc{sg} & singular \\
\textsc{supr}& superlative \\
\textsc{T} & theme-like argument in ditransitive clauses \\
\textsc{tam} & tense-aspect-mood \\
\textsc{tr}& transitive \\
\textsc{voc} & vocative \\
\end{tabularx}

\section*{Codes of example sources}
\subsection*{Main text corpus \citep{mohammadirad_speking_the_past}}


\begin{tabularx}{.45\textwidth}{lQ}
BP & Oral history  \\
DG & Local anecdote/myth\\
DP & Oral history \\
HB & Local anecdote/myth \\
JE & Process narrative \\
JH & Folktale \\
JM & Autobiography \\
JP & Oral history \\
\end{tabularx}
\begin{tabularx}{.45\textwidth}{lQ}
KŞ & Folktale \\
PM & Local anecdote/myth \\
RE & Process narrative \\
ŞC & Local anecdote, recent history \\
ZB & Local anecdote/myth \\
ZP & Oral history \\
ZQ & Local anecdote/myth\\
\\
\end{tabularx}

\subsection*{Codes of examples from the folktale corpus \citep{mohammadirad_folktale_inprep}}

\begin{tabularx}{.45\textwidth}{lQ}
BB& Folktale  \\
BM& Folktale \\
ÇH& Folktale \\
ÇK& Folktale \\
DB& Folktale \\
ED& Folktale \\
HJ& Folktale \\
HM& Local anecdote \\
HS& Folktale \\
HW& Folktale\\
JC& Folktale\\
JF& Folktale\\
JL& Folktale\\
KK& Folktale \\
KT& Folktale \\
ME& Folktale \\
\end{tabularx}
\begin{tabularx}{.45\textwidth}{lQ}
MF& Folktale \\
MM& Folktale\\
MP& Folktale \\
MR & Folktale \\
PK & Folktale \\
PP& Folktale\\
PW& Folktale \\
SH& Folktale \\
SK& Folktale \\
ŞE& Folktale \\
ŞŞ& Folktale\\
WL& Folktale\\
XŞ& Local anecdote \\
XX& Folktale \\
YX& Local anecdote \\
\\
\end{tabularx}
