\chapter{Syntax of the clause}
\begin{sloppypar}
This chapter surveys the syntax of different types of simple clauses and the argument structure associated with light verb constructions\is{light verb constructions}. The following topics are discussed: verbal clauses (\S\ref{sect:verbal_clasue}); copula clauses (\S\ref{sect:copula_clasue}), clauses with existential particles (\S\ref{sect:existentialclauses}), light verb constructions\is{light verb constructions} (\S\ref{sect:lvc-syntax}), reciprocal constructions (\S\ref{sect:reciprocal}); periphrastic causative constructions (\S\ref{sect:Periphrastic-causative}), passive clauses (\S\ref{sect:passive}), and interrogative clauses (\S\ref{sect:interr}).  

\section{Verbal clauses} \label{sect:verbal_clasue}

\subsection{Subject constituent}
The subject constituent exhibits typical properties of subjecthood, including control of reflexives. Recall that in the vernaculars of Silên and Nwên, the reflexive pronoun\is{reflexive pronouns} can take the bare\is{bare} form \textit{wê} in expressing certain functions. In (\ref{ex.reflexive1a}), the subject constituent controls the reference of the reflexive in the following clause. In (\ref{ex.reflexive1b}), it controls the reference of the reflexive in the subordinate clause\is{subordinate clause}.

\ea

\ea[]{
\textit{ayêç mila wê şarawe.} \\
\gll ayê=ç mi-l-a \textbf{wê} şar-a=we \\
\textsc{3pl=add} \textsc{ind-}go.\textsc{prs-3pl:S} \textsc{refl} hide.\textsc{prs.ind-3pl:A=compl} \\
\glt `They\textsubscript{\textit{i}} went [and] hid \textbf{themselves}\textsubscript{\textit{i}}.'  \label{ex.reflexive1a}
}
\ex[]{
\textit{zatşa mebo wê aşkera kera.} \\ 
\gll zat=şa me-b-o \textbf{wê} aşkera ker-a \\ 
 fear=\textsc{3pl:NC} \textsc{neg.ind-}be.\textsc{prs-3sg:S} \textsc{refl} disclosed do.\textsc{prs.ind-3pl:A} \\ 
\glt `(They\textsubscript{\textit{i}}) were afraid to make \textbf{themselves}\textsubscript{\textit{i}} visible.' \hfill[ÇK.67] \label{ex.reflexive1b}
}
\z
\z 

In the following example from the vernacular of Nwên, the bare\is{bare} form \textit{wê} has the same reference as the subject. 

\ea
\textit{lûwa zemînêweş da werû wê.} \\
\gll lûwa zemîn-êwe=ş da-{\O} wer-û \textbf{wê} \\
go.\textsc{pst.3sg:S} land.\textsc{indf=3sg:A} give.\textsc{pst.3sg:O} on-\textsc{ez.gen} \textsc{refl} \\
\glt `(He\textsubscript{\textit{i}}) went and put himself on a land [lit. gave a land on \textbf{himself}\textsubscript{\textit{i}}].' \\\hfill[ED.30]
\z 

\subsection{Word order\is{word order} configurations} \label{sect:wo_verbal_clasue}
The basic word order\is{word order} pattern is AOV. Despite having OV order, Hewramî\il{Hewramî} exhibits several head-initial configurations, including Noun-Adjective, Possessed-Possessor, Matrix clause-complement clause, Verb-Goal, and Verb-Recipient\is{recipient}, running against the predictions of the head-directionality hypothesis \citep[]{dryer_greenbergian_1992}{}. 

Taking oblique arguments\is{oblique arguments} into account, the word order\is{word order} is of the AOVX type. Following \citet{hawkins_ordering_2008}, the notation `X' indicates oblique arguments\is{oblique arguments}. It is seen later that the post-verbal X is especially the case for goals, recipients\is{recipient}, and addressees\is{addressee}. Similarly, the basic word order\is{word order} pattern in the neighbouring Iranian and Semitic languages has been reported to be AOVX (\citealt{haig_post_2022,HaigNoorlanderSchiborr+2025+159+184,mohammadirad_zagros_nodate}). In what follows, following the methodology in \citet{Haig2023}, some word order\is{word order} configurations in verbal clauses are surveyed. In some cases, I apply a gradient corpus-based method to describe word order\is{word order} properties of different arguments.

\subsubsection{Order of subject, object and verb}
The subject constituent, by default, precedes the direct object constituent, hence the order AOV. In the default AOV order, the subject is generally characterised by being topical, i.e., high in the animacy and definiteness scales, and expressing given information. The nuclear stress is generally placed on the direct object expressing new information (\ref{ex.sov2}), that is, on the immediate preverbal constituent. When the direct object expresses given information, the nuclear stress generally falls on the verb (\ref{ex.sov1}).

\ea
\textit{çêrhur zaroɫeke şot wero.} \\ 
\gll çêr=hur zaroɫe-(e)ke ş\`ot wer-o\,\suppipe{} \\ 
 under\textsc{=\textsc{post}} child\textsc{.m-def.m.sg.dir} milk\textsc{.m} eat\textsc{.prs.ind}\textsc{-3sg:A} \\ 
\glt `The baby drank [its] milk from below.' \hfill[ZB.45]  \label{ex.sov2}
\z


\ea
\textit{lalo kinaçekê maro.} \\ 
\gll lalo kinaç(ê)-ekê m-\`ar-o\,\suppipe{} \\ 
 maternal\_uncle\textsc{.m} girl\textsc{.f-def.f.sg} \textsc{ind-}bring\textsc{.prs-3sg:A} \\ 
\glt `Then, the [girl’s] uncle brought her (to Hewraman).' \hfill[ZP.49] \label{ex.sov1}
\z 

Occasionally, the subject constituent comes between the verb and its direct object, resulting in OAV order. This is especially true in past transitive constructions when the A argument is focal. By contrast, in clauses with AOV order, the A argument is generally given. Another difference is that in clauses with OAV order, the O argument is given, but this is not always true of clauses with AOV order. Recall from \S\ref{sect:differential-A-indexing} that the focality of the A argument in this construction triggers no-indexing of the A argument on the verb.

\ea
\textit{heywane awê berde} \\ 
\gll heywane awê berd-e \\ 
 animal\textsc{.f.sg.dir} water\textsc{.f.sg.obl} take\textsc{.pst-3sg}\textsc{.f:O} \\ 
\glt `The flood [lit. water] took away the animals.' \hfill[ZB.21]
\z


\ea
\textit{kuřû min řozgarya nekuşten.} \\
\gll kuř-û min řozgarî-a ne-kuşte=n \\
son-\textsc{ez.gen} \textsc{1sg} \textsc{pn-pl.obl} \textsc{neg-}kill.\textsc{pst.ptcp.m=cop.3sg.m:O} \\
\glt `My son, the Rozgaris did not kill him.' \hfill[HM.08]
\z 

The subject constituent may be expressed in the post-verbal position, tagged as an afterthought (\ref{ex.OVA11}). The effect of afterthought is to bind the clause with what precedes.

\ea

\ea[]{
\textit{qotêw aman.} \\ 
\gll qot(e)-êw ama=n \\ 
 coffin.\textsc{m-indf} come.\textsc{pst.ptcp.m=cop.3sg:S}\\ 
\glt `A coffin arrived [floating on the water].' \hfill[MF.75]
 }
\ex[]{ 
\textit{asawekeş bînan qotekey.} \\ 
\gll asaw-eke=ş bîna=n qote-(e)key \\ 
 mill.\textsc{def.m.sg.dir=3sg:A} block.\textsc{pst.ptcp.m=cop.3sg:O} coffin\textsc{-def.m.sg.obl} \\  
\glt `The coffin had blocked the [water stream in] the mill.' \hfill[MF.76] \label{ex.OVA11}
}
\z 
\z
As discussed in \S\ref{sect:DAM}, the majority of subjects are only expressed through indexing via mobile clitic pronouns. Of the total 395 past transitive clauses in the text corpus, 103 have overt A arguments, amounting to 26\% of overt A NPs arguments in the whole corpus. Overt subjects only occur in certain contexts, e.g., in contrastive topic constructions.

\ea
\textit{min řencim daw to berd!} \\
\gll \textbf{min} řenc=im da=w \textbf{to} berd \\
\textsc{1sg} toil=\textsc{1sg:A} give.\textsc{pst=}and \textsc{2sg} take.\textsc{pst} \\
\glt `I toiled, and you took (the credit)!' \hfill[YX.16] 
\z 

\subsubsection{Subject / Verb}
A lexical subject is, by default, positioned before the verb, hence the order SV. This order is associated with subjects with different referential properties, including definite subjects (\ref{ex.SV1})--(\ref{ex.SV2}), topical subjects marked by the additive\is{additive} particle (\ref{ex.SV3}), and indefinite subjects (\ref{ex.SV4}).

\ea
\textit{pîyake mêwe.} \\ 
\gll pîya-(e)ke m-ê=we \\ 
 man\textsc{.m-def.m.sg.dir} \textsc{ind-}come\textsc{.prs.3sg:S=compl} \\ 
\glt `The man returned.' \hfill[JP.57] \label{ex.SV1}
\z 


\ea
\textit{xanme witêne.} \\
\gll xanme witê=ne \\
woman\textsc{.f.sg.dir} sleep.\textsc{pst.ptcp.f=cop.3sg.f:S} \\
\glt `The lady slept.' \hfill[SH.109] \label{ex.SV2}
\z 


\ea
\textit{pîyaç toqo.} \\
\gll pîya=îç toq-o \\
man=\textsc{add} be\_terrified.\textsc{prs.ind-3sg:S} \\
\glt `The man is terrified.' \hfill[JL.62] \label{ex.SV3}
\z 


\ea
\textit{îna lûlejenê ama.} \\
\gll în(e)=a lûlejen-ê ama  \\
\textsc{dem.prox.dir.m.3sg=ptcl} flute\_player-\textsc{indf} come.\textsc{pst.3sg:S} \\
\glt `Look, a flute player came [to the palace].' \hfill[ED.113] \label{ex.SV4}
\z 

The subject constituent may be postposed under certain conditions, e.g., when the reference to the subject constituent is evoked in the previous discourse. The post-verbal subject in (\ref{ex.VSb}) reaffirms the reference of the subject of \textit{ama} `came'.

\ea

\ea[]{
\textit{kabra kot gelşa.} \\ 
\gll \textbf{kabra} kot-{\O} gel=şa \\ 
 man fall\textsc{.pst-3sg.m:S}\textsc{.m} with\textsc{=3pl} \\ 
\glt `The fellow accompanied them.' \hfill[HB.59]
 }
\ex[]{ 
\textit{ama asawekew şeşkî, kabrake.} \\ 
\gll ama asaw-eke-û şeşk-î \textbf{kabra-(e)ke} \\ 
 come\textsc{.pst.3sg:S} mill\textsc{.m-def.m.sg.dir-ez.gen} \textsc{pn-m.sg.obl} man\textsc{-def.m.sg.dir} \\  
\glt `The fellow came to the Shashk mill.' \hfill[HB.60] \label{ex.VSb}
}
\z 
\z

It is notable that the verbs \textit{fermaway} and \textit{watey}, both meaning `to say', may allow post-verbal subjects, especially with highly topical discourse participants in a narrative. Note that the post-verbal subject in (\ref{ex.addpost}) is followed by the additive clitic \textit{=îç}, which indicates a topic shift (see \S\ref{sect:additiveclitic} for details).

\ea \label{ex.addpost}
\textit{fermawo ađîç, `weɫa min pey şime girewû.'} \\ 
\gll fermaw-o \textbf{ađ=îç} weɫa min pey şime girew-û \\ 
 say\textsc{.prs.ind-3sg:A} \textsc{3sg.m.dir=add} indeed \textsc{1sg} for \textsc{2pl} cry\textsc{.prs.ind-1sg:S} \\ 
\glt `\textbf{He} said, ‘I’m crying for you.’' \hfill[BP.155]
\z

\subsubsection{Direct object / Verb}
Direct objects are, by default, positioned immediately before the verb, regardless of whether they are nominal or pronominal. Note, however, that direct objects are rarely expressed by independent pronouns, as they can be alternatively expressed by clitic pronouns (see \S\ref{sect:cliticpers}). The immediate preverbal position is where the sentence stress usually falls. The following examples illustrate preverbal nominal and pronominal objects.

\ea
\textit{meʕmûrê kîyano.} \\ 
\gll meʕmûr-ê kîyan-o \\ 
 officer\textsc{.m-indf} send\textsc{.prs.ind-3sg:A} \\ 
\glt `He (the uncle) sent a servant [to spy on Pir Shaliyar].' \hfill[JP.42]
\z 


\ea
\textit{kuřekey memew bizekêş ward.} \\ 
\gll kuř-ekey meme-û bize-(e)kê=ş ward-{\O} \\ 
 boy\textsc{.m-def.m.sg.obl} breast\textsc{.dir}\textsc{.m-ez.gen} goat\textsc{.f-def.f.sg=3sg:A} eat\textsc{.pst-3sg.m:O} \\ 
\glt `The boy fed from the goat’s udder.' \hfill[KŞ.31]
\z 


\ea
\textit{ađîşa mewêno.} \\ 
\gll ađîşa me-wên-o \\ 
 \textsc{3pl.obl} \textsc{neg.ind-}see\textsc{.prs-3sg:A} \\ 
\glt `He did not see them.' \hfill[JP.190]
\z 

Post-verbal objects are rare in the corpus. If they occur at all, they are limited to nominals with definite references which have been evoked in the previous discourse. They tend to occur in interrogative (\ref{ex.VO1}) and imperative (\ref{ex.VO2}) clauses. Note additionally that in both examples, the verb is focused, and the object NP is given.

\ea
\textit{maça, `şanat tomeke?’} \\ 
\gll m-aç-a şana=t tom-eke \\ 
\textsc{ind-}say\textsc{.prs-3pl:A} scatter\textsc{.pst.3sg:O=2sg:A} seed\textsc{-def.m.sg.dir}\\ 
\glt `They would say, `Did you plant the seeds?’' \hfill[JP.39] \label{ex.VO1}
\z 


\ea
\textit{mekojdê a kabray!} \\
\gll me-koj-dê a kabra-î \\
\textsc{proh-}kill.\textsc{prs-2pl:A} \textsc{dem.dist} fellow\textsc{-m.sg.obl} \\
\glt `Do not kill that man!' \hfill[SH.268] \label{ex.VO2}
\z 

In the following example, \textit{zerey} `money' is topical. It has been tagged as an afterthought, and resumed by the clitic pronoun.

\ea
\textit{be zor miđoş milû sey misefayre zerey.} \\ 
\gll be zor mi-đ(e)-o=\textbf{ş} mil-û sey misefa-î=re \textbf{zer-e-î} \\ 
by force \textsc{ind-}give\textsc{.prs-3sg:A=3sg:O} on\textsc{-ez.gen} \textsc{pn} \textsc{pn-m.sg.obl=post} money\textsc{.m-def-m.sg.obl} \\ 
\glt `He (the man from Hajij) did not take no for an answer [lit. with force.] [and] gave Sey Mustafa the money.' \hfill[JP.107]
\z 
\subsubsection{Goal / Verb}
Following \citet{haig_postpredicate_2024}, goal arguments refer to endpoint arguments of verbs of movement such as `go' and `come'. ``Caused goals" refer to endpoint arguments of verbs of caused motion such as `put', `send', and `bring', e.g., `Put the book on the table'. Goals and caused goals are strongly associated with post-verbal placement in the neighbouring Kurdish\il{Kurdish} and Gorani\il{Gorani} varieties (\citealt{haig_post_2022,mohammadirad_zagros_nodate}). The following examples illustrate the post-verbal positioning of goals of verbs of movement (\ref{ex.goal1})--(\ref{ex.goal2}) and goals of caused motion verbs (\ref{ex.goal3})--(\ref{ex.goal4}).

\ea
\textit{luw{ɛ}nê hewar.} \\ 
\gll luw{ɛ}=nê hewar \\ 
 go\textsc{.pst.ptcp.pl=cop.3pl:S} summer\_habitat\textsc{.m} \\ 
\glt `They (people) would go to the summer habitat.' \hfill[JE.8] \label{ex.goal1}
\z 


\ea
\textit{êtir milo law ađî.} \\ 
\gll êtir mi-l-o la-û ađî \\ 
 \textsc{disc.ptcl} \textsc{ind-}go\textsc{.prs-3sg:S} to\textsc{-ez.gen} \textsc{3sg.obl}\textsc{.m} \\ 
\glt `Anyhow, he went to him.' \hfill[JP.13] \label{ex.goal2}
\z 


\ea
\textit{beroş yaneşa.} \\ 
\gll ber-o=ş yane=şa \\ 
 take\textsc{.prs.ind-3sg:A=3sg:O} house\textsc{.m=3pl:PSR} \\ 
\glt `He took him (Hayas) to their home.' \hfill[JH.51] \label{ex.goal3}
\z 


\ea
\textit{minyoş qiřoɫû beřû darêwe.} \\ 
\gll mi-ny(e)-o=ş qiřoɫ-û beřû dar-êwe \\ 
\textsc{ind-}put\textsc{.prs-3sg:A=3sg:O} hollow\textsc{.m-ez.gen} oak\textsc{.m} tree\textsc{.m-indf} \\ 
\glt `He put him into a hollow of an oak tree.' \hfill[ZB.38] \label{ex.goal4}
\z 

Table \ref{tab:frequency-goal} illustrates the linear placement of goal arguments of `come' and `go' relative to the verb in the main text corpus \citep[]{mohammadirad_speking_the_past}. 

\begin{table}
    \begin{tabular}{lrr}
\lsptoprule
\textbf{Goals} & \textbf{N} & \textbf{\%} \\
\midrule
post-pred &149&95\% \\
pre-pred& 8 & 5\%\\
\lspbottomrule
    \end{tabular}
    \caption{Frequencies of post-verbal and pre-verbal goals of `come' and `go'}
    \label{tab:frequency-goal}
\end{table}

As seen from Table \ref{tab:frequency-goal}, the overwhelming majority of goal arguments of `come' and `go' are post-verbal. This figure is in line with the placement of goals in neighbouring Iranian and non-Iranian languages in the region \citep[]{haig_postpredicate_2024}{}, pointing to longstanding contact between genetically diverse languages. 

The preverbal goals in the text corpus are generally associated with the notion of refined motion. In the text corpus, they are attested with most complements of \textit{ta} `until' (\ref{ex.postvgoal1}) and endpoint arguments which are realised as question particles (\ref{ex.postvgoal2}). The refined motion goals may also be flagged as suggested by (\ref{ex.postvgoal4}). An important observation is that bare goals do not appear in the pre-predicative position, unless they are a question particle (\ref{ex.postvgoal2}).

\ea
\textit{hewramîyê ta meřeber am{ɛ}.} \\ 
\gll hewramî-ê \textbf{ta} \textbf{meřeber} am{ɛ}  \\ 
 Hewramî\textsc{-pl.dir} until \textsc{pn} come\textsc{.pst.3pl:S}  \\ 
\glt `The people of Hewraman came as far as Marabar.' \hfill[BP.115] \label{ex.postvgoal1}
\z 

\newpage
\ea
\textit{îse koge bilî?} \\ 
\gll îse \textbf{koge} bi-l-î \\ 
 now where \textsc{sbjv-}go\textsc{.prs-2sg:S} \\ 
\glt `Where might you be going?' \hfill[JH.17] \label{ex.postvgoal2}
\z 


\ea
\textit{pey law to am{ɛ}nmê.} \\ 
\gll \textbf{pey} \textbf{la-û} \textbf{to} am{ɛ}=nmê \\ 
 to side\textsc{-ez.gen} \textsc{2sg} come\textsc{.pst.ptcp.pl}\textsc{=cop}\textsc{.\textsc{1pl:S}} \\ 
\glt `We have come to you.' \hfill[ŞC.32] \label{ex.postvgoal3} 
\z


\ea
\textit{zemînegere luwan.} \\ 
\gll \textbf{zemîn-eke=re} luwa=n \\ 
 land\textsc{.m-def.m.sg.dir}\textsc{=post} go\textsc{.pst.ptcp.m=cop.3sg.m:S} \\ 
\glt `He had disappeared into the ground.' \hfill[BP.197] \label{ex.postvgoal4} 
\z 
\subsubsection{Recipient / Verb}
Nominal and independent pronominal recipients\is{recipient} of verbs of transfer such as `give' are, by default, positioned in the post-predicate position, as illustrated in (\ref{ex.rec1})--(\ref{ex.rec2}). Our survey shows that all nominal recipient\is{recipient} arguments of the verb `give' in the text corpus, viz., 12 out of 12, have a post-verbal placement. This reflects that recipients\is{recipient} are treated like goals of verbs of movement in terms of their placement. Additionally, it implies that in ditransitive constructions, the default word order\is{word order} is AOR, where R is the non-flagged recipient\is{recipient} of the verb `give'.

\ea \label{ex.rec1}
\textit{nanîç miđa to.} \\ 
\gll nan=îç mi-đ(e)-a \textbf{to} \\ 
 bread\textsc{.m=add} \textsc{ind-}give\textsc{.prs-3pl:A} \textsc{2sg} \\  
\glt `They will give you a meal.' \hfill[HB.40]
\z 


\ea \label{ex.rec2}
\textit{kinaçekêt miđey pî kuřî?} \\ 
\gll kinaç(ê)-ekê=t mi-đe-î \textbf{p=î} \textbf{kuř-î} \\ 
daughter\textsc{.f-def.f.sg=2sg:PSR} \textsc{ind-}give\textsc{.prs-2sg:A} to=\textsc{dem.prox} boy\textsc{.m-sg.obl} \\ 
\glt `Will you give your daughter to this boy [in marriage]?' \hfill[JE.75]
\z  

\subsubsection{Addressee\is{addressee} / Verb}
 Addressees\is{addressee} are arguments of verbs of speech such as `say' and `speak'. Like recipients\is{recipient} and goals, the addressee\is{addressee} argument of `say' is strongly associated with post-predicate placement. Examples (\ref{addr1}) and (\ref{addr2}) illustrate the post-positional and pre-positional placement of addressee arguments. In terms of flagging, sixteen of the total 19 addressee\is{addressee} arguments, including the preverbal one, are flagged by prepositions. Table \ref{tab:frequency-addr} illustrates positional preferences of nominal addressees\is{addressee}.

\ea \label{addr1}
\textit{maço be xanî.} \\ 
\gll m-aç-o be xan-î \\ 
 \textsc{ind-}say\textsc{.prs}\textsc{-3sg:A} to chief\textsc{.m-sg.obl}  \\ 
\glt `[He] said to the chief.' \hfill[KŞ.97]
\z 


\ea \label{addr2}
\textit{be min wateniş.} \\ 
\gll be min wate=n=iş \\ 
 to \textsc{1sg} say\textsc{.pst.ptcp.m=cop.3sg.m:O}\textsc{=3sg:A} \\ 
\glt `[He] told me.' \hfill[ZP.32]
\z 
 
\begin{table}
    \begin{tabular}{lrr}
\lsptoprule
\textbf{Addressees} & \textbf{N} & \textbf{\%} \\
\midrule
post-pred &18&95\% \\
pre-pred& 1 & 5\%\\
\lspbottomrule
    \end{tabular}
    \caption{Frequencies of post-verbal and pre-verbal nominal addressees of `say' and `tell' in the text corpus}
    \label{tab:frequency-addr}
\end{table}


Bare nominal addressees\is{addressee} are limited to imperative clauses, as suggested by the following example.

\ea
\textit{mêmanekey wat, `waçe dêdêt.'} \\ 
\gll mêman-ekey wat w\'aç-e dêdê=t \\ 
 guest\textsc{.m-def.m.sg.obl} say\textsc{.pst} say\textsc{.prs.imp-}\textsc{2sg:A} older\_sister\textsc{.f=2sg:PSR} \\ 
\glt `The guest said, `Tell your older sister!’' \hfill[JH.47]
\z

\subsubsection{Placement of adverbials}
The placement of adverbials in the clause cannot be captured by syntactic generalisation. Rather, it seems that information structure affects their positioning. Generally, three positions are available for adverbials: clause-initial (\ref{ex.adv1}), clause-medial (\ref{ex.adv2}), and clause-final (\ref{ex.adv3}). The clause-initial position is usually associated with setting the background and marking a new section in the discourse. By contrast, the latter two positions do not offer a major discourse break from what precedes. This is exemplified by the position of the temporal adverbial \textit{şewe} `night' in the following examples.

\ea \label{ex.adv1}
\textit{şewê wêş wermê wîno.} \\ 
\gll \textbf{şew(e)-ê} wê=ş werm-ê wîn-o \\ 
 night\textsc{-f.sg.obl} \textsc{refl=3sg:PSR} sleep\textsc{-indf} see\textsc{.prs.ind-3sg:A} \\  
\glt `One night, he (the king) had a dream.' \hfill[JP.153]
\z 


\ea \label{ex.adv2}
\textit{to mişo î birayte şewêne bere.} \\ 
\gll to mişo î bira-î=t=e \textbf{şew(e)-ê=ne} b\'er-e \\ 
 \textsc{2sg} \textsc{aux} \textsc{dem.prox} brother\textsc{.m-sg.obl=2sg:PSR=dem} night\textsc{-f.sg.obl=post} take\textsc{.prs.imp-2sg:A} \\ 
\glt `You should take this brother of yours [there] at night.' \hfill[DG.26]
\z


\ea \label{ex.adv3}
\textit{esb zînî kero peyşû piřne kero şewê.} \\ 
\gll esb zînî ker-o pey=ş=û piřne ker-o \textbf{şew(e)-ê}\\ 
horse saddle do\textsc{.prs.ind-3sg:A} for\textsc{=3sg:R}=and jumping do\textsc{.prs.ind-3sg:A} night\textsc{-f.sg.obl}\\ 
\glt `He saddled the horse for him. At night, he set off [quickly].' \\\hfill[ŞC.54]--[ŞC.55]
\z  

\section{Copula clauses} \label{sect:copula_clasue}
Copula clauses consist of the enclitic copula (see \S\ref{sect:copula-morph}), the predicate, and the subject. The basic word order\is{word order} in copula clauses consists of the subject followed by the predicate and the copula verb. This is the unmarked order in ascriptive (\ref{ex.cop.ascriptive}) and equational copula clauses (\ref{ex.cop.equational}).

\ea
\textit{xwa to heqnî.} \\ 
\gll xwa to heq=nî \\ 
 God\textsc{.m} \textsc{2sg} right\textsc{.m=cop.2sg:S} \\ 
\glt `God, you are right.' \hfill[KŞ.104] \label{ex.cop.ascriptive}
\z 


\ea
\textit{min wêm padşana.} \\ 
\gll min wê=m padşa=na \\ 
 \textsc{1sg} \textsc{refl}\textsc{=1sg:PSR} king\textsc{.m}\textsc{=cop}\textsc{.1sg:S} \\ 
\glt `I myself am the king.' \hfill[JP.160] \label{ex.cop.equational}
\z 
The enclitic copula is generally fixed in position. However, sentential stress on the subject constituent can affect the mobility of the enclitic copula. The movement of the copula here can be explained by narrow focus\is{narrow focus} movement.

\ea
\textit{anen pîya!} \\ 
\gll \`ane=\textbf{n} pîya\,\suppipe{} \\ 
 \textsc{dem.dist.m.3sg.dir}\textsc{=cop.3sg.m:S} man\textsc{.m} \\ 
\glt `\textbf{He} is the man [not you]!' \hfill[ŞC.19]
\z 


\ea
\textit{minna kuřû mîrî.} \\
\gll m\`in=\textbf{na} ku\v{r}=û mîr-î\,\suppipe{} \\
\textsc{1sg=cop.1sg:S} son=\textsc{ez.gen} prince-\textsc{obl.m} \\
\glt `\textbf{I} am the prince's son.' \hfill\citep[377]{khan_language_2023}
\z

In copular clauses where the predicate is a prepositional phrase, the copula may attach to the preposition rather than to the right edge of the phrase, depending on whether nuclear stress falls on the preposition or its complement. This occurs only when the prepositional phrase is headed by the preposition pêse ‘like’. With \textit{pêse} in focus\is{focus}, the copula moves on it, as seen in (\ref{like1}). If the nominal complement is focused, the copula lands on it (\ref{like2}).\footnote{It should be noted that the construction in (\ref{like1}) with the copula realised on the preposition occurs by default, which could point to the emergence of an innovative copula predicate based on the preposition \textit{pêse} `like'.} 


\ea \label{like1}
\textit{řisq ... pêsen miɫey.} \\
\gll řisq pês\`e=n miɫe-î\,\suppipe{} \\
rat like=\textsc{cop.3sg.m:S} mouse-\textsc{m.sg.obl} \\
\glt `A rat is like a mouse.' \hfill[PK.36]
\z 


\ea \label{like2}
\textit{pêse miɫeyn.} \\
\gll pêse miɫ\`e-î=n\,\suppipe{} \\
like mouse-\textsc{m.sg.obl=cop.3sg.m:S} \\
\glt `It's like a mouse.' \hfill[PK.35]
\z 

The subject constituent may be realised following the predicate as a right-dislocated topic, resulting in the order \textit{predicate subject-copula}. The focal post-predicate subject in (\ref{ex.predsbj-cop1}) has a reference that has been evoked in previous discourse.

\ea
\textit{ađîç wat, `a şêre minna.'} \\ 
\gll ađî=(î)ç wat a şêr=e min=na \\ 
 \textsc{3sg.obl.m=add} say\textsc{.pst} \textsc{dem.dist} lion\textsc{.m=dem} \textsc{1sg=cop.1sg:S} \\ 
\glt `He (the sultan) said, ‘\textbf{I} am that lion.’' \hfill[JH.116] \label{ex.predsbj-cop1}
\z 

The regular syntax of the copula clause applies as well when the predicate phrase is an interrogative pronoun (see \S\ref{secr:interogative_pronoun}). Thus, the copula follows the predicate, which in this case is an interrogative pronoun (\ref{ex.interr1})--(\ref{interr2}) and carries the nuclear stress.

\ea \label{ex.interr1}
\textit{ane çêşa? çêşa ane?} \\ 
\gll ane çêş=a çêş=a ane \\ 
\textsc{prsv} what\textsc{=cop.3sg.m:S} what\textsc{=cop.3sg.m:S} \textsc{prsv}  \\ 
\glt `What is going on? What is going on?' \hfill[ZP.113]
\z


\ea \label{interr2}
\textit{fermawo, `pîyew kêndê?'} \\ 
\gll fermaw-o pîye-û kê=ndê \\ 
say\textsc{.prs.ind-3sg} man\textsc{.m-ez.gen} who\textsc{=cop}.\textsc{2pl} \\ 
\glt `He said, ‘Whose men are you?’' \hfill[ŞC.33]
\z 

Example (\ref{ex.predsbj-cop2}) features a postposed subject.

\ea
\textit{konê î meʕmûrê mine çinnê wextên?}\\ 
\gll ko=nê \textbf{î} \textbf{meʕmûr-ê} min=e çinnê wext-ê=n \\ 
where\textsc{=cop.3pl:S} \textsc{dem.prox} officer\textsc{.m-pl.dir} \textsc{1sg=dem} some\textsc{.pl} time\textsc{.m-pl.dir=cop.3sg.m:S} \\
\glt `Where have my officers been during this time? [Lit. Where are my officers? It is some time.]' \hfill[BP.57] \label{ex.predsbj-cop2}
\z 

\subsection{Locational copula clauses}\label{sect:loc-cop}
Locational copula clauses are formed by the locational deictic particle \textit{îna}, which acts as the predicate. It is inflected identically to past intransitive verbs. The locational complement of \textit{îna} is, by default, realised post-predicatively.\footnote{The post-verbal placement of the locational complement in locational copula clauses has affected neighbouring Central Kurdish\il{Kurdish!Central} dialects as a substrate feature \citep[]{mohammadirad_zagros_nodate}{}.}

\ea
\textit{înanê çêge.} \\ 
\gll îna-(a)nê çêge \\ 
 \textsc{loc.deic.cop-1sg:S} here \\ 
\glt `I am here.' \hfill[JP.136]
\z 

\newpage
\ea
\textit{înaymê fiɫane yagê.} \\ 
\gll îna-îmê fiɫan-e yagê \\ 
 \textsc{loc.deic.cop-1pl:S} such\_and\_such\textsc{-ez.cmpd} place\textsc{.f} \\ 
\glt `We are at such-and-such a place.' \hfill[PM.12]
\z 

In the following example, the locational copula clause has split the subject \textit{neferê} from its prepositional phrase complement \textit{be namêw fiɫane kesî}.

\ea
\textit{neferê înarê ce menteqew hewramanatî be namêw fiɫane kesî.} \\ 
\gll nefer-ê îna-∅=rê ce menteqe-û hewraman-at-î be namê-û fiɫan-e kes-î \\ 
 person\textsc{.m-indf} \textsc{loc.deic.cop-3sg.m:S=povb} in region\textsc{-ez.gen} \textsc{pn-pl-m.sg.obl} by name\textsc{.f-ez.gen} such\_and\_such\textsc{-ez.cmpd} person\textsc{.m-sg.obl} \\ 
\glt `There is a person in the Hawram region called such-and-such.' \hfill[ZP.31]
\z 

\subsection{Predicate complements of `become'}
The verb `become’ has an identical morphology to `be', both being expressed by the stem \textit{b-}. Semantically, `become' is a change-of-state verb, while `be' expresses an equative copula. In terms of syntax, `become' differs from `be' in that the predicate complement of ‘become’ is generally realised post-predicatively if it is an NP. The nominal complements of `become' could be either a bare NP (\ref{ex.become1})--(\ref{ex.become2}) or be flagged by the preposition \textit{be} (\ref{ex.become3}). The preposition \textit{be} is glossed as \textsc{adp} when it heads the complement of `become'.\footnote{The post-verbal placement of the predicate complements of `become' has been recognised as one of the features shared among all varieties of Kurdish\il{Kurdish} (\citealt{haig_post_2022,mohammadirad_zagros_nodate}).}

\ea \label{ex.become1}
\textit{ađ bowe padşa.} \\ 
\gll ađ b-o=we padşa\\ 
 \textsc{3sg.m.dir} become\textsc{.prs.ind-3sg:S=compl} king\textsc{.m} \\ 
\glt `He (i.e., Jamsher Shah) became the king.' \hfill[DP.53]
\z


\ea \label{ex.become2}
\textit{tate bîyen çûwêw desû şuwaney.} \\ 
\gll tate bîye=n çû-ê-û des-û şuwane-î \\ 
 father\textsc{.m} become\textsc{.pst.ptcp.m=cop.3sg.m:S} stick\textsc{.m-indf-ez.gen} hand\textsc{.m-ez.gen} shepherd\textsc{.m-sg.obl} \\ 
\glt `The father has become like the stick in the shepherd’s hand.' \hfill[RE.41]
\z 


\ea \label{ex.become3}
\textit{bo be taqetê xeɫkî.} \\ 
\gll b-o be taqet-ê xeɫk-î \\ 
 become\textsc{.prs.ind-3sg:S} \textsc{adp} support\textsc{.m-indf} people\textsc{.m-sg.obl} \\  
\glt `He became an entertainer for people.' \hfill[KŞ.37]
\z 

The nominal complement of `become' is overwhelmingly realised in the post-predicate position. The following examples are some of the rare cases of the preverbal placement of the complements of `become'.

\ea
\textit{seʕbe řoj bowe.} \\ 
\gll seʕbe řoj b-o=we \\ 
 morning\textsc{.f} daylight\textsc{.m} become\textsc{.prs.ind-3sg:S=compl} \\ 
\glt `Daylight came. (Lit. Early morning became daylight.)' \hfill[ZB.26]
\z 


\ea
\textit{p-a kes-î b-û}\\
\gll \textbf{p-a} \textbf{kes-î} b-û  \\
\textsc{adp-dem.dist} person-\textsc{m.sg.obl} become.\textsc{prs-1sg:S}\\
\glt `May I become that person.' \hfill[MF.218] 
\z 

Adjective complements of `become', on the other hand, are generally preverbal (\ref{ex.become4})--(\ref{ex.become5}). The reverse positioning of adjectival complements follows from the fact that the combination `adjective + become' is treated like a light verb construction \citep[]{mohammadirad_zagros_nodate}{}.

\ea \label{ex.become4}
\textit{adiz bo mizyore.} \\ 
\gll adiz b-o mi-zy(e)-o=re \\ 
 upset become\textsc{.prs.ind-3sg:S} \textsc{ind-}go\_out\textsc{.prs-3sg:S=povb} \\ 
\glt `He (Hayas) became upset and went out (of the palace).' \hfill[JH.4]
\z 


\ea \label{ex.become5}
\textit{mêşhûr bîyen.} \\ 
\gll mêşhûr bîye=n \\ 
 famous become\textsc{.pst.ptcp.m=cop.3sg.m:S} \\ 
\glt `He had just become famous.' \hfill[ŞC.5]
\z 

In some rare cases, the adjective complement of `become' may appear in the post-predicate position. In addition, note that the overall frequency of post-predicate placement of adjectival complements of `become' is also very low compared to nominal complements in neighbouring languages \citep[]{mohammadirad_zagros_nodate}.

\ea
\textit{bîyê sîyawû çermew sûr.} \\ 
\gll bî-ê sîyaw=û çerme=û sûr \\ 
 become\textsc{.pst-3pl:S} black=and white=and red \\ 
\glt `They have become [one by one] black and white and red.' \hfill[JH.91]
\z 

\section{Clauses with existential particles} \label{sect:existentialclauses}

\subsection{Existential clauses\is{existential clause}}
The existential clauses in the present tense are based on the existential particle \textit{hen}, which is a frozen form consisting of the particle \textit{he-} and the \textsc{3sg.m} form of the copula \textit{=n}; see (\ref{ex.existclause1})--(\ref{ex.existclause2}). Like the locational copula constructions (\S\ref{sect:loc-cop}), the existential particle is inflected identically to past intransitive verbs (\S\ref{sect:existential-copula}). This analysis comes from the fact that the base \textit{hen} can also be used nominally (see \S\ref{sect:existential-copula}). 

\ea
\textit{mîyo meřêwe hene.} \\
\gll mi-đy(e)-o meř(e)-êwe hen-e \\
\textsc{ind}-notice.\textsc{prs-3sg:S} cave.\textsc{f-indf} \textsc{exist-3sg.f:S} \\
\glt `He noticed that there was a cave.' \hfill[MF.266] \label{ex.existclause1}
\z 


\ea
\textit{xaneqane ne nan hen, ne hardî henê.} \\ 
\gll xaneqa=ne ne nan hen-∅ ne hardî hen-ê \\ 
monastery\textsc{.m=post} neither bread\textsc{.m} \textsc{exist-3sg.m:S} nor flour\textsc{.f} \textsc{exist-3pl:S} \\ 
\glt `There is neither bread nor flour in the monastery.' \hfill[HB.2] \label{ex.existclause2}
\z 

In the past tense, the past form of the copula alone expresses existence; see (\ref{ex.existclause3}).

\ea
\textit{her sikeɫ bîyen.} \\ 
\gll her sikeɫ bîye=n\\ 
 just ember\textsc{.m.sg.dir} be\textsc{.pst.ptcp.m=cop.3sg.m:S} \\ 
\glt `There were only embers [to light the house].' \hfill[JE.40] \label{ex.existclause3}
\z 

When the subject of an existential clause\is{existential clause} is modified by a relative clause (\ref{ex.existclause4})--(\ref{ex.existclause5}), the relative clause is extraposed (see \S\ref{sect:rel-extrapos} for details).

\ea
\textit{yeknefer hen hermaneş kerû pey.} \\ 
\gll yek nefer hen-∅ hermane=ş ker-û pey \\ 
 one person \textsc{exist-3sg.m:S} work\textsc{.f=3sg:R} do.\textsc{prs.ind-1sg:A} for\\ 
\glt `There is a person for whom I work.' \hfill[ŞE.35] \label{ex.existclause4}
\z 


\ea
\textit{pîyayê hen namêş şê ʕeladînarê.} \\ 
\gll pîya-ê hen-∅ namê=ş şê ʕeladîn=a=rê\\ 
 man\textsc{.m-indf} \textsc{exist-3sg.m:S} name\textsc{.f=3sg:PSR} sheikh\textsc{.m} \textsc{pn=cop.3sg.m:S=povb} \\ 
\glt `There is a man called Sheikh Aladin.' \hfill[DG.24] \label{ex.existclause5}
\z 

\subsection{Predicative possessive constructions\is{predicative possessive constructions}}
The existential particle \textit{hen} is also used as the predicate in the expression of syntactic possession. It agrees in gender\is{gender agreement} and number\is{number agreement} with the possessed argument. The basic order in predicative possessive constructions\is{predicative possessive constructions} is Possessor-Possessed-Existential particle. As explained in \S\ref{sect:clitic-function}, the possessor in a predicative possessive construction is generally indexed by a clitic pronoun, as in (\ref{ex.poss-clc}), though not in (\ref{ex.pred-poss}).

\ea \label{ex.poss-clc}
\textit{min birayêwem hen.} \\
\gll min bira-êwe=m hen-∅ \\
 \textsc{1sg} brother\textsc{.m}\textsc{-indf}\textsc{=1sg:NC} \textsc{exist}\textsc{-3sg:S}\textsc{.m} \\
\glt `I have a brother.' \hfill[DG.34] \\
\z 


\ea
\textit{to henê çil kinaçê heqûdađê.} \\
\gll to hen-ê çil kinaçê heq=û dađ-ê \\
\textsc{2sg} \textsc{exist-3pl:S} forty daughter\textsc{.pl} right=and legitimate-\textsc{pl} \\
\glt `You have forty legitimate daughters.' \hfill[ME.162] \label{ex.pred-poss}
\z 

In rare cases, the possessor phrase is right-dislocated. This happens when the possessor has been evoked in previous discourse. 

\ea
\textit{êjdêhawêçiş hen padşaw semerqenî.} \\
\gll êjdêha-(e)wê=ç=iş hen-∅ padşa-û semerqen-î \\
dragon\textsc{.m-ind=add=3sg:NC} \textsc{exist-3sg.m:S} king-\textsc{ez.gen} \textsc{pn-m.sg.obl} \\
\glt `The king of Samarkand had a dragon too.' \hfill[ME.128]
\z 

The construction seen above entails a permanent relation of possession between the possessor and the possessed. When the possession is temporary, a locational copula construction consisting of the particle \textit{îna} expresses the relation of possession (\ref{ex.temp1}). Note that the particle \textit{îna} is not obligatory in this construction.

\ea \label{ex.temp1}
\textit{meɫayêwa. kitêbêwe gewreş îna pene.} \\ 
\gll meɫa-êw=a kitêb-êwe gewre=ş{\footnotemark} îna-∅ pene \\ 
 mullah\textsc{.m-indf=cop.3sg.m:S} book\textsc{.m-indf} big\textsc{.m=3sg:NC} \textsc{loc.deic.cop-3sg.m:S} with \\ 
\glt `It is a Mullah. He has a big book with him.' \hfill[KŞ.8]
\z\footnotetext{The clitic pronoun indexing `he' is obligatory, justifying the glossing as \textsc{nc}.}

Alternatively, temporary syntactic possession may be expressed in a copula clause containing a prepositional phrase headed by \textit{pene}.

\ea
\textit{milo serû hanêwe. esɫeheş pene bo.} \\ 
\gll mi-l-o ser-û han(e)-êwe esɫehe=ş pene b-o \\ 
 \textsc{ind-}go\textsc{.prs-3sg:S} on\textsc{-ez.gen} water\_spring\textsc{.m-indf} gun\textsc{.m=3sg:NC} with be\textsc{.prs.ind-3sg:S} \\ 
\glt `He goes to the edge of a spring. He has a gun with him.' \hfill[KŞ.4]
\z 


\ea \label{ex.temp2}
\textit{zeřeş penen.} \\ 
\gll zeř-e=ş pene=n \\ 
 money\textsc{.m-def=3sg:R} with\textsc{=cop.3sg.m:S} \\ 
\glt `He has money with him.' \hfill[JP.112]
\z 

\section{Light verb constructions\is{light verb constructions}} \label{sect:lvc-syntax}
As remarked in \S\ref{sect:alignment}, the transitivity of the clause is determined by the lexical transitivity of the verb and not the semantic transitivity. This is also true for light verb constructions (LVCs). Thus, \textit{fewt kerdey} `to pass away' [lit. death to do], \textit{kûç kerdey} `migrate' [migration to do], \textit{qesem wardey} `swear' [lit. oath to eat] are semantically intransitive as a whole, but syntactically considered transitive, given that the verbs `do' and `eat' are lexically transitive. Therefore, the transitivity of the clause is not determined semantically, but based on the lexical category of the verb, or in this case, the light verb in the LVC. In the following examples, three criteria show that the mentioned light verbs are syntactically transitive. (i) The subject NP appears in the oblique case; (ii) the clitic pronoun agrees with the subject; (iii) the light verb complement is considered the direct object, as it triggers agreement on the light verb (see also below). In this sense, semantically intransitive LVCs like the ones in (\ref{cpr1})--(\ref{cpr2}) are syntactically treated the same as typical simple transitive verbs (\ref{cpr3}).

\ea
\textit{patşay qesemiş wardebê.}\\
\gll patşa-î qesem=iş warde=b-ê \\
king-\textsc{m.sg.obl} oath\textsc{.m=3sg:A} eat.\textsc{pst.ptcp.m=}be.\textsc{prs-aug.3sg:O} \\
\glt `The king had sworn.' \hfill[DB.307] \label{cpr1}
\z


\ea
\textit{fewtiş kerd luwa.} \\ 
\gll fewt=iş kerd-∅ luwa \\ 
 death\textsc{.m=3sg:A} do\textsc{.pst-3sg.m:O} go\textsc{.pst.3sg:S} \\ 
\glt `He passed away. [Lit. He passed away and went.]' \\\hfill [ZQ.30] \label{cpr2}
\z 


\ea
\textit{î herî zûwaniş zana.} \\ 
\gll î her-î zûwan=iş zana-{\O} \\ 
 \textsc{dem.prox} donkey\textsc{-m.sg.obl} language\textsc{.m=3sg:A} know\textsc{.pst-3sg.m:O} \\ 
\glt `The donkey knew the [Sheikh’s] language.' \label{cpr3} \hfill[HB.71]
\z 
Light verb constructions\is{light verb constructions} are associated with special syntax in Hewramî\il{Hewramî}. The following discussion is based on \citet{Mohammadirad2023Frankfurt}. The non-verbal element within an LVC shows agreement in gender\is{gender agreement} and number\is{number agreement} with a preceding clausal argument. This agreement relationship is conditioned by the lexical class of the non-verbal element and the syntactic category of the argument controlling the agreement. As for the lexical class of the non-verbal element, the agreement is most productive with the adjective complement of a light verb\is{light verb}. The arguments triggering this agreement are S and P.\footnote{Note that predicative adjectives also carry gender and number agreement, but they do so only when the controller is S (see \S\ref{sect:predicative-adj}). On the other hand, in LVCs where the complement is an adjective, the latter shows agreement in gender and number with both S and P. Therefore, the agreement pattern in LVCs cannot be considered the same as the one with predicative adjectives.} The light verb construction in (\ref{ex.lvc1}) is \textit{neweş kewtey} `get ill', and the adjective \textit{neweş} agrees in gender\is{gender agreement} with the S. In (\ref{ex.lvc2}), the light verb construction is \textit{zamdar kerđey} `injure', and it can be seen that the adjective \textit{zamdar} agrees with the direct object \textit{ême}. Example (\ref{ex.lvc2a}) further shows that the agreement is not contingent on the properties of the S and O in terms of animacy.

\ea
\textit{kinaçêw padşaw misrî neweşe gino.} \\ 
\gll \textbf{kinaçê-û} \textbf{padşa-û} \textbf{misr-î} neweş-\textbf{e} gin-o \\ 
daughter\textsc{.f.dir-ez.gen} king\textsc{.m-ez.gen} \textsc{pn-m.sg.obl} ill\textsc{-f} fall\textsc{.prs.ind-3sg:S} \\ 
\glt `The king of Egypt’s daughter fell sick.' \hfill[ZP.25] \label{ex.lvc1}
\z 


\ea
\textit{ême zamdarê nekero.} \\ 
\gll \textbf{ême} zamdar-\textbf{ê} ne-ker-o \\ 
 \textsc{1pl} wounded\textsc{-pl} \textsc{neg.sbjv-}do\textsc{.prs-3sg:A} \\ 
\glt `He should not injure us' \hfill[DG.64] \label{ex.lvc2}
\z 


\ea
\textit{î pîfêşe weşe kerdêne.} \\ 
\gll î \textbf{pîfê}=ş=e weş-\textbf{e} kerdê=ne \\ 
\textsc{dem.prox} burr\textsc{.f=3sg:A}\textsc{=dem} well\textsc{-f} do\textsc{.pst.ptcp.f=cop.3sg.f:O} \\ 
\glt `He had made this burr.' \hfill[ZQ.42] \label{ex.lvc2a}
\z 

The agreement feature in light verb constructions can express information unspecified for certain controllers. In (\ref{ex.lvc3}), the controllers of agreement are the SAP pronouns, which are not inflected for gender\is{gender}. The agreement on the light verb constructions\is{light verb constructions} \textit{şîrîn kerđey} `sweeten' and \textit{sîyaw kerđey} `blacken' explicates the gender\is{gender} of the referents of the SAP pronouns. 

\ea
\textit{`ça toşa şîrîne kerđî ça çageyç minşa sîyaw kerđa.'} \\
\gll ça \textbf{to}=şa şîrîn-\textbf{e} kerđ-î ça çage=yç \textbf{min}=şa sîyaw-\textbf{\O} kerđ-a \\
there \textsc{2sg=3pl:A} sweet-\textsc{f} do.\textsc{pst-2sg:O} there there\textsc{=add} \textsc{1sg=3pl:A} black\textsc{-m} do.\textsc{pst-1sg:O} \\ 
\glt `[The husband said to his wife], `There where they sweetened you, they also blackened me.’' \hfill[XX.87] \label{ex.lvc3}
\z 

The agreement features extend by analogy to nominal complements of the light verb\is{light verb} that are similar in ending to adjectives. In (\ref{ex.lvc4}), the light verb construction\is{light verb constructions} is \textit{nigebanî kerđey} `to protect, guard'. The nominal element \textit{nigebanî} `guardianship' contains the derivational ending \textit{-î} added to the noun \textit{nigeban} `guard' (see \S\ref{nominal-affixation}). The gender agreement\is{gender agreement} on \textit{nigebanî} follows from the fact that the same derivational suffix forms adjectives from nouns (see \S\ref{sect:adj-derivation}). This has resulted in the extension of gender agreement\is{gender agreement} to \textit{nigebanî} by analogy with derived adjectives ending in \textit{-î}.

\newpage
\ea
\textit{min metawû îne\footnote{Note that the masculine\is{masculine} form of the demonstrative pronoun has been used to express a feminine\is{feminine} direct object. This follows from the decay in the inventory of demonstrative pronouns and the extension of the masculine\is{masculine} forms to the feminine\is{feminine} ones (see \S\ref{sect: dempro}).} nigebanîye kerû.} \\ 
\gll min me-taw-û îne nigebanî-e k\'er-û \\ 
 \textsc{1sg} \textsc{neg.ind-}can\textsc{.prs-1sg:A} \textsc{dem.prox.m.3sg.dir} guardian\textsc{-f} do\textsc{.prs.sbjv-1sg:A} \\ 
\glt `I cannot take care of her.' \hfill[ZP.103] \label{ex.lvc4}
\z 

Likewise, non-verbal elements with floating class membership between nouns and adjectives tend to show agreement with S and O. In the following examples, \textit{řed} `crossing' in \textit{řed bîyey} `pass by' and \textit{cemʕ} `addition' in \textit{cemʕ kerđey} `to collect' cannot be readily classified as prototypical nouns. The fact that they agree with S and O arguments may follow from their class membership floating between nouns and adjectives.

\ea
\textit{silênne řeđê ba.} \\ 
\gll silên=ne řeđ-ê b-a \\ 
 \textsc{pn=post} crossing\textsc{-pl.dir} be\textsc{.prs.ind-3pl:S} \\ 
\glt `They passed Silên.' \hfill[JP.167]
\z 


\ea
\textit{suɫtan selîmî cemʕê kerđê.} \\
\gll suɫtan selîm-î cemʕ-ê kerđ-ê \\
sultan \textsc{pn-m.sg.obl} addition-\textsc{pl} do.\textsc{pst-3pl:O} \\
\glt `Sultan Salim gathered them.'
\z 

On the other hand, number\is{number agreement} and gender agreement\is{gender agreement} does not occur when the non-verbal element in the LVC is a noun. In such cases, the non-verbal part appears in its bare\is{bare} form. In (\ref{ex.lvc5}), the LVC is \textit{mare kerdey} `to marry', with the masculine\is{masculine} noun \textit{mare} `marriage' not agreeing with the feminine\is{feminine} direct object. Similarly, in (\ref{ex.lvc6}), the nominal element \textit{încam} `accomplishment' is in its bare\is{bare} form and does not agree with the feminine\is{feminine} O. In (\ref{ex.lvc6c}), the noun \textit{wiş} `memory' does not agree in plural number\is{number agreement} with the plural O.

\ea
\textit{fîlfor î kinaçême peyş mare kerdê.} \\ 
\gll fîlfor î \textbf{kinaçê}=m=e pey=ş \textbf{mare} k\'er-dê \\ 
immediately \textsc{dem.prox} daughter\textsc{.f.sg=1sg:PSR=dem} to\textsc{=3sg:R} marriage\textsc{.m} do\textsc{.prs.imp-2pl:A} \\ 
\glt `Immediately marry my daughter to him.' \hfill[KŞ.80] \label{ex.lvc5}
\z 


\ea
\textit{çaştekê încam miđa.} \\ 
\gll \textbf{çaşt(î)-ekê} \textbf{încam} mi-đ(e)-a \\ 
 meal\textsc{.f-def.f.sg} accomplishment\textsc{.m} \textsc{ind-}give\textsc{.prs-3pl:A} \\ 
\glt `They made the meal.' \hfill[JP.254] \label{ex.lvc6}
\z 


\ea \label{ex.lvcn}
\textit{ta qewmekama wiş kermê ...} \\ 
\gll ta \textbf{qewm-eka}=ma \textbf{wiş} k\'er-mê \\ 
 until relative\textsc{.m-def.pl.obl=1pl:PSR} memory\textsc{.m} do\textsc{.prs.sbjv-1pl:A}\\ 
\glt `Until we inform our relatives ...' \hfill[JE.83] \label{ex.lvc6c}
\z 

The LVCs\is{light verb constructions} in (\ref{ex.lvc5})--(\ref{ex.lvcn}) allow a direct object in their argument structure. By contrast, in LVCs\is{light verb constructions} which don't allow a direct object in their argument structure, the nominal element within the LVC features some of the properties of direct objects. For example, it triggers agreement on the light verb\is{light verb} in TAMs built on past stem verbs in the same way nominal direct objects do. In other words, with LVCs\is{light verb constructions} which do not allow additional direct objects in their argument structure, the nominal complement of the LVC acts as a direct object, at least with regard to the control of agreement on the verb. In (\ref{ex.lvc7}), the LVC is \textit{koç kerđey} `to migrate' (lit. `migration to do'). It can be seen that \textit{koç} `migration' triggers agreement on the light verb\is{light verb}. In (\ref{ex.lvc8}), the LVC is \textit{ʕemrew xway kerdey} `to pass away' (lit. `to do the command of God'), containing the feminine\is{feminine} noun \textit{ʕemre} `command', which triggers the gender agreement\is{gender agreement} on the light verb\is{light verb}. In (\ref{ex.lvc9}), the LVC is \textit{duʕa kerdey} `to pray' containing the feminine\is{feminine} noun \textit{duʕa} `pray'.

\ea
\textit{koçşa kerden.} \\ 
\gll \textbf{koç}=şa kerde=\textbf{n} \\ 
 migration\textsc{.dir}\textsc{.m=3pl:A} do\textsc{.pst.ptcp.m=cop.3sg.m:O} \\ 
\glt `They migrated.' \hfill[JE.13] \label{ex.lvc7}
\z 


\ea
\textit{kabra merđû ʕemrew xwayş kerde.} \\ 
\gll kabra merđ-∅=û \textbf{ʕemre}-û xwa-î=ş kerd-\textbf{e} \\ 
fellow die.\textsc{pst-3sg.m:S=}and order\textsc{.f-ez.gen} God\textsc{.m-sg.obl=3sg:A} do\textsc{.pst}\textsc{-3sg}\textsc{.f:O} \\ 
\glt `The fellow died. He passed away. [Lit. He did the command of God.]' \\ \hfill[KT.16] \label{ex.lvc8}
\z 


\newpage
\ea
\textit{duʕaş kerde werû berew zatû heqî.} \\
\gll \textbf{duʕa}=ş kerd-\textbf{e} wer-û bere-û zat-û heq-î \\
prayer\textsc{.f=3sg:A} do.\textsc{pst-3sg.f:O} front-\textsc{ez.gen} door.\textsc{ez.g} essence\textsc{-ez.gen} God-\textsc{obl.m} \\
\glt `He prayed to God.' \hfill[SH.259] \label{ex.lvc9} 
\z 

\section{Reciprocal constructions}\label{sect:reciprocal}
Reciprocal constructions are based on the reciprocal pronoun \textit{yotirîn} `each other', which is grammatically derived from the numeral `one' and the superlative\is{superlative} suffix \textit{-tirîn} (see \S\ref{sect:reciprocalpronoun}). The reciprocal pronoun can, among other things, function as a direct object (\ref{ex.recp1}), a non-canonical subject\is{non-canonical subjects} (\ref{ex.recp2}) and a prepositional object (\ref{ex.recp3}). 

\ea
\textit{yotirînşa dî.} \\
\gll yotirîn=şa dî-∅ \\
\textsc{recp=3pl:A} see.\textsc{pst-3sg:O} \\
\glt `They saw each other.' \label{ex.recp1}
\z 


\ea
\textit{yotirînma gerek bîyen.} \\
\gll yotirîn=ma gerek bîye=n \\
\textsc{recp=1pl:NC} necessary be.\textsc{pst.ptcp.m=cop.3sg.m:S} \\
\glt `We liked [lit. wanted] each other.'  \hfill[JL.19] \label{ex.recp2}
\z 


\ea
\textit{ađê êtir şađê bîyê be yotirînî.} \\
\gll ađê êtir şađ-ê bî-ê be yotirîn-î \\
\textsc{3pl} well happy-\textsc{pl} be.\textsc{pst-3pl:S} with \textsc{recp-m.sg.obl} \\
\glt `They became happy with each other.' \hfill[ME.202] \label{ex.recp3}
\z

\section{Periphrastic causative constructions\is{periphrastic causative constructions}}\label{sect:Periphrastic-causative}
The causative\is{causative} is morphologically derived mainly from suffixing \textit{-n} (\textsc{prs}) and \textit{-na} (\textsc{pst}) to the present stem of the verb (\S\ref{sect:causative}).Additionally, Hewramî\il{Hewramî} employs a syntactic causative construction\is{periphrastic causative constructions}, where the causee is expressed by a prepositional phrase headed by the absolute preposition \textit{pene}. This implies that the causee occurs as a bound argument.

\newpage
\ea
\textit{ʕaɫifşa pene pêta.}\\
\gll ʕaɫif=şa \textbf{pene} pêt-\textbf{a} \\
fodder\textsc{.m.sg.dir=3sg:A} by gather.\textsc{pst-1sg:R} \\
\glt `They made me gather the fodder.' \hfill\citep[560]{khan_language_2023}
\z 


\ea
\textit{dřêma pene pêçênê.} \\
\gll dřê=\textbf{ma} \textbf{pene} pêç-ên-ê \\
prickle=\textsc{1pl:R} by twist\textsc{.prs-aug-3pl:A} \\
\glt `They would make us twist the pile of prickles.' \\ \hfill\citep[560]{khan_language_2023}
\z

The expression of causative\is{causative} may alternatively be conveyed by a periphrastic construction\is{periphrastic causative constructions}, where the causer's action and the causee's action are expressed in different clauses.

\ea
\textit{karêş pene kerdena be mehremû wêm qubuɫim kerden.} \\ 
\gll kar-ê=ş pene kerde=na be mehrem-û wê=m qubuɫ=im kerde=n \\ 
 task\textsc{.m-indf=3sg:A} to do\textsc{.pst.ptcp.m=cop.1sg:R} as kindred\textsc{-ez.gen} \textsc{refl=1sg:PSR} accepted\textsc{=1sg:A} do\textsc{.pst.ptcp.m=cop.3sg.m:O} \\ 
\glt `She made me accept her as my own kindred.' \hfill[JH.117]
\z 

\section{Passive\is{passive clause} clauses}\label{sect:passive}
The passive clause\is{passive clause} features only one preverbal argument. The agent is virtually never expressed.

\ea
\textit{a kinaçê to pey kuřû şuwaney niwîsyene.} \\ 
\gll a kinaçê to pey kuř-û şuwane-î niwîsye=ne \\ 
 \textsc{dem.dist} daughter\textsc{.dir}\textsc{.f} \textsc{2sg} to son\textsc{.m-ez.gen} shepherd\textsc{.m-sg.obl} write\textsc{.pst.pass=cop.3sg.f:S} \\ 
\glt `Your daughter was promised [lit. written] to the shepherd’s son.' \hfill[KŞ.99]
\z 

The passive agent\is{passive agent} may be expressed by a by-phrase at the end of the clause.

\ea
\textit{sewzê xel kiryan be sahêbîçiş.} \\
\gll sewzê xel kir-ya=n \textbf{be} \textbf{sahêb-î=ç=iş} \\
crop grain do.\textsc{prs-pass=cop.3sg.m:S} by owner-\textsc{obl.m=add=3sg:PSR} \\
\glt `The crop has been piled up [lit. made into corn] by its owner too.' \hfill[HR.16]
\z 

The passive\is{passive} may be expressed through an active impersonal construction. Here, the lexical subject is unexpressed, and the verb has \textsc{3pl} inflection. This construction is idiomatically rendered into English\il{English} by a passive\is{passive}.

\ea
\textit{beraşa dûrû mentêqew ewêgewe.} \\ 
\gll ber-a=şa dûr-û mentêqe-û ewêge=we \\ 
 take\textsc{.prs.ind-3pl:A=3pl:O} far\textsc{-ez.gen} region\textsc{.m-ez.gen} there\textsc{=post} \\ 
\glt `They (the king and his attendants) were taken to a region far from their own.' \hfill[KŞ.48]
\z 

\section{Interrogative clauses}\label{sect:interr}
The list of interrogative pronouns\is{interrogative pronouns} and their morphological forms were discussed in \S\ref{secr:interogative_pronoun}. This section gives a brief overview of interrogative clauses.

\subsection{Polar questions}\label{sect:polarq}
Polar questions are expressed by a rise in intonation towards the end of the clause. No specific question particle is used with polar questions.

\ea

\ea{
\textit{maça, `şanat tomeke?’} \\ 
\gll m-aç-a şana=t tom-eke \\ 
 \textsc{ind-}say\textsc{.prs-3pl:A} scatter\textsc{.pst}\textsc{=2sg:A} seed\textsc{-def.m.sg.dir}\\ 
\glt `They would say, ‘Did you plant the seeds?’'
}

\ex{ 
\textit{maço, `erê.'} \\ 
\gll m-aç-o erê \\ 
 \textsc{ind-}say\textsc{.prs-3sg:A} yes \\ 
\glt `He would say, ‘Yes.’' \hfill[JP.39]--[JP.40]
}
\z 
\z 

\subsection{Constituent (Wh-) questions\is{constituent (Wh-) question}}
Constituent (Wh-) questions\is{constituent (Wh-) question} are expressed by the interrogative pronouns\is{interrogative pronouns} set out in \S\ref{secr:interogative_pronoun}. Most interrogative phrases are not obligatorily clause-initial, as suggested by the following examples. This profile is shared by the languages in the region \citep[]{wals-93}{}.

\ea
\textit{maço, `to kênî?'} \\ 
\gll m-aç-o to \textbf{kê}=nî \\ 
 \textsc{ind-}say\textsc{.prs-3sg:A} \textsc{2sg} who\textsc{=cop.2sg:S} \\ 
\glt `He (the king) said, ‘Who are you?’' \hfill[KŞ.10]
\z


\ea
\textit{ey wêm kowe bilû?} \\ 
\gll ey wê=m \textbf{ko=we} bi-l-û \\ 
 \textsc{intj} \textsc{refl=1sg:PSR} where\textsc{=post} \textsc{sbjv-}go\textsc{.prs-1sg:S} \\ 
\glt `Where should I go myself?' \hfill[DP.23]
\z 


\ea
\textit{maça, `ême çî bilmêre ʕêraq wêma kermê zînan?'} \\ 
\gll m-aç-a ême \textbf{çî} bi-l-mê=re ʕêraq wê=ma k\'er-mê zînan \\ 
 \textsc{ind-}say\textsc{.prs-3pl:A} \textsc{1pl} why \textsc{sbjv-}go\textsc{.prs-1pl:S=povb} \textsc{pn} \textsc{refl=1pl:PSR} do\textsc{.prs.sbjv-1pl:A} prison\textsc{.m} \\  
\glt `They (the people) said, ‘Why should we go to Iraq and put ourselves in prison?’' \hfill[BP.101]
\z 

Constituent (Wh-) questions\is{constituent (Wh-) question} may be introduced by the question particle \textit{daxom}, expressing the speaker's uncertainty regarding the context of the question.

\ea
\textit{watşa, `daxom kê etik kerđêbo?'} \\
\gll wat=şa daxom kê etik kerđê=b-o \\
say.\textsc{pst=3pl:A} \textsc{q.ptcl} who offence do.\textsc{pst.ptcp.f=}be.\textsc{prs-3sg:O} \\
\glt `They said, `Who might have possibly offended her?’' \hfill[ED.195]
\z 

The attention-drawing particle \textit{erê} may be used with interrogative pronouns\is{interrogative pronouns} like \textit{çî} `what' to express an event which is unexpected from the speaker's point of view.

\ea
\textit{çiřoş hezretû şêxî fermawo, `erê řoɫem çî toryanî?'} \\ 
\gll çiř-o=ş hezret-û şêx-î fermaw-o erê řoɫe=m çî torya=nî? \\ 
 call\textsc{.prs.ind-3sg:A=3sg:O} his\_highness\textsc{.m-ez.gen} sheikh\textsc{.m-sg.obl} say\textsc{.prs.ind-3sg:A} \textsc{disc.ptcl} child\textsc{.m=1sg:PSR} why get\_offended\textsc{.pst.ptcp.m=cop.2sg:S} \\ 
\glt `His Highness the Sheikh summoned him [and] said, ‘Oh my son! Why were you offended?’' \hfill[HB.80]
\z 

In (\ref{ex.ere1}), the particle \textit{erê} has been used in a polar question, where it seems to express the speaker's uncertainty. Note that polar questions do not require an interrogative particle, as discussed in \S\ref{sect:polarq}.

\ea \label{ex.ere1}
\textit{erê řoɫê pîyake her îna mizgîne cuwaneke?}\\
\gll erê řoɫê pîya-(e)ke her îna mizgî=ne cuwan-eke\\
\textsc{q.ptcl} child man\textsc{-def.m.sg.dir} still \textsc{loc.deic.cop-3sg.m:S} mosque\textsc{.m=post} youth\textsc{-def.m.sg.dir}\\
\glt `Child, is the man, the youth, still in the mosque?' \hfill[HS.61]
\z 

\end{sloppypar}
